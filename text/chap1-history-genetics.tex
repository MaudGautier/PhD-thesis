\begin{savequote}[8cm]
“Our species, from the time of its creation, has been travelling onwards in pursuit of truth; and now that we have reached a lofty and commanding position, with the broad light of day around us, it must be grateful to look back on the line of our past progress; — to review the journey.”
	
\qauthor{--- William Whewell, \textit{\usebibentry{whewell1837history}{title}} \citeyearpar{whewell1837history} }
\end{savequote}

\chapter{\label{ch:1-history-genetics}A geneticist's history of genetics} 
%\otherpagedecoration

\minitoc{}



% Nouvelle decouverte donc champ disciplinaire se regroupe sous des nouveaux axiomes, voire se crée
% Le cas pour la genetique: les pioneering work de Mendel ont amene a de nouvelles facon de penser l'hérédité (à opposer a Lamarck)

Grand scientific discoveries sometimes lead a research field to completely reorganise around new principles or axioms. 
This was the case with the comprehension of heredity. 
Up until the late nineteenth century, the inheritance of acquired characters — the idea that an organism can transmit features that it has acquired through use or disuse during its lifetime to its progeny, — was a supposedly well-established fact that had been accepted by a plethora of philosophers and scientists, starting with Hippocrates (c. 460--c. 370 BC) \citep{zirkle1935inheritance}.
However, Mendel's pioneering work on hybridization questioned the latter paradigm and shaked the scientific community so well that it ended in the creation of a brand-new field in biology: genetics — which was first institutionalized in 1906 \citep{gayon2016mendel}.

In this chapter, I will review the main events of the genetics era that led to the concepts of recombination and gene conversion, which are of major interest for this thesis. \textbf{ET, SI JE PARLE DE DARWIN ADAPTATIONNISME A LA FIN, RAJOUTER QUE JE PARLE AUSSI DE CA}.
A reader who is not familiar with the vocable of recombination (such as “meiosis”, “gene conversion”, “post-meiotic segregation”, “negative interference”, etc…) may find this chapter slightly difficult, as these denominations will not be fully detailed here. I therefore send them back to the definitions at the beginning of this thesis, or to the subsequent chapters of this introduction where the terms will be fully described, whenever they come across one of them.

The historical developments that one can appreciate are nothing but the result of what was transmitted to us by our predecessors and I therefore entitled this chapter \textit{A geneticist's history of genetics} as a wink to what Richard Feynman (1918--1988), one of the most influential physicists of his time, wrote on this subject in his famous book on quantum physics \textit{\usebibentry{feynman2006qed}{title}}:

\begin{quote}
	\textit{‘By the way, what I have just outlined is what I call a “physicist’s history of physics,” which is never correct. What I am telling you is a sort of conventionalized myth-story that the physicists tell to their students, and those students tell to their students, and is not necessarily related to the actual historical development, which I do not really know!’}
\end{quote}






\section{Emergence of the concept of recombination}

\subsection{An abstruse exception to Mendel's laws of heredity}

Between 1857 and 1864, the Austrian monk Johann Gregor Mendel (1822--1884) undertook a series of hybridisation experiments on the garden pea plant \textit{Pisum sativum}. This led him to describe the idea of an “independent assortment of traits” \citep{mendel1996experiments}, thereby proving the existence of paired “elementary units of heredity” (i.e.\ genes) and establishing the statistical laws governing them.
His work remained unrecognised by the scientific community for several decades but was finally rediscovered in the early twentieth century when three botanists (Hugo de Vries (1845--1935), Carl Correns (1864--1963) and Erich von Tschermak (1871--1962)) independently confirmed his findings \citep{dunn2003gregor}.
Meanwhile, William Bateson (1861--1926) fiercely defended Mendel's thesis in \textit{\usebibentry{bateson1902mendel}{title}} \citep{bateson1902mendel} against his contemporary biometricians \citep[reviewed in][]{bateson2002william}, thus spreading Mendel's view into the scientific world.\\
% Meanwhile, as Mendel's attempts to explain the mechanisms of heredity lacked scientific support and contradicted Galton's “Law of Ancestral Heredity”, many scientists were skeptiks

A few years later, Bateson noticed exceptions to Mendel's principles of independent assortment: some crosses generated certain phenotypes in far excess from the expected Mendelian ratios \citep{bateson1905experimentalpea}. This led him and his collaborators to propose that certain traits were somehow coupled with one another, although they did not know how \citep{bateson1905experimental}. 
%They had discovered what is now called “genetic linkage”.

% Conclure: methode d'exploration de la genetique nee + desequilibre de liaison = on voit des differences aux lois donnees par Mendel. donc a comprendre.

% \textbf{Ajouter la decouverte de l'ADN comme support de l'information genetique?}

% \citet{suzuki1986introduction}
% \paragraph{Genetic information is encoded in DNA.}
% In the meantime,

% PLAN: DNA SUPPORT DE L'INFORMATION GENETIQUE:
% 1869: Friedrich Miescher: discovery of nuclein (=DNA) + Altmann 1889: named “nucleid acid”
% + 1883: Eduard Van Beneden: discovery of chromatin (=DNA+RNA+proteins that make up a chromosome)
% 1929: Phoebus Levene: DNA composed of four bases (ATCG)+phosphate+sugar
% + 1941: Chargaff's rule: A=T and c=G
% + 1953: James Watson and Francis Crick: model of double-helix DNA






\subsection{The chromosomal theory of inheritance}

In the meantime, it had been understood that cells derived from other cells, but the exact process was unknown. 
To understand it, Walther Flemming (1843--1905) used stains to intensify the contrasts of cell contents observed through microscopy and identified a substance located within the nucleus, which he named “chromatin” (from the greek word \textit{\textchi\textrho\textomega\textmu\textalpha}: “color”).
He described precisely the movements of chromosomes during cell division (which he termed “mitosis”), thus providing a mechanism for the distribution of nuclear material into daughter cells during mitosis \citep{flemming1879contributions}.

Theodor Boveri (1862--1915) went one step further by demonstrating the individuality of chromosomes in the roundworm \textit{Ascaris megalocephala}, which allowed him to suggest that the chromosomes of the germ cells are involved in heredity \citep{boveri1888zellen}.
In addition, he showed that the egg and the spermatozoon contribute the same number of chromosomes to the new individual, thus providing the first descriptions of meiosis \citep{boveri1890zellen}.
Walter Sutton (1877--1916) independently came to the same conclusion at about the same time: he enunciated the chromosomal theory of inheritance with the following words closing his \usebibentry{sutton1902morphology}{year} paper: “I may finally call attention to the probability that the association of paternal and maternal chromosomes in pairs and their subsequent separation during the reducing division […] may constitute the physical basis of the Mendelian law of heredity” \citep{sutton1902morphology}.

However, this theory was debated in the scientific community, because there was yet no direct proof of a link between the inheritance of traits and the segregation of chromosomes.\\ 

In parallel, based on cytological observations of chromosomes, Frans Janssens (1863--1924), a priest also known as the “microscopy wizard” for he mastered the process, developed the idea that the chromosomes' “filaments [chromatids] are involved in contacts that can modify their organization from one segment to the next” which “will generate new segmental combinations” in his \textit{Chiasmatype Theory} \citep{janssens1909theorie}.



\subsection{Morgan's theory of gene linkage and crossing-over}
% Morgan ne croit à aucune des theories. Pourtant, va etre le lien de ces trois. 
% Morgan decouvre une mutation chez la droso (explication globale dessus) => suspecte la liaison.
% Il decrit le mecansisme dans livre 1915.
% 
In 1909, Thomas Hunt Morgan (1856--1945) expressed his strong skepticism of the Mendelian theory of inheritance in his very derisive article \textit{\usebibentry{morgan1909factors}{title}} \citep{morgan1909factors} and doubted the chromosomal basis of heredity \citep[reviewed in][]{koszul2012centenary}.
Little did he know at the time that he was to become the main craftsman of the reconciliation of these two theories.\\




\begin{figure}[p]
	\centering
	\includegraphics[width = 1\textwidth]{figures/chap2/morgan-drosophila-cross-results.eps}
	% \includegraphics[width = 0.7\textwidth, trim = 0cm 0cm 11.65cm 0cm, clip]{figures/chap2/morgan-drosophila-cross-results.eps}
	\caption[Explanation of the results from reciprocal crosses between red-eyed and white-eyed \textit{Drosophila}]
	{\textbf{Explanation of the results from reciprocal crosses between red-eyed (red) and white-eyed (white) \textit{Drosophila}.} 
		\par In the first cross (left), a red-eyed purebred female is crossed with a white-eyed male, resulting in F1 hybrids made of heterozygous red-eyed females bearing both the dominant (w\textsuperscript{+}) and the recessive (w) alleles and red-eyed males bearing only the dominant (w\textsuperscript{+}) allele. The inbreeding of F1 individuals results in a F2 generation with a 3:1 ratio of red-eyed:white-eyed individuals, all white-eyed individuals being males.
		\par In the second cross (right), a white-eyed female is crossed with a red-eyed purebred male, resulting in F1 hybrids made of heterozygous red-eyed females bearing both the dominant (w\textsuperscript{+}) and the recessive (w) alleles and white-eyed males bearing only the recessive (w) allele. The inbreeding of F1 individuals results in a F2 generation with a 2:2 ratio of red-eyed:white-eyed individuals, half of white-eyed being males and half being females.
		\par The results of these two crosses show that the gene coding for eye color is located on the female sexual chrosome (X). The fact that results in the F2 progeny differ according to the direction of the cross (($\frac{w\textsuperscript{+}}{w\textsuperscript{+}}$) $\times$ (w) or ($\frac{w}{w}$) $\times$ (w\textsuperscript{+})) is a typical signature of linkage disequilibrium between the observed trait (eye color) and the sex chromosomes.
		\par This figure was reproduced from \citet{suzuki1986introduction} (Rights: WH Freeman \& Co).
	}
\label{fig:morgan-drosophila-cross-results}
\end{figure}

% \clearpage




In his famous “fly room” where he bred \textit{Drosophila melanogaster} fruit flies, he found an unusual male white-eyed individual. Crossing it with purebred red-eyed females yielded red-eyed males and females F1 hybrids, — a typical result proving that the white eye color is a recessive trait. Unexpectedly, after inbreeding the heterozygous F1 progeny, he discovered that the traits of the F2 offspring did not assort independently: all white-eyed flies were males (Figure~\ref{fig:morgan-drosophila-cross-results}, left). 
However, when he crossed the white-eyed male with F1 daughters, he found both male and female white-eyed flies (Figure~\ref{fig:morgan-drosophila-cross-results}, right), thus showing that the white eye color was not lethal for females.

He immediately hypothesised that eye color was connected to the sex determinant \citep{morgan1910sex} and, as these findings were consistent with the idea that genes were physical objects located on chromosomes, Morgan soon came up with the idea of genetic linkage, i.e.\ the fact that two genes closely associated on a chromosome do not assort independently \citep{morgan1911random}. 
He also suggested that this coupling dependended on the distance between genes: “we find coupling in certain characters, and little or no evidence at all of coupling in other characters; the difference depending on the linear distance apart of the chromosomal material that represent the factors.”\\

With three of his students (Alfred Sturtevant (1891--1970), Hermann Muller (1890--1967) and Calvin Bridges (1889--1938)), he summarized all the evidence in \textit{\usebibentry{morgan1915mechanism}{title}} which constitutes one of the most important books in the whole history of genetics \citep{gayon2016mendel}.
There were two major propositions in that book. 

First, the recognition that Mendelian factors — Morgan would soon call them “genes” — are physical portions of chromosomes. This brought a mechanistic support to Mendel's “law of segregation” (according to which the zygote inherits only one version of each gene from each parent) and to the so far unexplained exception to Mendel's “law of independent assortment fo traits”: when two genes are located on the same chromosome, they have to segregate together — and thus the law does not apply to this special case. 

Second, they proposed that the linkage between genes located on the same chromosome could sometimes break, through the process of what Morgan called “crossing-over” (Figure~\ref{fig:morgan-CO}). This was to take place at the positions of the chiasmata previously observed by Janssens \citep{janssens1909theorie}. Later, Edgar Wilson (1908--1992) and Morgan crafted structures of crossing-overs with clay to materialise how the crossing-over could physically form \citep{wilson1920chiasmatype}.\\

Altogether, with the ideas of recombination and crossing-over, Morgan had fused three theories: gene linkage (the major exception to Mendel's laws of heredity), the chromosomal theory of inheritance and the chiasmatype theory. This triggered a real revolution in biology and marked the commencement of genetics. His major contribution through his work on \textit{Drosophila} won him the \textit{Nobel Prize in Physiology or Medicine} in 1933.

It was only ten years later that Harriet Creighton (1909--2004) and Barbara McClintock (1902--1992) would bring the first proof of that theory by correlating cytological and genetic exchanges in maize \citep{creighton1931correlation}.

\begin{figure}[h]
	\centering
	\includegraphics[width = 1\textwidth]{figures/chap2/morgan-CO-1916.eps}
	% \includegraphics[width = 0.7\textwidth, trim = 0cm 0cm 11.65cm 0cm, clip]{figures/chap2/morgan-drosophila-cross-results.eps}
	\caption[Original drawing of crossing over in \textit{\usebibentry{morgan1915mechanism}{title}} \citep{morgan1915mechanism}]
	{\textbf{Original drawing of crossing over in \textit{\usebibentry{morgan1915mechanism}{title}} \citep{morgan1915mechanism}.} 
		\par Original legend by the authors: “At the level where the black and the white rod cross in A, they fuse and unite as shown in   D. The details of the crossing over are shown in B and C.”
		\par This drawing symbolises the reconciliation between Mendel's and the chromosomal theories of inheritance.
	}
\label{fig:morgan-CO}
\end{figure}

% \clearpage


% Conclure: on comprend que il existe des chromosomes et des recombinaisons, qui expliquent donc les deviations par rapport a l'attendu de Mendel. Le concept de recombinaison est ne.











\section{Emergence of the concept of gene conversion}

\subsection{The study of fungal products of meiosis}


% Why fungi
The next major advances on the comprehension of the recombination mechanism were to come through the study of fungi, soon adopted as model organisms for the multiple advantages they confer to genetics reseach.
First, as they take up little space and are easy and cheap to propagate, they can be studied in very large numbers.
% Second, it was reported early that they alternate haploid and diploid phases, i.e.\ phases with respectively one and two sets of chromosomes.

Second, it was reported early that they alternate haploid\footnote{Single set of chromosomes} and diploid\footnote{Two sets of chromosomes} phases. 
Indeed, the Czech scientist Jan \u{S}atava (1878--1938) managed to isolate the ascospores\footnote{Reproductive cells of a certain class of fungi (ascomycetes)} of a yeast and saw that they germinated without fusing other ascopores, thus giving rise to haploid cultures \citep[\cite{satava1918pohlavn}, reviewed in][]{barnett2007history}.
This feature, — haploidy of the progeny, — considerably facilitates the interpretation of the products of meiosis since the phenotype of each offspring is a direct manifestation of its genotype (contrary to diploid or higher-order of ploidy cases for which dominance and recessiveness may blur gene expression).

Third, in some fungi, the cells corresponding to the four products of meiosis remain grouped in a tetrad of four sexual spores, which makes the direct observation of a single meiosis possible.
The first study of this type, — a “tetrad analysis”, — was achieved by {{\O}}jvind Winge (1886--1964), the founder of yeast genetics \citep{winge1937two}.
In some ascomycetes, the meiotic products undergo one additional mitotic division, thus ending in “octads” of four pairs of identical spores (Figure~\ref{fig:spores-formation}). 

Last, in certain fungi, the spindles of the meiotic (and mitotic, if applicable) divisions are constrained in a tube-shaped ascus preventing them from overlapping, which leads the tetrads (or octads) to arrange linearly, and makes the interpretation of the behaviour of genes during meiosis (and mitosis) straightforward (Figure~\ref{fig:spores-formation}) \citep{casselton2002art}.


%Figure de: https://sites.google.com/site/portafolioelectronicobiologia/home/laboratorio-3-mitosis-y-meiosis
\begin{figure}[p]
	\centering
	\includegraphics[width = 1\textwidth]{figures/chap2/spores-formation.eps}
	% \includegraphics[width = 0.7\textwidth, trim = 0cm 0cm 11.65cm 0cm, clip]{figures/chap2/morgan-drosophila-cross-results.eps}
	\caption[Meiotic and post-meiotic mitotic segragations of chromosomes in a linear ascomycete tetrad]
	{\textbf{Meiotic and post-meiotic mitotic segragations of chromosomes in a linear ascomycete tetrad.}
		\par \textbf{REFAIRE MOI MEME EN AJOUTANT LES SPINDLES ET MONTRER LA LINEARITE}.
		\par shows results for a colored spore with or without crossing-over.
		\par from https://sites.google.com/site/portafolioelectronicobiologia/home/laboratorio-3-mitosis-y-meiosis
	}
\label{fig:spores-formation}
\end{figure}


% % Image neurospora - http://bioinfo.townsend.yale.edu/
% \begin{figure}[h]
%     \centering
%     \includegraphics[width = 0.75\textwidth]{figures/chap2/neurospora.eps}
%     \caption[Example of a \textit{Neurospora crassa} rosette of tetrads]
%     {\textbf{Example of a \textit{Neurospora crassa} rosette of tetrads.}
%         \par \textbf{VOIR SI JE GARDE OU PAS}.
%         \par Image from http://bioinfo.townsend.yale.edu/.
%     }
% \label{fig:neurospora-spores}
% \end{figure}
%


All these attributes and technical achievements rendered fungi superior model organisms for the study of recombination. And there began the dawn of the fungal genetics era.







\subsection{Four novel phenomena associated to recombination}

% Conversion genique
\subsubsection{Gene conversion}
% https://books.google.fr/books?id=CPzjqsU0CsgC&pg=PA423&lpg=PA423&dq=winckler+1930+gene+conversion&source=bl&ots=VER-oNiCs7&sig=ACfU3U2DWKklVaw1sqn5Tolg20NxH70OTQ&hl=fr&sa=X&ved=2ahUKEwiJz5aC99PgAhUHuRoKHWMzAyUQ6AEwAnoECAQQAQ#v=onepage&q=winckler%201930%20gene%20conversion&f=false

% Using them, Hans Winkler (1877--1945) observed 3+:1- and 1+:3- departures from the expected Mendelian segragation among tetrads of +/- diploids and invented the term “gene conversion” to describe such events \citep[\cite{winkler1930konversion}, reviewed by][]{roman1985gene}.
% This observation was later confirmed by Carl Lindegren (1896--1987), a former student of Morgan's, who obtained similar irregular ratios with frequencies of about 1\% in the budding yeast \textit{Saccharomyces cerevisiae} \citep{lindegren1953gene} as well as by Mary Mitchell (fl. 1950--1965) who found 2:6 segregations\footnote{A 2:6 segregation in the eight-spored \textit{Neurospora} ascus is equivalent to a 1:3 segregation in the four-spored ascus of the budding yeast \textit{Saccharomyces cerevisiae}.} of wild-type:recessive phenotypes in \textit{Neurospora} \citep{mitchell1955aberrant, mitchell1955further}.
% Such observation means that the information present on one chromatid is replaced by that of another chromatid \citep{orr1985fungal}.
% Since Winkler originally wrongly described gene conversion as a mutational mechanism (instead of a recombinational one), some authors suggested other nomenclature \citep[e.g.][]{roman1986early}, but his term persisted over the years and the concept it represented was substituted to a recombinational mechanism.\\
%

Using them, Hans Winkler (1877--1945) observed 3+:1- and 1+:3- departures from the expected Mendelian segragation among tetrads of +/- diploids \citep[\cite{winkler1930konversion}, reviewed in][]{roman1985gene}, which meant that the information present on one chromatid was replaced by that from another chromatid \citep{orr1985fungal}.
% He hypothesized that a mutational mechanism was at the origin of this replacement and invented the term “gene conversion” to describe it.

This observation was later confirmed by Carl Lindegren (1896--1987), a former student of Morgan's, who obtained similar irregular ratios with frequencies of about 1\% in the budding yeast \textit{Saccharomyces cerevisiae} \citep{lindegren1953gene} as well as by Mary Mitchell (fl. 1950--1965) who found 2:6 segregations\footnote{A 2:6 segregation in the eight-spored \textit{Neurospora} ascus is equivalent to a 1:3 segregation in the four-spored ascus of the budding yeast \textit{Saccharomyces cerevisiae}.} of wild-type:recessive phenotypes in \textit{Neurospora} \citep{mitchell1955aberrant, mitchell1955further}.\\

Originally, Winkler had hypothesized that a mutational mechanism was at the origin of this replacement and invented the term “gene conversion” to describe it. Although his interpretation turned out to be wrong (the mechanism is in fact purely recombinational, not mutational) and some authors suggested alternative nomenclature for it \citep[e.g.][]{roman1986early}, the term he had come up with persisted over the years and is still used today.




\begin{figure}[h]
	\centering
	\begin{subfigure}[b]{\textwidth}
		\includegraphics[width=1\textwidth]{figures/chap2/Olive-6to2.eps}
		\caption{Ascus containing 6 black and 2 white ascospores: gene conversion.}
\label{fig:sub-olive-6to2}
	\end{subfigure}

	\begin{subfigure}[b]{\textwidth}
		\includegraphics[width=1\textwidth]{figures/chap2/Olive-5to3.eps}
		\caption{Ascus containing 5 black and 3 white ascospores: post-meiotic segregation.}
\label{fig:sub-olive-5to3}
	\end{subfigure}

	\caption[Original photographies of aberrant octads in \textit{Sordaria fimicola} \citep{olive1959aberrant}]
	{\textbf{Original photographies of aberrant octads in \textit{Sordaria fimicola} \citep{olive1959aberrant}.}
	}
\label{fig:olive-octads}
\end{figure}




% PMS
\subsubsection{Post-meiotic segregation}

Soon after, Lindsay Olive (1917--1988) observed another type of aberrant segregation in the octads of \textit{Sordaria fimicola}: 5:3 segregation ratios (Figure~\ref{fig:sub-olive-5to3}) \citep{olive1959aberrant, kitani1962genetics}.
This result was puzzling, since it was not congruent with the models so far: 6:2 segregations were explainable on the basis of a non-directional transfer of information from one chromatid to another one, but this sole explanation could not account for the 5:3 segregation ratios. 
However, these results were totally reconciliable with the concept of a chromatid composed of two functional subunits, which had been proposed after autoradiographic studies on DNA \citep{taylor1957organization} in accordance with the Watson-Crick model of DNA \citep{watson1953molecular}.

This feature was again observed in \textit{Neurospora crassa} concomitantly with the finding that several alleles were converted concertedly \citep{case1964allelic}. 
Such co-conversion of alleles was also found in \textit{S. cerevisiae}, together with the finding that the frequency of co-conversion decreases with increasing distance between the alleles \citep[\cite{fogel1969informational}, reviewed by][]{orr1985fungal}.\\

% At the same time, co-conversion of several alleles in addition to post-meiotic segregation\footnote{Segregation occuring after the end of meiosis, during the mitotic division (Figure~\ref{fig:spores-formation})} of the sites was observed in \textit{Neurospora crassa} \citep{case1964allelic}.
% Such co-conversion of alleles was also found in \textit{S. cerevisiae}, together with the finding that the frequency of co-conversion decreases with increasing distance between the alleles \citep[\cite{fogel1969informational}, reviewed by][]{orr1985fungal}.

Altogether, these findings indicated the presence of “heteroduplex DNA”, i.e.\ a DNA portion where the two strands composing it contain different information for the segregating marker.
Such heteroduplex DNA cannot be detected genetically until an additional round of DNA replication produces two duplexes, each expressing the information from one of the strands of the heteroduplex.
These segregations, occuring after the end of meiosis, are called “post-meiotic segregations” (PMS).
The additional observation that markers are co-converted at frequencies dependent on their distance suggested that heteroduplex DNA (and thus, gene conversion) could span hundreds of nucleotides \citep{orr1985fungal}.





% Conversion polarity
\subsubsection{Conversion polarity}

In addition, it was found that gene conversion frequencies vary linearly from one end of a gene to the other \citep[reviewed in][]{nicolas1994polarity}: this discovery was made in both \textit{Ascobolus immeraus} \citep{lissouba1960existence,lissouba1962fine} and in \textit{Neurospora crassa} \citep{murray1960complementation} at approximately the same time. 
This phenomenon was observed again in \textit{Aspergillus nidulans} \citep{siddiqi1962mutagenic} and in other mutants of \textit{Neurospora} \citep{stadler1963recombination}, and was designated as “conversion polarity” or “polarized recombination”.

Later, one of its discoverers, Lady Noreen Murray (1935--2011) demonstrated that this polarity was due to elements located close to the gene, as opposed to being imposed by the orientation of the gene with respect to the centromere \citep{murray1968polarized}. This led to the idea that recombination initiates on “pseudofixed sites”, the erstwhile concept for what we now call “recombination hotspots”.
% This finding later led to the idea that recombination initiated from pseudofixed sites (which we now call “recombination hotspots”).\\





% Negative interference
\subsubsection{Negative interference}

One last important observation made during this decade came from a study on \textit{Aspergillus nidulans} \citep{pritchard1955linear}.
The authors looked at four linked marker genes, whose recessive alleles will here be designated as ‘y’, ‘11’, ‘8’ and ‘bi’, and whose dominant alleles will here be designated as ‘+’ in all four cases. 
%The authors looked at four linked marker genes, for which I will note the four recessive alleles ‘y’, ‘11’, ‘8’ and ‘bi’, and the dominant alleles ‘+’ in all four cases. 
They crossed a strain of genotype (y+8+) with a strain of genotype (+11+bi) to obtain a F1 hybrid of genotype ($\frac{y}{+}\frac{+}{11}\frac{8}{+}\frac{+}{bi}$) and found that the largest proportion of recombinants from this hybrid was of genotype (y++11), while all other combinations ((y+++), (+++bi) and (++++)) were under-represented \citep[reviewed in][]{whitehouse1965crossing}.
Similar observations of this phenomenon were made in \textit{Neurospora crassa} \citep{mitchell1956consideration}.

These observations suggested that recombination between alleles (in this case, between the second and third marker) are negatively associated with recombination in neighbouring regions (in this case, between the first and second, and between the third and fourth markers). This feature was designated as “negative interference”.







\subsection{The first theories on the recombinational mechanism}

% Correaltion between gene conversion and PMS donc processus mechanistically linked.

% CONCLUSION: lien et tehories qui vont suivre.
To sum up, over the course of the 1950's and of the early 1960's, numerous studies evidenced that crossing-over was associated with gene conversion, PMS, polarized recombination and negative interference.

It was soon proposed that all these processes were somehow mechanistically linked \citep{perkins1962frequency} and from that point on, several scientists conjectured theories reuniting these observations.
One important one, the “copy-choice hypothesis”, was postulated by Joshua Lederberg (1925--2008) \citep{lederberg1955recombination}. According to this (wrong) theory, the process of replication switches from copying one parental chromosome to the other — the switch occuring when both chromosomes are closely paired.
An alternative hypothesis, “the hybrid DNA hypothesis”, was proposed \citep{whitehouse1963theory}, allegedly inspired from the model of Robin Holliday (1932--2014) \citep{holliday2011recombination} which the latter would publish the following year \citep{holliday1964mechanism}.

The Holliday model \citep{holliday1964mechanism,holliday1968genetic}, which was in accordance with the then recent discovery of the double-stranded structure of DNA \citep{franklin1953molecular,watson1953molecular}, happened to be the first widely accepted molecular explanation for the phenomena with which crossing-over had been found to be associated, namely aberrant segregation (i.e.\ gene conversion and PMS) and polarized recombination. 
Briefly, this model rested on the formation of a DNA break, separation of the two DNA strands and base pairing with complementary segments to form symmetric heteroduplex DNA, and the formation and resolution of a Holliday junction.\\

%Evidence in favour of correction of heterozygous DNA: \citep{kitani1962genetics}: frequency of aberrant lower than expected by this model.
% Evidence for DNA synthesis at pachytene, (as postulated on this hypothesis of crossing-over) found with phosphorylated thymine (Hotta and Stern 1961).

% With this, evident that there was a connection between segregation and crossing-over. (because, in 48 instances out of 52 when the chromatif involved in aberrant segregation could be identified, it also took part in the crossing-over. — reviewed by whitehouse)

\textbf{PARLER AUSSI DE MESELSON-RADDLING ET DE DSBR??}
\textbf{Voir Szostak}
\textbf{FAIRE AUSSI UNE MINI PARTIE SUR DARWIN / IDEES EN EVOLUTION AVEC NEUTRALISME VS ADAPTATIONNISME, CAR THESE SUR GENOMIQUE EVOLUTIVE}
%file:///Users/maudgautier/Documents/These/_RANGE/00_PhD_dissertation/bibliography_additional/Chap2/03_Models/Szostak-1983.pdf
% CONCLUSION FIN DE CHAPITRE
\textbf{CE PARAGRAPHE DE CONCLUSION CLIAREMENT A REVOIR, en fonction de si je parle de Meselson-raddling ou pas, et de si je parle de evolution ou pas.}
While research carries on, new evidence is found. 
Some of these have invalidated the Holliday model, and several other models saw the day to account for the new findings. I will review them in the next chapter.
Hopefully, future progress in the field will allow to definitively validate one of them.

Although much progress has been made altogether on the understanding of recombination, it is curious that the main points that remain to clarify concern some of the first to be ever identified: pairing, crossing-over, segregation…


\textbf{FIN DU CHAPITRE 1}

% Even though much progress have been made altogether on the understanding of recombination, it is funny to note that the main remaining points to clarify concern

%Although a great deal is known about meiosis, there is still much to be learned. In fact, it is curious that, despite the wealth of molecular knowledge that is available on how cells carry out their day-to-day affairs, the processes that were some of the first ever to be identified by geneticists (pairing, crossing-over, and segregation) remain among the most mysterious of cellular processes.

% Mais, toujours des incertitudes importantes. Dans les prochains chapitres, je vais revenir plus en detail sur l'ensemble des phenomenes decrits ici. Afin, dans mes resultats de mieux decrire ces phenomenes manquants.

% Les modeles toujours que des modeles (cf partie 3)
% Encore bcp de points d'interrogation sur la meiose et la recombinaison et le role precis des molecules impliquees. Donc toujours de la recherche dessus.
% En particulier, le BGC qu'on etudie dans cette these
% Role de la recombinaison comme reparation des cassures pour replication  (cf topo Daniel)


% Preciser dans l'introduction de la partie Historique de la recombinaison au chapitre 1 que j'explique la naissance des differents termes ici. Mais qu'ils seront explicites au cours des chaptires qui suivent: modeles de recombinaison, conversion genique, meiose…































% Et whitehouse Gene conversion and crossovers. dire que mene a des modeles de recombinaison a deux process, ou on a gene conversion et segregation en deux etapes









% %%%%% Mon plan (Ordonne par theme)
% %HEREDITY:
% 1865: Mendel: experience des pois a expliquer.
% + Bateson 1905: linkage diesquilibrium in certain cases. (ou dans meiose et recombinaison)
%
% %DNA SUPPORT DE L'INFORMATION GENETIQUE:
% 1869: Friedrich Miescher: discovery of nuclein (=DNA) + Altmann 1889: named “nucleid acid”
% + 1883: Eduard Van Beneden: discovery of chromatin (=DNA+RNA+proteins that make up a chromosome)
% 1929: Phoebus Levene: DNA composed of four bases (ATCG)+phosphate+sugar
% + 1941: Chargaff's rule: A=T and c=G
% + 1953: James Watson and Francis Crick: model of double-helix DNA
%
% %MEIOSE AND RECOMBINATION:
% 1911/1913 Thomas Hunt Morgan: concept de linkage desiquilibrium + crossover. Overall, fusion of three theories: “the chromosome theory of inheritance,” imparted by Wilhelm Roux, Walther Flemming, Theodor Boveri, and Walter Sutton; “gene linkage,” an exception to Mendel’s law of independent assortment, first reported by Carl Correns; and the “chiasmatype theory,” derived from Frans Janssens' cytological observations of meiotic chromosomes.
% + Les trois autres theories avant.
% 1931: Proof of the crossover theory came from Harriet Creighton and Barbara McClintock (Creighton and McClintock 1931), who were able to correlate cytological and genetic exchanges in maize.
%
% %CONSEQUENCES(BGC):
% fungal genetics experiments (to get all four products of meiosis)
% our key concepts that formed the foundation of molecular models of recombination:gene conversion, an exception to Mendel’s principle of segregation, signaled a local nonreciprocal transfer of genetic information (Winkler 1930; Lindergren 1953; Mitchell 1955); postmeiotic segregation (PMS) indicated the presence of heteroduplex DNA (Olive 1959; Kitani et al. 1962); polarity gradients of gene conversion lead to the idea that recombination initiated from pseudofixed sites (Lissouba and Rizet 1960; Murray 1960); and the strong correlation between gene conversion/PMS events and crossing-over led to the proposal that these processes were mechanistically linked (Kitani et al. 1962; Perkins 1962; Whitehouse 1963).
%
% %MINI CONCLU:
% Les modeles toujours que des modeles (cf partie 3)
% Encore bcp de points d'interrogation sur la meiose et la recombinaison et le role precis des molecules impliquees. Donc toujours de la recherche dessus.
% En particulier, le BGC qu'on etudie dans cette these
% Role de la recombinaison comme reparation des cassures pour replication  (cf topo Daniel)

