\begin{savequote}[8cm]

	‘When Oldspeak had been once and for all superseded, the last link with the past would have been severed.’
	\qauthor{--- George Orwell, \textit{\usebibentry{orwell1949nineteen}{title}} \citeyearpar{orwell1949nineteen} }

\end{savequote}

\chapter*{\label{ch:preamble}}
\fancyhead[LO]{\emph{Preamble}}
\fancyhead[RE]{\emph{Preamble}}

% Add blank lines before starting
\begin{parse lines}[\noindent]{#1\\}



\end{parse lines}

While Charles Darwin (1809--1882) was advocating an evolutionary interpretation of vestigial structures\footnote{A vestigial structure is an anatomical feature or behaviour that has lost part or all of its initial function and that thus no longer seems to have a purpose in the current species. For instance, the human appendix and coccyx are two such vestigial organs.}
in his groundbreaking opus \textit{On the Origin of Species} \citeyearpar{darwin1859origin}, he drew a parallel between the work of linguists and that of evolutionary biologists:

\begin{quote}
	\textit{‘Rudimentary organs may be compared with the letters in a word, still retained in the spelling, but become useless in the pronunciation, but which serve as a clue in seeking for its derivation.’}
\end{quote}


Nowadays, with the rise of sequencing technologies, the meaningfulness of his analogy is just as topical as ever: evolutionary biologists can now directly ‘read’ DNA and search for its ‘etymology’ by analysing the series of its ‘letters’. 
Ultimately, their goal is to uncover the kinship ties between species, just like linguists would disclose the paths through which words have travelled by examining the remnants of unpronounced letters within them.

Indeed, the discovery of DNA in the mid-twentieth century \citep{franklin1953molecular,watson1953molecular,wilkins1953molecular} brought about a real revolution in the study of evolution and even led to the establishment of a new research field to which this thesis belongs: molecular evolution — now rather called evolutionary genomics for whole genomes, rather than single genes, get analysed.
I will therefore open the introduction in Part~\ref{part:introduction} with Chapter~\ref{ch:1-history-genetics} devoted to tracing back the scientific findings in genetics that directly led to the emergence of this research field aiming at understanding genome evolution.

But, precisely, why and how do genomes evolve?
Three main evolutionary forces are classically invoked in this process: mutation, natural selection and genetic drift.
Though, a couple of decades ago, a fourth force made an entrance in the evolutionary scene: biased gene conversion (BGC).
This driver of genome evolution is a direct consequence of recombination — a process essential to meiotic cell division in sexually-reproducing organisms. 
I will thus review the mechanism of meiotic recombination in Chapter~\ref{ch:2-recombination-mechanistics} and the sources of recombination rate variation in Chapter~\ref{ch:3-recombination-variation}. 
This will lead me, in Chapter~\ref{ch:4-gBGC}, to go over the knowledge acquired so far on the fourth evolutionary force of interest for this thesis.

From that point on, I will focus on the puzzling observation which laid the foundation for this work and will set, in Part~\ref{part:objectives}, the objectives we wanted to address.


The results presented in Part~\ref{part:results} will then be divided into four chapters.
In Chapter~\ref{ch:5-methodology}, I will describe the unprecedentedly powerful approach we implemented to detect recombination events at high resolution in single individuals.
Next, I will show how we used this method to precisely characterise mouse recombination patterns in Chapter~\ref{ch:6-recombination-parameters} and to quantify biased gene conversion in Chapter~\ref{ch:7-quantification-BGC}.
Last, in Chapter~\ref{ch:8-HFM1}, I will detail how we adapted our method to other studies of recombination with more complex experimental designs involving several genomic introgressions.
All the developments presented in this part are the result of a collaboration with Bernard de Massy and Frédéric Baudat, and those of Chapter~\ref{ch:8-HFM1} also involved Valérie Borde and Corinne Grey: the totality of the experimental work necessary for this study (mouse crosses and DNA extraction) was carried out by them.
As for me, I contributed to this project by designing and implementing the bioinformatic procedures allowing to detect and quantify recombination and biased gene conversion and by analysing the ensuing results.

Finally, Part~\ref{part:discussion} will be dedicated to discussing this work: I will first consider the scientific implications of our study in Chapter~\ref{ch:9-discussion} and will then share ideas related to it in the broader fields of epistemology, philosophy of science and sociology of knowledge in Chapter~\ref{ch:10-philosophising}.




