\begin{savequote}[8cm]
	
	% ‘You are always taught to do as much as you can. Always put checks in. Always look for exceptions. Always handle the most general case. Always give the user the best advice. Always print a meaningful error message. Always this. Always that. You have so many things in the background that you're supposed to do, there's no room left to think. I say, forget all that and ask yourself, “What's the simplest thing that could possibly work?”
	% I think the advice got turned into a command: “Do the simplest thing that could possibly work.” That's a little more confusing, because there isn't this notion that as soon as you've done it, we'll evaluate it.’
	
	% "The Simplest Thing that Could Possibly Work: A Conversation with Ward Cunningham, Part V". Interview with Bill Venners, www.artima.com. January 19, 2004.
	% https://www.artima.com/intv/simplest3.html
	
	
	% Talk is cheap. Show me the code. — Linus Torvalds

	‘The assumption is that when something turns out to not be ideal, it will be refactored again. Everything is subject to refactoring.’
	% https://www.artima.com/intv/ownership.html
	\qauthor{--- Ward Cunningham, \textit{\usebibentry{cunningham2003collective}{title}} \citeyearpar{cunningham2003collective} }
	
\end{savequote}

% \chapter{\label{ch:8-HFM1}Impact of HFM1 knock-out on recombination in mice}
% \chapter{\label{ch:8-HFM1}Adaptation to studies of knock-out mutants}
% \chapter{\label{ch:8-HFM1}Adaptation to other studies of knock-out mutants}
% \chapter{\label{ch:8-HFM1}Adaptation of the method to other studies on recombination: Hfm1 knock-out mutants}
% \chapter{\label{ch:8-HFM1}Adaptation of the method to other studies on recombination}
% \chapter{\label{ch:8-HFM1}Methodological adaptations to study recombination-dependent gene knock-out mutants}
% \chapter{\label{ch:8-HFM1}Methodological adaptations to study recombination-dependent genes}
\chapter{\label{ch:8-HFM1}Methodological adaptations to other studies of recombination}
% \chapter{\label{ch:8-HFM1}Methodological adaptations to study \textit{Hfm1} knock-out mutants}

%\otherpagedecoration


\minitoc{}

{\small{} \itshape{}

\paragraph{This chapter in brief —}

The classical approach to understand the specific role of a given recombination-dependent gene consists in studying individuals with an inactivated version of the gene.
% Even though many of the actors of the recombination process are known, the specific role of some of them remains to be elucidated.
% Classically, this can be done by studying individuals for which a given recombination-dependent gene has been inactivated.
Such introgression of a knocked-in (KI) gene into a hybrid can be done by crossing one of its ancestors with a third strain containing the inactivated version of the gene.
Thus, recombination events should be searched in the hotspots of a F2 individual containing three distinct genetic backgrounds.
Here, we adapted the method we implemented for simple F1 hybrids to such design.
Basically, we had to distinguish the polymorphic sites expressing variation between the two parental genomes from those originating from the third introgressed genome.
% In the end, our method was as powerful as the original one for simple F1 hybrids.
This implementation was as powerful as the original method and we could thus study the role of \textit{Hfm1}, a DNA helicase important for CO control: we found that its KI leads to increased recombination rate and shortened CO conversion tracts.

}

\newpage

Even though it is nowadays largely acknowledged that the formation of recombination events (COs and NCOs) depends on a complex network of proteins (see Chapter~\ref{ch:2-recombination-mechanistics}), the specific role and impact of many of these actors has not yet been elucidated.
In particular, the yeast gene \textit{MER3} (mouse homologue: \textit{Hfm1}) codes for a meiosis-specific DNA helicase \citep{nakagawa1999saccharomyces,nakagawa2002saccharomyces} that participates in CO control and in DNA heteroduplex extension \citep{mazina2004saccharomyces,nakagawa2002mer3}.
This gene is also essential to CO formation in other fungi \citep{sugawara2009coprinus}, plants \citep{mercier2005two,chen2005arabidopsis}, humans \citep{tanaka2006hfm1} and mice \citep{guiraldelli2013mouse}.
Though, even if it has been shown that it diminishes the rate in mice \citep{guiraldelli2013mouse}, its impact on other parameters of recombination, like the length of conversion tracts, has not been established.

Because the method that I described in Chapter~\ref{ch:5-methodology} showed unprecedented power to detect recombination events in single individuals, it became possible to study the impact of \textit{Hfm1} on recombination by comparing the recombinational outcome of wild-type and \textit{hfm1} mutant (\textit{Hfm1\textsuperscript{-/-}}) mice.
In this chapter, I detail how the mutant gene was introgressed into F2 mice descending from B6 and CAST strains, how we refactored our method to study recombination in these F2 individuals and what were the primary results of this analysis.




\section{Experimental design}
\subsection{Introgression of the mutant \textit{hfm1} allele}
% knock-out (KO) gene}



\begin{table}[p]
	\begin{subtable}[h]{\textwidth}
		
		\subcaption{Ancestry of S28353 and S28355.} 
		\centering
		\begin{adjustbox}{width = 1\textwidth}
			\begin{tabular}{rrrrrrrr}

				\toprule
				\textbf{Mouse ID} & \textbf{Relationship} & \textbf{\% B6} & \textbf{\% DBA2} & \textbf{\% CAST} & \textbf{HFM1} & \textbf{Mother} & \textbf{Father} \\

				\midrule
				39856   & Maternal grandmother  & 0.0   & 0.0   & 100.0 & WT/WT & N/A    & N/A \\
				28130   & Maternal grandfather  & 75.0  & 25.0  & 0.0   & WT/KI & 72205 & N/A \\
				F0\#2 (72205)   & Paternal grandmother  & 50.0  & 50.0  & 0.0   & WT/KI & N/A    & N/A \\
				N/A      & Paternal grandfather  & 100.0 & 0.0   & 0.0   & WT/WT & N/A    & N/A \\
				\midrule
				22228   & Mother                & 37.5  & 12.5  & 50.0  & WT/KI & 39856 & 28130 \\
				28196   & Father                & 75.0  & 25.0  & 0.0   & WT/KI & 72205 & N/A \\
				\midrule
				28353   & Mutant analysed       & 56.25 & 18.75 & 25.0  & KI/KI & 22228 & 28196 \\
				28355   & WT analysed           & 56.25 & 18.75 & 25.0  & WT/WT & 22228 & 28196 \\
				\bottomrule

			\end{tabular}
		\end{adjustbox}
\label{tab:ancestry-28353-28355}
	\end{subtable}
	\vspace{2cm}


	\begin{subtable}[h]{\textwidth}

		\subcaption{Ancestry of S28367.} 
		\centering
		\begin{adjustbox}{width = 1\textwidth}
			\begin{tabular}{rrrrrrrr}

				\toprule
				\textbf{Mouse ID} & \textbf{Relationship} & \textbf{\% B6} & \textbf{\% DBA2} & \textbf{\% CAST} & \textbf{HFM1} & \textbf{Mother} & \textbf{Father} \\

				\midrule
				F0\#3 (72212)  & Maternal grandmother  & 50.0  & 50.0  & 0.0   & WT/KI & N/A    & N/A \\
				N/A      & Maternal grandfather  & 100.0 & 0.0   & 0.0   & WT/WT & N/A    & N/A \\
				28163   & Paternal grandmother  & 75.0  & 25.0  & 0.0   & WT/KI & 72205 & N/A \\
				39978   & Paternal grandfather  & 0.0   & 0.0   & 100.0 & WT/WT & N/A    & N/A \\
				\midrule
				28172   & Mother                & 75.0  & 25.0  & 0.0   & WT/KI & 72212 & N/A \\
				28238   & Father                & 37.5  & 12.5  & 50.0  & WT/KI & 28163 & 39978 \\
				\midrule
				28371   & Mutant analysed       & 56.25 & 18.75 & 25.0  & WT/WT & 28172 & 28238 \\
				\bottomrule

			\end{tabular}
		\end{adjustbox}
\label{tab:ancestry-28367}
	\end{subtable}
	\vspace{2cm}


	\begin{subtable}[h]{\textwidth}

		\subcaption{Ancestry of S28371.}
		\centering
		\begin{adjustbox}{width = 1\textwidth}
			\begin{tabular}{rrrrrrrr}

				\toprule
				\textbf{Mouse ID} & \textbf{Relationship} & \textbf{\% B6} & \textbf{\% DBA2} & \textbf{\% CAST} & \textbf{HFM1} & \textbf{Mother} & \textbf{Father} \\

				\midrule
				39856   & Maternal grandmother  & 0.0   & 0.0   & 100.0 & WT/WT & N/A    & N/A \\
				28130   & Maternal grandfather  & 75.0  & 25.0  & 0.0   & WT/KI & 72205 & N/A \\
				F0\#2 (72205)  & Paternal grandmother  & 50.0  & 50.0  & 0.0   & WT/KI & N/A    & N/A \\
				N/A      & Paternal grandfather  & 100.0 & 0.0   & 0.0   & WT/WT & N/A    & N/A \\
				\midrule
				28250   & Mother                & 37.5  & 12.5  & 50.0  & WT/KI & 39856 & 28130 \\
				28198   & Father                & 75.0  & 25.0  & 0.0   & WT/KI & 72205 & N/A \\
				\midrule
				28371   & WT analysed           & 56.25 & 18.75 & 25.0  & WT/WT & 28250 & 28198 \\
				\toprule

			\end{tabular}
		\end{adjustbox}
\label{tab:ancestry-28371}
	\end{subtable}
	\vspace{1cm}


	\caption[Genealogy of the four mice analysed]
	{\textbf{Genealogy of the four mice analysed.}
		\par The genealogies (parents and grandparents) of each of the two mutant mice (IDs: 28353 and 28371) and of the two wild-type (WT) mice (IDs: 28355 and 28367) analysed in this study, as well as the characteristics (background composition in B6, CAST and DBA2 genomes, and the \textit{Hfm1} alleles carried: either the knocked-in (KI), i.e.\ inactivated, or the WT allele) of all the individuals involved in the ancestry are reported in the following subtables: \textbf{(a)} 28353 and 28353; \textbf{(b)} 28367; \textbf{(c)} 28371.
	}
\label{tab:genealogies}
\end{table}



A point mutation on \textit{Hfm1} (\textit{Hfm1\textsuperscript{KI}}) was introduced in the zygote of a cross between two F1 mice deriving from hybridisations between two \textit{Mus musculus domesticus} strains: strain C57BL/6J, hereafter called B6 and strain DBA/2J, hereafter called DBA2.
The resulting founder mice (F0\#2 and F0\#3) were thus heterozygous for the \textit{Hfm1} gene (\textit{Hfm1\textsuperscript{WT/KI}}) and their genetic backgrounds were composed of 50\% DBA2 and 50\% B6 genomes.

Further crosses with other B6 and \textit{Mus musculus castaneus} (strain CAST/EiJ, hereafter called CAST) mice resulted in individuals carrying either two KI (i.e.\ inactivated) alleles for \textit{Hfm1} (\textit{Hfm1\textsuperscript{KI/KI}} or \textit{Hfm1\textsuperscript{-/-}}), two WT (i.e.\ activated) alleles (\textit{Hfm1\textsuperscript{WT/WT}} or \textit{Hfm1\textsuperscript{+/+}}) or one allele of each (\textit{Hfm1\textsuperscript{WT/KI}} or \textit{Hfm1\textsuperscript{+/-}}).
The genetic backgrounds for these mice were composed of a mixture of B6, DBA2 and CAST genomes (Table~\ref{tab:genealogies}).

Of these, two \textit{hfm1} mutant (28353 and 28367) and two WT (28355 and 28371) male mice were selected for further analysis: their sperm DNA was extracted and sonicated to produce fragments of a mean size of 450 bp.



\subsection{Target selection, DNA capture and sequencing}
% \subsection{Selection of target hotspots}

Like in Chapter~\ref{ch:5-methodology}, we selected hotspots from the list identified by \citet{baker2015prdm9} on the basis of PRDM9 ChIP-seq peak detection.
We used the same criteria as before: a minimum of 4 SNPs in the 300-bp central region, a strict maximum of 60 sites with low sequence quality in the 1-kb central region and at least 90\% of identity between the B6 and the CAST reference genome on at least 80\% of the selected region.

Though, since the main aim of this analysis was to test for any effect of the \textit{Hfm1} mutation on CO CT length, we extended the width of our selected hotspots to 3 kb.
Thus, the third selection criterium discarded a larger number of candidate hotspots than in Chapter~\ref{ch:5-methodology}, since identity was required on 3 kb instead of 1 kb.
In the end, 890 3-kb long hotspots were retained, and the same 500 control regions as in Chapter~\ref{ch:5-methodology} were added to that list of targets.\\



% \subsection{DNA capture, sequencing and mapping}

For the efficiency of DNA capture to be identical in both haplotypes, two baits were designed for each of the 1,390 targets: one corresponding to the CAST haplotype and one to the B6 haplotype.
We then performed two successive rounds of DNA capture on each of the four DNA samples from the four mice.
Libraries were then sequenced by an Illumina device using a 250-bp paired-end protocol, and the sequenced reads were mapped onto the B6 and the CAST reference genomes as described in Chapter~\ref{ch:5-methodology}.
Overall, read mapping statistics and capture efficiency were similar to what was found in Chapter~\ref{ch:5-methodology} (Table~\ref{tab:HFM1-statistics-seq-mapping-capture}).




\begin{table}[t]
    \centering
	\begin{adjustbox}{width = 1\textwidth}
    \begin{tabular}{rrrrrrr}
        \toprule
		\multicolumn{2}{c}{\textbf{Sample}} & \multicolumn{2}{c}{\textbf{Mapping (\%)}} & \multicolumn{3}{c}{\textbf{Capture efficiency}} \\
		\cmidrule(l){1-2} \cmidrule(l){3-4} \cmidrule(l){5-7}
        \textbf{Library} & \textbf{Library} & \textbf{Ref.} & \textbf{Ref.} & \textbf{\# Filtered} & \textbf{\% in} & \textbf{\# in} \\
        \textbf{ID} & \textbf{size} & \textbf{B6} & \textbf{CAST} & \textbf{Fragments} & \textbf{targets} & \textbf{targets} \\
		\midrule
        % \cmidrule(l){1-1} \cmidrule(l){2-7} \cmidrule(l){8-9} \cmidrule(l){10-12}
        28355 & 14,977,880 & 99.87 & 99.76 & 7,206,235 & 86.09 & 6,203,730 \\
        28371 & 14,977,880 & 99.87 & 99.76 & 7,206,235 & 86.09 & 6,203,730 \\
		\textbf{Total WT} & \textbf{} & \textbf{99.71} & \textbf{99.68} & \textbf{} & \textbf{48.06} & \textbf{161,564,743} \\
		\\
        28353 & 14,977,880 & 99.87 & 99.76 & 7,206,235 & 86.09 & 6,203,730 \\
        28367 & 14,977,880 & 99.87 & 99.76 & 7,206,235 & 86.09 & 6,203,730 \\
		\textbf{Total mutants} & \textbf{} & \textbf{99.48} & \textbf{99.44} & \textbf{} & \textbf{47.86} & \textbf{186,150,465} \\
        % \textbf{Total} & \textbf{-} & \textbf{-} & \textbf{-} & \textbf{-} & \textbf{-} & \textbf{976,826,736} & \textbf{99.68} & \textbf{99.37} & \textbf{468,513,483} & \textbf{71.63} & \textbf{335,605,734} \\
        \bottomrule
    \end{tabular}
	\end{adjustbox}
    \caption[Sequencing, mapping and capture-efficiency summary metrics]
    {\textbf{Sequencing, mapping and capture-efficiency summary metrics.}
        \par Reads were mapped onto the B6 and CAST reference genomes, and fragments were filtered as described in Chapter~\ref{ch:5-methodology}.
		The lines in bold represent the totals for the two WT and the two mutant mice.
    }
\label{tab:HFM1-statistics-seq-mapping-capture}
\end{table}





% \section{Detection of recombination in F2 individuals}
\subsection{Expected genetic background composition}


The point mutation on \textit{Hfm1} originated from B6/DBA2-background founder mice (F0\#2 and F0\#3) and was introgressed into a B6xCAST hybrid \textit{via} two consecutive crosses: 
on the one hand, the founder mice were crossed with B6/B6-background mice, thus yielding one 75\%-B6/25\%-DBA2 parent;
on the other hand, other 75\%-B6/25\%-DBA2 mice were crossed with CAST mice to yield a second parent with a background composed of 37.5\% B6, 12.5\% DBA2 and 50\% CAST genomes (Table~\ref{tab:genealogies}).
Each of the four selected mice (28353, 28355, 28367 and 28371) were then obtained by crossing the two aforementioned parents together.
Thus, their background encompassed 56.25\% B6, 18.75\% DBA2 and 25\% CAST genomes.

More precisely, the expected genomic proportion (and thus, the expected proportion of hotspots) of each genetic background were those reported in Table~\ref{tab:background-expected}.


\begin{table}[t]
    \centering
    % \begin{adjustbox}{width = 1\textwidth}
    \begin{tabular}{rrrr}

        \toprule
		\multicolumn{2}{c}{\textbf{Detailed}} & \multicolumn{2}{c}{\textbf{Simplified}} \\
		\cmidrule(l){1-2} \cmidrule(l){3-4}
        \textbf{Background} & \textbf{\% expected} & \textbf{Background} & \textbf{\% expected} \\

        % \midrule
		\cmidrule(l){1-2} \cmidrule(l){3-4}
        B6/B6     & 28.125 & \multirow{3}{*}{BD/BD} & \multirow{3}{*}{50.0} \\
        B6/DBA2   & 18.750 & & \\
        DBA2/DBA2 & 3.125 & & \\
		% \midrule
		\cmidrule(l){1-2} \cmidrule(l){3-4}
        B6/CAST   & 37.500 & \multirow{2}{*}{BD/CAST} & \multirow{2}{*}{50.0} \\
        DBA2/CAST & 12.500 & & \\
        % \midrule
		\cmidrule(l){1-2} \cmidrule(l){3-4}
		CAST/CAST & 0.000 & CAST/CAST & 0.0 \\
        \bottomrule

    \end{tabular}
    % \end{adjustbox}
    \caption[Expected distribution of genetic backgrounds in the mice analysed]
	{\textbf{Expected distribution of genetic backgrounds in the mice analysed.}
		\par Because the B6 and DBA2 genomes present high sequence conservation \citep{davis2005genomewide}, we regrouped them under the label ‘BD’.
		The expected genomic proportion (and thus proportion of hotspots) in each of the six possible ‘detailed’ backgrounds were reported on the left panel and the expected proportions in each of the three ‘simplified’ backgrounds were reported on the right panel.
	} 
\label{tab:background-expected}
\end{table}




% \begin{table}[t]
%     \centering
%     % \begin{adjustbox}{width = 1\textwidth}
%     \begin{tabular}{rrr}
%
%         \toprule
%         \textbf{Background (simplified)} & \textbf{Background (detailed)} & \textbf{\% expected} \\
%
%         \midrule
%         BD/BD       & B6/B6     & 28.125 \\
%         BD/BD       & B6/DBA2   & 18.750 \\
%         BD/BD       & DBA2/DBA2 & 3.125 \\
%         BD/CAST     & B6/CAST   & 37.500 \\
%         BD/CAST     & DBA2/CAST & 12.500 \\
%         CAST/CAST   & CAST/CAST & 0.000 \\
%         \bottomrule
%
%     \end{tabular}
%     % \end{adjustbox}
%     \caption[Expected distribution of genetic backgrounds in the four mice analysed]
%     {\textbf{Expected distribution of genetic backgrounds in the four mice analysed.}
%         \par
%     }
% \label{tab:background-expected}
% \end{table}
%
%Frequence de
%- background B6/B6: 28.125
%- background B6/DBA2: 18.75
%- background DBA2/DBA2: 3.125
%- background B6/CAST: 37.5
%- background CAST/CAST: 0
%- background DBA2/CAST: 12.5






The recombination events of interest in this analysis were those arising between a CAST haplotype and a B6 or a DBA2 haplotype.
What's more, the B6 and the DBA2 genomes present high sequence conservation — presumably because they inherited large genomic regions from a relatively recent common ancestor \citep{davis2005genomewide}, — whereas the B6 and CAST strains which diverged about 350,000 to 500,000 years ago \citep{geraldes2008inferring} present a much higher genome-wide divergence of 1 SNP every 150 bp \citep{keane2011mouse}.
Thus, for the sake of siplicity, we chose not to distinguish between the B6 and DBA2 background and grouped these two haplotypes under the label ‘BD’.
With this simplified notation, an expected 50\% of the hotspots of the analysed mice should be in a BD/BD background, and the other 50\% in a BD/CAST background (Table~\ref{tab:background-expected}).\\


The background composition of hotspots constituted one difference with the dataset analysed in Chapter~\ref{ch:5-methodology}, in which all hotspots were heterozygous because the individual analysed was a F1 hybrid.
But the most critical difference was the presence of a third genomic origin (DBA2): the polymorphic sites between DBA2 and B6 would undoubtedly lead to spurious genotype calls and thus, false positive discoveries.
Therefore, prior to searching for recombination events in this dataset, it was necessary to distinguish the B6-DBA2 polymorphic sites from the BD-CAST polymorphic sites, in order to base the genotyping solely on the latter class of markers.







\section{Detection of recombination in F2 individuals}
\subsection{Inferrence of the origin of polymorphic sites}

Because the B6 and the DBA2 genomes were much more highly conserved \citep{davis2005genomewide} than the B6 and the CAST genomes \citep{keane2011mouse}, we used the simplifying assumption that any polymorphic site between the CAST and B6 genomes would also appear between the CAST and DBA2 genomes. 
In other words, we assumed that the B6 and the DBA2 strains carried the same allele at BD-CAST polymorphic sites.
When mapped against the CAST reference genome, the ‘reference’ (REF) allele at any such BD-CAST polymorphic site was necessarily the CAST allele and the ‘alternate’ (ALT) allele was the BD (B6 and DBA2) allele.
Thus, given that no hotspot could be in a CAST/CAST background (Table~\ref{tab:background-expected}), there was no possible \textit{scenario} for which an individual could be genotyped by the variant-caller as REF/REF at a BD-CAST polymorphic site.

In contrast, the polymorphic sites between the B6 and the DBA2 genomes could be, in certain cases, genotyped as REF/REF\@.
Indeed, at such B6-DBA2 polymorphic sites, the allele carried by the B6 strain was, by definition, different than that carried by the DBA2 strain.
According to the principle of parsimony, one of them (either the B6 or the DBA2 strain) should carry the same allele as the CAST strain.
If the B6 (resp.\ DBA2) allele was identical to the CAST allele, the B6 (resp.\ DBA2) allele would be the labeled as the reference (REF) allele and the DBA2 (resp.\ B6) allele as the alternate (ALT), when mapped against the CAST reference genome.
Therefore, an individual with a B6/CAST- or a B6/B6- background (resp.\ DBA2/CAST- or DBA2/DBA2-background) would be genotyped REF/REF at such a B6-DBA2 polymorphic site.

To distinguish between the two types of markers (B6-DBA2 or BD-CAST), we thus followed the following rule: 
if, in at least one of the four mice, a marker was genotyped REF/REF, we inferred that it was a B6-DBA2 marker.
Otherwise, we inferred that it was a BD-CAST marker.
We note that this led to erroneous calls (a B6-DBA2 marker mistakenly labeled as a BD-CAST marker) in cases where all four mice had the B6/B6 or the B6/CAST (resp.\ the DBA2/DBA2 or the DBA2/CAST) background. Though, in these \textit{scenarii}, the DBA2 and the B6 haplotypes are strictly undiscernable from one another and hus this would not lead to any erroneous genotyping error.





\subsection{Identification of the genetic background}

After labelling the origin of all the polymorphic sites, we inferred the genetic background in each target for each of the four mice as follows: 
if more than 90\% of the markers classified as BD/CAST displayed a REF/ALT genotype, a BD/CAST-background was inferred;
if more than 90\% of the markers classified as BD/CAST displayed a REF/ALT genotype, a BD/BD-background was inferred;
in any other case, the background was not inferred.
Last, we refined the analysis for the targets whose genetic background could not be inferred with the criterium aforementioned: if they were flanked on both the 5’ and the 3’ sides by at least 5 targets with a BD/BD (resp.\ BD/CAST) genetic background, we attributed them the BD/BD (resp.\ BD/CAST) background.

This \textit{modus operandi} ended in a mosaic of BD/BD and BD/CAST genetic backgrounds consistent with 0 or 1 (and sometimes 2) crossing-overs per chromosome in each individual (Figure~\ref{fig:mosaic-backgrounds} and Appendix~\ref{app:data-and-figs}), which provided strong support that our inferrence was correct.
All in all, across all 1,390 loci of the 4 mice, 7 were incongruent with the genetic background around (either because they were subject to a double crossing-over, or because of genotyping errors). We thus chose to remove these from the analysis.
Altogether, the proportion of heterozygous (i.e.\ BD/CAST-background) targets (Table~\ref{tab:proportion-het-backgrounds}) was close to the expected 50\% (Table~\ref{tab:background-expected}).
To further verify that these observed proportions fitted what was expected, we simulated a BD/CASTxBD/BD cross in which COs (number given by the sex-averaged genetic length) were drawn randomly along each chromosome. We found that the distribution of the expected proportion of heterozygous targets fitted with the observations (data not shown).




\begin{table}[t]
    \centering
    \begin{tabular}{rrrrrrr}
        \toprule
        \textbf{Category} & \textbf{Sample} & \textbf{\# BD/CAST} & \textbf{\# BD/BD} & \textbf{\% of Het.} \\
        \textbf{} & \textbf{} & \textbf{background} & \textbf{background} & \textbf{targets} \\
	    
		\midrule
		\multirow{2}{*}{WT}      & 28355    & 764   & 561  & 57.66 \\ 
		& 28371    & 845   & 461  & 64.70 \\ 
        \midrule
        \multirow{2}{*}{Mutant}  & 28353    & 663   & 669  & 49.77 \\ 
        & 28367    & 624   & 693  & 47.38 \\ 
        \midrule
		\multicolumn{2}{r}{\textbf{Average}} & \textbf{724} & \textbf{596} & \textbf{54.85} \\
        \bottomrule
    \end{tabular}
    \caption[Proportion of heterozygous targets in each mouse analysed]
	{\textbf{Proportion of heterozygous targets in each mouse analysed.}
		\par The background for each hotspot was inferred as described in the main text, for wild-type (WT) mice (top panel) and mutant mice (bottom panel). The line in bold represents the average across all four mice.
	}
\label{tab:proportion-het-backgrounds}
\end{table}






\begin{table}[b!]
    \centering
    \begin{tabular}{rrrrrr}
        \toprule
        \textbf{Target} & & \textbf{Nb of} & \textbf{Nb of} & \textbf{Nb of} & \textbf{Event rate} \\

        \textbf{category} & \textbf{Sample} & \textbf{targets} & \textbf{fragments} & \textbf{events} & \textbf{($\times$ 10\textsuperscript{-6})} \\

        \midrule
        \multirow{7}{*}{\textbf{\textit{Hotspots}}} & 28355    & 485 & 28,181,748  & 1,298 & 46.1 \\ 
		& 28371    & 552 & 34,015,365  & 1,847 & 54.3 \\ 
		& \textbf{Tot.\ WT} &  \textbf{1037} & \textbf{62,197,113}  & \textbf{3,145} & \textbf{50.6} \\ 
		\\
		& 28353 & 429 & 25,598,721 & 3,486 & 136 \\
		& 28367 & 390 & 30,863,121  & 2,082 & 67.4 \\
		& \textbf{Tot.\ mutants} & \textbf{819} & \textbf{56,461,842}  & \textbf{5,568} & \textbf{98.6} \\
		\midrule
        
		\multirow{7}{*}{\textbf{\textit{Controls}}} & 28355 & 279 & 15,206,411  & 34 & 2.24 \\ 
		& 28371 & 293 & 16,997,729  & 58 & 3.41 \\ 
		& \textbf{Tot.\ WT} & \textbf{572} & \textbf{32,204,140}  & \textbf{92} & \textbf{2.86} \\ 
		\\
		& 28353 & 234 & 13,658,994 & 33 & 2.42 \\
		& 28367 & 234 & 17,565,253  & 25 & 1.42 \\
		& \textbf{Tot.\ mutants} & \textbf{468} & \textbf{31,224,247}  & \textbf{58} & \textbf{1.86} \\ 
	    \midrule
        \multicolumn{1}{r}{\textbf{FP rate}} & \multicolumn{5}{r}{\textbf{3.22 \%}} \\
        \bottomrule
    \end{tabular}
    \caption[Number of events detected in hotspot and control targets]
    {\textbf{Number of events detected in hotspot and control targets.}
        \par Events (false positives (FPs) or genuine recombination events) were detected using the unique-molecule genotyping pipeline described in Chapter~\ref{ch:5-methodology}.
        All fragments or events overlapping at least 1 bp with a given target are counted in this table.
        The event rate corresponds to the ratio of candidate recombination events over the total number of fragments.
        The maximum false positive (FP) rate is the ratio of the event rate in control targets over that in hotspots.
		The lines in bold represent the totals for the two WT and the two mutant mice.
    }
\label{tab:HFM1-FP-rate}
\end{table}
% TOT NB FRAGS HOTSPOTS
% >>> 56461842+62197113
% 118658955
% TOT NB EVENTS HOTSPOTS
% >>> 5568+3145
% 8713
% >>>
% TOT NB FRAGS CONTROL
% >>> 31224247+32204140
% 63428387
% TOT NB EVENTS CONTROL
% >>> 58+92
% 150
% RATE CONTRO
% >>> 150/63428387
% 2.3648717411022922e-06
% RATE FRAGS
% >>> 8713/118658955
% 7.342892915246051e-05
% RATIO (FP RATE)
% >>> 2.3648717411022922e-06/7.342892915246051e-05
% 0.03220626758960502




\subsection{Detection of events in heterozygous hotspots}

Finally, we applied the unique-molecule genotyping pipeline described in Chapter~\ref{ch:5-methodology} to all the heterozygous targets and, by comparing the number of events found in control regions and hotspots, the maximum FP error rate was similar to that from Chapter~\ref{ch:5-methodology}: 3.22\% (Table~\ref{tab:HFM1-FP-rate}).

Altogether thus, our procedure was efficient to detect recombination events in the F2 individuals.
From this point on, we could thus assess the impact of the \textit{hfm1} mutation on several aspects of the recombination process.



\section{Impact of \textit{Hfm1} on recombination}
\subsection{Impact on the recombination rate (ou CO rate)}
\label{chap8:recombination-rate}

- observe que le taux de recombi est bcp plus grand chez non mut.
- coherent avec litt.
- mais, dans litt, surtout affecte les CO, nous peut pas mais regarde les Rec-1S en comparant deux a deux sur les memes sets de hotspots.

% Les ref trouvees dans Guiraldelli
This effect (diminished CO rate) is similar to that found in \textit{Arabidopsis} \textit{mer3} mutants where the formation of COs is reduced by 38\% \citep{mercier2005two} and in \textit{Saccharomyces cerevisiae} \textit{mer3} mutants where it is reduced by 50--60\% \citep{borner2004crossover,jessop2006meiotic}.

% Trouvailles de Guiraldelli chez souris
% In a majority of Hfm12/2 spermatocytes the number of initial recombination events appears comparable to that observed in wild-type mice, but recombination is defective and the number of bivalents at diakinesis is reduced to ,20%. These results are consistent with current models of recombination in which HFM1 participates in the major CO pathway. Although CO formation is severely defective in Hfm12/2 spermatocytes, chromosomes pair and initial synapsis is apparently normal. Our results suggest that early recombination transactions are sufficient for homologous recognition, pairing and initial synapsis. However, synapsis is incomplete for all but a small number of chromosomes in Hfm12/2 spermatocytes.



(Table~\ref{tab:HFM1-FP-rate})

\subsection{Pairwise comparison of the CO rate at shared hotspots} Ou sans pairwise et recombination rate
refaire la manip de laurent et conclure
Inexpliquee la variation entre individus. Donc besoin d'avoir plus d'individus pour comprendre le determinant de ce taux de CO / recombi chez les mutants, et essayer d'identifier un facteur en commun.

\subsection{Impact on CO tract length}

Finally, because, the \textit{Hfm1} yeast homologue (\textit{MER3}) has been shown to play a role in DNA heteroduplex extension \citet{mazina2004saccharomyces}, we wanted to assess whether tract lengths differed between the \textit{hfm1} mutant and the WT mice.

Because CO and NCO CT lengths are not directly observable from the data, we performed an approximate bayesian computation (ABC) similar to what was described in Chapter~\ref{ch:6-recombination-parameters}, based on 50,000 simulations reproducing this experiment (thus, as compared to Chapter~\ref{ch:6-recombination-parameters}, we modified hotspot width, fragment start and stop positions and polymorphic sites to fit with this experiment).\\



\begin{table}[b!]
    \centering
	\begin{adjustbox}{width = 1\textwidth}
    \begin{tabular}{rrrrrrr}


        \toprule
		& \multicolumn{3}{c}{\textbf{WT}} & \multicolumn{3}{c}{\textbf{Mutant}} \\
		\cmidrule(l){2-4} \cmidrule(l){5-7}
        \textbf{Parameter} & \textbf{Both} & \textbf{28355} & \textbf{28371} & \textbf{Both} & \textbf{28353} & \textbf{28367} \\

        \midrule
        \textbf{\textit{CO:NCO ratio}}      & 0.108 [0.009--0.189] & 0.095 & 0.098 & 0.092 [0.0003--0.40] & 0.051 & 0.166 \\
        \textbf{\textit{CO CT length}}\\
        $Mean$                              & 744 [219--2790] & 539 & 654 & 236 [145--478] & 238 & 253 \\
        $Sd$                                & 582 [101--765] & 514 & 759 & 292 [30--397] & 416 & 232 \\
        \textbf{\textit{NCO CT length}}\\
        $Mean$                              & 34 [5--47] & 35 & 31 & 30 [0.75--397] & 49 & 32 \\
        $Sd$                                & 43 [1--101] & 38 & 57 & 108 [13--260] & 130 & 66 \\
        
        \bottomrule

    \end{tabular}
	\end{adjustbox}
	\caption[Recombination parameters inferred from an approximate bayesian computation for WT and mutant mice]
	{\textbf{Recombination parameters inferred from an approximate bayesian computation for WT and mutant mice.} 
		\par Parameters (CO:NCO ratio and CO and NCO conversion tract (CT) length reported in bp) were estimated for both WT and both mutant mice.
	95\% confidence intervals were reported between brackets.
	Because the observed recombination rate varied greatly between the two mutants (Subsection~\ref{chap8:recombination-rate}), we also reported than point estimates for all single individuals.
	}
\label{tab:HFM1-ABC-results}
\end{table}




Altogether, the CO:NCO ratio and the NCO CT length estimated for WT mice were strikingly close to those measured on the WT mice of Chapter~\ref{ch:6-recombination-paremeters} (Table~{tab:ABC-results}).
CO CTs were slightly longer (albeit \textit{not} significantly) than those found in the previous ABC, which likely comes from the fact that the targeted regions were wider in this experiment whereas the maximum distance between DSB sites and CO switch points was limited in the previous one.

Interestingly, we found a clear CO CT length reduction in \textit{hfm1} mutant mice as compared to WT mice (Table~\ref{tab:HFM1-ABC-results}).
Because the observed recombination rates varied greatly between the two mutants (Subsection~\ref{chap8:recombination-rate}), we checked whether this effect was also visible in single individuals and found that, indeed, the inferred conversion tract lengths were stable, no matter the recombination rate.
This observation was consistent with the idea that, in mice, the \textit{Hfm1} gene plays a role in extending the DNA heteroduplex.



PUIS UNE CONCLU UN PEU PLUS GENERALE ex: donc finalement, la methode qu'on a mis au point a permis de gagner new insights sur la recombi, mais aussi le BGC, cependant, il existe toutefois des limitations que je discute dans le prochain chapitre, ainsi que les implications scientifiques.





