\begin{savequote}[8cm]
“[…] if there is one event in the whole evolutionary sequence at which my own mind lets my awe still overcome my instinct to analyse, and where I might concede that there may be a difficulty in seeing a Darwinian gradualism hold sway throughout almost all, it is this event — the initiation of meiosis.”
% “[…] if there is one event in the whole evolutionary sequence at which my own mind lets my awe still overcome my instinct to analyse […], it is this event — the initiation of meiosis.”
	
\qauthor{--- W. D. “Bill” Hamilton, \textit{\usebibentry{hamilton1996narrow}{title}} \citeyearpar{hamilton1996narrow} }

	% \textlatin{Neque porro quisquam est qui dolorem ipsum quia dolor sit amet, consectetur, adipisci velit...}

% There is no one who loves pain itself, who seeks after it and wants to have it, simply because it is pain...
  % \qauthor{--- Cicero's \textit{de Finibus Bonorum et Malorum}}
\end{savequote}

\chapter{\label{ch:2-recombination-mechanistics}Meiotic recombination, the essence/substrate of heredity} 
%\otherpagedecoration

\minitoc{}


% \subsection*{Preamble — Laying the foundations for research on recombination}
% \subsection*{Preamble — Laying the grounds for research on recombination}
% \subsection*{Preamble — From understanding heredity to discovering recombination}
% \subsection*{Preamble — From the mystery of heredity to the discovery of recombination}
% \subsection*{Preamble — The discovery of recombination}
% \subsection*{Preamble — How the concept of recombination emerged (a little of history)}
% \subsection*{History of sciences — How the concept of recombination emerged}

\subsection*{Preamble — How the concept of recombination emerged}

\paragraph{An unexplained exception to Mendel's laws of heredity.}
Between 1857 and 1864, the Austrian monk Gregor Mendel (1822--1884) undertook a series of hybridisation experiments on the garden pea plant \textit{Pisum sativum}. This led him to describe the idea of an “independent assortment of traits” \citep{mendel1996experiments}, thereby proving the existence of paired “elementary units of heredity” (i.e.\ genes) and establishing the statistical laws governing them.
His work remained unrecognised by the scientific community for several decades but was finally rediscovered in the early twentieth century when three botanists (Hugo de Vries (1845--1935), Carl Correns (1864--1963) and Erich von Tschermak (1871--1962)) independently confirmed his findings \citep{dunn2003gregor}.
Meanwhile, William Bateson (1861--1926) fiercely defended Mendel's thesis in \textit{\usebibentry{bateson1902mendel}{title}} \citep{bateson1902mendel} against his contemporary biometricians \citep[reviewed in][]{bateson2002william}, thus spreading Mendel's view into the scientific world.\\
% Meanwhile, as Mendel's attempts to explain the mechanisms of heredity lacked scientific support and contradicted Galton's “Law of Ancestral Heredity”, many scientists were skeptiks

A few years later, Bateson noticed exceptions to Mendel's principles of independent assortment: some crosses generated certain phenotypes in far excess from the expected Mendelian ratios \citep{bateson1905experimentalpea}. This led him and his collaborators to propose that certain traits were somehow coupled with one another, although they did not know how \citep{bateson1905experimental}. 
%They had discovered what is now called “genetic linkage”.

% Conclure: methode d'exploration de la genetique nee + desequilibre de liaison = on voit des differences aux lois donnees par Mendel. donc a comprendre.

% \textbf{Ajouter la decouverte de l'ADN comme support de l'information genetique?}

% \citet{suzuki1986introduction}
% \paragraph{Genetic information is encoded in DNA.}
% In the meantime,

% PLAN: DNA SUPPORT DE L'INFORMATION GENETIQUE:
% 1869: Friedrich Miescher: discovery of nuclein (=DNA) + Altmann 1889: named “nucleid acid”
% + 1883: Eduard Van Beneden: discovery of chromatin (=DNA+RNA+proteins that make up a chromosome)
% 1929: Phoebus Levene: DNA composed of four bases (ATCG)+phosphate+sugar
% + 1941: Chargaff's rule: A=T and c=G
% + 1953: James Watson and Francis Crick: model of double-helix DNA






\paragraph{The chromosomal theory of inheritance.}
In the meantime, it had been understood that cells derived from other cells, but the exact process was unknown. 
To understand it, Walther Flemming (1843--1905) used stains to intensify the contrasts of cell contents observed through microscopy and identified a substance located within the nucleus, which he named “chromatin” (from the greek word \textit{\textchi\textrho\textomega\textmu\textalpha}: “color”).
He described precisely the movements of chromosomes during cell division (which he termed “mitosis”), thus providing a mechanism for the distribution of nuclear material into daughter cells during mitosis \citep{flemming1879contributions}.

Theodor Boveri (1862--1915) went one step further by demonstrating the individuality of chromosomes in the roundworm \textit{Ascaris megalocephala}, which allowed him to suggest that the chromosomes of the germ cells are involved in heredity \citep{boveri1888zellen}.
In addition, he showed that the egg and the spermatozoon contribute the same number of chromosomes to the new individual, thus providing the first descriptions of meiosis \citep{boveri1890zellen}.
Walter Sutton (1877--1916) independently came to the same conclusion at about the same time: he enunciated the chromosomal theory of inheritance with the following words closing his \usebibentry{sutton1902morphology}{year} paper: “I may finally call attention to the probability that the association of paternal and maternal chromosomes in pairs and their subsequent separation during the reducing division […] may constitute the physical basis of the Mendelian law of heredity” \citep{sutton1902morphology}.

However, this theory was debated in the scientific community, because there was yet no direct proof of a link between the inheritance of traits and the segregation of chromosomes.\\ 

In parallel, based on cytological observations of chromosomes, Frans Janssens (1863--1924), a priest also known as the “microscopy wizard” for he mastered the process, developed the idea that the chromosomes' “filaments [chromatids] are involved in contacts that can modify their organization from one segment to the next” which “will generate new segmental combinations” in his \textit{Chiasmatype Theory} \citep{janssens1909theorie}.



\paragraph{Thomas Hunt Morgan's theory of gene linkage and crossing-over.}
% Morgan ne croit à aucune des theories. Pourtant, va etre le lien de ces trois. 
% Morgan decouvre une mutation chez la droso (explication globale dessus) => suspecte la liaison.
% Il decrit le mecansisme dans livre 1915.
% 
In 1909, Thomas Hunt Morgan (1856--1945) expressed his strong skepticism of the Mendelian theory of inheritance in his very derisive article \textit{\usebibentry{morgan1909factors}{title}} \citep{morgan1909factors} and doubted the chromosomal basis of heredity \citep[reviewed in][]{koszul2012centenary}.
Little did he know at the time that he was to become the main craftsman of the reconciliation of these two theories.




\begin{figure}[h]
	\centering
	\includegraphics[width = 1\textwidth]{figures/chap2/morgan-drosophila-cross-results.eps}
	% \includegraphics[width = 0.7\textwidth, trim = 0cm 0cm 11.65cm 0cm, clip]{figures/chap2/morgan-drosophila-cross-results.eps}
	\caption[Explanation of the results from reciprocal crosses between red-eyed and white-eyed \textit{Drosophila}]
	{\textbf{Explanation of the results from reciprocal crosses between red-eyed (red) and white-eyed (white) \textit{Drosophila}.} 
		\par In the first cross (left), a red-eyed purebred female is crossed with a white-eyed male, resulting in F1 hybrids made of heterozygous red-eyed females bearing both the dominant (w\textsuperscript{+}) and the recessive (w) alleles and red-eyed males bearing only the dominant (w\textsuperscript{+}) allele. The inbreeding of F1 individuals results in a F2 generation with a 3:1 ratio of red-eyed:white-eyed individuals, all white-eyed individuals being males.
		\par In the second cross (right), a white-eyed female is crossed with a red-eyed purebred male, resulting in F1 hybrids made of heterozygous red-eyed females bearing both the dominant (w\textsuperscript{+}) and the recessive (w) alleles and white-eyed males bearing only the recessive (w) allele. The inbreeding of F1 individuals results in a F2 generation with a 2:2 ratio of red-eyed:white-eyed individuals, half of white-eyed being males and half being females.
		\par The results of these two crosses show that the gene coding for eye color is located on the female sexual chrosome (X). The fact that results in the F2 progeny differ according to the direction of the cross (($\frac{w\textsuperscript{+}}{w\textsuperscript{+}}$) $\times$ (w) or ($\frac{w}{w}$) $\times$ (w\textsuperscript{+})) is a typical signature of linkage disequilibrium between the observed trait (eye color) and the sex chromosomes.
		\par This figure was reproduced from \citet{suzuki1986introduction} (Rights: WH Freeman \& Co).
	}
\label{fig:morgan-drosophila-cross-results}
\end{figure}

\clearpage




In his famous “fly room” where he bred \textit{Drosophila melanogaster} fruit flies, he found an unusual male white-eyed individual. Crossing it with purebred red-eyed females yielded red-eyed males and females F1 hybrids, — a typical result proving that the white eye color is a recessive trait. Unexpectedly, after inbreeding the heterozygous F1 progeny, he discovered that the traits of the F2 offspring did not assort independently: all white-eyed flies were males (Figure~\ref{fig:morgan-drosophila-cross-results}, left). 
However, when he crossed the white-eyed male with F1 daughters, he found both male and female white-eyed flies (Figure~\ref{fig:morgan-drosophila-cross-results}, right), thus showing that the white eye color was not lethal for females.

He immediately hypothesised that eye color was connected to the sex determinant \citep{morgan1910sex} and, as these findings were consistent with the idea that genes were physical objects located on chromosomes, Morgan soon came up with the idea of genetic linkage, i.e.\ the fact that two genes closely associated on a chromosome do not assort independently \citep{morgan1911random}. 
He also suggested that this coupling dependended on the distance between genes: “we find coupling in certain characters, and little or no evidence at all of coupling in other characters; the difference depending on the linear distance apart of the chromosomal material that represent the factors.”

With three of his students (Alfred Sturtevant (1891--1970), Hermann Muller (1890--1967) and Calvin Bridges (1889--1938)), he summarized all the evidence in \usebibentry{morgan1915mechanism}{title} which constitutes one of the most important books in the whole history of genetics \citep{gayon2016mendel}. 
There were two major propositions in that book. 
First, the recognition that Mendelian factors — Morgan would soon call them “genes” — are physical portions of chromosomes. This brought a mechanistic support to Mendel's “law of segregation” (according to which the zygote inherits only one version of each gene from each parent) and to the so far unexplained exception to Mendel's “law of independent assortment fo traits”: when two genes are located on the same chromosome, they have to segregate together — and thus the law does not apply to this special case. 
Second, they proposed that the linkage between genes located on the same chromosome could sometimes break, through the process of what Morgan called ”crossing over” (Figure~\ref{fig:morgan-CO}). This was to take place at the positions of the chiasmata previously observed by Janssens \citep{janssens1909theorie}. Morgan, together with Edgar Wilson (1908--1992) later crafted structures of crossing overs with clay \citep{wilson1920chiasmatype}.\\

Altogether, with the idea of recombination and crossing over, Morgan had fused three theories: gene linkage (the major exception to Mendel's laws of heredity), the chromosomal theory of inheritance and the chiasmatype theory. This triggered a real revolution in biology and marked the real commencement of genetics. His major contribution through his work on \textit{Drosophila} won him the \textit{Nobel Prize in Physiology or Medicine} in 1933.

It was only ten years later that Harriet Creighton (1909--2004) and Barbara McClintock (1902--1992) would bring the first proof of that theory by correlating cytological and genetic exchanges in the maize \citep{creighton1931correlation}.

\begin{figure}[h]
	\centering
	\includegraphics[width = 1\textwidth]{figures/chap2/morgan-CO-1916.eps}
	% \includegraphics[width = 0.7\textwidth, trim = 0cm 0cm 11.65cm 0cm, clip]{figures/chap2/morgan-drosophila-cross-results.eps}
	\caption[Original drawing of crossing over in \textit{\usebibentry{morgan1915mechanism}{title}} \citep{morgan1915mechanism}]
	{\textbf{Original drawing of crossing over in \textit{\usebibentry{morgan1915mechanism}{title}} \citep{morgan1915mechanism}.} 
		\par Original legend: “At the level where the black and the white rod cross in A, they fuse and unite as shown in   D. The details of the crossing over are shown in B and C.”
		This drawing symbolises the reconciliation between Mendel's and the chromosomal theories of inheritance.
	}
\label{fig:morgan-CO}
\end{figure}

\clearpage


% Conclure: on comprend que il existe des chromosomes et des recombinaisons, qui expliquent donc les deviations par rapport a l'attendu de Mendel. Le concept de recombinaison est ne.












\paragraph{The study of fungal products of meiosis led to the key concepts of the recombination mechanism.}


% Conclure: on voit des differences dans la recombinaison. Donc a expliquer. En particulier, BGC.



\paragraph{And research still goes on…}

% Des inconnues sur l'origine evolutive de la meiose. Des inconnues sur le role de la recombinaison...






%%%%% Mon plan (Ordonne par theme)
%HEREDITY:
1865: Mendel: experience des pois a expliquer.
+ Bateson 1905: linkage diesquilibrium in certain cases. (ou dans meiose et recombinaison)

%DNA SUPPORT DE L'INFORMATION GENETIQUE:
1869: Friedrich Miescher: discovery of nuclein (=DNA) + Altmann 1889: named “nucleid acid”
+ 1883: Eduard Van Beneden: discovery of chromatin (=DNA+RNA+proteins that make up a chromosome)
1929: Phoebus Levene: DNA composed of four bases (ATCG)+phosphate+sugar
+ 1941: Chargaff's rule: A=T and c=G
+ 1953: James Watson and Francis Crick: model of double-helix DNA

%MEIOSE AND RECOMBINATION:
1911/1913 Thomas Hunt Morgan: concept de linkage desiquilibrium + crossover. Overall, fusion of three theories: “the chromosome theory of inheritance,” imparted by Wilhelm Roux, Walther Flemming, Theodor Boveri, and Walter Sutton; “gene linkage,” an exception to Mendel’s law of independent assortment, first reported by Carl Correns; and the “chiasmatype theory,” derived from Frans Janssens' cytological observations of meiotic chromosomes.
+ Les trois autres theories avant. 
1931: Proof of the crossover theory came from Harriet Creighton and Barbara McClintock (Creighton and McClintock 1931), who were able to correlate cytological and genetic exchanges in maize.

%CONSEQUENCES(BGC):
fungal genetics experiments (to get all four products of meiosis)
our key concepts that formed the foundation of molecular models of recombination:gene conversion, an exception to Mendel’s principle of segregation, signaled a local nonreciprocal transfer of genetic information (Winkler 1930; Lindergren 1953; Mitchell 1955); postmeiotic segregation (PMS) indicated the presence of heteroduplex DNA (Olive 1959; Kitani et al. 1962); polarity gradients of gene conversion lead to the idea that recombination initiated from pseudofixed sites (Lissouba and Rizet 1960; Murray 1960); and the strong correlation between gene conversion/PMS events and crossing-over led to the proposal that these processes were mechanistically linked (Kitani et al. 1962; Perkins 1962; Whitehouse 1963).


%MINI CONCLU:
Les modeles toujours que des modeles (cf partie 3)
Encore bcp de points d'interrogation sur la meiose et la recombinaison et le role precis des molecules impliquees. Donc toujours de la recherche dessus.
En particulier, le BGC qu'on etudie dans cette these
Role de la recombinaison comme reparation des cassures pour replication  (cf topo Daniel)


% Dans partie meiose (partie 2):
Origine evolutive de la meiose debattue (transfo bacterienne ou mitose?)
Greek word






\section{Cytological aspects of meiosis}
+ 2 possibilities. EIther meiosis comes from mitosis, or from transfomration https://academic.oup.com/bioscience/article/60/7/498/234118

Stages + chromatin state + checkpoints + chiasmata + recombination nodules + synaptonemal complex

\section{Chronology of meiotic recombination}
cf Baudat et de Massy + ma presentation a ce sujet
\subsection{Initiation of recombination}
\subsection{Meiotic DSB repair}
\subsection{Resolution}

\section{Models of recombination}

\section{The importance of meiotic recombination}


+ Parler de Gene conversion
+ Parler de Male vs Female meiosis  https://cellbiology.med.unsw.edu.au/cellbiology/index.php/Meiosis
+ errors in meiosis
+ regulation of meiois (cyclins)






\section*{Sorte de plan / Liste d'idees}
historic
initiation of recombination: 
- Homologue pairing / interhomolg interactions
- determinsation localisation
- formation DSB / programmed DSB formation
Meiotic DSB repair:
- Homoology search
- synapsis between homologues
- models of recombination
- Resolution into CO/NCO
- Dissolution of the SC
Crossover control: 
- assurance / interference
- differentiation CO/NCO
Chromatin state: shapes the recombination landscape (ou dans position des hotspots): nucleosome occupancy + meiotic chromosome architecture. 
Importance of meiotic recombination: Genetic disorders otherwise + exemple de l'un qui a perdu PRDM9 mais qui n'est pas stérile pour autant. 
Checkpoints
strand asymmetry
DNA polymerases
Gene conversion ici? + non-allelic gene conversion (et un impact sur la détection de)



Deuxieme chapitre:
Methodological approaches to study recombination
Variation of recombination rates wtithin genomes and among species
evolvability of recombination rates



\section*{Notes temporaires}

%%%%% Modele de Odenthal-Hesse (chapitre sur recombinaison)

% \section{Chronology of meiotic recombination}
% \subsection{Programmed DSB formation}
% \subsection{Strand invasion and junction molecule formation}
% \subsection{Mismatch repair}
% \subsection{Resolution}
%
% \section{Models of recombination}




%%%%% Modele de Papier Baudat de Massy 2013

% \section{Initiation of recombination}
% \subsection{Homologue pairing}
% \subsection{Programmed DSB formation}
%
% \section{Meiotic DSB repair}
% \subsection{Homology search}
% \subsection{Synapsis between homologues}
% \subsection{Models of DSB repair}
%
% \section{Resolution CO/NCO}
% \subsection{Differentiation CO/NCO}
% \subsection{CO interference}



Plan chronologique
\begin{itemize}
	\item Mammalian meiosis (overview of the cycle)
	\item (Zoom sur Prophase 1)
	\item Leptotene stage: Initiation of recombination (Homologue pairing before DSB + Determination localisation DSB + DSB) + Miotic DSB repair (homology search)
	\item Zygotene stage: Meiotic DSB repair (Synapsis between homologues + Start resolution CO/NCO)
	\item Pachytene stage: Meiotic DSB repair (Resolution CO/NCO)
	\item Pachytene + Zygotene stages: Preparation to metaphase I (dissolution of SC)

\end{itemize}

Plan Neil Hunter the essence of heredity
\begin{itemize}
	\item Meiosis and the roots of recombination research
	\item Molecular models of meiotic recombination
	\item Interhomolog interactions
	\item Programmed DSB formation
	\item Crossover control (Assurance and interference, differentiation CO/NCO, pro-CO role of the synaptonemal complex, recombination associated DNA synthesis)
	\item Resolving, disolving and unwinding joint molecules to implement CO and NCO fates (differential timing and regulation of CO and NCO formation, MutL and EXO1=CO-specific resolving factor, MUS81 enzymes = role in meiotic joint molecule processing, STR/BTR ensemble as master regulators of meiotic joint molecule metabolism, SLX4-associated endonucleases and he GEN1 resolvase, SMC complex facilitates joint molecules formation and resolution, implementing NCO formation)
	\item Clinical significance of meiotic recombination (Aneuploidy CO and advancing maternal age, meiotic recombination and genomic disorders, defective recombination and infertility)


\end{itemize}

Plan Mammalian Meiotic Recombination: A Toolbox for Genome Evolution (https://www.karger.com/Article/FullText/452822):
\begin{itemize}
	\item Recombination and he repair of DSBs (Organization of meiotic chromosomes: importance of chromosomal axes, molecular events involved in he formation and repair of DSBs)
	\item Methodological approaches to he study of recombination
	\item Genetic and epigenetic marks of DSBs and recombination hotspots 
	\item Variation of recombination rates within genomes and among species (Variability at the chromosomal level, variation of fine-scale recombination maps)
	\item Evolvability of recombination rates (Chromosomal rearrangements as recombination modifiers)
\end{itemize}

Plan de Hotposts for initiation of meiotic recombination (https://www.ncbi.nlm.nih.gov/pmc/articles/PMC6237102/)
\begin{itemize}
	\item Defining DSB hotspots
	\item Chromatin shapes the meiotic DSB landscape (Nucleosome occupancy, meiotic chromosome architecture)
	\item Meiotic DSB and crossover distributions
	\item PRDM9 and H3K4me3
	\item The hotspot paradox
	\item Recombination initiation in repetitive sequences
	\item Byond hotspots: DSB-dependent spatial regulation


\end{itemize}


Mechanismes moleculaires precis + molecules impliquees
\begin{itemize}
	\item Appariement des chromosomes
	\item Formation du DSB
	\item Reparation CO/NCO
	\item Tous les modeles de resynthese des brins
	\item Observation des parametres de recombinaison chez la levure, souris
	% \item https://www.cell.com/current-biology/pdf/S0960-9822(06)01257-7.pdf: surveillance of breaks = checkpoints
	\item interference des CO
	\item Notion de strand asymmetry
	\item DNA polymerases (https://www.ncbi.nlm.nih.gov/pmc/articles/PMC5295669/)
	\item https://www.karger.com/Article/FullText/452822: mammalian eiotic recombination: a toolbox for genome evolution

\end{itemize}


Molecules tres importantes sur lesquelles insister:
\begin{itemize}
	\item DMC1
	\item RAD51
	\item (PRDM9)
	\item Spo11
	\item MUS81
	\item MLH1
	\item HFM1
\end{itemize}

Autre:
\begin{itemize}
	\item Non-allelic gene conversion
	\item Recombining without hotspots (https://www.ncbi.nlm.nih.gov/pmc/articles/PMC4684701/)
	\item Knockout of PRDM9 (http://science.sciencemag.org.inee.bib.cnrs.fr/content/352/6284/474)

\end{itemize}




