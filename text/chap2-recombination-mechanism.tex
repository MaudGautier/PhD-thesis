\begin{savequote}[8cm]
“[…] if there is one event in the whole evolutionary sequence at which my own mind lets my awe still overcome my instinct to analyse, and where I might concede that there may be a difficulty in seeing a Darwinian gradualism hold sway throughout almost all, it is this event — the initiation of meiosis.”
% “[…] if there is one event in the whole evolutionary sequence at which my own mind lets my awe still overcome my instinct to analyse […], it is this event — the initiation of meiosis.”
	
\qauthor{--- W. D. “Bill” Hamilton, \textit{\usebibentry{hamilton1996narrow}{title}} \citeyearpar{hamilton1996narrow} }
\end{savequote}

\chapter{\label{ch:2-recombination-mechanistics}Meiotic recombination, the essence of heredity} 
%\otherpagedecoration

\minitoc{}


\begin{quote}
\textit{‘Why all this silly rigmarole of sex? Why this gavotte of chromosomes? Why all these useless males, this striving and wasteful bloodshed, these grotesque horns, colors… and why, in the end, novels, like }Cancer Ward\textit{, about love?’}

\qauthor{--- W. D. Hamilton, \textit{\usebibentry{hamilton1975review}{title}} \citeyearpar{hamilton1975review}}
\end{quote}

This is how the fanciful Bill Hamilton (1936—2000) sums up the mystery of sexual reproduction (or simply, “sex”) that has been puzzling biologists for over a century and which, to this day, remains unanswered \citep{de2007evolution, otto2009evolutionary}.

This so-called “paradox of sex” finds its roots in that most theoretical arguments plead an elevated cost of sex as compared to asexual modes of reproduction \citep{otto2002evolution,lehtonen2012many}.
First, females invest half their reproductive resources in the production of males which, in turn, invest minimally into the progeny, as epitomised by the uncommonness of paternal care when it is not beneficial to the male \citep{smith1977parental,fromhage2007stability} — a concept known as the “twofold cost of sex” or “cost of meiosis” \citep{bell1982masterpiece}.
Second, the sexual act itself wastes time and energy to find and attract a sexual partner, and exposes the individual to the risks of contracting diseases and of being predated (sometimes by the mate itself), thus making sex a pearilous and unprofitable endeavour.

Nevertheless, only 80 \citep{vrijenhoek1989list,neaves2011unisexual} of the 70,000 vertebrate species discovered so far \citep{iucn2019} and as little as 0.1\% of all named animals \citep{vrijenhoek1998animal} reproduce otherwise than sexually. %More generally, although asexuality often arises, it rarely persists for long, thus suggesting that it is an evolutionary dead-end or that sex is hard to quit \citep{normarck2003genomic}.
Such pervasiveness of sex in nature constitutes indisputable proof of its evolutionary success. 

But, given its considerable drawkbacks, how come sex has superseded all other forms of reproduction?
% What then, given all the costs of sex, explains its pervasiveness in nature and where does its evolutionary success come from?
% How come then, given all its costs, that sex is so widespread in nature and what explains its evolutionary success?
Over 20 theories have been put forward to answer this question \citep{kondrashov1993classification}, but the most generally claimed advantages revolve around the idea that sex both eliminates deleterious mutations and brings up more favourable combinations of alleles \citep{normarck2003genomic, speijer2016can}.
This defensibly profitable reshuffling of alleles is called “recombination” and occurs during meiosis, the cellular process leading to the formation of gametes.

This chapter — named after a review on the subject \citep{hunter2015meiotic} — explores the details of the cytological features of meiosis and recombination with a particular focus on mammals, before venturing into the multiple models of meiotic recombination which still demand to be evidenced.

% \textbf{+ dire que tout n'a pas ete elucie et toujours du travail dessus + dire que les choses sont conservees chez les eucaryotes + meiose vient du grec. + citation victor hensen dans weisman? + parler des origines controversees de la meiose? (mais mecanisme assez bien elucide) + evidence pour l'avantage sont toujours scarce + focus sur mammalian meiosis and recombination}
% \textbf{citation victor hensen dans weisman?}



% After over a century of investigation, one persisting mystery of biology is why sexual reproduction (or simply, “sex”) is so widespread in nature \citep{de2007evolution, otto2009evolutionary}:
% % After over a century of investigation, the reason why sexual reproduction (or simply, “sex”) is so widespread in nature remains a mystery \citep{de2007evolution, otto2009evolutionary}:
% % After over a century of investigation, the mystery remains on the reason why sexual reproduction (or simply, “sex”) is so widespread in nature \citep{de2007evolution, otto2009evolutionary}:
% only 80 \citep{vrijenhoek1989list,neaves2011unisexual} of the 70,000 vertebrate species discovered so far \citep{iucn2019} and as little as 0.1\% of all named animals \citep{vrijenhoek1998animal} reproduce exclusively asexually.
% This is puzzling because most theoretical arguments plead an elevated cost of sex as compared to asexual modes of reproduction \citep{otto2002evolution,lehtonen2012many}.
% Indeed, females invest half their reproductive resources in the production of males which, in turn, only invest minimally into the progeny: paternal care is uncommon in many — though not all \citep{smith1977parental} — species. This is known as the



% - Sexual reproduction widespread in most eukaryotes.
% - Meiosis forms the gametes, and then are fused through fertilisation.
% % - However, largely costly, because search for mate... + requires to have a doublecost.
% - So, why so widespread? One largely shared thought on this matter is the fact that recombination, which occurs during, confers a selective advantage.
% - In this chapter, which I termed after a review by Neil Hunter, I will go through the details of the mechanistic features of meiosis, and of that of recombination. So far, not elucidated and several models to explain the recombination process. I will go through all the known recombination mechanisms so far. Last, go through a rapid overview of the importance of recombination by showing the defects that can occur in cases where it does not work.
%



% INTRO
% titre de Neil Hunter.
% Une fois les bases de la génétique posées, qui sont une comprehension de l'hérédité => pour comprendre l'hérédité, besoin de comprendre les mécanismes (meiose + recombinaison).
%
% OU
% Ceux qui ont étudié la génétique, sont partis sur des observations sur l'hérédité. Et la base de l'éhérédité est la recombinaison et la méiose. Donc j'explicite ce qu'on sait dessus d'abord.
% Le titre vient d'une revue de Neil Hunter.
%
% D'abord parle de la méiose, qui est l'événement qui permet de mener à la formation des gamètes, et donc qui est l'étape clé qui explique l'hérédité.
% Cette méiose assez complexe, avec un grand nombre d'étapes.
%
% Ensuite, parle d'un aspect essentiel de la méiose: la recombinaison homologue. Quelle est la chronologie de cet événement complexe (qui est encore loin d'être totalement élucidé). Et quels sont les modèles proposés pour le comprendre (mais qui ne sont que des modèles).
%
%
% Question à moi-même (pour Laurent): si DSB, pourquoi la partie cassée du chromosome ne part pas ailleurs dans le cytoplasme?
%
%









% Biblio souris - meiose
% O’Bryan, M. K. & Kretser, D. Mouse models for genes involved in impaired spermatogenesis. Int. J. Androl. 29, 76–89 (2006).

\section{Meiosis in the context of gametogenesis}

% \subsection{The production of gametes via two cellular divisions}
\subsection{A two-step division process to form gametes}

Most sexually-reproducing organisms have diploid cells, \textit{i.e.} cells that contain two sets of chromosomes: one from each parent.
The transmission of half this genetic material to their progeny goes through the formation of specialised haploid cells (\textit{i.e.} cells containing a single set of chromosomes) called “gametes”.
%, which later fuse with another coming from the second parent during a stage called “fertilisation” to form a “zygote”, the fertilised cell that will lead to the formation of a new individual. 
Such transition from diploidy to haploidy occurs during a particular type of cell division called “meiosis” (from the Greek word \textit{\textgreek{μ}\textepsilon\textgreek{ίωσις}}: “lessening”).

Despite some clear differences between organisms \citep{hunter2003synaptonemal, loidl2016conservation}, the most salient features of meiosis are conserved among eukaryotes, which suggests that their last common ancestor had already acquired it \citep{cavalier2002origins,ramesh2005phylogenomic,speijer2015sex} through a process that is still largely debated \citep{wilkins2009evolution,bernstein2010evolutionary,bernstein2011meiosis}.\\


% IMPORTANTS
% https://www.ncbi.nlm.nih.gov/books/NBK26840/
% Figure vient de: http://www.macmillanhighered.com/BrainHoney/Resource/6716/digital_first_content/trunk/test/hillis2e/hillis2e_ch07_5.html
% https://www.khanacademy.org/science/high-school-biology/hs-reproduction-and-cell-division/hs-meiosis/a/hs-meiosis-review
% https://www.khanacademy.org/science/biology/cellular-molecular-biology/meiosis/a/phases-of-meiosis

% LEFT PAGE
\begin{sidewaysfigure}[p]
	\centering
	\leftskip-3.4cm
	\rightskip-2.7cm
	\rotfloatpagestyle{empty}
	\includegraphics[width = 1.25\textwidth, trim = 0cm 0.61cm 0cm 0cm, clip]{figures/chap2/meiosis-with-cytology.eps}
	\captionsetup{width=1.22\textwidth, margin={-2.2cm, -3.3cm}}
	\caption{MMy caption bla bla bla y caption bla bla bla y caption bla bla bla y caption bla bla bla y caption bla bla bla y caption bla bla blapy caption bla bla bla y caption bla bla bla y caption bla bla bla y caption bla bla bla y caption bla bla bla y caption bla bla blapy caption bla bla bla y caption bla bla bla y caption bla bla bla y caption bla bla bla y caption bla bla bla y caption bla bla blapy caption bla bla bla y caption bla bla bla y caption bla bla bla y caption bla bla bla y caption bla bla bla y caption bla bla blapy caption bla bla bla y caption bla bla bla y caption bla bla bla y caption bla bla bla y caption bla bla bla y caption bla bla blapy caption bla bla bla y caption bla bla bla y caption bla bla bla y caption bla bla bla y caption bla bla bla y caption bla bla blapy caption bla bla bla y caption bla bla bla y caption bla bla bla y caption bla bla bla y caption bla bla bla y caption bla bla blap
		\citep{hillis2012principles}
	}
\label{fig:meiosis-cytological-steps}
\end{sidewaysfigure}

% % RIGHT PAGE
% \begin{sidewaysfigure}[p]
%     \centering
%     \leftskip-2.4cm
%     \rightskip-2.4cm
%     \rotfloatpagestyle{empty}
%     \includegraphics[width = 1.25\textwidth, trim = 0cm 0.61cm 0cm 0cm, clip]{figures/chap2/meiosis-with-cytology.eps}
%     \captionsetup{width=1.25\textwidth, margin={-2.2cm,-3.3cm}}
%     \caption{MMMy caption bla bla bla y caption bla bla bla y caption bla bla bla y caption bla bla bla y caption bla bla bla y caption bla bla blapy caption bla bla bla y caption bla bla bla y caption bla bla bla y caption bla bla bla y caption bla bla bla y caption bla bla blapy caption bla bla bla y caption bla bla bla y caption bla bla bla y caption bla bla bla y caption bla bla bla y caption bla bla blapy caption bla bla bla y caption bla bla bla y caption bla bla bla y caption bla bla bla y caption bla bla bla y caption bla bla blapy caption bla bla bla y caption bla bla bla y caption bla bla bla y caption bla bla bla y caption bla bla bla y caption bla bla blapy caption bla bla bla y caption bla bla bla y caption bla bla bla y caption bla bla bla y caption bla bla bla y caption bla bla blapy caption bla bla bla y caption bla bla bla y caption bla bla bla y caption bla bla bla y caption bla bla bla y caption bla bla blap y caption
%     \citep{hillis2012principles}
% }
% \label{fig:meiosis-cytological-steps}
% \end{sidewaysfigure}





Concretely, meiosis is preceded by a unique round of chromosome duplication occurring during the interphase of diploid germinal cells (ovocytes in females and spermatocytes in males).
% Basically, meiosis consists of two successive cell divisions preceded by a unique round of chromosome duplication occurring during the interphase of diploid germinal cells (ovocytes in females and spermatocytes in males).
Thence, before entering meiosis, each homologous chromosome (or “homologue”) \textit{i.e.} each parental copy, is formed of two identical double-helix DNA molecules called “sister chromatids” which are physically attached to one another at a point called the “centromere”. 
Therefrom, the two successive cell divisions that compose meiosis proceed to distribute the chromatids into four gametes.

The first meiotic division is also known as the “reductional division” because it separates the pairs of homologues to form two haploid cells. 
It is classically divided into four stages: prophase, metaphase, anaphase and telophase (Figure~\ref{fig:meiosis-cytological-steps}, top).
Prophase I, described more extensively in Section~\ref{chap2:prophase-I}, stages the pairing of homologous chromosomes along with recombination.
Next, the meiotic splindle bonds the paired homologues and lines them up on the equatorial plate during metaphase I, before separating them during anaphase I. 
Such segregation of homologues is achieved thanks to the existence of two opposite forces that stabilise the chromosomes until they are correctly oriented: first, the chiasmata that maintain the homologues attached and second, the meiotic spindle that creates a poleward tension \citep{petronczki2003menage}.
Segregation \textit{per se} terminates at telophase I during which the chromosomes decondense and a nuclear enveloppe (NE) forms around the nuclei.
At the end of the first meiotic division, each of the two haploid daughter cells (“secondary gametocytes”) contains one pair of sister chromatids corresponding either to the paternal or to the maternal homologue.

Following a short interkinesis during which DNA does not replicate, the second meiotic division splits the sister chromatids in a manner much similar to a haploid mitosis. 
This division is also known as the “equational division” since the number of chromosomes remains equal before and after it.
Like the first one, it is partitioned into four stages (Figure~\ref{fig:meiosis-cytological-steps}, bottom) which are executed synchroneously in the two secondary gametocytes.
During prophase II, the NEs break down and the chromatids recondense. In the meantime, the centrosomes duplicated during interkinesis move towards opposite poles while a new meiotic spindle forms between them and starts to capture chromatids.
The single chromosomes line up across the equational plates of each cell during metaphase II and the two sister chromatids segregate towards opposing poles during anaphase II\@.
At telophase II, the chromosomes begin to decondense and new NEs form around them, thus producing the final set of four genetically-unique haploid gametes.\\

% process of meiosis:
% https://cnx.org/contents/GFy_h8cu@9.87:GYZS3DDP@8/The-Process-of-Meiosis
% https://eggsnchromosomes.com/what-is-meiosis/
% https://www.khanacademy.org/science/biology/cellular-molecular-biology/meiosis/a/phases-of-meiosis
% http://www.macmillanhighered.com/BrainHoney/Resource/6716/digital_first_content/trunk/test/hillis2e/hillis2e_ch07_5.html
% https://www.ncbi.nlm.nih.gov/books/NBK26840/
% https://www.khanacademy.org/science/high-school-biology/hs-reproduction-and-cell-division/hs-meiosis/a/hs-meiosis-review
% https://www.britannica.com/science/meiosis-cytology


Even though these general features of meiosis are shared, its timing and the products it forms are sexually dimorphic in mammals \citep[reviewed in][]{handel2010genetics}. 
Indeed, male meiosis ends in the formation of four gametes (spermatids) whereas female meiosis ends in a single functional gamete and three non functional haploid cells called “polar bodies”.

As for the timing, in males, spermatogonia mature into spermatocytes which next initiate meiosis all along adulthood, thus resulting in a continuous production of sperm.
In contrast, one common conception in females is that the integrality of oogonia mature into ovocytes during fetal development \citep{pearl1921studies,zuckerman1951number}, even though recent findings suggest that oocyte production may be sustained in the postnatal ovaries \citep{johnson2004germline,johnson2005oocyte}.
In any case, female meiotic prophase I — initiated and arrested right after the production of ovocytes — is resumed in small batches of ovocytes at periodic intervals during the reproductive lifespan, but halts once again at metaphase II, until fertilisation by a spermatozoid (if it ever occurs) triggers the finalisation of the process.\\

While the transition from plain cell cycle to meiotic entry is managed by a complex body of checkpoints \citep[reviewed in][]{marston2005meiosis}, the metronomic completion of meiotic subprocesses is abundantly warranted by the capacity of chromosomes to respond to cell cycle controls \citep[reviewed in][]{mckim1995chromosomal}.












\section{Chronology of meiotic recombination}
cf Baudat et de Massy + ma presentation a ce sujet

cf aussi Neil hunter: interhomolg interaction + programmed DSB formation + CO control

\subsection{Initiation of recombination}
\subsection{Meiotic DSB repair}
\subsection{Resolution}


% https://www.ncbi.nlm.nih.gov/books/NBK21986/ — tetrad analysis utilisee pour mapper les doubles CO
% https://www.ncbi.nlm.nih.gov/books/NBK22106/ — Mitotic crossing-over

\section{Models of recombination}

Voir aussi: orr-weaver et szostak
Et parler de asymmetric VS symmetric heteroduplex DNA (orr-weaver and szostak — section Polarity)


% https://books.google.fr/books?id=7V0N6Tt8fUwC&pg=PA43&lpg=PA43&dq=murray+1960+polarity&source=bl&ots=mtj-qfJ1ZM&sig=ACfU3U1rKTqzCqEtcJkNw4ex96F_KPI87Q&hl=fr&sa=X&ved=2ahUKEwiG0b39-tfgAhUJ0RoKHRn4CWsQ6AEwB3oECAkQAQ#v=onepage&q=murray%201960%20polarity&f=false
Sur la polarité des gene conversion DONC des sites precis ou la recombinaison demarre (a mettre dans les points chauds de recombinaison).

N. Saitou, Introduction to Evolutionary Genomics, Computational Biology 17,
% DOI 10.1007/978-1-4471-5304-7_2, © Springer-Verlag London 2013
file:///Users/maudgautier/Downloads/9781447153030-c2.pdf
Recombination was discovered by Thomas Hunt Morgan and his colleagues in the
early twentieth century. The concept of “gene conversion” was fi rst proposed by
Winkler in 1930 [ 11  ], but it was not fully accepted for a long time, until studies on
fungi clearly showed conversion events [ 12, 13  ]. Holliday (1964; [ 14  ]) proposed the
“Holliday structure” model (Fig. 2.14) to connect gene conversion, or nonreciprocal
transfer of DNA fragment, and recombination.


Early studies on gene conversion were mostly restricted to fungal genetics. As
molecular evolutionary studies of multigene family started, unexpected similarity
of tandemly arrayed rRNA genes was found [ 15  ]. This phenomenon was termed
“concerted evolution,” and gene conversion or unequal crossing-over was proposed
to explain this characteristic of some multigene families (e.g., [ 16  ]). New statistical
methods were developed to detect gene conversion between homologous
sequences [ 17, 18  ]. Program GENECONV developed by Sawyer [ 19  ] became the
standard tool for analyzing gene conversions. We now know that conversion can
occur in any genomic region irrespective of genes (DNA regions having function)
or nongenic regions (e.g., [ 20  ]). However, “gene conversion” as technical jargon is
currently widely accepted, and I follow this nomenclature. Gene conversion can be
classifi ed into two types: intragenic or between alleles and intergenic or between
duplicated genes. 


When Winkler [ 11  ] proposed gene conversion in 1930, it was a deviation from
the Mendelian ratio. Later, detailed observations on baker’s yeast and Neurospora
[ 12, 13  ] established gene conversion, and Holliday’s [ 14  ] model transformed
gene conversion from phenomenon to mechanism. Nowadays several enzymes
are known to be involved in DNA strand exchanges [ 30  ]. Abundant genome
sequence data and their computational analyses again turned gene conversion or
more fl atly homogenization of homologous sequences from mechanism to phenomenon. We should be careful of any prejudice to a particular phenomenon when we
try to interpret them with certain mechanism. One phenomenon, such as homologous sequence homogenization, may occur not only via gene conversion but with
some other mechanisms, including one unknown to us at this moment. It is obvious
that we should grasp molecular mechanism of gene conversion, including enzymatic
machineries. 


\section{The importance of meiotic recombination}

Meiose essentiels, car certains mutatns qui empechent la mutation sont non viables (orr-weaver and szostak)

+ Parler de Gene conversion
% + Parler de Male vs Female meiosis  https://cellbiology.med.unsw.edu.au/cellbiology/index.php/Meiosis
+ errors in meiosis
+ regulation of meiois (cyclins)



Quand parle des maps de linkage: voir citation Muller %1920:98-101
“[I]t has never been claimed, in the theory of linear linkage, that the per cents of crossing over are actually proportional to the map distances: what has been stated is that the per cents of crossing overs are calculable from the map distances…”





\section*{Sorte de plan / Liste d'idees}
historic
initiation of recombination: 
- Homologue pairing / interhomolg interactions
- determinsation localisation
- formation DSB / programmed DSB formation
Meiotic DSB repair:
- Homoology search
- synapsis between homologues
- models of recombination
- Resolution into CO/NCO
- Dissolution of the SC
Crossover control: 
- assurance / interference
- differentiation CO/NCO
Chromatin state: shapes the recombination landscape (ou dans position des hotspots): nucleosome occupancy + meiotic chromosome architecture. 
Importance of meiotic recombination: Genetic disorders otherwise + exemple de l'un qui a perdu PRDM9 mais qui n'est pas stérile pour autant. 
Checkpoints
strand asymmetry
DNA polymerases
Gene conversion ici? + non-allelic gene conversion (et un impact sur la détection de)



Deuxieme chapitre:
Methodological approaches to study recombination
Variation of recombination rates wtithin genomes and among species
evolvability of recombination rates



\section*{Notes temporaires}

%%%%% Modele de Odenthal-Hesse (chapitre sur recombinaison)

% \section{Chronology of meiotic recombination}
% \subsection{Programmed DSB formation}
% \subsection{Strand invasion and junction molecule formation}
% \subsection{Mismatch repair}
% \subsection{Resolution}
%
% \section{Models of recombination}




%%%%% Modele de Papier Baudat de Massy 2013

% \section{Initiation of recombination}
% \subsection{Homologue pairing}
% \subsection{Programmed DSB formation}
%
% \section{Meiotic DSB repair}
% \subsection{Homology search}
% \subsection{Synapsis between homologues}
% \subsection{Models of DSB repair}
%
% \section{Resolution CO/NCO}
% \subsection{Differentiation CO/NCO}
% \subsection{CO interference}



Plan chronologique
\begin{itemize}
	\item Mammalian meiosis (overview of the cycle)
	\item (Zoom sur Prophase 1)
	\item Leptotene stage: Initiation of recombination (Homologue pairing before DSB + Determination localisation DSB + DSB) + Miotic DSB repair (homology search)
	\item Zygotene stage: Meiotic DSB repair (Synapsis between homologues + Start resolution CO/NCO)
	\item Pachytene stage: Meiotic DSB repair (Resolution CO/NCO)
	\item Pachytene + Zygotene stages: Preparation to metaphase I (dissolution of SC)

\end{itemize}

Plan Neil Hunter the essence of heredity
\begin{itemize}
	\item Meiosis and the roots of recombination research
	\item Molecular models of meiotic recombination
	\item Interhomolog interactions
	\item Programmed DSB formation
	\item Crossover control (Assurance and interference, differentiation CO/NCO, pro-CO role of the synaptonemal complex, recombination associated DNA synthesis)
	\item Resolving, disolving and unwinding joint molecules to implement CO and NCO fates (differential timing and regulation of CO and NCO formation, MutL and EXO1=CO-specific resolving factor, MUS81 enzymes = role in meiotic joint molecule processing, STR/BTR ensemble as master regulators of meiotic joint molecule metabolism, SLX4-associated endonucleases and he GEN1 resolvase, SMC complex facilitates joint molecules formation and resolution, implementing NCO formation)
	\item Clinical significance of meiotic recombination (Aneuploidy CO and advancing maternal age, meiotic recombination and genomic disorders, defective recombination and infertility)


\end{itemize}

Plan Mammalian Meiotic Recombination: A Toolbox for Genome Evolution (https://www.karger.com/Article/FullText/452822):
\begin{itemize}
	\item Recombination and he repair of DSBs (Organization of meiotic chromosomes: importance of chromosomal axes, molecular events involved in he formation and repair of DSBs)
	\item Methodological approaches to he study of recombination
	\item Genetic and epigenetic marks of DSBs and recombination hotspots 
	\item Variation of recombination rates within genomes and among species (Variability at the chromosomal level, variation of fine-scale recombination maps)
	\item Evolvability of recombination rates (Chromosomal rearrangements as recombination modifiers)
\end{itemize}

Plan de Hotposts for initiation of meiotic recombination (https://www.ncbi.nlm.nih.gov/pmc/articles/PMC6237102/)
\begin{itemize}
	\item Defining DSB hotspots
	\item Chromatin shapes the meiotic DSB landscape (Nucleosome occupancy, meiotic chromosome architecture)
	\item Meiotic DSB and crossover distributions
	\item PRDM9 and H3K4me3
	\item The hotspot paradox
	\item Recombination initiation in repetitive sequences
	\item Byond hotspots: DSB-dependent spatial regulation


\end{itemize}


Mechanismes moleculaires precis + molecules impliquees
\begin{itemize}
	\item Appariement des chromosomes
	\item Formation du DSB
	\item Reparation CO/NCO
	\item Tous les modeles de resynthese des brins
	\item Observation des parametres de recombinaison chez la levure, souris
	% \item https://www.cell.com/current-biology/pdf/S0960-9822(06)01257-7.pdf: surveillance of breaks = checkpoints
	\item interference des CO
	\item Notion de strand asymmetry
	\item DNA polymerases (https://www.ncbi.nlm.nih.gov/pmc/articles/PMC5295669/)
	\item https://www.karger.com/Article/FullText/452822: mammalian eiotic recombination: a toolbox for genome evolution

\end{itemize}


Molecules tres importantes sur lesquelles insister:
\begin{itemize}
	\item DMC1
	\item RAD51
	\item (PRDM9)
	\item Spo11
	\item MUS81
	\item MLH1
	\item HFM1
\end{itemize}

Autre:
\begin{itemize}
	\item Non-allelic gene conversion
	\item Recombining without hotspots (https://www.ncbi.nlm.nih.gov/pmc/articles/PMC4684701/)
	\item Knockout of PRDM9 (http://science.sciencemag.org.inee.bib.cnrs.fr/content/352/6284/474)

\end{itemize}




