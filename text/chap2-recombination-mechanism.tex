\begin{savequote}[8cm]
“[…] if there is one event in the whole evolutionary sequence at which my own mind lets my awe still overcome my instinct to analyse, and where I might concede that there may be a difficulty in seeing a Darwinian gradualism hold sway throughout almost all, it is this event — the initiation of meiosis.”
% “[…] if there is one event in the whole evolutionary sequence at which my own mind lets my awe still overcome my instinct to analyse […], it is this event — the initiation of meiosis.”
	
\qauthor{--- W. D. “Bill” Hamilton, \textit{\usebibentry{hamilton1996narrow}{title}} \citeyearpar{hamilton1996narrow} }

	% \textlatin{Neque porro quisquam est qui dolorem ipsum quia dolor sit amet, consectetur, adipisci velit...}

% There is no one who loves pain itself, who seeks after it and wants to have it, simply because it is pain...
  % \qauthor{--- Cicero's \textit{de Finibus Bonorum et Malorum}}
\end{savequote}

\chapter{\label{ch:2-recombination-mechanistics}Meiotic recombination, the essence/substrate of heredity} 
%\otherpagedecoration

\minitoc{}


Quand parle des maps de linkage: voir citation Muller %1920:98-101
“[I]t has never been claimed, in the theory of linear linkage, that the per cents of crossing over are actually proportional to the map distances: what has been stated is that the per cents of crossing overs are calculable from the map distances…”

% Dans partie meiose (partie 2):
Origine evolutive de la meiose debattue (transfo bacterienne ou mitose?)
Greek word


\section{Cytological aspects of meiosis}
+ 2 possibilities. EIther meiosis comes from mitosis, or from transfomration https://academic.oup.com/bioscience/article/60/7/498/234118

Stages + chromatin state + checkpoints + chiasmata + recombination nodules + synaptonemal complex

\section{Chronology of meiotic recombination}
cf Baudat et de Massy + ma presentation a ce sujet
\subsection{Initiation of recombination}
\subsection{Meiotic DSB repair}
\subsection{Resolution}


% https://www.ncbi.nlm.nih.gov/books/NBK21986/ — tetrad analysis utilisee pour mapper les doubles CO
% https://www.ncbi.nlm.nih.gov/books/NBK22106/ — Mitotic crossing-over

\section{Models of recombination}

Voir aussi: orr-weaver et szostak
Et parler de asymmetric VS symmetric heteroduplex DNA (orr-weaver and szostak — section Polarity)


% https://books.google.fr/books?id=7V0N6Tt8fUwC&pg=PA43&lpg=PA43&dq=murray+1960+polarity&source=bl&ots=mtj-qfJ1ZM&sig=ACfU3U1rKTqzCqEtcJkNw4ex96F_KPI87Q&hl=fr&sa=X&ved=2ahUKEwiG0b39-tfgAhUJ0RoKHRn4CWsQ6AEwB3oECAkQAQ#v=onepage&q=murray%201960%20polarity&f=false
Sur la polarité des gene conversion DONC des sites precis ou la recombinaison demarre (a mettre dans les points chauds de recombinaison).

N. Saitou, Introduction to Evolutionary Genomics, Computational Biology 17,
% DOI 10.1007/978-1-4471-5304-7_2, © Springer-Verlag London 2013
file:///Users/maudgautier/Downloads/9781447153030-c2.pdf
Recombination was discovered by Thomas Hunt Morgan and his colleagues in the
early twentieth century. The concept of “gene conversion” was fi rst proposed by
Winkler in 1930 [ 11  ], but it was not fully accepted for a long time, until studies on
fungi clearly showed conversion events [ 12, 13  ]. Holliday (1964; [ 14  ]) proposed the
“Holliday structure” model (Fig. 2.14) to connect gene conversion, or nonreciprocal
transfer of DNA fragment, and recombination.


Early studies on gene conversion were mostly restricted to fungal genetics. As
molecular evolutionary studies of multigene family started, unexpected similarity
of tandemly arrayed rRNA genes was found [ 15  ]. This phenomenon was termed
“concerted evolution,” and gene conversion or unequal crossing-over was proposed
to explain this characteristic of some multigene families (e.g., [ 16  ]). New statistical
methods were developed to detect gene conversion between homologous
sequences [ 17, 18  ]. Program GENECONV developed by Sawyer [ 19  ] became the
standard tool for analyzing gene conversions. We now know that conversion can
occur in any genomic region irrespective of genes (DNA regions having function)
or nongenic regions (e.g., [ 20  ]). However, “gene conversion” as technical jargon is
currently widely accepted, and I follow this nomenclature. Gene conversion can be
classifi ed into two types: intragenic or between alleles and intergenic or between
duplicated genes. 


When Winkler [ 11  ] proposed gene conversion in 1930, it was a deviation from
the Mendelian ratio. Later, detailed observations on baker’s yeast and Neurospora
[ 12, 13  ] established gene conversion, and Holliday’s [ 14  ] model transformed
gene conversion from phenomenon to mechanism. Nowadays several enzymes
are known to be involved in DNA strand exchanges [ 30  ]. Abundant genome
sequence data and their computational analyses again turned gene conversion or
more fl atly homogenization of homologous sequences from mechanism to phenomenon. We should be careful of any prejudice to a particular phenomenon when we
try to interpret them with certain mechanism. One phenomenon, such as homologous sequence homogenization, may occur not only via gene conversion but with
some other mechanisms, including one unknown to us at this moment. It is obvious
that we should grasp molecular mechanism of gene conversion, including enzymatic
machineries. 


\section{The importance of meiotic recombination}

Meiose essentiels, car certains mutatns qui empechent la mutation sont non viables (orr-weaver and szostak)

+ Parler de Gene conversion
% + Parler de Male vs Female meiosis  https://cellbiology.med.unsw.edu.au/cellbiology/index.php/Meiosis
+ errors in meiosis
+ regulation of meiois (cyclins)






\section*{Sorte de plan / Liste d'idees}
historic
initiation of recombination: 
- Homologue pairing / interhomolg interactions
- determinsation localisation
- formation DSB / programmed DSB formation
Meiotic DSB repair:
- Homoology search
- synapsis between homologues
- models of recombination
- Resolution into CO/NCO
- Dissolution of the SC
Crossover control: 
- assurance / interference
- differentiation CO/NCO
Chromatin state: shapes the recombination landscape (ou dans position des hotspots): nucleosome occupancy + meiotic chromosome architecture. 
Importance of meiotic recombination: Genetic disorders otherwise + exemple de l'un qui a perdu PRDM9 mais qui n'est pas stérile pour autant. 
Checkpoints
strand asymmetry
DNA polymerases
Gene conversion ici? + non-allelic gene conversion (et un impact sur la détection de)



Deuxieme chapitre:
Methodological approaches to study recombination
Variation of recombination rates wtithin genomes and among species
evolvability of recombination rates



\section*{Notes temporaires}

%%%%% Modele de Odenthal-Hesse (chapitre sur recombinaison)

% \section{Chronology of meiotic recombination}
% \subsection{Programmed DSB formation}
% \subsection{Strand invasion and junction molecule formation}
% \subsection{Mismatch repair}
% \subsection{Resolution}
%
% \section{Models of recombination}




%%%%% Modele de Papier Baudat de Massy 2013

% \section{Initiation of recombination}
% \subsection{Homologue pairing}
% \subsection{Programmed DSB formation}
%
% \section{Meiotic DSB repair}
% \subsection{Homology search}
% \subsection{Synapsis between homologues}
% \subsection{Models of DSB repair}
%
% \section{Resolution CO/NCO}
% \subsection{Differentiation CO/NCO}
% \subsection{CO interference}



Plan chronologique
\begin{itemize}
	\item Mammalian meiosis (overview of the cycle)
	\item (Zoom sur Prophase 1)
	\item Leptotene stage: Initiation of recombination (Homologue pairing before DSB + Determination localisation DSB + DSB) + Miotic DSB repair (homology search)
	\item Zygotene stage: Meiotic DSB repair (Synapsis between homologues + Start resolution CO/NCO)
	\item Pachytene stage: Meiotic DSB repair (Resolution CO/NCO)
	\item Pachytene + Zygotene stages: Preparation to metaphase I (dissolution of SC)

\end{itemize}

Plan Neil Hunter the essence of heredity
\begin{itemize}
	\item Meiosis and the roots of recombination research
	\item Molecular models of meiotic recombination
	\item Interhomolog interactions
	\item Programmed DSB formation
	\item Crossover control (Assurance and interference, differentiation CO/NCO, pro-CO role of the synaptonemal complex, recombination associated DNA synthesis)
	\item Resolving, disolving and unwinding joint molecules to implement CO and NCO fates (differential timing and regulation of CO and NCO formation, MutL and EXO1=CO-specific resolving factor, MUS81 enzymes = role in meiotic joint molecule processing, STR/BTR ensemble as master regulators of meiotic joint molecule metabolism, SLX4-associated endonucleases and he GEN1 resolvase, SMC complex facilitates joint molecules formation and resolution, implementing NCO formation)
	\item Clinical significance of meiotic recombination (Aneuploidy CO and advancing maternal age, meiotic recombination and genomic disorders, defective recombination and infertility)


\end{itemize}

Plan Mammalian Meiotic Recombination: A Toolbox for Genome Evolution (https://www.karger.com/Article/FullText/452822):
\begin{itemize}
	\item Recombination and he repair of DSBs (Organization of meiotic chromosomes: importance of chromosomal axes, molecular events involved in he formation and repair of DSBs)
	\item Methodological approaches to he study of recombination
	\item Genetic and epigenetic marks of DSBs and recombination hotspots 
	\item Variation of recombination rates within genomes and among species (Variability at the chromosomal level, variation of fine-scale recombination maps)
	\item Evolvability of recombination rates (Chromosomal rearrangements as recombination modifiers)
\end{itemize}

Plan de Hotposts for initiation of meiotic recombination (https://www.ncbi.nlm.nih.gov/pmc/articles/PMC6237102/)
\begin{itemize}
	\item Defining DSB hotspots
	\item Chromatin shapes the meiotic DSB landscape (Nucleosome occupancy, meiotic chromosome architecture)
	\item Meiotic DSB and crossover distributions
	\item PRDM9 and H3K4me3
	\item The hotspot paradox
	\item Recombination initiation in repetitive sequences
	\item Byond hotspots: DSB-dependent spatial regulation


\end{itemize}


Mechanismes moleculaires precis + molecules impliquees
\begin{itemize}
	\item Appariement des chromosomes
	\item Formation du DSB
	\item Reparation CO/NCO
	\item Tous les modeles de resynthese des brins
	\item Observation des parametres de recombinaison chez la levure, souris
	% \item https://www.cell.com/current-biology/pdf/S0960-9822(06)01257-7.pdf: surveillance of breaks = checkpoints
	\item interference des CO
	\item Notion de strand asymmetry
	\item DNA polymerases (https://www.ncbi.nlm.nih.gov/pmc/articles/PMC5295669/)
	\item https://www.karger.com/Article/FullText/452822: mammalian eiotic recombination: a toolbox for genome evolution

\end{itemize}


Molecules tres importantes sur lesquelles insister:
\begin{itemize}
	\item DMC1
	\item RAD51
	\item (PRDM9)
	\item Spo11
	\item MUS81
	\item MLH1
	\item HFM1
\end{itemize}

Autre:
\begin{itemize}
	\item Non-allelic gene conversion
	\item Recombining without hotspots (https://www.ncbi.nlm.nih.gov/pmc/articles/PMC4684701/)
	\item Knockout of PRDM9 (http://science.sciencemag.org.inee.bib.cnrs.fr/content/352/6284/474)

\end{itemize}




