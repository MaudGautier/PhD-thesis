\begin{savequote}[8cm]
	‘I don't claim to be a methodologist, but I act like one only because I do methodology to protect myself from crazy methodologists.’
	
	\qauthor{--- Ward Cunningham, \textit{\usebibentry{cunningham2004geek}{title}} \citeyearpar{cunningham2004geek} }
	
\end{savequote}

\chapter{\label{ch:5-methodology}High-resolution detection of recombination in single individuals}
%\otherpagedecoration

\minitoc{}

{\small{} \itshape{}

\paragraph{This chapter in brief —}

Because the existing approaches to study recombination at high resolution are extremely resource-intensive, we implemented a novel approach based on the unique-molecule genotyping of recombination-enriched sperm DNA from single highly heterozygous individuals.
We found that the main source of errors when genotyping unique recombinant molecules of DNA did not come from sequencing errors, but from alignment ambiguities — for the aligners are biased towards the reference genome.
Thus, searching for events after mapping fragments onto both parental genomes proved to be the most critical step of our pipeline.
In the end, our approach proved 100 times more powerful than current methods to detect recombination: it allowed to identify several thousands of recombination events in single individuals, with a false positive rate below 5\%.

}

\newpage


The existing approaches to study recombination events at high resolution are limited (see Chapter~\ref{ch:3-recombination-variation}).
For instance, the total number of events detectable with approaches like pedigree analyses is capped by the restricted number of meioses that can be analysed (one meiosis per individual sequenced).
In addition, since whole genomes are sequenced to retrieve these events, the cost/benefit ratio is particularly elevated for species with large genomes, like mammals.


Here, we propose a different procedure which rests on the unique-molecule genotyping of recombination-enriched sperm DNA from single highly heterozygous individuals (Figure~\ref{fig:experimental-design}).
In this chapter, I first describe how our experimental design led to an enrichment in detectable recombination events and how we implemented our unique-genotyping pipeline to identify such events and then discuss the impact of each component of our workflow onto the detectability of events.

\begin{figure}[p]
	\centering
	\includegraphics[width = 0.9\textwidth]{figures/inkscape/experimental-design-with-text.eps}
	\caption[Overview of the experimental design]
	{\textbf{Overview of the experimental design.}
		\par We performed a cross between a \textit{Mus musculus domesticus} (B6) and a \textit{Mus musculus castaneus} (CAST) mouse individual to obtain a F1 hybrid, from which we extracted sperm DNA, i.e.\ the substrate of recombination products.
		We then performed two rounds of DNA capture to target the 1,018 hotspots and 500 control regions selected, and sequenced captured DNA with an Illumina device, using a 250-pb paired-end protocol.
		At the end of this process, the pool of DNA was enriched in recombination events.
		B6 chromosomes and fragments of DNA are coloured in red and CAST chromosomes and fragments of DNA in yellow.
	}
\label{fig:experimental-design}
\end{figure}



\section{Overview of the experimental design}
\subsection{Acquisition of highly polymorphic individuals}

Detecting recombination events rests on one indispensable prerequisite: the presence of markers (i.e.\ polymorphic sites).

Therefore, we performed a cross between two subspecies of mice that present a high level of heterozygosity (1 SNP every 150 bp) \citep{keane2011mouse,yalcin2012nextgeneration} and that are known to hybridise naturally \citep{orth1998natural}: \textit{Mus musculus domesticus} (strain C57BL/6J, hereafter called B6) and \textit{Mus musculus castaneus} (strain CAST/EiJ, hereafter called CAST).
This cross resulted in F1 hybrid mice (B6xCAST), of which two males were selected. 
Sperm DNA was then collected from these two individuals and kindly given to us by D. Bourch'is (Institut Curie, Paris).

The extracted DNA from both biological replicates was then sonicated to produce fragments of a mean size of 350 bp.


\subsection{Enrichment in detectable recombination events}%Selection of target loci and DNA capture}

Since the recombination rate is relatively weak genome-wide, we wanted to target specifically recombination hotspots, i.e.\ regions of the genome where recombination massively occurs.
This required two steps: selecting hotspots, and performing DNA capture (i.e.\ hybridisation-based targeted-DNA enrichment) of these loci \citep[reviewed in \citealp{horn2012target}]{gnirke2009solution,hodges2007genomewide}.

\subsubsection{Selection of targets}

The recombination hotspots of B6xCAST mice had previously been identified \textit{via} chromatin immunoprecipitation followed by sequencing (ChIP-seq) of the PRDM9 protein \citep{baker2015prdm9}.
We restricted this known list of 6,758 hotspots to those (1) displaying a high marker density in the vicinity of the PRDM9 binding site (so as to increase the chance of detecting recombination events) and (2) aligning on their whole length on both the CAST and the B6 reference genome (so as to restrain mapping artifacts).

In practice, the selection criterium on heterozygosity (a minimum of 4 SNPs in the 300-bp central region of the locus centred on the PRDM9 peak summit) was the most stringent: it cut down the original list of 6,758 hotspots to only 1,261 hotspots.
The other two criteria on mappability (a strict maximum of 60 sites with low sequence quality in the 1-kb central region, and the absence of a large indel by ensuring that a minimum of 800 bp in the 1-kb from the B6 genome shared at least 90\% identity with the CAST genome) respectively discarded 205 and 38 additional loci.
Altogether thus, a total of 1,018 1-kb long regions centred on the summit of the PRDM9 ChIP-seq peaks were selected.
These were positioned randomly both across and along chromosomes (Figure~\ref{fig:hotspots-along-chromosomes}).



% RIGHT PAGE
\begin{sidewaysfigure}[p]
	\centering
	\leftskip-2.4cm
	\rightskip-2.4cm
	\rotfloatpagestyle{empty}
	\includegraphics[width = 1.25\textwidth]{figures/chap5/Recombination_events_along_chromosomes_bw.eps}
	\captionsetup{width=1.25\textwidth, margin={-2.2cm, -3.3cm}}
	\caption[Distribution of recombination events across the 1,018 selected hotspots positioned randomly along chromosomes]
	{\textbf{Distribution of recombination events across the 1,018 selected hotspots positioned randomly along chromosomes.}
		\par Chromosomes are represented in grey and oriented so that the centromere is on the bottom side of the figure (mouse chromosomes are acrocentric).
		The total number of recombination events identified is given by the length of the horizontal black bar at the position of each of the 1,018 selected hotspots.
	}
\label{fig:hotspots-along-chromosomes}
\end{sidewaysfigure}





In addition, we selected 500 1-kb control regions which displayed genomic characteristics similar to those of the 1,018 hotspots (in terms of GC-content, SNP density, sequence quality and content in transposable elements) but which did not belong to the list of known recombination hotspots.

\subsubsection{DNA capture}

\begin{figure}[b!]
	\centering
	\includegraphics[width = 1\textwidth]{figures/chap5/Absence_capture_bias_non_recombinant_B6.eps}
	\caption[Absence of capture bias between the B6 and CAST haplotypes]
	{\textbf{Absence of capture bias between the B6 and CAST haplotypes.}
		\par All fragments exclusively containing B6-typed markers were designed as non-recombinant fragments coming from the B6 haplotype. 
		The distribution of the proportion of such fragments across targets is reported in this figure. 
		The dashed line corresponds to the median proportion of B6-genotyped fragments across targets and the two dotted lines correspond to the 2.5 and 97.5 percentiles (i.e.\ the delimitation of the proportion for 95\% of targets).
	}
\label{fig:prop-B6-in-targets}
\end{figure}


To enrich the sequencing data in fragments coming from the 1,518 aforementioned loci (hereafter called targets), we performed either one or two rounds of DNA capture targeting them.
Since our final aim was to detect recombination events, i.e.\ fragments carrying both a portion of the B6 haplotype and a portion of the CAST haplotype, it was essential that the efficiency of the capture be similar for both haplotypes.
We thus designed two baits (one for each of the two haplotypes) for every target.

We next monitored the existence of any capture bias by looking at the origin of all the non-recombinant fragments.
Indeed, as recombination is rare, the vast majority of sequenced fragments do not correspond to recombination events and thus, half of all non-recombinant fragments should come from the B6 haplotype (and consequently, the other half from the CAST haplotype).
We found that this was indeed the case since the proportion of fragments containing only \textit{B6}-typed markers (i.e.\ coming from the B6 haplotype) revolved around 50\% for nearly all targets: 95\% of targets held in a [43.4\%; 55.6\%] range (Figure~\ref{fig:prop-B6-in-targets}).

Thus, although a small fraction of hotspots displayed a haplotype bias (possibly because one of the two baits better matched one of the two haplotypes), overall, there was no systematic bias favouring the capture of one haplotype relative to the other.
	



\subsection{Ultra deep-sequencing and mapping of captured DNA}

Libraries were sequenced by an Illumina device using a 250-bp paired-end protocol, except for 4 small libraries (out of 18) which contributed to 6\% of the total number of fragments and which were sequenced as a pilot experiment using a 100-bp paired-end protocol (Table~\ref{tab:characteristics-seq-mapping-capture}). 

We then mapped the sequenced reads to both the GRCm38/mm10 version of the B6 genome (\url{ftp://ftp-mouse.sanger.ac.uk/ref/}) and to the CAST/EiJ draft reference genome (\url{ftp://ftp-mouse.sanger.ac.uk/REL-1509-Assembly/}), using BWA-MEM \citep{li2009fast, li2013aligning} with default parameters and marking shorter split hits as secondary.
PCR duplicates were marked thanks to picardTools (version 1.98(1547)) \citep{picard2018toolkit} and pairs of reads which were either marked as unmapped, as secondary alignment\footnote{BWA marks a read as secondary-aligned in cases where it can align at several locations. The best hit (i.e.\ location with the best alignment score) is marked as primary alignment, while all others are marked as secondary alignment.} or as mapping in an improper pair\footnote{A proper pair flag is attributed by the aligner (here, BWA) to a pair of reads if the reads are oriented in an inward-facing direction and are mapped within 4 standard-deviations of the mean insert size of the block of $10^6$ read pairs to which they belong.} were filtered out, for they were not likely to be real fragments.\\

Overall, sequenced reads mapped equally well to both the B6 and the CAST reference genome assemblies (Table~\ref{tab:characteristics-seq-mapping-capture}).
In addition, DNA capture was efficient since 72\% of sequenced fragments mapped within the selected targets.
This resulted in a substantial coverage of the targeted loci: the mean sequencing depth on 1-kb long targets was 97,177 $x$, and it raised up to above 941,737 $x$ in certain regions.

The variation in coverage across hotspots was similar to that across control regions (data not shown) and it was rather limited as the variation in coverage of 80\% of hotspots held in a 5:1 ratio ([10; 90] quantiles = [71,745; 385,412] reads).
Nonetheless, we found that the variation in coverage across hotspots was highly correlated to the mean GC-content of the targets (Pearson correlation: ${r = -0.641}$; \textit{p}-val $< 2.2 \times 10^{-16}$).
Thus, apart from a GC-content effect, there was no large capture nor mapping bias across hotspots.


\begin{sidewaystable}[p]
    \centering
	\begin{adjustbox}{width = 1\textwidth}
    \begin{tabular}{rrrrrrrrrrrr}
        \toprule
        \textbf{} & \multicolumn{6}{c}{\textbf{Sample and sequencing characteristics}} & \multicolumn{2}{c}{\textbf{Mapping (\%)}} & \multicolumn{3}{c}{\textbf{Capture efficiency}} \\
		\cmidrule(l){2-7} \cmidrule(l){8-9} \cmidrule(l){10-12}
        \textbf{Library} & \textbf{Mouse} & \textbf{\# DNA} & \textbf{Rep.} & \textbf{Lane} & \textbf{Read} & \textbf{Library} & \textbf{Ref.} & \textbf{Ref.} & \textbf{\# Filtered} & \textbf{\% in} & \textbf{\# in} \\
        \textbf{ID} & \textbf{ID} & \textbf{capture} & \textbf{} & \textbf{ID} & \textbf{length} & \textbf{size} & \textbf{B6} & \textbf{CAST} & \textbf{Fragments} & \textbf{targets} & \textbf{targets} \\
        % \midrule
		\cmidrule(l){1-1} \cmidrule(l){2-7} \cmidrule(l){8-9} \cmidrule(l){10-12}
		A-1 & 1 & 2 & \textit{-} & \textit{-} & 100 bp & 14,977,880 & 99.87 & 99.76 & 7,206,235 & 86.09 & 6,203,730 \\
        A-2 & 1 & 1 & \textit{-} & \textit{-} & 100 bp & 12,457,816 & 99.75 & 99.27 & 5,813,649 & 26.86 & 1,561,461 \\
        A-3 & 2 & 2 & \textit{-} & \textit{-} & 100 bp & 16,000,000 & 99.85 & 99.73 & 7,631,724 & 84.88 & 6,478,370 \\
        A-4 & 2 & 1 & \textit{-} & \textit{-} & 100 bp & 13,291,526 & 99.74 & 99.24 & 6,110,086 & 24.29 & 1,484,199 \\
        % \midrule
		\cmidrule(l){1-1} \cmidrule(l){2-7} \cmidrule(l){8-9} \cmidrule(l){10-12}
		B-1 & 1 & 2 & \textit{-} & \textit{-} & 250 bp & 51,923,148 & 99.76 & 99.74 & 24,887,319 & 85.95 & 21,391,551 \\
        B-2 & 1 & 1 & \textit{-} & \textit{-} & 250 bp & 64,260,092 & 99.72 & 99.66 & 29,732,709 & 26.67 & 7,927,136 \\
        B-3 & 2 & 2 & \textit{-} & \textit{-} & 250 bp & 98,238,822 & 99.64 & 99.61 & 46,831,049 & 84.87 & 39,749,391 \\
        B-4 & 2 & 1 & \textit{-} & \textit{-} & 250 bp & 130,482,992 & 99.60 & 99.52 & 59,942,764 & 24.52 & 14,700,518 \\
        % \midrule
		\cmidrule(l){1-1} \cmidrule(l){2-7} \cmidrule(l){8-9} \cmidrule(l){10-12}
        C-1 & 1 & 2 & 1 & 1 & 250 bp & 67,775,154 & 99.79 & 99.75 & 33,221,010 & 82.54 & 27,421,183 \\
        C-2 & 1 & 2 & 1 & 2 & 250 bp & 69,564,748 & 99.78 & 99.75 & 34,102,906 & 82.65 & 28,185,446 \\
        C-3 & 1 & 2 & 2 & 1 & 250 bp & 79,002,218 & 99.79 & 99.75 & 38,818,821 & 84.05 & 33,013,536 \\
        C-4 & 1 & 2 & 2 & 2 & 250 bp & 81,074,012 & 99.77 & 99.74 & 39,841,630 & 85.14 & 33,921,167 \\
        C-5 & 2 & 2 & 1 & 1 & 250 bp & 60,911,138 & 99.71 & 99.68 & 29,876,369 & 85.45 & 25,530,140 \\
        C-6 & 2 & 2 & 1 & 2 & 250 bp & 62,042,362 & 99.71 & 99.67 & 30,437,699 & 85.55 & 26,039,414 \\
        C-7 & 2 & 2 & 2 & 1 & 250 bp & 66,489,166 & 99.77 & 99.74 & 32,701,170 & 86.18 & 28,182,084 \\
        C-8 & 2 & 2 & 2 & 2 & 250 bp & 68,382,888 & 99.76 & 99.73 & 33,636,524 & 86.28 & 29,022,692 \\
		C-9 & \textit{-} & 2 & 1\&2 & 1 & 250 bp & 9,796,678 & 83.27 & 84.44 & 3,801,189 & 62.12 & 2,361,438 \\
		C-10 & \textit{-} & 2 & 1\&2 & 2 & 250 bp & 10,156,096 & 82.77 & 84.93 & 3,920,630 & 62.04 & 2,432,278 \\
		% \midrule
		\cmidrule(l){1-1} \cmidrule(l){2-7} \cmidrule(l){8-9} \cmidrule(l){10-12}
		\textbf{Total} & \textbf{-} & \textbf{-} & \textbf{-} & \textbf{-} & \textbf{-} & \textbf{976,826,736} & \textbf{99.68} & \textbf{99.37} & \textbf{468,513,483} & \textbf{71.63} & \textbf{335,605,734} \\
        \bottomrule
    \end{tabular}
	\end{adjustbox}
    \caption[Sequencing, mapping and capture-efficiency summary metrics]
	{\textbf{Sequencing, mapping and capture-efficiency summary metrics.}
		\par The sperm from the two individuals was sequenced three times (horizontal panels): ‘A-libraries’ correspond to a pilot where reads were sequenced with a 2 $\times$ 100-bp protocol at a low coverage; ‘B-libraries’ were enriched in recombination events \textit{via} either one or two rounds of DNA-capture and were then sequenced deeply with a 2 $\times$ 250-bp protocol; ‘C-libraries’ were sequenced \textit{a posteriori} (after having analysed results of the ‘B-libraries’), using only samples with two rounds of DNA-capture to increase the total number of recombination events.
		\par The first vertical panel recaps the biological and sequencing characteristics of the samples. 
		The library size is the number of sequenced reads.
		\par The second vertical panel recaps the percentage of reads mapping on either of the two reference genomes (B6 and CAST).
		\par The third vertical panel recaps statistics on the efficiency of DNA-capture.
		Capture efficiency corresponds to the percentage of filtered fragments (i.e.\ fragments remaining after the removal of unmapped and secondary-aligned reads) that map within the targeted sites.
	}
\label{tab:characteristics-seq-mapping-capture}
\end{sidewaystable}





\section{The unique-molecule genotyping pipeline}
\label{sec:pipeline}

Since recombination ends in the juxtaposition of DNA from the two parental haplotypes, discerning recombination events comes back to spotting fragments presenting both \textit{B6}-typed and \textit{CAST}-typed genetic markers.
This requires two steps: disclosing the position of polymorphic sites and identifying the allele carried by a given fragment at all the markers it overlaps.


\subsection{Identification of polymorphic sites}

We performed variant-calling (i.e.\ the process of identifying variant (a.k.a.\ polymorphic) sites on a genome) with GATK\footnote{Other routine manipulations of files and visualisation of alignments were performed using the following tools and versions: SAMTools (version 1.4) \citep{li2009sequence}, BEDTools (version 2.26.0) \citep{quinlan2010bedtools}, JVarKit \citep{lindenbaum2015jvarkit}, the IGV interface (version 2.3\_88) \citep{robinson2011integrative}.} (version 3.3) \citep{mckenna2010genome}.

Basically, GATK performs four main steps: local insertion/deletion (indel) realignment, base quality score recalibration (BQSR), variant-calling \textit{per se} and variant quality score recalibration (VQSR).
Briefly, local indel realignment consists in transforming regions with indel-based misalignments into clean reads containing a consensus indel; BQSR consists in applying a score correction accounting for sources of systematic technical errors by modelling sequencing miscalls empirically; variant-calling allows to call both SNPs and indels; and VQSR provides an estimate of the probability that a called variant is a true genetic variant thanks to the establishment of an empirical model linking the latter likelihood to metrics describing the variants.\\

For all these steps, the GATK team recommends a number of best practices \citep{depristo2011framework, vanderauwera2013fastq} which, in many instances, require several external datasets of ‘true’ (i.e.\ validated by several independent studies) and ‘false’ variants which are available for human genomes only.
Therefore, we adapted the variant-calling process to mouse genomes as described hereunder, by using other types of datasets as close as possible to the recommendations from the GATK team. 

First, to perform local indel realignment, our list of known indels was made of all the indels found between the B6 strain and any of the other 35 strains of the version 5 release (\url{ftp://ftp-mouse.sanger.ac.uk/REL-1505-SNPs_Indels/}) of the mouse genomes project (MGP) \citep{keane2011mouse}.

Second, to perform base quality score recalibration (BQSR), our list of known indels was the same as that used for local indel realignment, and that of known SNPs was made of all the SNPs found between the B6 strain and any of the other 35 strains of the MGP\@.

Third, to perform variant quality score recalibration (VQSR), our list of true variant sites was made of all the sites (both SNPs and indels) found to vary between the B6 and the CAST strains by the MGP, and our list of both true variants and false positives of all the sites (both SNPs and indels) found to vary between the B6 strain and any of the other 16 strains of the MGP\@. 
The annotations we specified as covariates for the model were: the quality by read depth (QD), the overall mapping quality of reads supporting the variants called (MQ), the rank sum test for mapping qualities (MQRankSum), the rank sum test for the distance from the end of the reads (ReadPosRankSum), and two measures of strand bias (FS and SOR).
Last, since VQSR classifies variants according to their confidence, we discarded all those which were marked as ‘LowQuality’.




\subsection{Genotyping of individual DNA fragments}

Genotyping basically consists in comparing the allele carried by the analysed fragment at a given polymorphic site with those of the parental genomes.
Though, the accuracy of genotyping is subject to two main types of errors.

First, the variant-calling step can output a small proportion of false positives (FPs), for instance because of mapping artifacts in the vicinity of indels.
In that case, even if a fragment is correctly sequenced, a genotyping error may arise from these FP markers.
Since recombination is rare ($\ll 1\%$), the allelic frequencies at genuine polymorphic sites should comply with the Mendelian transmission of alleles.
To avoid any error coming from the aforementioned FPs, we thus applied a hard filter on these frequencies: only sites with allelic frequencies within the [36\%; 64\%]\footnote{The values of that range were deliberately set as relatively large to account for any difference in capture efficiency at individual hotspots (Figure~\ref{fig:prop-B6-in-targets})} were retained.
We additionally applied a hard filter on read coverage: any called variant supported by fewer than 100 reads was automatically discarded.

Second, sequencing errors directly lead to genotyping errors.
To avoid that, one can use the information provided by the sequencer: the Phred quality score which is logarithmically related to the probability for the base call to be incorrect \citep{ewing1998basecalling,ewing1998basecallinga}.
However, the Phred scores produced by the sequencing machines are subject to various sources of systematic technical error and sequencing machines generally underestimate the probability of error \citep{gatk2012base}.
Thus, we used the base quality scores recalibrated by GATK to filter out base calls with high probabilities of error: all sites with a recalibrated quality below 20 (i.e.\ with a probability to be miscalled greater than 1\%) were discarded.

We then genotyped each fragment at every of the remaining high-confidence variant sites that it overlapped by comparing its base call (or sequence of base calls in the case of indels) with that of the reference genome.
Whenever the fragment carried an allele distinct from that of the reference genome, we checked that the allele carried was that of the other parental genome to avoid misgenotyping any remaining erroneous base call.



\subsection{Identification of recombination events}

A simple way to test for the accuracy of our genotyping was to monitor the polymophic sites overlapped by the two reads of a given fragment: in principle, such markers should have the same genotype call on both reads.
In our data, only 0.3\%\footnote{Markers overlapped by both reads correspond to markers that are located at the end of reads. Since read extremities are more prone to both misalignments and sequencing errors \citep{kircher2009improved,minoche2011evaluation,abnizova2012analysis,wang2012estimation,laehnemann2016denoising}, the genotyping error rate provided here is likely to be overestimated.} (97 out of 32114) of all such markers were genotyped discordingly. 
Even if this seems to be a low error rate, it is not negligible in view of the scarcity of recombination events.
%cut -f13 $DATA/2_dBGC/6_ThirdSequencing/05_Analyses_of_Recombinants/02_Recombinants_and_False_Positives_Dataset/Recombinants_dataset.txt|sed '1,1d' | paste -s -d";"| awk '{print gsub(/1.01/, "")}'
%32114
%cut -f10 $DATA/2_dBGC/6_ThirdSequencing/05_Analyses_of_Recombinants/02_Recombinants_and_False_Positives_Dataset/Recombinants_dataset.txt|sed '1,1d' |paste -s -d";" | awk '{print gsub(/DISCORDANT/, "")}'
%97
%97/32114 = 0.003020489506134396
Therefore, to avoid false positives (FPs) due to genotyping errors, we identified fragments as recombination events when they bore a minimum of 2 \textit{CAST}- and 2 \textit{B6}-typed markers.

Last, since targets were sequenced deeply, a non-negligible portion of the fragments sequenced were likely to have arisen from PCR duplicates. 
Therefore, we discarded all events which showed an homologue both starting and ending at the same genomic position, so as to be sure to retain only one copy of any given recombination event in our dataset.\\

Finally, aside from sequencing errors, alignment ambiguities can lead to false positive calls (see Section~\ref{sec:determinants}).
These depend on the aligner and its parameters, — among which the reference genome.
Thus, we performed the whole procedure (mapping, variant-calling, marker selection, recombination event identification) twice: once using the B6 parental genome as reference, and once using the CAST parental genome as reference.


\section{The determinants of sensitivity and specificity}
\label{sec:determinants}

\subsection{An unprecedentedly powerful approach}


Since none of the 500 control loci correspond to known recombination hotspots, they should host few — or no — recombination events. 
Therefore, the number of recombination events detected in these control regions provides an upper limit for the number of false positives (FP) and, as hotspots and controls share similar genomic characteristics, the FP rate is expected to be comparable in both backgrounds.

All in all, 18,821 recombination events were retrieved in the 1,018 selected hotspots, and we estimated the maximum FP rate to be 3.73\% (Table~\ref{tab:FP-rate}).




\begin{table}[h]
    \centering
    \begin{tabular}{rrrrr}
        \toprule
        \textbf{Target} & \textbf{Nb of} & \textbf{Nb of} & \textbf{Nb of} & \textbf{Event rate} \\

		\textbf{category} & \textbf{targets} & \textbf{fragments} & \textbf{events} & \textbf{($\times$ 10\textsuperscript{-6})} \\

        \midrule
        Hotspots & 1,018 & 228,984,512 & 18,821 & 82.2 \\ 
        Controls & 500 & 106,850,906 & 328 & 3.07 \\
        \midrule
        \multicolumn{1}{r}{\textbf{FP rate}} & \multicolumn{4}{r}{\textbf{3.73 \%}} \\
        \bottomrule
    \end{tabular}
    \caption[Number of events detected in hotspot and control targets]
	{\textbf{Number of events detected in hotspot and control targets.}
		\par Events (false positives (FPs) or genuine recombination events) were detected using the unique-molecule genotyping pipeline described in Section~\ref{sec:pipeline}.
		All fragments or events overlapping at least 1 bp with a given target are counted in this table.
		The event rate corresponds to the ratio of candidate recombination events over the total number of fragments.
		The maximum false positive (FP) rate is the ratio of the event rate in control targets over that in hotspots.
	}
\label{tab:FP-rate}
\end{table}

Altogether, our approach displayed a much better efficiency/cost ratio than comparable methods to characterise recombination events at high resolution.

Indeed, in a recent study carried on mice by \citet{li2018highresolution}, the sequencing of 119 genomes of mice at a 12--30-\textit{x} coverage (which corresponds to about 6,742 Gb sequenced) ended in the identification of 4,075 recombination events.
In contrast, our approach required the sequencing of a total of 980 million 250-bp long reads (which corresponds to 244 Gb sequenced) to retrieve 18,821 recombination events.
Thus, the number of recombination events detected per Gb sequenced was over 100 times superior with our method (77.1 events/Gb) than in that by \citet{li2018highresolution} (0.604 events/Gb).

Despite the fact that humans and flycatchers have a recombination rate respectively twice and six times as high as mice \citep{kawakami2014highdensity,kawakami2017wholegenome}, our approach remained largely more powerful than what two other recent studies achieved on these species \textit{via} a pedigree analysis.
Indeed, \citet{halldorsson2016rate} sequenced 530 whole human genomes with a sequencing depth of over 30-\textit{x} (which corresponds to approximately 50,000 Gb sequenced) and detected 485 recombination events (after applying very stringent selection criteria); 
and \citet{smeds2016highresolution} sequenced the genomes of 11 birds at a mean 42-\textit{x} coverage (which, since the flycatcher genome is 1.1 Gb long \citep{ellegren2012genomic}, corresponds to about 500 Gb sequenced) and identified 592 events.
Therefore, their approaches respectively led to the detection of only 0.00970 and 1.18 events per Gb sequenced.

Altogether thus, our approach was indisputably much more powerful than comparable studies in detecting recombination events.



\subsection{The critical step: mapping onto both genomes}


Performing the whole procedure twice (once for each of the two parental genomes used as reference) was absolutely critical to the specificity of our approach.

Indeed, when several alignment alternatives exist for a given fragment to map at a particular genomic location, aligners (like BWA) are programmed to select the alternative with the best score.
But, for the similarity between the mapped fragment and the reference genome to be maximal \citep{smith1981identification}, the penalty associated to opening a gap (i.e.\ for an indel) is generally higher than that associated to a sequence of several mismatches.
As a consequence, read extremities, especially when they encompass an indel, are generally misaligned in a way that better matches the sequence of the reference genome than they truly do.
In other words, mapping is biased towards the reference genome.

In principle, local realignment around indels corrects a large part of these reference-biased misalignments.
Nevertheless, in view of the rarity of recombination events, a non-negligible portion of these misalignments remain and are likely to lead to spurious detections of recombination events.
Therefore, to counterbalance this mapping bias, we decided to perform the whole procedure using consecutively the two parental genomes as the mapping reference.\\

We found this \textit{modus operandi} to be crucial to the specificity of our method.
Indeed, all else being equal (same values for all the other filters), performing the procedure using only the B6 genome as a reference resulted in 1,088,237 events found in the hotspots.
This meant that only 1.7\% (18,821 out of 1,088,237) of all the events detected on the B6 genome were validated on the CAST genome and thus, that over 98\% of the events detected with only one genome used as a reference likely corresponded to FPs.
% >>> 1088237/176039928
% 0.006181762355640137
% >>> 686595/89697868
% 0.00765452975983777
% >>> 0.006181762355640137/0.00765452975983777
% 0.807595312787856
% >>> 0.00765452975983777/0.006181762355640137
% 1.238243937483929

In contrast, when the procedure was repeated onto the other reference genome (CAST), the FP rate dropped down to below 5\% (Table~\ref{tab:FP-rate}).
Thus, identifying events based on the mapping onto \textit{both} parental genomes was, by far, the most crucial step to the specific detection of recombination events.



\subsection{Impact of the filters on the false positive (FP) rate} 

The shortness of sequenced read pairs circumscribed the number of polymorphic sites accessible on each fragment ($median = 7$; $mean = 7.66$) and, to the difference of pedigree analyses where all fragments carry the same allele because they all arise from the same recombination event, the DNA fragments in the sperm we analysed originated from millions of distinct meioses and were thus to be genotyped individually.
Therefore, any sequencing error made by the Illumina device — which occurs at low \citep{fox2014accuracy,pfeiffer2018systematic} but non-negligible rates as compared to that of recombination events — may be fatal to the accurate genotyping of unique molecules.
The several filtering steps we added all along our pipeline to ensure genotyping a high accuracy all had an impact on the sensitivity and the specificity of our method, as discussed hereunder.\\


First, the filters set for variant selection (no strong departure from the Mendelian transmission of alleles and a minimal number of reads supporting the variant) were not stringent: only 7\% of all markers across all sequenced fragments were eliminated, and these corresponded principally to variants located at the extremities of target regions\footnote{The targets spanned 1 kb, but our analysis extended to an additional 1 kb on both the 5’- and the 3’-end of each target, so as to include all sequenced fragments overlapping at least 1 bp of the target (\textit{NB}: the maximum fragment length was 800 bp).}.

The filter on the base quality score had a greater impact on the specificity of our method.
Indeed, all else being equal, not setting this filter resulted in a 32\% rate of FPs (as compared to the 3.7\% rate when the filter was on).
% >>> (717/182991)/(3492/281537)
% 0.31590075853166194
% >>> (155/173494)/(2693/263187)
% 0.0873122778380181
% >>> (17/67747)/(1046/116352)
% 0.027912646876675138
This shows that most genotyping errors, aside from those originating from misalignments, arose from sequencing miscalls.

As for the removal of PCR duplicates, it divided the total number of events by a factor 2.
We note that the fragments we discarded at this step (i.e.\ those starting and ending at the exact same genomic locations) were likely to be genuine PCR duplicates since the vast majority ($> 95\%$) of pairs of identically-located fragments were found inside the same sperm sample.



\begin{table}[t]
	\centering
	\begin{adjustbox}{width = 1\textwidth}
		\begin{tabular}{rrrrrrrr}
			\toprule
			\textbf{} & \multicolumn{3}{c}{\textbf{Hotspot targets}} & \multicolumn{3}{c}{\textbf{Control targets}} & \textbf{} \\
			\cmidrule(l){2-4} \cmidrule(l){5-7} 
			\textbf{Threshold} & \textbf{\# Fragments} & \textbf{\# Events} & \textbf{\textperthousand} & \textbf{\# Fragments} & \textbf{\# Events} & \textbf{\textperthousand} & \textbf{FP (\%)} \\


			\midrule
			1+1  & 28,282,623 & 172,414 & 6.10 & 14,154,263 & 59,550 & 4.21 & 69 \\
			\textbf{2+2}  & \textbf{19,965,597} & \textbf{11,059} & \textbf{0.554} & \textbf{10,563,573} & \textbf{75} & \textbf{0.0071} & \textbf{1.3} \\
			3+3  & 12,105,005 & 3,101 & 0.256 & 6,734,622 & 6 & 0.00089 & 0.35 \\
			4+4  & 6,378,813 & 874 & 0.137 & 3,840,375 & 0 & 0.00 & 0.0 \\
			\bottomrule
		\end{tabular}
	\end{adjustbox}
	\caption[Impact of the minimum requirement of \textit{B6}- and \textit{CAST}-typed markers on the FP rate]
	{\textbf{Impact of the minimum requirement of \textit{B6}- and \textit{CAST}-typed markers on the FP rate.}
		\par A threshold $n_1+n_2$ corresponds to a minimum requirement of $n_1$ \textit{B6}-typed and $n_2$ \textit{CAST}-typed markers for a fragment to be identified as an event (FP or genuine recombination event).
		The event rate corresponds to the ratio of candidate recombination events over the total number of fragments.
		The maximum false positive (FP) rate is the ratio of the event rate in control targets over that in hotspots.
		The line in bold corresponds to the selected threshold: 2 \textit{B6}-typed and 2 \textit{CAST}-typed markers.
		The values reported in this table correspond to the results obtained on a subset (100,000,000 fragments) of our whole dataset. At this stage of the process, PCR duplicates have not been removed.
	}
\label{tab:threshold-recombinant}
\end{table}



Last but not least, we considered that a fragment was a recombination event if at least two of its markers were \textit{B6}-typed and two were \textit{CAST}-typed.
This was a minimum requirement, since applying a less stringent filter of only one \textit{B6}-typed and one \textit{CAST}-typed marker led to a much higher FP rate of 69\% (Table~\ref{tab:threshold-recombinant}).
This means that, in spite of all the filters that were set, the genotyping error rate remained sufficiently high to call spurious recombination events and thus needed to be double-checked (by requiring at least two genotype calls).\\



However, the obvious limitation of this filter is that it prevents the retrieval of events with conversion tracts overlapping only one polymorphic site.
More generally, the detectability of recombination events depends on the polymorphism: the lower the SNP density, the fewer events are detectable.
Therefore, the recombination parameters that can be directly observed are likely to differ from the real recombination parameters.
This limitation calls for the use of inferential methods to obtain the real (and undetectable) parameters of recombination, as will be described in the following chapter.









