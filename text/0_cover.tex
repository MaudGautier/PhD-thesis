%%%%%%%%%%%%%%%%
% modèle de page de garde pour une thèse de l'Université de Lyon (version de mars 2016)
% Attention ! Encodage UTF8 ! Sinon adapter en conséquence...
%%%%%%%%%%%%%%%%

\newgeometry{inner=2cm, outer=2cm, top=2cm, bottom=2cm}
% \newgeometry{top=2.5cm, bottom=2.5cm}

% pour ne pas conserver les parametres lies a cette page specifique dans le reste du doc (notamment fontsize et parindent)
% https://tex.stackexchange.com/questions/45028/cancel-undo-a-setlength-parindent3ex
\begingroup

% \selectlanguage{french}



\setlength{\parindent}{0pt}
\thispagestyle{empty}


\begin{center}
\includegraphics[height=3cm]{figures/logo_centred} %le fichier "logo" doit être dans le même dossier que le fichier tex
\end{center}


\fontsize{11pt}{13pt}\selectfont
% N\textsuperscript{o} d'ordre NNT: \textbf{A INDIQUER}%2016LYSE1334

\vspace{1cm}

\begin{center}
\fontsize{14pt}{16pt}\selectfont
\textbf{\uppercase{Thèse de doctorat de l'université de Lyon}}\\
\fontsize{12pt}{14pt}\selectfont
opérée au sein de\\
\textbf{l'Université Claude Bernard Lyon 1}

\vspace{0.5cm}

\textbf{École Doctorale ED341\\% rectifier le numéro d'accréditation
Evolution Ecosyst\`emes Microbiologie Modélisation (E2M2)}% nom complet de l'école doctorale

\vspace{0.5cm}

\textbf{Spécialité de doctorat : Génomique évolutive}\\
% Discipline : Génomique évolutive % éventuellement


\vspace{1.5cm}

Soutenue publiquement le 25 septembre 2019, par :\\
\vspace{0.1cm}
\fontsize{14pt}{16pt}\selectfont
\textbf{Maud GAUTIER}

\vspace{1.5cm} % adapter à la longueur du titre

\rule[20pt]{\textwidth}{0.5pt}

\fontsize{25pt}{28pt}\selectfont
% \textbf{Recombination as a driver of genome evolution: characterisation of biased gene conversion in mice}
\textbf{La recombinaison comme moteur de l'\'evolution des g\'enomes : caract\'erisation de la conversion g\'enique biais\'ee chez la souris}

\rule{\textwidth}{0.5pt}

\vspace{2cm} % adapter à la longueur du titre
\end{center}

\fontsize{12pt}{14pt}\selectfont
Devant le jury composé de :
\bigskip\bigskip

\fontsize{11pt}{13pt}\selectfont
%
Mme Dominique \textsc{Mouchiroud}, Professeur, Universit\'e Claude Bernard Lyon 1 \hfill Pr\'esident du jury\bigskip % mention "président" à ne préciser qu'après la soutenance
%
% \bigskip

Mme Val\'erie \textsc{Borde}, Directrice de Recherche, CNRS/Institut Curie \hfill  Rapporteur\\
%
M. Adam \textsc{Eyre-Walker}, Professeur, Universit\'e de Sussex (Royaume-Uni) \hfill Rapporteur\\
%
Mme Gwenael \textsc{Piganeau}, Directrice de Recherche, CNRS/BIOM \hfill Rapporteur\\
%
M. Bertrand \textsc{Llorente}, Directeur de Recherche, CNRS/CRCM \hfill Examinateur\bigskip

M. Laurent \textsc{Duret}, Directeur de Recherche, CNRS/LBBE \hfill Directeur de th\`ese\\
%
%Nom Prénom, grade/qualité, établissement/entreprise \hfill Invité(e) % le cas échéant

% %%%%
\newpage\thispagestyle{empty}
\null
%\strut ou ~ ou \mbox{} ou \null
\newpage
% % \cleardoublepage
% \thispagestyle{empty}
% \begin{minipage}{0.95\textwidth}
% \begin{center}
% \begin{large}\textbf{UNIVERSITE CLAUDE BERNARD - LYON 1}\end{large}
% \vspace{0.8cm}
%
% \begin{footnotesize}
% \begin{tabular}{p{8cm}p{6.75cm}}
% \textbf{Président de l'Université} & \textbf{M. le Professeur Frédéric FLEURY} \\
% Prédisent du Conseil Académique & M. le Professeur Hamda BEN HADID\\
% Vice-président  du Conseil d'Administration & M. le Professeur Didier REVEL\\
% Vice-président du Conseil Formation et Vie Universitaire & M. le Professeur Philippe CHEVALIER\\
% Vice-président de la Commission Recherche & M. Fabrice VALLÉE\\
% Directeur Général des Services & M. Alain HELLEU\\
% \end{tabular}
% \end{footnotesize}
% \end{center}
% \end{minipage}
% \vfill
% \begin{minipage}{0.95\textwidth}
% \begin{center}
% \vspace{0.5cm}
% \textit{\textbf{COMPOSANTES SANTE}}
% \vspace{0.2cm}
% \begin{footnotesize}
% \begin{tabular}{p{8cm}p{6.75cm}}
% Faculté de Médecine Lyon Est – Claude Bernard & Directeur : M. le Professeur J. ETIENNE\\
% Faculté de Médecine et de Maïeutique Lyon Sud – Charles Mérieux & Directeur : Mme la Professeure C. BURILLON\\
% Faculté d’Odontologie  & Directeur : M. le Professeur D. BOURGEOIS\\
% Institut des Sciences Pharmaceutiques et Biologiques & Directeur : Mme la Professeure C. VINCIGUERRA\\
% Institut des Sciences et Techniques de la Réadaptation & Directeur : M. X. PERROT\\
% Département de formation et Centre de Recherche en Biologie Humaine & Directeur : Mme. la Professeure A-M. SCHOTT\\
% \end{tabular}
% \end{footnotesize}
% \end{center}
% \end{minipage}
% \vfill
% \begin{minipage}{0.95\textwidth}
% \begin{center}
% \vspace{0.5cm}
% \textit{\textbf{COMPOSANTES ET DEPARTEMENTS DE SCIENCES ET TECHNOLOGIE}}
% \vspace{0.2cm}
% \begin{footnotesize}
% \begin{tabular}{p{8cm}p{6.75cm}}
% Faculté des Sciences et Technologies & Directeur : M. le Professeur F. DE MARCHI\\
% Département Biologie & Directeur : M. le Professeur F. THEVENARD\\
% Département Chimie Biochimie & Directeur : Mme C. FELIX\\
% Département GEP & Directeur : M. H. HAMMOURI\\
% Département Informatique & Directeur : M. le Professeur S. AKKOUCHE\\
% Département Mathématiques & Directeur : M. le Professeur G. TOMANOV\\
% Département Mécanique & Directeur : M. le Professeur H. BEN HADID\\
% Département Physique & Directeur :  M. le Professeur J-C PLENET \\
% UFR Sciences et Techniques des Activités Physiques et Sportives & Directeur : M. Y.VANPOULLE   \\
% Observatoire des Sciences de l’Univers de Lyon & Directeur : M. B. GUIDERDONI \\
% Polytech Lyon & Directeur : M. le Professeur E. PERRIN\\
% École Supérieure de Chimie Physique Électronique & Directeur : M. G. PIGNAULT\\
% Institut Universitaire de Technologie de Lyon 1 & Directeur : M. le Professeur C. VITON\\
% Institut Universitaire de Formation des Maîtres & Directeur : M. le Professeur A. MOUGNIOTTE\\
% Institut de Science Financière et d'Assurances & Directeur : M. N. LEBOISNE
% \end{tabular}
% \end{footnotesize}
% \end{center}
% \end{minipage}
% \newpage\thispagestyle{empty}
% \null
% %\strut ou ~ ou \mbox{} ou \null
% \newpage
%
%%%%%%%%%%%%%%%%%%%%%%%%
\endgroup
\restoregeometry
