

\section*{Abstract}

During meiosis, recombination hotspots host the formation of DNA double-strand breaks (DSBs). DSBs are subsequently repaired through a process which, in a wide range of species, is biased towards the favoured transmission of G and C alleles: GC-biased gene conversion (gBGC). This fundamental distorter of meiotic segregation mimics the action of positive selection and plays a major role in the evolution of base composition in the vicinity of recombination hotspots.

Here, we coupled capture-seq and bioinformatic techniques to implement an approach that proved 100 times more powerful than current methods to detect recombination. With it, we identified 18,821 crossing-over (CO) and non-crossover (NCO) events at very high resolution in single individuals and could thus precisely characterise patterns of recombination in mice.
In this species, recombination hotspots are targeted by PRDM9 and are therefore subject to a second type of biased gene conversion (BGC): DSB-induced BGC (dBGC). Quantifying both dBGC and gBGC with our data brought to light the fact that, in cases of structured populations, past gBGC from the parental lineages is hitchhiked by dBGC when the populations cross.
Next, we dissociated the hitchhiking effect to directly measure the intensity of gBGC (b) in both COs and NCOs. We observed that, in male mice, only NCOs — and more particularly single-marker NCOs — contribute to b. In contrast, in humans, both NCOs and at least a portion of COs (those with complex conversion tracts) distort allelic frequencies. This suggests that the DSB repair machinery leading to gBGC has evolved extremely rapidly within the mammalian clade. 
Our findings are also consistent with the hypothesis of a selective pressure restraining the intensity of the deleterious gBGC process at the population-scale: this would materialise through a multi-level compensation of the effective population size by the recombination rate, the length of conversion tracts and the transmission bias.

Altogether, our work has allowed to better comprehend how recombination and biased gene conversion proceed in the mammalian clade. 


\newpage
\section*{Résumé en français}

Au cours de la méiose, les points chauds de recombinaison sont le siège de la formation de cassures double-brin de l’ADN. Ces dernières sont ensuite réparées par un processus qui, chez de nombreuses espèces, favorise la transmission des allèles G et C : la conversion génique biaisée vers GC (gBGC). Cet important distorteur de la ségrégation méiotique mime les effets de la sélection positive et joue un rôle majeur dans l’évolution de la composition nucléotidique au voisinage des points chauds.

Dans cette étude, en couplant des développements bioinformatiques à une technique de capture ciblée d’ADN suivie de séquençage haut-débit (capture-seq), nous avons réussi à mettre au point une approche qui s’est révélée 100 fois plus performante pour détecter les événements de recombinaison que les méthodes existant actuellement. Ainsi, nous avons pu identifier 18 821 crossing-overs (COs) et non-crossovers (NCOs) à très grande résolution chez des individus uniques, ce qui nous a permis de caractériser minutieusement la recombinaison chez la souris.
Chez cette espèce, les points chauds de recombinaison sont ciblés par la protéine PRDM9 et sont donc soumis à une deuxième forme de conversion génique biaisée (BGC) : le biais d’initiation (dBGC). La quantification du dBGC et du gBGC à partir de nos données nous a permis de mettre en lumière le fait que, au moment où des populations structurées s’hybrident, le gBGC des lignées parentales est propagé par un phénomène d’auto-stop génétique (genetic hitchhiking) provenant du dBGC.
Ensuite, nous avons dissocié ce phénomène d’auto-stop pour mesurer directement l’intensité du gBGC (b) dans les COs et les NCOs. Nous avons vu que, chez les souris m\^ales, seuls les NCOs — et plus particulièrement les NCOs contenant un seul marqueur génétique— contribuent au b. En comparaison, chez l’Homme, à la fois les NCOs et au moins une part des COs (ceux qui présentent des tracts de conversion complexes) distordent les fréquences alléliques. Ceci suggère que la machinerie de réparation des cassures double-brin qui induit le biais de conversion génique (BGC) a évolué extrêmement rapidement au sein des mammifères. Nos résultats sont aussi en accord avec l’hypothèse selon laquelle une pression de sélection limiterait l’intensité de ce processus délétère à l’échelle de la population. Cela se traduirait par une compensation de la taille efficace de population à de multiples niveaux : par le taux de recombinaison, par la longueur des tracts de conversion et par le biais de transmission.

Somme toute, notre travail a permis de mieux comprendre la façon dont la recombinaison et la conversion génique biaisée opèrent chez les mammifères.

% \newpage
% \section*{Résumé étendu en français}
%
%
%
% inkscape
% contour rouge
% 310a33ff
% couleur rouge
% 993333ff
% couleur jaune
% efbc00ff
% contour jaune
% d59c00ff
% Image souris
% https://www.google.com/imgres?imgurl=https%3A%2F%2Funixtitan.net%2Fimages%2Fmice-clipart-silhouette-2.png&imgrefurl=https%3A%2F%2Funixtitan.net%2Fexplore%2Fmice-clipart-silhouette%2F&docid=WP3tqoNXVwdgBM&tbnid=lJVmSkpwVUXs6M%3A&vet=12ahUKEwjLgcqR1IbjAhVOzYUKHdNQC8c4rAIQMygVMBV6BAgBEBY..i&w=591&h=410&bih=848&biw=1860&q=house%20mouse&ved=2ahUKEwjLgcqR1IbjAhVOzYUKHdNQC8c4rAIQMygVMBV6BAgBEBY&iact=mrc&uact=8#h=410&imgdii=lJVmSkpwVUXs6M:&vet=12ahUKEwjLgcqR1IbjAhVOzYUKHdNQC8c4rAIQMygVMBV6BAgBEBY..i&w=591
% https://omaharentalads.com/explore/mice-clipart-silhouette/
%
%
%
%
%
