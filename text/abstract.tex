

\section*{Abstract}

During meiosis, recombination hotspots host the formation of DNA double-strand breaks (DSBs). DSBs are subsequently repaired through a process which, in a wide range of species, is biased towards the favoured transmission of G and C alleles: GC-biased gene conversion (gBGC).
The intensity of this fundamental distorter of meiotic segregation strongly varies between species but the factors dictating its evolution are not known.
We thus aimed at directly quantifying the transmission bias in mice and comparing the parameters on which it depends with other mammals.
% To better understand this fundamental distorter of meiotic segregation, we aimed at directly quantifying the transmission bias in mice and comparing the parameters on which in depends with other mammals.
% Better understanding this fundamental distorter of meiotic segregation thus requires to directly quantify the transmission bias and compare the parameters on which in depends across several species.
% To identify the latter and thus better understand this fundamental distorter of meiotic segregation, we aimed at directly quantifying the transmission bias in mice and comparing the parameters on which in depends with other mammals.
% This fundamental distorter of meiotic segregation
% proceeds with distinct intensities
% plays a major role in genome evolutiona
% mimics the action of positive selection and plays a major role in the evolution of base composition in the vicinity of recombination hotspots.

Here, we coupled capture-seq and bioinformatic techniques to implement an approach that proved 100 times more powerful than current methods to detect recombination. With it, we identified 18,821 crossing-over (CO) and non-crossover (NCO) events at very high resolution in single individuals and could thus precisely characterise patterns of recombination in mice.
In this species, recombination hotspots are targeted by PRDM9 and are therefore subject to a second type of biased gene conversion (BGC): DSB-induced BGC (dBGC). Quantifying both dBGC and gBGC with our data brought to light the fact that, in cases of structured populations, past gBGC from the parental lineages is hitchhiked by dBGC when the populations cross.
% Next, we dissociated the hitchhiking effect to directly measure the intensity of gBGC ($b$) in both COs and NCOs.
We next observed that, in male mice, only NCOs — and more particularly single-marker NCOs — contribute to the intensity of gBGC. In contrast, in humans, both NCOs and at least a portion of COs (those with complex conversion tracts) distort allelic frequencies. 
This suggests that the DSB repair machinery leading to gBGC varies across mammals.
% has evolved extremely rapidly within the mammalian clade.
Our findings are also consistent with the hypothesis of a selective pressure restraining the intensity of the deleterious gBGC process at the population-scale: this would materialise through a multi-level compensation of the effective population size by the recombination rate, the length of conversion tracts and the transmission bias.

Altogether, our work has allowed to better comprehend how recombination and biased gene conversion proceed in the mammalian clade.\\

\textbf{Keywords:} Recombination, Biased gene conversion, PRDM9, Hotspots, Genomics, Molecular evolution, Mammals, Sperm-typing.


\newpage
\section*{Résumé en français}

Au cours de la méiose, les points chauds de recombinaison sont le siège de la formation de cassures double-brin de l’ADN. Ces dernières sont ensuite réparées par un processus qui, chez de nombreuses espèces, favorise la transmission des allèles G et C : la conversion génique biaisée vers GC (gBGC). 
L'intensité de cet important distorteur de la ségrégation méiotique varie fortement entre espèces mais les facteurs déterminant son évolution sont toujours inconnus. 
Nous avons donc voulu quantifier directement le biais de transmission chez la souris et comparer les paramètres dont il dépend avec d'autres mammifères.
% Cet important distorteur de la ségrégation méiotique mime les effets de la sélection positive et joue un rôle majeur dans l’évolution de la composition nucléotidique au voisinage des points chauds.

Dans cette étude, en couplant des développements bioinformatiques à une technique de capture ciblée d’ADN suivie de séquençage haut-débit (capture-seq), nous avons réussi à mettre au point une approche qui s’est révélée 100 fois plus performante pour détecter les événements de recombinaison que les méthodes existant actuellement. Ainsi, nous avons pu identifier 18 821 crossing-overs (COs) et non-crossovers (NCOs) à très grande résolution chez des individus uniques, ce qui nous a permis de caractériser minutieusement la recombinaison chez la souris.
Chez cette espèce, les points chauds de recombinaison sont ciblés par la protéine PRDM9 et sont donc soumis à une deuxième forme de conversion génique biaisée (BGC) : le biais d’initiation (dBGC). La quantification du dBGC et du gBGC à partir de nos données nous a permis de mettre en lumière le fait que, au moment où des populations structurées s’hybrident, le gBGC des lignées parentales est propagé par un phénomène d’auto-stop génétique (genetic hitchhiking) provenant du dBGC.
% Ensuite, nous avons dissocié ce phénomène d’auto-stop pour mesurer directement l’intensité du gBGC ($b$) dans les COs et les NCOs.
Nous avons ensuite pu observer que, chez les souris m\^ales, seuls les NCOs — et plus particulièrement les NCOs contenant un seul marqueur génétique— contribuent à l'intensité du gBGC. En comparaison, chez l’Homme, à la fois les NCOs et au moins une part des COs (ceux qui présentent des tracts de conversion complexes) distordent les fréquences alléliques. 
Ceci suggère que la machinerie de réparation des cassures double-brin qui induit le biais de conversion génique (BGC) présente des variations au sein des mammifères.
%a évolué extrêmement rapidement au sein des mammifères. 
Nos résultats sont aussi en accord avec l’hypothèse selon laquelle une pression de sélection limiterait l’intensité de ce processus délétère à l’échelle de la population. Cela se traduirait par une compensation de la taille efficace de population à de multiples niveaux : par le taux de recombinaison, par la longueur des tracts de conversion et par le biais de transmission.

Somme toute, notre travail a permis de mieux comprendre la façon dont la recombinaison et la conversion génique biaisée opèrent chez les mammifères.\\

\textbf{Mots-clés:} Recombinaison, Conversion génique biaisée, PRDM9, Points chauds, Génomique, \'Evolution moléculaire, Mammifères, Sperm-typing.

\newpage
\section*{Résumé étendu en français}

{
\setstretch{1.15}

Lorsque l'on traite de l'évolution des génomes, trois forces sont classiquement invoquées : la mutation, la sélection naturelle et la dérive génétique.
% La première est la source
% La deuxième
% La troisième
%
Toutefois, depuis une vingtaine d'année, une autre force a fait son entrée sur la scène évolutive~: la conversion génique biaisée, que nous noterons ‘BGC’ (de l'anglais \textit{biased gene conversion}).
Ce phénomène est une conséquence directe du processus de recombinaison méiotique chez les espèces à reproduction sexuée.

Chez les mammifères en effet, après s'être fixée à certains loci cibles appelés ‘points chauds de recombinaison’, la protéine PRDM9 recrute la machinerie de formation de cassures double-brin et marque, de ce fait, l'initiation d'un événement de recombinaison \citep{baudat2010prdm9,myers2010drive,parvanov2010prdm9}.
Ce dernier doit ensuite être réparé en utilisant le chromosome homologue comme matrice, ce qui mène à ce qu'on appelle un événement de conversion génique, c'est-à-dire le transfert non-réciproque d'une information de séquence d'ADN\@.

Toutefois, si PRDM9 présente une plus grande affinité de liaison avec la séquence de l'un des deux chromosomes (que nous appellerons ‘haplotype’), la cassure s'initiera préférentiellement sur cet haplotype, et l'événement de conversion génique se fera donc préférentiellement dans un sens donné : c'est ce qu'on appelle le biais d'initiation, aussi appelé conversion génique biaisée induite par cassure double brin et noté ‘dBGC’ (de l'anglais \textit{double-strand break-induced biased gene conversion}).
Du fait de ce phénomène, les points chauds finissent nécessairement par s'éroder : comme l'haplotype portant le motif ciblé par PRDM9 est le siège de la cassure, il est systématiquement converti par l'autre haplotype, et voué à dispara\^itre \citep{boulton1997hotspot}.

Il existe une deuxième forme de conversion génique biaisée : la conversion génique biasée vers GC, que l'on notera ‘gBGC’ (de l'anglais \textit{GC-biased gene conversion}).
En effet, il a été observe chez plusieurs espèces 
de façon directe \citep{mancera2008highresolution, si2015widely, williams2015noncrossover, halldorsson2016rate, keith2016high, smeds2016highresolution}
ou indirecte \citep{escobar2011gcbiased,pessia2012evidence,figuet2014biased}
que la réparation des cassures double-brin favorise les allèles G et C par rapport aux allèles A et T\@.\\


% les événements de recombinaison sont initiés par la formation de cassures double-brin, dont la position est déterminée par la protéine PRDM9.
% Après s'être fixée à certains loci selon son affinité de liaison avec ceux-ci, cette dernière recrute la machinerie de formation des cassures double-brin.
% Suite à cela,
% Cette dernière se lie à certains loci spécifiques d'autant plus fortement que son affinité de liaison avec eux est élevée et
%
%
% En effet, les événements de recombinaison sont initiés par la formation d'une cassure double-brin qui est ensuite réparée gr\^ace à l'action d'une machinerie de réparation de ces cassures.
% Or, la position de ces cassures est déterminée par la protéine PRDM9 qui se lie d'autant plus fortement à certains loci que son affinité de liaison avec ceux-ci est forte, et recrute
% en fonction de son affinité de liaison avec certains loci,

% - une consequence : si des mutations sur un des haplotypes, PRDM9 se lie plus sur celui non mute, et donc, le mute est donneur dans conversion. C'est le paradoxe des hotspots
% - cela mene a dBGC = biais d'initiation, dont une des consequences est lerosion des hotspots

% - aussi une autre forme: lors de la reparation, il a ete observe que l'allele GC est souvent favorise chez many species: on appelle ca le gBGC






La quantification du coefficient de conversion génique biaisée à l'échelle des populations ($B$) chez un grand nombre de métazoaires \citep{galtier2018codon} a mis en évidence un résultat étonnant: 
l'intensité du gBGC ne varie que dans une gamme de valeurs très restreinte.
Par exemple, chez les mammifères placentaires, $B$ reste dans une fourchette de 0 à 7 \citep{lartillot2013phylogenetic}.
\'Etant donné que $B$ correspond au produit de la taille efficace de population ($N_e$) par le coefficient de gBGC ($b$) et que la taille efficace peut varier sur plusieurs ordres de grandeurs parmi les métazoaires, $b$ ne peut mécaniquement pas être identique chez toutes les espèces.
Au contraire, un ou plusieurs des paramètres dont $b$ dépend (le taux de recombinaison $r$, la longueur des tracts de conversion $L$ et le biais de transmission $b_0$) varient nécessairement inversement à la taille efficace.


Cependant, peu de données sont disponibles pour comprendre la base de la dépendance entre $N_e$ et $b$: le biais de transmission ($b_0$) n'a été mesuré que chez quelques espèces \citep{mancera2008highresolution, si2015widely, williams2015noncrossover, halldorsson2016rate, keith2016high, smeds2016highresolution} et, parmi les mammifères, la seule espèce chez qui ce biais a été mesuré de façon directe (\textit{Homo sapiens}) présente une très faible taille efficace d'environ 10,000 \citep{takahata1993allelic,erlich1996hla,harding1997archaic,charlesworth2009fundamental,yu2004nucleotide}.

Afin d'apporter un éclairage nouveau sur l'interaction entre $b$ et $N_e$, nous avons donc voulu quantifier le gBGC chez une autre espèce de mammifères présentant une taille efficace beaucoup plus grande que celle de l'Homme \citep{geraldes2008inferring,phifer-rixey2012adaptive,davies2015factors}: la souris \textit{Mus musculus}.\\


Pour pouvoir quantifier précisément le gBGC, il est nécessaire de disposer d'un grand nombre d'événements de recombinaison.
Or, la méthode généralement utilisée pour détecter ces événements — l'analyse de pedigrees — est extrêmement gourmande en ressources : 
elle requiert le séquençage de génomes complets d'un grand nombre d'individus et permet de détecter seulement un nombre limité de recombinants.
Nous avons donc mis au point une nouvelle approche permettant de détecter plusieurs milliers de recombinants à très haute résolution chez des individus uniques.

% De plus, afin de maximiser le nombre de recombinants détectables, nous avons sélectionné 1 018 points chauds de recombinaison particulièrement denses en marqueurs hétérozygotes
% Concrètement, notre approche repose sur le génotypage de molécules d'ADN uniques issues du sperme de souris hybrides.
% Afin de maximiser le nombre de recombinants détectables, nous avons, au préalable, réalisé une étape de capture ciblée d'ADN provenant de 1 018 points chauds de recombinaison particulièrement denses en marqueurs hétérozygotes.
% Brièvement, notre approche repose sur le génotypage de molécules d'ADN uniques issues du sperme de souris hybrides enrichi en événements de recombinaison gr\^ace au ciblage spécifique de 1 018 points chauds de recombinaison denses en sites polymorphes.
Concrètement, notre approche repose sur deux étapes principales.
Premièrement, puisque la recombinaison n'est identifiable qu'à partir du génotypage de marqueurs hétérozygotes, nous avons croisé deux races de souris (C57BL/6J que nous noterons ‘B6’ et CAST/EiJ que nous appellerons ‘CAST’) issues de deux sous-espèces (\textit{Mus musculus domesticus} et \textit{Mus musculus castaneus}) présentant un fort taux de polymorphisme de 0.74\% \citep{keane2011mouse,yalcin2012nextgeneration}.
Les points chauds de recombinaison chez l'hybride F1 qui résulte de ce croisement (B6xCAST) ont déjà été identifiés par d'autres que nous \citep{baker2015prdm9}.
Afin de maximiser le nombre de recombinants détectables, nous en avons donc sélectionné 1 018 qui sont particulièrement denses en marqueurs hétérozygotes.
Nous avons ensuite enrichi l'ADN du sperme de cet hybride en fragments provenant de ces loci gr\^ace à une technique de ciblage spécifique suivie de séquençage haut-débit (capture-seq).

La deuxième étape de notre procédure consiste à génotyper les molécules séquencées de façon individuelle, et d'identifier, parmi ces dernières, celles correspondant à des événements de recombinaison.
Toute la difficulté de cette analyse réside dans le fait que les molécules sont uniques: dès lors, toute erreur de séquençage ou toute ambiguïté d'alignement peut devenir une source d'erreur à l'origine de faux positifs (i.e.\ de fragments détectés comme recombinants alors qu'ils ne le sont pas).
Lors de la mise en œuvre de notre approche, nous nous sommes rendus compte que les anomalies les plus critiques à cet égard provenaient de l'étape d'alignement car celui-ci est biaisé vers le génome de référence.
L'étape cruciale de notre méthode a donc été d'effectuer la procédure en utilisant successivement les deux génomes parentaux comme référence.

Au final, notre approche s'est révélée extrêmement performante.
A titre de comparaison, les études récentes ayant obtenu des cartes de recombinaison à haute résolution chez l'Homme, la souris ou l'oiseau \citep{halldorsson2016rate,smeds2016highresolution,li2018highresolution} se sont montrées plus de cent fois moins puissantes que notre approche pour détecter ces événements.\\


L'approche que nous avons mise au point nous a permis de détecter 18 821 événements de recombinaison chez la souris et donc de caractériser précisément la recombinaison sur environ un millier de points chauds (jusqu'alors, ceci n'avait été fait que sur une poignée de points chauds).

En premier lieu, nous avons pu observer l'étendue de la variation du taux de recombinaison entre les points chauds et identifier quelques uns de ses déterminants.
En particulier, l'affinité de liaison entre la protéine PRDM9 et son motif cible est parfaitement proportionnelle à l'activité recombinationnelle du point chaud.
Toutefois, les points chauds dont les deux haplotypes (celui venant de B6 et celui venant de CAST) présentent un différentiel d'affinité à PRDM9 important (les points chauds dits ‘asymétriques’) ont un taux de recombinaison fortement réduit (d'un facteur deux à quatre) par rapport à l'attendu basé sur l'intensité du signal PRDM9.

Un certain nombre d'événements de recombinaison (en particulier ceux dont le tract de conversion ne chevauche aucun marqueur polymorphe) ne sont pas détectables.
Dès lors, les paramètres de recombinaison observés — comme la longueur des tracts de conversion, le taux de recombinaison et le ratio de COs et de NCOs — ne sont pas forcément représentatifs des paramètres de recombinaison réels.
Pour pouvoir estimer ces paramètres réels, il est donc nécessaire de passer par des méthodes inférentielles telles que la méthode bayésienne approchée (\textit{approximate bayesian computation}) qui consiste à simuler le processus biologiques avec différents paramètres et à sélectionner les simulations dont le résultat est proche des observations biologiques.
Par ce biais, nous avons pu estimer de façon indirecte les paramètres de recombinaison chez la souris : les tracts de conversion des COs mesurent 450 paires de bases en moyenne contre 35 pour les NCOs, et le taux de recombinaison moyen sur l'ensemble des points chauds que nous avons étudié est de 30 cM/Mb.\\


Ensuite, en cherchant à quantifier le biais de transmission ($b_0$) des allèles GC et donc l'intensité du gBGC ($b$) chez la souris, nous avons remarqué que, dans un dispositif expérimental tel que le nôtre, ce biais était affecté par l'autre forme de conversion génique: le biais d'initiation (dBGC).
En effet, prenons le cas de deux populations possédant deux allèles \textit{Prdm9} distincts évoluant donc de façon indépendante dans leurs lignées respectives.
Dans chacune des lignées, les points chauds ciblés par l'allèle présent s'érodent sous l'effet du dBGC et s'enrichissent en même temps en allèles G et C sous l'effet du gBGC\@.
Lorsque l'on croise deux individus issus de ces deux lignées, l'allèle \textit{Prdm9} initie la cassure double-brin sur l'haplotype pour lequel il a la plus grande affinité, c'est-à-dire l'haplotype de la lignée avec laquelle il n'a \textit{pas} co-évolué, puisque celle dans laquelle il se trouvait a vu ses points chauds s'éroder.
Ainsi, c'est l'haplotype de sa lignée d'origine — qui est localement enrichi en GC — qui sera systématiquement le donneur lors de l'événement de conversion génique.
De ce fait, le gBGC qui a eu lieu dans les lignées parentales est propagé par un phénomène d’auto-stop génétique (\textit{genetic hitchhiking}) provenant du dBGC\@.

Pour pouvoir quantifier le gBGC correctement, il fallait donc contrôler pour cet effet d'auto-stop, ce que nous avons fait en sous-échantillonnant les tracts de conversion analysés pour égaliser le nombre d'événements de conversion ayant un donneur B6 à ceux ayant un donneur CAST\@.
Dès lors, nous avons pu quantifier le gBGC et observer que le biais de transmission ($b_0$) est nul pour les COs et extrêmement faible chez les NCOs contenant plusieurs marqueurs génétiques (NCO-2+). 
En revanche, le biais est très élevé pour les NCOs contenant un seul marqueur (NCO-1) : l'intensité du biais est comparable à ce qui a été observé chez l'humain \citep{halldorsson2016rate}.\\

% En effet, lorsque deux populations possédant deux allèles \textit{Prdm9} distincts évoluent de façon indépendante pendant un laps de temps suffisamment long pour que les points chauds ciblés par chaque allèle s'érodent dans leurs lignées respectives, croiser deux individus issus de ces deux lignées amènera forcément à une situation dans laquelle le g
%
% En effet, si deux populations possédant deux allèles \textit{Prdm9} distincts évoluent de façon indépendante pendant longtemps (relativement à la vitesse d'évolution)
%
% if two populations with distinct Prdm9 alleles have evolved independently during a length of time sufficient for the hotspots targeted by each allele to erode specifically in their lineage, crossing them together will end in dBGC hitchhiking past gBGC
%

% Cette quantification de l'intensité du biais chez la souris, qui est une espèce à forte taille efficace, nous a
A partir de là, nous avons pu comparer la relation entre l'intensité du gBGC ($b$) et la taille efficace de population ($N_e$) chez les deux espèces de mammifères pour lesquelles le biais de transmission ($b_0$) a été quantifié de façon directe : la souris et l'Homme.
Nos analyses indiquent que le taux de recombinaison et la longueur des tracts de conversion participent tous deux à limiter l'intensité du gBGC ($b$) chez la souris par rapport à l'Homme et, bien que les données disponibles à l'heure actuelle soient insuffisantes pour le confirmer, il semblerait le biais de transmission des COs y participe également.
% nous avions de nouveaux éléments permettant d'apporter quelques réponses à la question originelle de cette étude : la relation entre l'intensité du gBGC ($b$) et la taille efficace de population ($N_e$).
% Chez la souris

Globalement, ces observations sont compatibles avec l'hypothèse selon laquelle une pression de sélection limiterait l’intensité de ce processus délétère à l’échelle de la population par le biais d'une compensation de la taille efficace de population à de multiples niveaux : par le taux de recombinaison, par la longueur des tracts de conversion et, peut-être, par le biais de transmission des COs.\\


Enfin, la méthode de détection des recombinants à l'échelle d'individus uniques est tout indiquée pour étudier le rôle individuel de gènes impliqués dans le processus de recombinaison.
Pour ce faire, il faut analyser des individus homozygotes pour une version inactivée du gène d'intérêt mais présentant tout de même un haut niveau d'hétérozygotie pour que la recombinaison soit détectable.
Comme des individus F2 issus du croisement de trois lignées distinctes peuvent présenter de telles caractéristiques alors que des individus F1 issus d'un unique croisement ne le peuvent pas, il nous a fallu adapter notre méthode à un tel schéma de croisement.
% Concrètement, nous avons dû distinguer les marqueurs génétiques

Suite à cela, nous avons pu analyser le rôle du gène \textit{Hfm1}, une hélicase d'ADN essentielle à la résolution des cassures double-brin en COs : nous avons observé que son inactivation menait à un taux de recombinaison plus élevé et à des tracts de conversion de COs sensiblement plus courts que chez les individus non mutants.\\



Somme toute, notre travail a mené à la mise au point d'une approche originale de détection de la recombinaison à haute résolution et à faible coût chez des individus uniques.
Cette approche ouvre la voie à l'étude plus poussée des gènes impliqués dans le processus de recombinaison et nous a permis de mieux comprendre la façon dont la recombinaison et la conversion génique biaisée opèrent chez les mammifères.


% Ainsi, notre étude a permis d'observer l'étendue de la variation du taux de recombinaison entre les points chauds et

}





%
%
% inkscape
% contour rouge
% 310a33ff
% couleur rouge
% 993333ff
% couleur jaune
% efbc00ff
% contour jaune
% d59c00ff
% Image souris
% https://www.google.com/imgres?imgurl=https%3A%2F%2Funixtitan.net%2Fimages%2Fmice-clipart-silhouette-2.png&imgrefurl=https%3A%2F%2Funixtitan.net%2Fexplore%2Fmice-clipart-silhouette%2F&docid=WP3tqoNXVwdgBM&tbnid=lJVmSkpwVUXs6M%3A&vet=12ahUKEwjLgcqR1IbjAhVOzYUKHdNQC8c4rAIQMygVMBV6BAgBEBY..i&w=591&h=410&bih=848&biw=1860&q=house%20mouse&ved=2ahUKEwjLgcqR1IbjAhVOzYUKHdNQC8c4rAIQMygVMBV6BAgBEBY&iact=mrc&uact=8#h=410&imgdii=lJVmSkpwVUXs6M:&vet=12ahUKEwjLgcqR1IbjAhVOzYUKHdNQC8c4rAIQMygVMBV6BAgBEBY..i&w=591
% https://omaharentalads.com/explore/mice-clipart-silhouette/
%
%
%
%
%
