\begin{savequote}[8cm]
	
	‘So my antagonist said, “Is it impossible that there are flying saucers? Can you prove that it's impossible?” “No”, I said, “I can't prove it's impossible. It's just very unlikely”. At that he said, “You are very unscientific. If you can't prove it impossible then how can you say that it's unlikely?” But that is the way that is scientific. It is scientific only to say what is more likely and what less likely, and not to be proving all the time the possible and impossible.’
	
	\qauthor{--- Richard Feynman, \textit{\usebibentry{feynman1964character}{title}} \citeyearpar{feynman1964character} }
	
\end{savequote}

\chapter{\label{ch:6-recombination-parameters}Characterisation of recombination in mouse autosomal hotspots}
%\otherpagedecoration


\minitoc{}

{\small{} \itshape{}

\paragraph{This chapter in brief —}

Even if recombinational activity is known to vary by orders of magnitude across individual hotspots, the properties determining this variation are still poorly understood.
Further progress in comprehending the basis of these fluctuations can only arise with the thorough examination of individual hotspots but, in mammals, only a handful of these have been directly characterised at high resolution.
Here, thanks to the 18,821 events that we detected with the approach developed in Chapter~\ref{ch:5-methodology}, we identified some of the main factors governing the recombinational activity of individual hotspots, we precisely described recombination in over a thousand hotspots and we estimated the hidden biological parameters of recombination through an inferential approach.
Overall, this study provides the first global picture of recombination patterns in mouse autosomal hotspots.

}

\newpage


Although many of the molecular details of the recombination process have been dissected (see Chapter~\ref{ch:2-recombination-mechanistics}) and the recombinational activity is known to vary by orders of magnitude across mouse individual hotspots \citep{paigen2008recombinational}, the properties determining this variation are still poorly understood.
So far, in mice, only a handful of recombination hotspots have been directly characterised by sequencing recombination products in sperm or oocytes \citep{yauk2003highresolution,bois2007highly,baudat2007cis,ng2008quantitative,cole2010comprehensive,cole2014mouse}.

Recently, several genome-wide hotspot maps have been obtained, either by ChIP-seq against either PRDM9 \citep{baker2015prdm9}, RAD51 or DMC1 \citep{smagulova2011genomewide} or thanks to the sequencing-based detection of DMC1-bound ssDNA \citep{khil2012sensitive, brick2012genetic} or SPO11 oligos \citep{lange2016landscape}.
However, all these techniques give only indirect information on recombination:
ChIP-seq against PRDM9 reflects its binding affinity to a given locus but does not indicate the associated recombination rate;
ChIP-seq against DMC1 reveals both the DSB rate and the repair efficiency, but the two phenomena are indistinctable with this sole method;
and the sequencing-based detection of SPO11 oligos requires an extremely large amount of material and, thus, so far, is only available for one dataset of \textit{Mus musculus domesticus} mice.
As such, all these approaches only provide information on the intermediary steps of recombination, but none at all on its outcome.
Therefore, to characterise recombination, it appeared essential to use another method allowing to directly study its products.\\

% on the intermediary steps of recombination, but none on its outcome.
% On top of that, the DMC1 ChIP-seq reflects indistinctly the DSB rate and the repair efficiency, while the sequencing-based detection of SPO11 oligos requires an extremely large amount of material and, thus, is only avaliable for one dataset of \textit{Mus musculus domesticus} mice.
% Therefore, to characterise recombination, it appeared essential to use another approach allowing to directly study the products of recombination.

% - Pour voir les hotspots, on peut utiliser un ChIP-seq PRDM9: affinite mais ne donne pas d'info sur le taux de CO
% - Spo11: taux DSB mais lourd a mettre en oeuvre
% - DMC1 taux DSB+repair efficiency
% - tous en amont de la recombinaison, donc il est important de mettre au point une approche qui permette de voir les produits de la recombinaison et non pas les etapes intermediaires.
%


% But, thanks to the recent development of techniques allowing to precisely map recombination hotspots, — like ChIP-seq against DMC1 and RAD51 \citep{smagulova2011genomewide}, sequencing-based detection of DMC1-bound ssDNA \citep{khil2012sensitive, brick2012genetic} and sequencing-based detection of SPO11 oligos \citep{lange2016landscape}, — high-resolution genome-wide maps of recombination initiation hotspots have been obtained in several mouse strains and F1 hybrids \citep{smagulova2016evolutionary}.

% Recent techniques like ChIP-seq against DMC1 and RAD51 \citep{smagulova2011genomewide} and sequencing-based detection of DMC1-bound ssDNA \citep{khil2012sensitive, brick2012genetic} and of SPO11 oligos \citep{lange2016landscape} allow to precisely map recombination hotspots.
% Thanks to these, high-resolution genome-wide maps of recombination initiation hotspots have been obtained in several strains and F1 hybrid mice \citep{smagulova2016evolutionary}.
%
% But only a handful of mouse recombination hotspots have been directly characterised \citep{yauk2003highresolution,bois2007highly,baudat2007cis,ng2008quantitative,cole2010comprehensive,cole2014mouse}
% But the recombination activity can vary by orders of magnitude \citep{paigen2008recombinational}.
% This calls for understanding the parameters that modify this magnitude.

Here, to better understand the extent of the variation in recombinational activity and the factors governing it, we precisely characterised recombination in 1,018 hotspots of a mouse F1 hybrid descended from a cross between \textit{Mus musculus domesticus} (strain C57BL/6J, hereafter called B6) and \textit{Mus musculus castaneus} (strain CAST/EiJ, hereafter called CAST).
% we propose to account for this lack by precisely recombination activity in over 1,000 mouse hotspots.
In this chapter, I show how the set of recombination events detected with the approach developed in Chapter~\ref{ch:5-methodology} allowed us to describe some of the determinants of recombinational activity, better characterise recombination and infer its hidden parameters \textit{via} inferential approaches.
% describe what the determinants of recombinational activity are
% show that the set of recombination events that we found with the approach developed in Chapter~\ref{ch:5-methodology} is reliable to characterise recombination and infer its hidden parameters \textit{via} inferrential approaches.
%




% \protect\pagebreak
% \section{Confidence in the detected events}
% \section{A high-confidence set of recombination events}
\section{Determinants of recombinational activity}

% \subsection{Consistency with observations on one chromosome}
\subsection{A high-confidence set of recombination events}

\begin{figure}[b!]
    \centering
    \includegraphics[width = 1\textwidth]{figures/chap6/Correl-paigen-log-events-vs-RR-correction-Laurent-transformed_log_instead_log10_label.eps}
	\caption[Correlation between the expected recombination rate and the observed number of events on the 33 intervals analysed by \citet{paigen2008recombinational}]
	{\textbf{Correlation between the expected recombination rate and the observed number of events on the 33 intervals analysed by \citet{paigen2008recombinational}.}
		\par We compared the recombination rates of 33 intervals defined by \citet{paigen2008recombinational} to the total number of events we detected on these intervals, brought back to the length of each interval (see main text).
		The intervals selected were those which exclusively encompassed hotspots that were analysed in our study.
		The Pearson correlation between the two measures was extremely high both with raw ($R^2 = 0.974$; \textit{p}-val $< 2.2 \times 10^{-16}$) and log-transformed ($R^2 = 0.934$; \textit{p}-val $< 2.2 \times 10^{-16}$) measures.
    }
\label{fig:correlation-paigen}
\end{figure}


To determine whether the recombination rates we observed with our approach were quantitatively accurate, we aimed at comparing our results with those of more classical approaches.
We thus used data from \citet{paigen2008recombinational} who examined in detail the recombinational activity of chromosome 1 in a mouse exhibiting the same genetic background as ours (B6xCAST).
They particularly focused on the telomere-proximal 24.7 Mb, which was cut into 65 intervals ($median$ $length = 205$~kb) encompassing a total of 130 \textit{Prdm9} hotspots. 
Under the asumption that the recombination rate is null outside hotspots, the recombination rate they measured should equal:

\begin{equation}
	\label{eq:paigen}
	r_i = \frac{\sum\limits_{h=1}^{n_{h}^{i}} ( r_h \times L_h)}{L_i}
\end{equation}

where $r$ and $L$ respectively represent the recombination rate and the length of the region considered, the subscripts $i$ and $h$ respectively stand for the interval defined by \citet{paigen2008recombinational} and the 1-kb hotspots we defined, and $n_h^i$ corresponds to the number of hotspots in interval $i$.

Among the 65 intervals of their study, there were 33 for which all \textit{Prdm9} hotspots were included in our dataset.
We thus compared the CO rates that \citet{paigen2008recombinational} measured on these intervals to the total number of recombination events we observed in the 37 hotspots comprised in these intervals, brought back to the length of the interval as given in Equation~\ref{eq:paigen} (Figure~\ref{fig:correlation-paigen}).
We found that both measures correlated extremely well (Pearson correlation: $R^2 = 0.974$; \textit{p}-val $< 2.2 \times 10^{-16}$).
Therefore, the recombination events we detected are highly reliable since they concord exceptionally well with those identified by this independent study.

% We thus brought back the total number of events we detected to the length of the interval and found that it correlated extremely well with their data (Pearson correlation: $R^2 = 0.974$; \textit{p}-val $< 2.2 \times 10^{-16}$).

% TAILLE DES INTERVALLES (65 intervalles)
% cut -f10 $DATA/2_dBGC/6_ThirdSequencing/08_Compare_with_Paigen/01_Original_Paigen_Data/PositiveControls.txt|sed '1,1d' | sed 's/,/./'|Rscript -e 'summary(as.numeric(readLines("stdin")))'
%   Min. 1st Qu.  Median    Mean 3rd Qu.    Max.
%    1.8   185.9   205.7   251.8   372.4   877.2

% NOMBRE DE P9PEAKS
% cut -f9 $DATA/2_dBGC/6_ThirdSequencing/08_Compare_with_Paigen/01_Original_Paigen_Data/PositiveControls.txt|sed '1,1d'|paste -sd+|bc
% 130


% \subsection{PRDM9 and DMC1 signals predict hotspot intensity}%Correlation with PRDM9 hotspot intensity}
\subsection{Predictors of hotspot intensity}%Correlation with PRDM9 hotspot intensity}
% \subsection{PRDM9 binding and recombinational activity}%Correlation with PRDM9 hotspot intensity}
% \subsection{PRDM9 hotspot intensity: a good predictor of activity}%Correlation with PRDM9 hotspot intensity}



\begin{figure}[p]
    \centering
    \begin{subfigure}[b]{0.75\textwidth}
        % \subcaption{PRDM9 binding affinity as a predictor of hotspot intensity}
		% \includegraphics[width = 1\textwidth]{figures/chap6/Correlation_intensity_PRDM9-corrected_CO_x2.eps}
		\includegraphics[width = 1\textwidth]{figures/chap6/Correlation_intensity_PRDM9_with_lm_0.eps}
    \end{subfigure}

    \vspace{0.5cm}
	\begin{subfigure}[b]{0.75\textwidth}
        % \subcaption{DMC1 binding affinity as a predictor of hotspot intensity}
		% \includegraphics[width = 1\textwidth]{figures/chap6/Correlation_intensity_DMC1-corrected_CO_x2.eps}
		\includegraphics[width = 1\textwidth]{figures/chap6/Correlation_intensity_DMC1_with_lm_0.eps}
	\end{subfigure}

	\caption[Proportionality between the recombination rate and either PRDM9 or DMC1 binding intensity]
	{\textbf{Proportionality between the recombination rate and either PRDM9 (top) or DMC1 (bottom) binding intensity.}
		\par All 1,018 hotspots were divided into 10 classes of either increasing PRDM9 signal (top), i.e.\ the number of PRDM9 ChIP-seq tags on each PRDM9 ChIP-seq peak, brought back to the width of the peak (PRDM9 ChIP-seq data on B6xCAST hybrid mice from \citealp{baker2015prdm9}), or increasing DMC1 signal (bottom), i.e.\ the number of DMC1 ChIP-seq tags on each PRDM9 ChIP-seq peak (DMC1 ChIP-seq data on B6xCAST hybrid mice from \citealp{smagulova2016evolutionary}).
		The observed number of recombination events identified per sequenced Mb (left y-axis) was converted into a CO rate (right y-axis) as detailed in Subsection~\ref{chap6:extrapolation-CO-rate}.
		The points and error bars respectively represent the mean number of events (or CO rate) and the standard error on the mean for hotspots of each class.
		The linear regression model for PRDM9 (slope $=1047$; intercept $=0$; \textit{p}-val~$= 4.21 \times 10^{-8}$) and DMC1 (slope $=0.027$; intercept $=15$; \textit{p}-val~$= 3.8 \times 10^{-5}$) were drawn as dotted lines.
    }
\label{fig:correlation-PRDM9-standard-error}
\end{figure}

Next, since PRDM9 \citep{baker2015prdm9} and DMC1 \citep{smagulova2016evolutionary} ChIP-seq data are available for mice exhibiting the same genetic background (B6xCAST) as ours, we analysed the relationship between these two signals on the one hand, and the recombinational activity of the hotspots we selected on the other hand (Figure~\ref{fig:correlation-PRDM9-standard-error}).

Overall, we found that both PRDM9 and DMC1 binding affinity (both proxies of the propensity for a hotspot to form DSBs) are accurate predictors of recombinational activity.
Of note, the relationship with PRDM9 binding affinity is linear (regression: \textit{p}-val $= 4.2 \times 10^{-8}$) but that with DMC1 signal is not (Figure~\ref{fig:correlation-PRDM9-standard-error}.b.), possibly because DMC1 ChIP-seq reflects not only the DSB rate but also the efficiency of the repair of the DSB which varies among hotspots \citep{lange2016landscape,davies2016reengineering}.


\subsection{Lower recombination rate of asymmetric hotspots}
% \subsection{DMC1 hostpot intensity: a good predictor of activity}%Correlation with DMC1 hotspot intensity}
% $DATA/2_dBGC/6_ThirdSequencing/05_Analyses_of_Recombinants/03_General_Statistics/05_Distribution_of_recombinants/Intensity_and_counts_good_classes_and_DMC1.txt


In a B6xCAST hybrid, the PRDM9 target motif located on the homologous chromosome originating from the B6 parent (hereafter called B6 haplotype) may differ from that located on the homologue originating from the CAST parent (hereafter called CAST haplotype) because of the accumulation of mutations along the separate lineages (B6 or CAST) since their common ancestor \citep{davies2016reengineering,smagulova2016evolutionary}.
Consequently, in certain hotspots, PRDM9 may bind preferentially one of the two haplotypes while, in other hotspots, it may bind both haplotypes equally.
The former class of hotspots is referred to as ‘asymmetric’ and the latter as ‘symmetric’.

\citet{li2018highresolution} previously identified that such variations in hotspot asymmetry explain part of the variations in recombination rates for a given DMC1 signal.
To check whether this pattern was also observed in our dataset, we distinguished between symmetric and asymmetric hotspots based on the strand-specific PRDM9 ChIP-seq data from \citet{baker2015prdm9} (see Chapter~\ref{ch:7-quantification-BGC}).
We found that, for a given PRDM9 or DMC1 signal, the number of recombination events was greater for symmetric than for asymmetric hotspots (two to four times greater) (Figure~\ref{fig:asymmetry-and-PRDM9}).
% We also found a similar relationship with strand-specific DMC1 signal (Appendix~\ref{app:data-and-figs}).

As hypothesised by \citet{li2018highresolution}, this relationship can be explained if these asymmetric hotspots are repaired using the sister chromatid~instead~of~the~homologue:
since one haplotype is not bound by PRDM9 in such asymmetric hotspots, it is possible that the presence of PRDM9 on both homologues may play a role in homology search.
Yet, hotspot asymmetry does not account for all the variation (see the width of boxplots in Figure~\ref{fig:asymmetry-and-PRDM9}): 
instead, the sampling variance (i.e.\ the limited number of events per hotspot) and most likely a biological factor not yet identified may explain the residual variation.
% instead, the random sampling of hotspots, or another (unknown) source of variability may explain the residual variation.

% Yet, there remains a residual variation that cannot be explained by hotspot asymmetry, and may alternatively come from the random sampling of hotspots or from a large genuine variability between hotspots.
% De plus, on l'observe directement avec PRDM9 alros que eux DMC1, ce qui veut dire qu'il ne s'agit pas d'un effet uniquement du temps de reparation, mais bien un effet des hotspots asymetriques qui sont moins forts.
% Pourquoi seraient moins forts?



\begin{figure}[p]
	\centering
	\begin{subfigure}[b]{0.75\textwidth}
		\includegraphics[width = 1\textwidth]{figures/chap6/Recombination_rates_2_axes_sym_vs_Asym_PRDM9_with_legend_and_outliers-corrected_CO_x2.eps}
	\end{subfigure}

	\vspace{0.5cm}

	\centering
	\begin{subfigure}[b]{0.75\textwidth}
		\includegraphics[width = 1\textwidth]{figures/chap6/Recombination_rates_2_axes_sym_vs_Asym_DMC1_with_legend_and_outliers-corrected_CO_x2.eps}
	\end{subfigure}    

	\caption[Asymmetric hotspots display lower recombinational activity than expected by their PRDM9 or DMC1 binding affinity]
	{\textbf{Asymmetric hotspots display lower recombinational activity than expected by their PRDM9 (top) or DMC1 (bottom) binding affinity.}
		\par All 1,018 hotspots were divided into 10 classes of increasing PRDM9 signal (top), i.e.\ the number of PRDM9 ChIP-seq tags brought back to the width of the peak (data from \citealp{baker2015prdm9}), or increasing DMC1 signal (bottom) (data from \citealp{smagulova2016evolutionary}).
		The observed number of recombination events identified per sequenced Mb (left y-axis) was converted into a CO rate (right y-axis) as detailed in Subsection~\ref{chap6:extrapolation-CO-rate}.
		Symmetric hotspots (green, $N = 650$) were distinguished from asymmetric hotspots (orange, $N=236$) as detailed in Chapter~\ref{ch:7-quantification-BGC}.
		The linear regression model with the PRDM9 signal for symmetric (slope~$= 18$; intercept~$=0$; \textit{p}-val~$< 2.2 \times 10^{-16}$) and asymmetric (slope~$= 7.9$; intercept~$=0$; \textit{p}-val~$= 9.21 \times 10^{-7}$) hotspots and with the DMC1 signal for symmetric (slope~$= 1.8$; intercept~$=0$; \textit{p}-val~$< 2.2 \times 10^{-16}$) and asymmetric (slope~$= 8.6$; intercept~$=0$; \textit{p}-val~$< 2.2 \times 10^{-16}$) hotspots are drawn as dotted lines.
    }
\label{fig:asymmetry-and-PRDM9}
\end{figure}









\section{Observable recombination parameters}
\subsection{Definition of observable conversion tracts}

\begin{figure}[b!]
	\centering
	\includegraphics[width = 1\textwidth]{figures/inkscape/Terminology-sauvetage-inkscape.eps}
	\caption[Terminology used to characterise recombination events]
	{\textbf{Terminology used to characterise recombination events.}
		\par The read coverage of the PRDM9 ChIP-seq peak (data from \citealp{baker2015prdm9}) is drawn in the top panel (‘PRDM9 ChIP-seq’).
		\textit{B6} (red) and \textit{CAST} (yellow) reference alleles are reported in the middle panel (‘Markers’).
		Examples of sequenced fragments that are or are not recombination events are drawn in the bottom panel (‘Fragments’).
		See main text for the description of each annotation.
	}
\label{fig:terminology}
\end{figure}


To characterise the recombination events we observed, we needed to localise the position of their conversion tracts (CTs).
Though, the latter are not directly observable in the data, but they can be \textit{inferred} from ‘haplotype switches’, i.e.\ changes of haplotype along a DNA fragment.
To avoid any confusion between the real CT and the inferred CT, we decided to denote the latter CT\textsuperscript{$\star$}.

To infer the position of CTs\textsuperscript{$\star$}, we defined ‘switch intervals’ (black segments in Figure~\ref{fig:terminology}) as sequence segments delineated by two consecutive markers with distinct genotypes (\textit{B6}-\textit{CAST} or \textit{CAST}-\textit{B6}).
% , one of which being a \textit{B6} variant and the other a \textit{CAST} variant.
We also defined the ‘switch point’ as the midpoint of the switch interval.\\

We could distinguish three types of recombination events in our data: 
11,665 events including strictly one switch point, which we called ‘single-switch recombination events’ (‘Rec-1S’);
5,932 events including strictly two switch points, which we called ‘double-switch recombination events’ (‘Rec-2S’);
and 1,224 events including more than two switch points, which we called ‘multiple-switch recombination events’ (‘Rec-MS’).

Rec-2S events most probably correspond to NCO events. 
Rec-1S events may correspond to either CO events or to fragments that partially overlap a NCO\@.
Rec-MS correspond to complex events (COs or NCOs) and represent only a small fraction (6.5\%) of recombinant fragments.\\

For Rec-2S events, we simply defined the CT\textsuperscript{$\star$} as the region between its two switch points.

For Rec-1S events, only one edge of the CT can be detected (the one corresponding to the switch point). 
In principle, it is necessary to compare the four products of meiosis to be able to detect the extent of CO CTs.
However, previous studies have shown that, in the vast majority of cases, CTs overlap the DSB site \citep{cole2014mouse}.
Hence, the segment between the switch point and the DSB is, in most cases, included in the CT\@.
Thus, for Rec-1S events, we defined the CT\textsuperscript{$\star$} as the region between the switch point and PRDM9 ChIP-seq peak, which colocates precisely with DSB sites \citep{lange2016landscape}.
It should be noted that for Rec-1S events, the CT\textsuperscript{$\star$} corresponds to only one end of the CT (the edge located on the other side of the DSB cannot be detected).
Furthermore, when the DSB site is located within the switch interval, the CT\textsuperscript{$\star$} cannot be inferred.

Finally, for Rec-MS fragments, the CT\textsuperscript{$\star$} cannot be inferred either.




% The terms used to describe recombination events are illustrated in Figure~\ref{fig:terminology}.
% An ‘interval’ (i.e.\ a sequence delineated by two consecutive markers) is either delimited by two \textit{B6} variants, two \textit{CAST} variants or one \textit{B6} and one \textit{CAST} variant.
% On the one hand, intervals positioned in between two \textit{B6} (resp.\ \textit{CAST}) markers were considered to originate from the B6 (resp.\ CAST) haplotype.
% On the other hand, intervals positioned in between one \textit{B6} variant and one \textit{CAST} variant were not assignable to either haplotype but necessarily comprised one end of the conversion tract (CT).
% The latter types of intervals were called ‘switch intervals’ (grey segments in Figure~\ref{fig:terminology}).
%
% We used a proxy for the genuine position of the CT edge: the ‘switch point’, defined as the midpoint of the switch interval.
% Recombination events including strictly one switch point were called ‘single-switch recombination events’ (‘Rec-1S’); those including strictly two switch points were called ‘double-switch recombination events’ (‘Rec-2S’) and those including more than two switch points were called ‘multiple-switch recombination events’ (‘Rec-MS’).
%
% For Rec-1S events, one edge of the CT was known: the position of the only switch point.
% We approximated that its second edge corresponded to the inferred position of the DSB site, i.e.\ the summit of the PRDM9 ChIP-seq peak (data from \citealp{baker2015prdm9}).
% Thus, the inferred (or observable) CT (noted CT\textsuperscript{$\star$}) for Rec-1S events corresponded to the region between the switch point and the inferred DSB position.
% As for Rec-2S events, the CT\textsuperscript{$\star$} corresponded to the region between the two switch points and, for Rec-MS events, the position of the CT\textsuperscript{$\star$} could not be inferred.
%


% Last, we infer that the haplotype from which the variants included in the CT\textsuperscript{$\star$} originate correspond to the donor in the gene conversion event. When the CT\textsuperscript{$\star$} includes no variant (which is the case for a portion of the Rec-1S) and when the CT\textsuperscript{$\star$} is not defined (case for Rec-MS), we cannot infer the donor.



\subsection{Identification of the gene-conversion donor}


\begin{figure}[b!]
    \centering
    \includegraphics[width = 1\textwidth]{figures/chap6/CorrelationDMC1_FINAL_BIS_on_lab_computer.eps}
	\caption[Correlation between the expected and the observed proportions of CAST-donor fragments across hotspots displaying at least 5 events]
	{\textbf{Correlation between the expected and the observed proportions of CAST-donor fragments across hotspots displaying at least 5 events.}
		\par The expected proportion of CAST-donor fragments (x-axis) was based on the probability that the DSB initiates on the B6 haplotype from DMC1 ssDNA-sequencing (SSDS) data by \citet{smagulova2016evolutionary} (see main text).
		Only the 582 hotspots displaying a minimum of 5 recombination events were reported in this figure.
		The Pearson correlation between the two measures gave: $R^2 = 0.66$; {\textit{p}-val $< 2.2 \times 10^{-16}$}.
    % \par The expected proportion of CAST-donor fragments (x-axis) is computed based on the probability of the DSB to initiate on the B6 haplotype (see \ref{par:MM-DMC1-analysis} for details on the analysis of DMC1 ssDNA-sequencing (SSDS) data by \citep{smagulova2016evolutionary}). The observed proportion of CAST-donor fragments (y-axis) is calculated as described in the main text (see above). Solely the 582 hotspots displaying a minimum of 5 recombination events are reported in this graph. The coefficient of correlation between the expected and observed proportions of CAST-donor fragments is $r = 0.81$ ($p-value < 2.2\e{-16}$).}
}
\label{fig:correl-donor-DMC1}
\end{figure}


The conversion tracts that we inferred (CT\textsuperscript{$\star$}) directly determine which haplotype (B6 or CAST) is the donor in the gene conversion event (Figure~\ref{fig:terminology}).

\citet{smagulova2016evolutionary} measured the relative proportion of DSB initiation on each haplotype of thousands of hotspots in a mouse exhibiting the same genetic background as ours (B6xCAST).
Thus, to assess the accuracy of our inference, we wanted to compare the proportion of B6- and CAST-donor fragments we inferred in each of the hotspot we studied to what would be expected based on their DMC1 ssDNA-sequencing (SSDS) data (Figure~\ref{fig:correl-donor-DMC1}).

Overall, we found a strong positive correlation between the expected and the observed proportions of CAST-donor fragments (Pearson correlation: $R^2 = 0.66$; {\textit{p}-val $< 2.2 \times 10^{-16}$}).
Our simple and intuitive way of assigning its donor to each fragment was thus sufficient to explain, at the hotspot-scale, most of the variance due to DSB initiation bias.



% % - pour verifier si inf correctes, on compare l'inf a des donnees experimentales
%
% After a gene conversion event, the receiver (cold) haplotype is converted into the donor (hot) haplotype all along the CT\@.
% Thus, when all markers located within the CT\textsuperscript{$\star$} were typed \textit{B6} (resp.\ \textit{CAST}), we infered that the donor was the B6 (resp.\ CAST) haplotype.
% If, however, the CT\textsuperscript{$\star$} included no variant (which was the case for a portion of Rec-1S events) or if it was not defined (which was the case for Rec-MS events), we could not infer which haplotype was the donor.\\
%
% For a given hotspot, the fraction of CAST-donor (resp.\ B6-donor) recombination events should be proportional to the probability that the DSB initiates on the B6 (resp.\ CAST) haplotype.
% Thus, to assess the accuracy of our inference, we used DMC1 ssDNA-sequencing (SSDS) data from a B6\textsubscript{f}xCAST\textsubscript{m} hybrid \citep{smagulova2016evolutionary} to estimate the relative proportion of DSB initiation on each haplotype:
% we remapped the raw data (SRR2961459, SRR2961460) on both the B6 and the CAST reference genomes and we genotyped all the variants of each fragment.
% % When the totality of its variants were typed \textit{B6}, we assigned the B6 haplotype to the fragment and, respectively, we assigned the CAST haplotype to it when the totality of its variants were typed \textit{CAST}.
% When the totality of its variants were typed \textit{B6}, we assigned the B6 haplotype to the fragment and, respectively, the CAST haplotype when they were typed \textit{CAST}.
% We then compared this probability for each hotspot to initiate a DSB on the B6 haplotype (i.e.\ the expected proportion of CAST-donor events) to the actual fraction of CAST-donor events inferred with the approach described above (Figure~\ref{fig:correl-donor-DMC1}).
%
%
%
% Overall, we found a strong positive correlation between the expected and the observed proportions of CAST-donor fragments (Pearson correlation: $R^2 = 0.66$; {\textit{p}-val $< 2.2 \times 10^{-16}$}).
% % Therefore, our donor-inference method is sufficient to explain, at the hotspot scale, most of the variance due to DSB initiation bias. This is a strong proof that our very simple and intuitive way of assigning its donor to each fragment is trustworthy.
% % Therefore, our donor-inference method is sufficient to explain, at the hotspot scale, most of the variance due to DSB initiation bias and our very simple and intuitive way of assigning its donor to each fragment is trustworthy.
% Our simple and intuitive way of assigning its donor to each fragment was thus sufficient to explain, at the hotspot-scale, most of the variance due to DSB initiation bias and should also be trustworthy at the scale of single fragments.
%

\subsection{Description of the recombination events} 

Among the 18,821 recombination events detected across 898 hotspots (median~$= 10$; max $= 327$ events per hotspot), 11,665 corresponded to Rec-1S events, 5,932 to Rec-2S events and 1,224 to Rec-MS events.
The CTs\textsuperscript{$\star$} of Rec-1S events (median~$=~97$~bp; mean~$= 142$~bp) were longer — and, consequently, somewhat more spread (Figure~\ref{fig:example-recombinants} and Appendix~\ref{app:data-and-figs}) — than the CTs\textsuperscript{$\star$} of Rec-2S events (median~$= 78$ bp; mean~$=~90$~bp).
% awk '$26=="NCO" {print}' $DATA/2_dBGC/6_ThirdSequencing/05_Analyses_of_Recombinants/02_Recombinants_and_False_Positives_Dataset/Recombinants_dataset.txt|cut -f24|Rscript -e 'summary(as.numeric(readLines("stdin")))'
%    Min. 1st Qu.  Median    Mean 3rd Qu.    Max.
%    9.00   54.00   78.00   90.36  113.00  475.00
% awk '$26=="CO" {print}' $DATA/2_dBGC/6_ThirdSequencing/05_Analyses_of_Recombinants/02_Recombinants_and_False_Positives_Dataset/Recombinants_dataset.txt|cut -f24|Rscript -e 'summary(as.numeric(readLines("stdin")))'
%    Min. 1st Qu.  Median    Mean 3rd Qu.    Max.
%     1.0    36.0    97.0   142.7   216.0  1142.0
These features of Rec-1S and Rec-2S are reminiscent of those of COs and NCOs, respectively.

% However, the ratio of Rec-1S:Rec-2S (1.97:1) was much different from that expected for CO:NCO (1:9).
% This arises from two limitations of the direct observations.
% First, Rec-1S events may not comprise exclusively COs, but also NCOs.
% Indeed, everytime one edge of a sequenced fragment falls into the middle of a NCO CT, this event will necessarily be detected as a Rec-1S.
% Second, the detection of events is limited by the density in polymorphic sites and this effect is much greater for NCOs since their CTs are generally short.
% some NCOs are intrinsically undetectable due to
%

However, our data revealed about twice as many Rec-1S as Rec-2S events — an observation much different from the expected CO:NCO ratio.
Indeed, in mice, among the 200--300 DSBs formed per meiosis, 20 are expected to be repaired as COs and the remaining 180--280 as NCOs \citep{baudat2007regulating, martinez-perez2009distribution}. 
Since NCOs affect only one of four chromatids (while COs affect two), one would \textit{a priori} expect to identify only one quarter of NCOs (i.e.\ 45--70) and half of COs ($\approx$10), hence a CO:NCO ratio ranging between 1:4.5 and 1:7. 

Two non-mutually exclusive reasons justify the gap between the observed and the expected ratios.
%Rec-1S:Rec-2S and the expected CO:NCO ratio. 
On the one hand, the Rec-1S:Rec-2S ratio does not directly reflect the CO:NCO ratio.
Indeed, everytime one edge of a sequenced fragment falls into the middle of a NCO CT, this event is necessarily detected as a Rec-1S. 
Thus, Rec-1S events do not exclusively comprise COs: a portion of them correspond to NCOs.
On the other hand, for NCOs to be detected with our approach, their CTs must be long enough to overlap at least two markers (see Chapter~\ref{ch:5-methodology}).
Though, since NCO CTs are only a few base pairs to a few tens of base pairs long \citep{cole2014mouse}, one would \textit{a priori} expect a non-negligible (unknown at this stage, but see Subsection~\ref{chap6:extrapolation-CO-rate}) proportion of them to be intrinsically undetectable, especially in regions with low SNP density.\\

Therefore, to characterise recombination regardless of these two limitations of direct observations, it appeared necessary to use inferential approaches to uncover the \mccorrect{true\footnote{I use ‘true’ as opposed to ‘observed’, but the parameters that are inferred correspond to the most likely ones and not necessarily to the exact real ones.}} recombination parameters. This is developed in the following section.



% RIGHT PAGE
\begin{sidewaysfigure}[p]
    \centering
    \leftskip-2.4cm
    \rightskip-2.4cm
    \rotfloatpagestyle{empty}
    \includegraphics[width = 1.25\textwidth, trim = 0cm 0cm 0cm 3.94cm, clip]{figures/chap6/P9peak_chr11_10175985_10177504_10177504.eps}
    \captionsetup{width=1.25\textwidth, margin={-2.2cm, -3.3cm}}
	\abovecaptionskip-0.1cm
    \caption[Recombination events in a PRDM9\textsuperscript{CAST}-targeted hotspot located on chromosome 11 (chr11:10175985--10177504)]
	{\textbf{Recombination events in a PRDM9\textsuperscript{CAST}-targeted hotspot located on chromosome 11 (chr11:10175985--10177504).}
		\par The figure is centred on the PRDM9 ChIP-seq peak summit.
		The top panel depicts the maximum number of detectable Rec-1S (light grey) and Rec-2S (dark grey) switch intervals.
		%, which partially accounts for variation in recombination rate along and across hotspots.
		The two middle panels indicate variation in read coverage and the positions of markers (filled circles).
		The bottom panel pictures the detected Rec-1S (upper board) and Rec-2S (bottom board) events.
		\textit{B6}-typed markers and intervals are coloured in red, \textit{CAST}-typed markers and intervals in yellow, and switch intervals in grey.
	}
\label{fig:example-recombinants}
\end{sidewaysfigure}




% \subsection{CO:NCO ratios and polymorphism}
% Mettre une figure en lien, facon de montrer que juste lie a la detectabilite (ou si lien existe, pas detectable).
% Et donc la detectabilite a un role majeur.





\section{Inferred recombination parameters}
\subsection{Approximate bayesian computation (ABC)}

In order to discover which range of values of the biological parameters were compatible with our observations, we implemented an approximate bayesian computation (ABC) approach \citep{csillery2010approximate,sunnaker2013approximate}.
In short, an ABC consists in creating a simulator that reproduces at best the biological experiment, performing a large number of simulations with variable input parameters and assessing which range of values are biologically relevant by confronting the summary statistics representative of the output of the simulations to the biological observations.


% (here: the recombination parameters given in Table~\ref{tab:ABC-results}).
% Summary statistics, representative of the output, are calculated to be confronted with the observations (here: the distributions of Rec-1S and Rec-2S observed CT (CT\textsuperscript{$\star$}) lengths and the observed Rec-1S:Rec-2S ratio).
% In the end, the recombination parameters are estimated by selecting the simulated sets of parameters that end in the identification of recombination events displaying characteristics close to the ones obtained with the experimental dataset of recombination events.



\subsubsection{Implementation of the simulator}
We built a simulator that mimicked the formation of recombination events, their sequencing and their genotyping.
Briefly, all simulated recombination events were distributed across the 1,018 hotspots proportionately to their predicted propensity to form DSBs, which we approximated by their PRDM9 signal intensity (i.e.\ the number of tags per kb on each PRDM9 ChIP-seq peak from \citealp{baker2015prdm9}).
For each simulated hotspot, the ratio of CO over NCO recombination events was $r_{CO:NCO}$.

SNPs, insertions and deletions were positioned along each hotspot at the exact locations were they were found by variant-calling on our sequencing data (see Chapter~\ref{ch:5-methodology}). 
CO CT lengths were drawn from a normal distribution of mean $m_{CO}$ and standard deviation $sd_{CO}$, and NCO CT lengths were drawn from a gamma distribution of alpha $\frac{m_{NCO}^{2}}{sd_{NCO}^{2}}$ and beta $\frac{sd_{NCO}}{m_{NCO}}$ (i.e.\ a distribution with mean $m_{NCO}$ and standard deviation $sd_{NCO}$). 
The middle point of the CT for both COs and NCOs was positioned at the inferred DSB site (i.e.\ the summit of the PRDM9 ChIP-seq peak) and each recombination event was assigned a donor (either B6 or CAST) under a binomial distribution with probability 0.5.

For each simulated recombination event, we randomly selected one of the two gametes involved in the recombination event and simulated $n_{fragments}$ sequenced fragment, whose start and end positions were drawn from the real positions of the fragments in our experimental dataset. 
We ran our unique-molecule genotyping pipeline (see Chapter~\ref{ch:5-methodology}) on all the simulated fragments to identify those that would be detected as recombination events.



\subsubsection{Selection of the simulations compatible with the experimental data}
In total, we simulated 100,000 datasets $\mathcal{D^{*}}$ by assigning a value taken from the following prior distributions to each of the input parameters: 
\begin{itemize}
    \item $m_{CO} \hookrightarrow \mathcal{U}([100, 1000])$ bp, 
    \item $sd_{CO} \hookrightarrow \mathcal{U}([50, 300])$ bp,
    \item $m_{NCO} \hookrightarrow \mathcal{U}([1, 300])$ bp, 
    \item $sd_{NCO} \hookrightarrow \mathcal{U}([1, 100])$ bp,
    \item and $r_{CO:NCO} = 10^{r}$ with $r \hookrightarrow \mathcal{U}([-2, 1])$,
\end{itemize}
where $\mathcal{U}$ represents the uniform distribution.\\

For each simulated dataset as well as for the experimental dataset, we summarised the results of the recombination events found with the following summary statistics: the observed Rec-1S:Rec-2S ratio $r_{Rec-1S:Rec-2S}^{obs}$, the observed mean and quartiles of Rec-1S CT\textsuperscript{$\star$} lengths ($l_{Rec-1S}^{mean}$, $l_{Rec-1S}^{0.25}$, $l_{Rec-1S}^{0.5}$, $l_{Rec-1S}^{0.75}$) and the observed mean and quartiles of Rec-2S CT\textsuperscript{$\star$} lengths ($l_{Rec-2S}^{mean}$, $l_{Rec-2S}^{0.25}$, $l_{Rec-2S}^{0.5}$, $l_{Rec-2S}^{0.75}$). 

We then used the R package ‘abc’ \citep{csillery2012abc} to select the simulated datasets $\mathcal{D^{*}}$ that ended in summary statistics $\mathcal{S^{*}}$ close to the summary statistics $\mathcal{S}$ of the experimental dataset $\mathcal{D}$, with a tolerance threshold ($\epsilon$) of 5\% (i.e.\ $\mathcal{D^{*}}$ was retained if $d(\mathcal{S^{*}}, \mathcal{S}) \leq \epsilon$). 





\subsection{Comparison with direct observations}


\begin{table}[b!]
	\centering
	\begin{adjustbox}{width = 1\textwidth}
		\begin{tabular}{rrr}

			\toprule
			\textbf{Parameter} & \textbf{Literature} & \textbf{ABC approach} \\

			\midrule
			\textbf{\textit{CO:NCO ratio}}      & 0.1\textsuperscript{[1]}  & 0.119 [0.014--0.20] \\
			\textbf{\textit{CO CT length (bp)}}\\
			$Mean$                              & 566\textsuperscript{[2]}  & 447 [245--874] \\
			$Sd$                                & 277\textsuperscript{[2]}  & 363 [92--471] \\
			\textbf{\textit{Detectable NCO markers CT\textsuperscript{$\star$} length (bp)}}\\
			$Mean$                              &  94\textsuperscript{[2]}  & 95 [74--110] \\
			$Sd$                                &  62\textsuperscript{[2]}  & 49 [30--60] \\

			\textbf{\textit{Real NCO CT length (bp)}}\\
			$Mean$                  &  \textit{-}  & 36 [4--54] \\
			$Sd$                    &  \textit{-}  & 45 [3--86] \\

			\bottomrule

		\end{tabular}
	\end{adjustbox}
	\caption[Consistency between the recombination parameters inferred \textit{via} our ABC approach and those directly measured by independent studies]
	{\textbf{Consistency between the recombination parameters inferred \textit{via} our ABC approach and those directly measured by independent studies.}
		\par References from which the values were extracted are given inside superscript brackets: [1]~corresponds to \citet{cole2010comprehensive} and [2] corresponds to \citet{cole2014mouse}. 
		CT stands for ‘conversion tract’ and CT\textsuperscript{$\star$} for ‘inferred (or observable) conversion tract’. 
		Only a portion of NCOs are detectable by tetrad analyses (those whose CT\textsuperscript{$\star$} overlaps at least 1 marker). 
Thus, we report the CT\textsuperscript{$\star$} length of both this subset of detectable NCOs that have been analysed by tetrad analyses, but also report the mean CT length of all NCOs (both detectable and undetectable). 
		For the ABC, the 95\% confidence intervals are reported between brackets.
	}
\label{tab:ABC-results}
\end{table}




CO and NCO CT lengths had previously been measured in two hotspots \textit{via} the analysis of mouse tetrads \citep{cole2014mouse} and the mouse CO:NCO ratio had been determined \textit{via} cytological estimates of DSB numbers: among about 250 DSBs arising in each meiosis, around 23 are repaired as COs \citep{baudat2007regulating, martinez-perez2009distribution}, which, with the assumption that the remaining ones are repaired as NCOs, leads to a rough estimate for the CO:NCO ratio of 0.1.\\

To assess the correctness of the parameter ranges identified by the ABC, we compared them to the aforementioned estimates (Table~\ref{tab:ABC-results}).
Altogether, we found that these two sets of parameter ranges were strikingly close.
This adequacy was particularly impressive regarding the CO:NCO ratio, considering the fact that we did not set any prior constraint on any of the simulated parameters.

Similarly, the length of the CTs\textsuperscript{$\star$} of detectable NCOs (i.e.\ those with a CT overlapping at least one marker) estimated by the ABC was almost identical to that directly observed by tetrad analyses.
However, this reported CT\textsuperscript{$\star$} length did not take into account that of undetectable NCO CTs (i.e.\ those too short to overlap any marker). As such, the actual mean CT length for all NCO events is necessarily shorter than that reported by direct observations and can only be provided by the ABC\@: we estimated it to be around 36 bp (Table~\ref{tab:ABC-results}).

As for COs, even if the 95\% confidence interval from the ABC included it, the value reported in the literature was slightly higher than the punctual estimate from the ABC\@.
This was likely due to the fact that the summary statistics were compared to the observations of CTs spreading onto a maximum of 500 bp (as the 1-kb hotspots were centred on the DSB site). 
If, instead, observations had been extended to larger regions, the estimated CTs would surely have been longer (as in Chapter~\ref{ch:8-HFM1}).\\

All in all thus, the ABC allowed to estimate the mean recombination parameters for the 1,018 hotspots we had selected and thus provided a broad insight of recombination patterns in mice.





\subsection{Extrapolation of recombination parameters}
\label{chap6:extrapolation-CO-rate}

Next, we used the results of the ABC to extrapolate other pieces of information on recombination: the CO rate and the composition in COs and NCOs of the observed Rec-1S events. 

\subsubsection{Estimation of the average CO rate}
Applying our unique-molecule genotyping pipeline on simulated recombination events (as was done with the ABC) allowed us to estimate the proportion of events that are detectable.

We defined the detectability ($d$) as the ratio of detected recombination events ($n$) over the total number of recombining gametes that were simulated ($N_r$):

\begin{equation} \label{eq:detectability}
    d = \frac{n}{N_r}
\end{equation}

As for the recombination rate ($R$), it corresponded to the proportion of recombining gametes ($N_r$) among all the gametes analysed ($N_g$):

\begin{equation} \label{eq:recombination-rate}
    R = \frac{N_r}{N_g}
\end{equation}

% Only a part ($n$) of these recombining gametes are detected as recombination events. We will refer to the proportion that we discovered as the detectability ($d$):
%
% \begin{equation} \label{eq:detectability}
%     d = \frac{n}{N_r}
% \end{equation}

Combining equations~\ref{eq:detectability} and~\ref{eq:recombination-rate}, we get:

\begin{equation*}
    R = \frac{n}{d \times N_g}
\end{equation*}

In the 4,997 simulations selected by the ABC, 6.7\% of simulated recombination events were discovered, which gave us a direct estimate for $d$. 
As for $n$ and $N_g$, we observed 18,821 recombination events out of 228,984,512 fragments analysed. 
Using these values, we found that the recombination rate in 1-kb long hotspots was $1.23 \times 10^{-3}$.
Since 0.119 of all recombination events corresponded to COs (see Table~\ref{tab:ABC-results}), the CO rate in 1-kb long hotspots was $1.46 \times 10^{-4}$ per gamete or also $2.92 \times 10^{-4}$ per bivalent, i.e.\ an average recombination rate of 29.2 cM/Mb across all analysed hotspots.

\begin{figure}[p]
	\centering
	\includegraphics[width = 1\textwidth]{figures/chap6/Correl_CO_rate_extrapolation-corrected_CO_x2.eps}
	\caption[Adequacy between two independent manners of extrapolating the CO rate]
	{\textbf{Adequacy between two independent manners of extrapolating the CO rate.}
		\par For each of the 1,018 hotspots, the CO rate was extrapolated based on the detectability inferred by the ABC (x-axis) and compared to the CO rate extrapolated from the number of informative fragments (y-axis).
		The calculus for the extrapolation of the CO rate \textit{via} the ABC was the following: $R = \frac{n}{d \times N_g} \times f_{CO}$, where $n$ is the number of recombination events detected, $d$ the detectability (inferred by the ABC), $N_g$ the total number of fragments analysed and $f_{CO}$ the proportion of COs in all the recombination events.
		This extrapolated CO rate was converted into cM/Mb by multiplying it by $10^2 \times \frac{10^{-3}}{10^{-6}}$ (the $10^{-3}$ multiplying factor came from the fact that the CO rate was measured on 1-kb long hotspots).
		As for the extrapolation from the number of informative fragments, our calculus was the following: $R = \frac{n_{Rec-1S}}{L^{i}_{seq} \times N^{i}_{f}}$, where $n_{Rec-1S}$ represents the number of Rec-1S events, $L^{i}_{seq}$ the length sequenced on each fragment and $N^{i}_{f}$ the number of informative fragments (i.e.\ those overlapping a minimum of 4 markers).
		This rate was then converted into cM/Mb by multiplying it by $10^2 \times 10^6$.
		The two measures correlated extremely well (Pearson correlation: $R^2 = 0.84$; \textit{p}-val $< 2.2 \times 10^{-16}$) and the slope of the linear regression equalled 0.9342 (\textit{p}-val $< 2 \times 10^{-16}$).
	}
\label{fig:extrapolation-CO-rate}
\end{figure}



Using the simplifying assumption that the CO:NCO ratio is similar in all hotspots, we used the process just described to transform the number of events in a given hotspot to its recombination rate: this is how the right y-axis (CO rate in cM/Mb) was calculated for Figures~\ref{fig:correlation-PRDM9-standard-error} and~\ref{fig:asymmetry-and-PRDM9}.\\
 % extrapolation of CO rates based on the total number of events directly detected and the detectability inferred \textit{via} the ABC

Alternatively, the CO rate could be extrapolated independently of the results of the ABC\@: assuming that Rec-1S events mainly correspond to COs, the CO rate would equal the fraction of Rec-1S events per sequenced base pair where events are detectable.
Indeed, recombination events can only be detected in ‘informative’ fragments, i.e.\ fragments overlapping at least 4 markers, and this irregularity should not be counted into the CO rate.
Thus, when based on the number of informative fragments ($N^{i}_{f}$), their sequenced length ($L^{i}_{seq}$) and the number of Rec-1S detected ($n_{Rec-1S}$), the recombination rate $R$ equals:

\begin{equation*}
    R = \frac{n_{Rec-1S}}{L^{i}_{seq} \times N^{i}_{f}}
\end{equation*}\\



We found a remarkable adequacy between those two independent ways of extrapolating the CO rate (Figure~\ref{fig:extrapolation-CO-rate}): the slope of the linear regression between the two was extremely close to 1 (slope $ =0.9342$; \textit{p}-val $< 2 \times 10^{-16}$).


% However, when comparing either of these extrapolated CO rates to the CO rate measured by \citet{paigen2008recombinational} on the 33 intervals of chromosome 1,
However, even if these two independent extrapolated estimates of CO rates concorded well, they were 10 times lower than those measured by \citet{paigen2008recombinational} on chromosome 1 (data not shown). 
As such, our extrapolation may underestimate the actual CO rate by a factor 10. Though, if it is indeed the case, we do not know where the gap comes from.






\subsubsection{CO:NCO composition of Rec-1S events}

As estimated by the ABC (Table~\ref{tab:ABC-results}), for every 1,000 recombination events repaired as NCOs, 119 are repaired as COs and, since NCOs affect only one chromatid when COs affect two, only 500 NCOs are expected to be seen. 
In the simulations selected by the ABC, the detectability for COs equalled 0.105 while that for NCOs equalled 0.0548. 
Based on these estimates, one would expect to detect 27.4 NCOs (out of the 500 chromatids affected) and 12.5 COs (out of the 119 chromatids affected).

Because they encompass only one switch point, all COs should be detected as Rec-1S events.
NCOs, however, could be detected as either Rec-1S or Rec-2S events. 
In the simulations selected by the ABC, 49.9\% of all the NCOs detected were detected as Rec-1S. 
Thus, among the 27.4 NCOs expected, 13.7 should be detected as Rec-1S events and 13.7 as Rec-2S events. 

All in all thus, we would expect to detect 26.2 Rec-1S events (13.7 (52.3\%) NCOs + 12.5 (47.7\%) COs) and 13.7 Rec-2S (all NCOs), i.e.\ a Rec-1S:Rec-2S ratio of 1.91, thus very close the ratio found experimentally (1.96).




% As a matter of fact, the results of the ABC evidence that Rec-2S all correspond to NCOs while Rec-1S represent a mixture of 52.3\% NCOs and 47.7\% COs (see \ref{par:MM-CO-NCO-estimates-in_Recs}).
% NCOs are less detectable than COs (3 times less according to the results of the ABC), since, to be detected, NCO CTs (which are only a few tens of base pairs long) must overlap at least two markers.
% Donc si on fait le total des deux, on revient bien dans un ratio de 1:6 comme attendu (1:4.5 a 1:7).







