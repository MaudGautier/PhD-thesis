% First parameter can be changed eg to "Glossary" or something. / List of Abbreviations
% Second parameter is the max length of bold terms.

% \begin{savequote}[8cm]
	% ‘“What's the use of their having names,” the Gnat said, “if they won't answer to them?”
	% “No use to them,” said Alice; “but it's useful to the people who name them, I suppose. If not, why do things have names at all?”’
    %
    % \qauthor{--- Stephen Jay Gould, \textit{\usebibentry{gould1993eight}{title}} \citeyearpar{gould1993eight} }
% \end{savequote}


	% \item[C57BL/6J] Mouse strain C57 black 6.
	% \item[CAST/EiJ]
\begin{mclistof}{Abbreviations}{3.2cm}

\item[AA] Amino acid.
\item[ABC] Approximate bayesian computation.
\item[ALT] Alternate (\textit{versus} REF\@: reference).
\item[ATM (kinase)] Ataxia telangiectasia mutated (kinase).
\item[A] Adenine.
\item[$B$] Population-scaled gBGC coefficient.
\item[$b$] gBGC coefficient.
\item[$b_0$] Transmission bias.
\item[$b_{dBGC}$] dBGC coefficient.
\item[B6] Mouse strain C57BL/6J\@.
\item[BD] B6 or DBA2.
\item[BER] Base excision repair.
\item[BGC] Biased gene conversion.
\item[BLM] Bloom syndrome RecQ helicase-like.
\item[BQSR] Base quality score recalibration.
\item[BWA] Burrows-Wheeler aligner (bioinformatic tool).
\item[C2H2] Cys\textsubscript{2}-His\textsubscript{2}.
\item[CAST] Mouse strain CAST/EiJ\@.
\item[CCO] Complex CO\@.
\item[CE] Central element.
\item[CI] Confidence interval.
\item[cM] Centimorgan.
\item[chB] Cold haplotype: B6.
\item[chC] Cold haplotype: CAST\@.
\item[COA] Crossing-over assurance.
\item[COH] Crossing-over homeostasis.
\item[COI] Crossing-over interference.
\item[CO] Crossing-over (or crossover).
\item[CT\textsuperscript{$\star$}] Observed (or inferred) conversion tract.
\item[CT] Conversion tract.
\item[CUB] Codon usage bias.
\item[C] Cytosine.
\item[ChIP-seq] Chromatin immunoprecipitation followed by sequencing.
\item[D-loop] Displacement loop.
\item[DAFS] Derived allele frequency spectrum (a.k.a.\ site frequency spectrum, SFS).
\item[DAF] Derived allele frequency.
\item[DBA2] Mouse strain DBA/2J\@.
\item[dBGC] DSB-induced biased gene conversion.
\item[dHJ] Double-Holliday junction.
\item[DMC1] DNA meiotic recombinase 1 (or Dosage Suppressor of Mck1 homologue).
\item[DNA] Deoxyribonucleic acid.
\item[DNM] \textit{De novo} mutation.
\item[DSBR] Double-strand break repair.
\item[DSB] Double-strand break.
\item[dsDNA] Double-stranded DNA\@.
\item[EME1] Essential meiotic structure-specific endonuclease 1 (Yeast homologue: Mms4).
\item[EM] Expectation maximisation.
\item[F1 hybrid] First filial generation of offspring of distinct parental types.
\item[F2] Second filial generation. Results from a F1 $\times$ F1 cross.
\item[F3, F4, etc] Subsequent filial generations.
\item[FIMO] Find individual motif occurrences (bioinformatic tool).
\item[FP] False positive.
\item[GATK] Genome analysis toolkit (bioinformatic tool).
\item[gBGC] GC-biased gene conversion.
\item[GC\textsubscript{3}] GC-content at third codon position.
\item[GC\textsuperscript{*}] Stationary (or equilibrium) GC-content.
\item[GRC] Genome reference consortium.
\item[GRCm38] Genome reference consortium mouse build 38 (synonym: mm10).
\item[G] Guanine.
\item[Gb] Giga base pairs.
\item[H3K36me3] Addition of 3 methyl groups to the lysine 36 on the histone H3 protein.
\item[H3K4me3] Addition of 3 methyl groups to the lysine 4 on the histone H3 protein.
\item[H\textsubscript{$n1$}K\textsubscript{$n2$}me\textsubscript{$n3$}] Addition of $n3$ methyl groups to the lysine $n2$ on the histone $n1$ protein.
\item[HACNS] Human accelerated conserved non-coding sequence (a.k.a.\ HAR).
\item[HAR] Human accelerated region (a.k.a.\ HACNS).
\item[HFM1] Helicase for meiosis 1 (yeast homologue: Mer3).
\item[HJ] Holliday junction.
\item[HORMAD1] HORMA domain-containing protein 1.
\item[HRR] Homologue recognition region.
\item[HR] Homologous recombination.
\item[Hop1] Homologue pairing 1 (mouse ortholog HORMAD1).
\item[INM] Inner nuclear membrane.
\item[Indel] Insertion or deletion.
\item[kb] Kilo base pairs.
\item[KRAB] Kr\"uppel-associated box.
\item[$L$] Conversion tract length.
\item[LCA] Last common ancestor.
\item[LD] Linkage disequilibrium.
\item[LE] Lateral element.
\item[MCM8,9] Minichromosome maintenance deficient 8, 9.
\item[MEI1] Meiosis inhibitor protein 1.
\item[MEI4] Meiosis inhibitor protein 4.
\item[MEME] Multiple EM for motif elicitation (bioinformatic tool).
\item[MGP] Mouse genomes project.
\item[MHC] Major histocompatibility complex.
\item[MLH1] MutL protein homologue 1.
\item[mm10] \textit{Mus musculus} genome build 10 (synonym: GRCm38).
\item[MMR] Mismatch repair.
\item[MRE11] Meiotic Recombination 11 (yeast homologue: Mre11).
\item[mRNA] Messenger RNA\@.
\item[MSH4,5] MutS protein homologue 4, 5.
\item[MUS81] Crossover junction endonuclease 81.
\item[M] Morgan.
\item[Mb] Mega base pairs.
\item[Mer2] Yeast recombination protein 107 (alias: Rec107).
\item[MHC] Major histocompatibility complex.
\item[Mms4] Methyl methanesulfonate sensitivity 4 (Mouse homologue: EME1).
\item[NBS1] Nibrin (yeast homologue: Xrs2).
\item[NCO-1] Single-marker NCO\@.
\item[NCO-2+] Multiple-marker NCO\@.
\item[NCO] Non crossing-over (or non-crossover).
\item[NDR] Nucleosome-depleted region.
\item[$Ne$] Effective population size.
\item[NE] Nuclear enveloppe.
\item[NHEJ] Non-homologous end-joining.
\item[NOV] Novel.
\item[ONM] Outer nuclear membrane.
\item[PAR] Pseudoautosomal region.
\item[PCR] Polymerase chain reaction.
\item[PC] Pairing centre.
\item[PMS] Post-meiotic segregation.
\item[pot-NCO-1] Potential single-marker NCO\@.
\item[PRDM9] Positive regulatory (PR) domain zinc finger protein 9.
\item[$r$] Recombination rate.
\item[RAD50] RAD50 double-strand break repair protein (yeast homologue: Rad50).
\item[RAD51] RAD51 double-strand break repair protein (yeast homologue: Rad51).
\item[Rec114] Yeast recombination protein 114.
\item[REC8] Meiotic recombination protein 8.
\item[REF] Reference (\textit{versus} ALT\@: alternate).
\item[RFLP] Restriction fragment length polymorphism.
\item[RNA] Ribonucleic acid.
\item[RNAi] RNA interference.
\item[RNF212] RING finger protein 212 (yeast homologue: Zip4.)
\item[RPA] Replication protein A\@.
\item[RR] Recombination rate.
\item[S] Strong nucleotide (G or C).
\item[SCP1] Synaptonemal complex protein 1.
\item[SCP2] Synaptonemal complex protein 2.
\item[SCP3] Synaptonemal complex protein 3.
\item[SC] Synaptonemal complex.
\item[SDSA] Synthesis-dependent strand annealing.
\item[SEI] Single-end invasion.
\item[SFS] Site frequency spectrum (a.k.a.\ derived allele frequency spectrum, DAFS).
\item[SNP] Single-nucleotide polymorphism.
\item[SPO11] SPO11 initiator of meiotic double-strand breaks.
\item[ssDNA] Single-stranded DNA\@.
\item[SSRXD] Synovial sarcoma X repression domain.
\item[sym] Symmetric.
\item[SW (S~$\rightarrow$~W)] Mutation from a ‘strong’ (S) (i.e.\ G or C) to a ‘weak’ (W) nucleotide (i.e.\ A or T). Alternatively noted GC~$\rightarrow$~AT\@.
\item[SYCE1] Synaptonemal complex central element 1.
\item[SYCE2] Synaptonemal complex central element 2.
\item[Srs2] Yeast suppressor of Rad six 2.
\item[tB] Targeted by PRDM9\textsuperscript{Dom2}.
\item[tC] Targeted by PRDM9\textsuperscript{Cst}.
\item[TEX11] Testis-expressed sequence 11 (yeast homologue: Zip3).
\item[TE] Transposable element.
\item[TF] Transverse filaments.
\item[TSS] Transcription start site.
\item[T] Thymine.
\item[tRNA] Transfer RNA\@.
\item[UTR] Untranslated region.
\item[VQSR] Variant quality score recalibration.
\item[W] Weak nucleotide (A or T).
\item[WS (W~$\rightarrow$~S)] Mutation from a ‘weak’ (W) (i.e.\ A or T) to a ‘strong’ (S) nucleotide (i.e.\ G or C). Alternatively noted AT~$\rightarrow$~GC\@.
\item[WT] Wild-type.
\item[Xrs2] DNA repair protein (Mouse homologue: NBS1).
\item[Zip3,4] Homologues of mammalian TEX11 and RNF212
\item[Znf] Zinc finger.
\item[\textit{Prdm9\textsuperscript{Cst}}] \textit{Prdm9} allele carried by CAST mice.
\item[\textit{Prdm9\textsuperscript{Dom2}}] \textit{Prdm9} allele carried by B6 mice.

	% pour les yeasts: https://www.yeastgenome.org/locus/S000001334
	% pour les mice: https://www.genecards.org/Search/Keyword?queryString=Srs2


	%CO–crossover, DSB–double strand break, HR–homologous recombination, HRR–homology recognition region, IR–ionizing radiation, NCO–non-crossover, NHEJ–non-homologous end joining, PC–pairing center, SC–synaptonemal complex



\end{mclistof}








% List of definitions
\begin{alwayssingle}
	%  \begin{savequote}[8cm]
	%
	%     ‘We can’t define anything precisely. If we attempt to, we get into that paralysis of thought that comes to philosophers… One saying to the other: “You don’t know what you are talking about!”.
	%     The second one says: “What do you mean by talking? What do you mean by you? What do you mean by know?”’
	%     \qauthor{--- f, \textit{\usebibentry{flaubert1889correspondance}{title}} \citeyearpar{flaubert1889correspondance} }
	%
	% \end{savequote}
	%
	\chapter*{Definitions}

	\addcontentsline{toc}{chapter}{Definitions}
	\thispagestyle{plain}
	\pagestyle{fancy}
	\fancyhead[LO]{\emph{Definitions}}
	\fancyhead[RE]{\emph{Definitions}}
	\fancyhead[RO,LE]{\emph{\thepage}}
	\setlength{\baselineskip}{\frontmatterbaselineskip}
	\begin{description}

		\item[Achiasmy] The phenomenon where autosomal recombination is completely absent in one sex of a species.
		\item[Allele] A variant form of a given gene.
		\item[Anaphase] Third stage of mitosis, meiosis I and meiosis II\@.
		\item[Ascospore] Reproductive cells of a certain class of fungi (ascomycetes).
		\item[Asymmetric hotspot] Hotspot for which one of the two haplotypes is more likely to host the double-strand break.
		\item[Apoptosis] Programmed cell death (from the Greek word \textgreek{ἀπόπτωσις}: ‘falling off’).
		\item[Biased gene conversion] Process by which gene conversion is biased towards a given outcome. It occurs when one haplotype has a higher probability of being the donor.
		\item[C-terminus] End of an amino acid chain terminated by a free carboxyl group.
		\item[Centimorgan] Unit of genetic distance: 1 cM corresponds to a frequency of crossing-overs of 1\%.
		\item[ChIP-sequencing] Method used to analyse protein interaction with DNA\@.
		\item[Chiasma (\textit{pl.} chiasmata)] An exchange (crossing-over) between paired chromatids, observed cytologically between diplotene and the first meiotic anaphase, from the Greek word \textit{\textgreek{χίασμα}}: ‘X-shaped cross’.%, which refers to two lines placed cross-wise, like an “X”.
		\item[Chromatid] A DNA molecule associated to proteins and forming one half of the two identical copies of a replicated chromosome.
		\item[Codon usage bias] Unequal frequency of the alternative codons that specify the same amino acid.
		\item[Codon] Sequence of three nucleotides coding for a given amino acid.
		\item[Cold haplotype] The haplotype that is most often the donor in the gene conversion event.
		% \item[Conversion polarity (or polarised recombination)]
		\item[CpG (or CG) site] Region of DNA where a cytosine nucleotide is followed by a guanine nucleotide in the 5’-to-3’ direction.
		\item[CpG island] Region with a high frequency of CpG sites.
		% \item[Crossing-over interference, assurance, homeostasis]
		\item[Crossing-over] \mccorrect{Recombination event leading to the reciprocal exchange of the DNA sequences flanking the crossing-over point}.
		\item[DNA capture] Hybridisation-based targeted-DNA enrichment.
		\item[DSB-induced biased gene conversion] The form of biased gene conversion due to the differential formation of double-strand breaks on the two haplotypes.
		\item[Diploid] Organism (or phase) displaying a ploidy of 2 ($n=2$), i.e.\ two sets of chromosomes (which are paired).
		\item[Ectopic gene conversion] Gene conversion between copies of a gene family.
		\item[Effective population size] The number of individuals in a population who contribute to the next generation.
		\item[GC-biased gene conversion] The process by which the GC-content increases because of biased gene conversion.
		\item[GC-content] The percentage of G or C nucleotidic bases in a DNA sequence.
		\item[Gamete] Product of meiosis.
		\item[Gene conversion] A non-reciprocal recombination process that results in one sequence being converted into the other.
		\item[Genetic drift] The random fluctuation in allele frequencies due to random sampling of individuals.
		\item[Genetic distance] Distance between DNA markers on a chromosome measured as the amount of crossing-overs between them.
		\item[Genetic interference] The fact that the formation of a recombination event can affect that of others in adjacent regions.
		\item[Genetic linkage] Non-independent assortment of genes.
		\item[Genetic marker] A known site of heterozygosity.
		\item[Genotyping] The process by which DNA is analyzed to determine which genetic variant (allele) is present for a given marker.
		\item[Haploid] Organism (or phase) displaying a ploidy of 1 ($n=1$), i.e.\ a single set of chromosomes.
		\item[Haplotype] (In the context of this thesis, used to define the background of the PRDM9 motif)
		\item[Heterochiasmy] The differential recombination rates between the sexes of a species.
		\item[Heteroduplex DNA] A DNA portion where the two strands composing it contain different information for the segregating marker.
		\item[Holocentric] Chromosome devoid of any major centromeric constriction. 
		\item[Homologues] A set of one paternal and one maternal chromosomes that pair up during meiosis (a.k.a.\ homologous chromosomes).
		\item[Homologous recombination] The process through which segments of DNA are exchanged between two DNA duplexes with high \mccorrect{sequence} similarity.
		\item[Hot haplotype] The haplotype that most often hosts the double-strand break.
		\item[\textit{in silico}] In a computing context.
		\item[\textit{in vitro}] Outside the normal biological context.
		\item[\textit{in vivo}] Inside the normal biological context.
		\item[Interphase] Period of cell growth before cell division.
		\item[Locus (\textit{pl.} loci)] Fixed position of a genetic marker on a chromosome (from the Latin word \textit{locus}: ‘place’).
		\item[Linkage disequilibrium] Non-random associations between loci.
		\item[Meiosis] Specialised cell division that reduces the chromosome number by half and leads to the formation of gametes.
		\item[Metaphase] Second stage of mitosis, meiosis I and meiosis II\@.
		\item[\textit{n}-fold degenerate codon] A position of a codon is said to be \textit{n}-fold degenerate if \textit{n} of the four nucleotides possibleat this position (A, T, C, G) end in the same amino acid (AA). By extension, a codon is said to be \textit{n}-fold degenerate if \textit{n} different three-nucleeotide sequences will code for the same AA\@.
		\item[N-terminus] End of an amino acid chain terminated by a free amine group.
		\item[Non-crossover] \mccorrect{Recombination event without the exchange of flanking DNA sequences.}
		\item[Nonself haplotype] The haplotype that did not co-evolve with a given \textit{Prdm9} allele.
		\item[Nonsynonymous substitution] Substitution that does not modify the amino acid produced.
		\item[Outgroup] Distantly related group of organisms that serves as the ancestral reference for the studied group (or ingroup).
		\item[Pedigree] A family tree drawn with standard genetic symbols, showing inheritance patterns for specific phenotypic characters.
		\item[Phenotype] The composite of observable traits.
		\item[Ploidy] The number of complete sets of chromosomes ($n$) in a cell. 
		\item[Polymerase chain reaction] Molecular biology method used to make copies of a specific DNA fragment.
		\item[Polymorphic] Which presents several forms. In other words: subject to inter-individual variability.
		\item[Post-meiotic segregation] Segregation occuring after the end of meiosis, during the mitotic division (Figure~\ref{fig:spores-formation}).
		\item[Primer] Short single-stranded nucleic acid used to initiate DNA synthesis.
		\item[Prophase] First stage of mitosis, meiosis I and meiosis II\@.
		\item[Pseudogene, pseudogeneisation (pas utilise)]
		\item[Purebred] Bred from members of a recognised breed, strain, or kind without admixture of other blood over many generations.
		\item[Reciprocal cross] Breeding experiment designed to test the role of parental sex on a given inheritance pattern.
		\item[Recombination hotspot] Region of the genome with an elevated rate of recombination.
		\item[Recombination] Exchange of DNA sequence information.
		\item[Self haplotype] The haplotype that co-evolved with a given \textit{Prdm9} allele.
		\item[Sister chromatids] The two chromatids originating from the same chromosome (after a replication event).
		\item[Stationary GC-content (GC\textsuperscript{$\star$})] The GC-content that sequences would reach at equilibrium if patterns of substitution remained constant over time.
		\item[Symmetric hotspot] Hotspot for which the two haplotypes are equally likely to host the double-strand break.
		\item[Synapsis] Pairing of homologues.
		\item[Synonymous substitution] Substitution that modifies the amino acid produced.
		\item[Telophase] Fourth stage of mitosis, meiosis I and meiosis II\@.
		\item[Tetrad analysis] Analysis of the four products (gametes) resulting from one single meiosis event.
		\item[Transition] Mutation between two nucleotidic bases of the same family (purine or pyrimidine), i.e.\ either a A~$\leftrightarrow$~G or a C~$\leftrightarrow$~T mutation.
		\item[Transversion] Mutation involving a change of nucleotidic family (from a purine to a pyrimidine or the other way round), i.e.\ either a A~$\leftrightarrow$~C, a A~$\leftrightarrow$~T, a G~$\leftrightarrow$~C or a G~$\leftrightarrow$~T mutation.
		\item[Variant-calling] The process of identifying variant (a.k.a.\ polymorphic) sites on a genome.
		\item[ZMM complex] A set of \mccorrect{conserved} yeast proteins Zip1, Zip2, Zip3, Zip4, Mer3, Msh4, Msh5 \mccorrect{and Spo11} (a.k.a.\ synapsis initiation complex, SIC).



			% NOTE\@: pour pas mal de definitions, voir http://www.lifesci.sussex.ac.uk/CSE/members/aeyrewalker/pdfs/EWNRG01%201.pdf

			% ajouter la definition de ZMM (p38 CO pathway)
			% a set of yeast proteins (Zip1, Zip2, Zip3, Zip4/Spo22, Mer3, Msh4, and Msh5, termed the SIC or ZMM proteins)

			% https://incenp.org/notes/2012/gene-nomenclature.html



	\end{description}
\end{alwayssingle}
\mtcaddchapter{}


