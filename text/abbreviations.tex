% First parameter can be changed eg to "Glossary" or something. / List of Abbreviations
% Second parameter is the max length of bold terms.

\begin{mclistof}{Abbreviations}{3.2cm}

\item[F1 hybrid] First filial generation of offspring of distinct parental types.
\item[F2] Second filial generation. Results from a F1 $\times$ F1 cross.
\item[F3, F4, etc] Subsequent filial generations.
	% \otherpagedecoration


\item[CO] Crossing-over (or crossover).
\item[NCO] Non crossing-over (or non-crossover).
\item[PMS] Post-meiotic segregation.
\item[DNA] 
\item[DSB] Double-strand break.
\item[NHEJ]
\item[HR] Homologous recombination.
\item[HRR] Homologue recognition region.
\item[SC] 
\item[NE] Nuclear enveloppe.
\item[A]
\item[C]
\item[G]
\item[T]
\item[MMR]
\item[BER]
\item[kb]
\item[Mb]
\item[Gb]
\item[INM] Inner nuclear membrane.
\item[ONM] Outer nuclear membrane.
\item[PC] Pairing centre.
\item[DSB]
\item[ssDNA]
\item[dsDNA]
\item[LCA] Last common ancestor.
\item[PCR] Polymerase chain reaction.
\item[COA] Crossing-over assurance.
\item[COI] Crossing-over interference.
\item[COH] Crossing-over homeostasis.
\item[MHC] Major histocompatibility complex.
\item[RNA]
\item[RNAi] RNA interference.
\item[NDR] Nucleosome-depleted region.
\item[TSS] Transcription start site.
\item[DNM] \textit{De novo} mutation.
\item[UTR]
\item[CpG island]
\item[H3K4me3]
\item[H3K36me3]
\item[PRDM9]
\item[C2H2] Cys\textsubscript{2}-His\textsubscript{2}.
\item[Znf] Zinc finger.
\item[BGC]
\item[dBGC]
\item[gBGC]
\item[KRAB]
\item[SSRXD]
\item[Znf]
\item[PR/SET]
\item[H3K4]
\item[H3K36]
\item[general: Hn1Kn2men3]
\item[TE] Transposable element.
\item[CUB] Codon usage bias.
\item[HAR] HUman accelerated region.
\item[Ne]
\item[MMR] Mismatch repair.
\item[BER] Base excision repair.
\item[GC\textsubscript{3}] GC-content at third codon position.
\item[PAR] Pseudoautosomal region.
\item[GC\textsuperscript{*}] Stationary (or equilibrium) GC-content.
\item[CpG site] Dans definition?? 5’-CG-3’ dinucleotide.
\item[CUB] Codon usage bias.
\item[HAR] Human accelerated region.
\item[AA] Amino acid.
\item[tRNA]
\item[mRNA]
\item[$Ne$]
\item[HAR] Human accelerated region.
\item[HACNS] Human accelerated conserved non-coding sequence.
\item[WS (W~$\rightarrow$~S)] Mutation from a ‘weak’ (W) (\textit{i.e.} A or T) to a ‘strong’ (S) nucleotide (\textit{i.e.} G or C). Alternatively noted AT~$\rightarrow$~GC\@.
\item[SW (S~$\rightarrow$~W)] Mutation from a ‘strong’ (S) (\textit{i.e.} G or C) to a ‘weak’ (W) nucleotide (\textit{i.e.} A or T). Alternatively noted GC~$\rightarrow$~AT\@.
\item[DAF] Derived allele frequency.
\item[SFS] Site frequency spectrum (a.k.a.\ derived allele frequency spectrum, DAFS).
\item[DAFS] Derived allele frequency spectrum (a.k.a.\ site frequency spectrum, SFS).
\item[KO] Knock-out (ou knocking-out??)
\item[KI] Knock-in.
\item[FP] False positive.
\item[INDEL]
\item[BQSR] Base quality score recalibration.
\item[VQSR] Variant quality score recalibration.
\item[BWA]
\item[BWA-MEM]
\item[MEME]
\item[FIMO]
\item[GATK]
\item[B6]
\item[CAST]
\item[GRCm38]
\item[mm10]
\item[ChIP-seq]
\item[C57BL/6J]
\item[CAST/EiJ]
\item[FP] False positive.
\item[MGP] Mouse genomes project.
	ANNOTATIONS GATK
\item[CI] Confidence interval.
\item[CT] Conversion tract.
\item[CT\textsuperscript{$\star$}] Observed (or inferred) conversion tract.
\item[dBGC]
\item[gBGC]
\item[BGC]
\item[B6]
\item[CAST]
\item[\textit{Prdm9\textsuperscript{Dom2}}] \textit{Prdm9} allele carried by B6 mice.
\item[\textit{Prdm9\textsuperscript{Cst}}] \textit{Prdm9} allele carried by CAST mice.
\item[$b$]
\item[$b_{dBGC}$]
\item[$b_0$]
\item[$L$]
\item[$r$]
\item[$B$]
\item[WS, S, W, A, T, C, G, W\textrightarrow{} S, S\textrightarrow{} W…]
\item[ABC]
\item[FP] False positive.
\item[NCO-1]
\item[pot-NCO-1]
\item[NCO-2+]
\item[ZMM complex] acronym for yeast proteins Zip1/Zip2/Zip3/Zip4, Msh4/Msh5, Mer3
\item[D-loop] (definition)
\item[WT] Wild-type.
\item[CCO] Complex crossing-over.
\item[BD]
\item[DBA2]
\item[SNP]
\item[REF] Reference (\textit{versus} ALT\@: alternate).
\item[ALT] Alternate (\textit{versus} REF\@: reference).

	Synaptonemal complex associated
\item[LE] Lateral element.
\item[CE] Central element.
\item[TF] Transverse filaments.
\item[SCP1,2,3]
\item[SYCE1,2]


	Recombination proteins
\item[ATM (kinase)] Ataxia telangiectasia mutated (kinase).
\item[MEI1,4]
\item[RPA]
\item[DMC1]
\item[RAD50, RAD51] yeast Rad50, Rad51
\item[MRE11] yeast Mre11
\item[NBS1]
\item[HORMAD1] HORMA domain-containing protein 1.
\item[Mer2,3] YEAST
\item[Rec114] YEAST
\item[SPO11]
\item[NBS1] Yeast Xrs2
\item[REC8]
\item[BLM] Bloom syndrome RecQ helicase-like.
\item[TEX11] Testis-expressed sequence 11 (yeast homologue: Zip3).
\item[Zip3,4] Homologues of mammalian TEX11 and RNF212
\item[RNF212] RING finger protein 212 (yeast homologue: Zip4.)
\item[MCM8,9] Minichromosome maintenance deficient 8, 9.
\item[MSH4,5] MutS protein homologue 4, 5.
\item[MLH1] MutL protein homologue 1.
\item[TEX11]
\item[HFM1]
\item[MUS81]
\item[Mms4] Mouse homologue: EME1
\item[EME1] Yeast homologue: Mms4
\item[Srs2] only yeast, pas chez mammif
\item[Hop1] (mouse ortholog HORMAD1)
\item[PCR]

Recombination models
\item[HJ] Holliday junction.
\item[dHJ] Double-Holliday junction.
\item[SDSA]
\item[DSBR]
\item[NHEJ]
\item[SEI] Single-end invasion.
\item[D-loop]


	Genetic distances
\item[M] Morgan.
\item[cM] Centimorgan.
\item[SNP]


%CO–crossover, DSB–double strand break, HR–homologous recombination, HRR–homology recognition region, IR–ionizing radiation, NCO–non-crossover, NHEJ–non-homologous end joining, PC–pairing center, SC–synaptonemal complex



\end{mclistof}




% List of definitions
\begin{alwayssingle}\chapter*{Definitions}
	\addcontentsline{toc}{chapter}{Definitions}
	\thispagestyle{plain}
	\pagestyle{fancy}
	\fancyhead[LO]{\emph{Definitions}}
	\fancyhead[RE]{\emph{Definitions}}
	\fancyhead[RO,LE]{\emph{\thepage}}
	\setlength{\baselineskip}{\frontmatterbaselineskip}
	\begin{description}

		\item[Purebred] Bred from members of a recognised breed, strain, or kind without admixture of other blood over many generations.
		\item[Reciprocal cross] Breeding experiment designed to test the role of parental sex on a given inheritance pattern.
		\item[Gene conversion] A non-reciprocal recombination process that results in an alteration of the sequence of a gene to that of its homologue.
	\item[Chiasma] (plural chiasmata) an exchange (crossing-over) between paired chromatids, observed cytologically between diplotene and the first meiotic anaphase, from the Greek word \textit{\textgreek{χίασμα}}: “X-shaped cross”.%, which refers to two lines placed cross-wise, like an “X”.
		\item[Tetrad analysis] Analysis of the four products (gametes) resulting from one single meiosis event.
		\item[Haploid] Organism (or phase) displaying a ploidy of 1 ($n=1$), i.e.\ a single set of chromosomes.
		\item[Diploid] Organism (or phase) displaying a ploidy of 2 ($n=2$), i.e.\ two sets of chromosomes (which are paired).
		\item[Ploidy] The number of complete sets of chromosomes ($n$) in a cell. 
		\item[Phenotype]
		\item[Genotype]
		\item[Tetrad analysis]
		\item[Meiosis]
		\item[Mitosis]
		\item[Recombination]
		\item[Ascospore]
		\item[(Genetic) marker] % A molecular marker is a site of heterozygosity for some type of silent DNA variation not associated with any measurable phenotypic variation (Popa????)
		\item[Heteroduplex DNA]
		\item[Negative interference]
		\item[Conversion polarity (or polarised recombination)]
		\item[Post-meiotic segregation]
		\item[Gene conversion]
		\item[Gene linkage]
		\item[Crossing-over]
		\item[Gamete]
		\item[Interphase]
		\item[Homologous chromosomes]
		\item[Sister chromatids]
		\item[Chromosome]
		\item[Chromatid]
		\item[Crossing-over]
		\item[Non-crossover]
		\item[Homologous chromosomes]
		\item[Sister chromatids]
		\item[Allele]
		\item[Gene]
		\item[Muller's Ratchet]
		\item[Hill-Robertson]
		\item[C terminus, N terminus]
		\item[Locus (\textit{pl.} loci)]
		\item[Genetic marker]
		\item[Polymorphic]
		\item[Allele] (In the context of this thesis, used to define which PRDM9 allele)
		\item[Haplotype] (In the context of this thesis, used to define the background of the PRDM9 motif)
		\item[Pedigree] A family tree drawn with standard genetic symbols, showing inheritance patterns for specific phenotypic characters.
		\item[Polymerase chain reaction]
		\item[ChIP-seq]
		\item[Heterochiasmy] The differential recombination rates between the sexes of a species.
		\item[Achiasmy] The phenomenon where autosomal recombination is completely absent in one sex of a species.
		\item[Holocentric]
		\item[Red Queen, hotspot paradox]
		\item[Meiotic drive]
		\item[\textit{in vitro}, \textit{in vivo}, \textit{in silico}]
		\item[Recombination hotspot]
		\item[Pseudogene, pseudogeneisation (pas utilise)]
		\item[Crossing-over interference, assurance, homeostasis]
		\item[Heteroduplex, heteroduplexed DNA]
		\item[Codon usage bias (CUB)]
		\item[effective population size (Ne)]
\item[GC-content]
\item[codon]
\item[\textit{n}-fold degenerate codon] A position of a codon is said to be \textit{n}-fold degenerate if \textit{n} of the four nucleotides possibleat this position (A, T, C, G) end in the same amino acid (AA). By extension, a codon is said to be \textit{n}-fold degenerate if \textit{n} different three-nucleeotide sequences will code for the same AA\@.
\item[Outgroup]
\item[Purine]
\item[Pyrimidine]
\item[Transition]
\item[Transversion]
\item[Knocking-out (ou knock-out???)]
\item[DNA capture]
\item[ChIP-sequencing]
\item[Phred quality score]
\item[Symmetric/Asymmetric hotspot]
\item[GC-biased gene conversion]
\item[DSB-induced biased gene conversion]
\item[Gene conversion]
\item[Principle of parsimony]

Meiosis-linked definitions
\item[Prophase]
\item[Metaphase]
\item[Anaphase]
\item[Telophase]
\item[Cellular division]
\item[Equatorial plate]


NOTE\@: pour pas mal de definitions, voir http://www.lifesci.sussex.ac.uk/CSE/members/aeyrewalker/pdfs/EWNRG01%201.pdf 


	\end{description}
\end{alwayssingle}
\mtcaddchapter{}


