% First parameter can be changed eg to "Glossary" or something. / List of Abbreviations
% Second parameter is the max length of bold terms.

\begin{mclistof}{Abbreviations}{3.2cm}

\item[F1 hybrid] First filial generation of offspring of distinct parental types.
\item[F2] Second filial generation. Results from a F1 $\times$ F1 cross.
\item[F3, F4, etc] Subsequent filial generations.
	% \otherpagedecoration


\item[CO] Crossing-over (or crossover).
\item[NCO] Non crossing-over (or non-crossover).
\item[PMS] Post-meiotic segregation.
\item[DNA] 
\item[DSB] Double-strand break.
\item[NHEJ]
\item[HR] Homologous recombination.
\item[HRR] Homologue recognition region.
\item[SC] 
\item[NE] Nuclear enveloppe.
\item[A]
\item[C]
\item[G]
\item[T]
\item[MMR]
\item[BER]
\item[kb]
\item[Mb]
\item[Gb]
\item[INM] Inner nuclear membrane.
\item[ONM] Outer nuclear membrane.
\item[PC] Pairing centre.
\item[DSB]
\item[ssDNA]
\item[dsDNA]
\item[LCA] Last common ancestor.
\item[PCR] Polymerase chain reaction.
\item[COA] Crossing-over assurance.
\item[COI] Crossing-over interference.
\item[COH] Crossing-over homeostasis.
\item[MHC] Major histocompatibility complex.
\item[RNA]
\item[RNAi] RNA interference.
\item[NDR] Nucleosome-depleted region.
\item[TSS] Transcription start site.
\item[DNM] \textit{De novo} mutation.
\item[UTR]
\item[CpG island]
\item[H3K4me3]

	Synaptonemal complex associated
\item[LE] Lateral element.
\item[CE] Central element.
\item[TF] Transverse filaments.
\item[SCP1,2,3]
\item[SYCE1,2]


	Recombination proteins
\item[ATM (kinase)] Ataxia telangiectasia mutated (kinase).
\item[MEI1,4]
\item[RPA]
\item[DMC1]
\item[RAD50, RAD51]
\item[MRE11]
\item[NBS1]
\item[HORMAD1] HORMA domain-containing protein 1.
\item[MER2,3]
\item[REC114]
\item[SPO11]
\item[REC8]
\item[BLM] Bloom syndrome RecQ helicase-like.
\item[TEX11] Testis-expressed sequence 11.
\item[ZIP3,4]
\item[RNF212] RING finger protein 212.
\item[MCM8,9] Minichromosome maintenance deficient 8, 9.
\item[MSH4,5] MutS protein homologue 4, 5.
\item[MLH1] MutL protein homologue 1.
\item[TEX11]
\item[HFM1]
\item[MUS81]
\item[MMS4]
\item[SRS2]


Recombination models
\item[HJ] Holliday junction.
\item[dHJ] Double-Holliday junction.
\item[SDSA]
\item[DSBR]
\item[NHEJ]
\item[SEI] Single-end invasion.
\item[D-loop]


	Genetic distances
\item[M] Morgan.
\item[cM] Centimorgan.
\item[SNP]


%CO–crossover, DSB–double strand break, HR–homologous recombination, HRR–homology recognition region, IR–ionizing radiation, NCO–non-crossover, NHEJ–non-homologous end joining, PC–pairing center, SC–synaptonemal complex



\end{mclistof}




% List of definitions
\begin{alwayssingle}\chapter*{Definitions}
	\addcontentsline{toc}{chapter}{Definitions}
	\thispagestyle{plain}
	\pagestyle{fancy}
	\fancyhead[LO]{\emph{Definitions}}
	\fancyhead[RE]{\emph{Definitions}}
	\fancyhead[RO,LE]{\emph{\thepage}}
	\setlength{\baselineskip}{\frontmatterbaselineskip}
	\begin{description}

		\item[Purebred] Bred from members of a recognized breed, strain, or kind without admixture of other blood over many generations.
		\item[Reciprocal cross] Breeding experiment designed to test the role of parental sex on a given inheritance pattern.
		\item[Gene conversion] A non-reciprocal recombination process that results in an alteration of the sequence of a gene to that of its homologue.
	\item[Chiasma] (plural chiasmata) an exchange (crossing-over) between paired chromatids, observed cytologically between diplotene and the first meiotic anaphase, from the Greek word \textit{\textgreek{χίασμα}}: “X-shaped cross”.%, which refers to two lines placed cross-wise, like an “X”.
		\item[Tetrad analysis] Analysis of the four products (gametes) resulting from one single meiosis event.
		\item[Haploid] Organism (or phase) displaying a ploidy of 1 ($n=1$), i.e.\ a single set of chromosomes.
		\item[Diploid] Organism (or phase) displaying a ploidy of 2 ($n=2$), i.e.\ two sets of chromosomes (which are paired).
		\item[Ploidy] The number of complete sets of chromosomes ($n$) in a cell. 
		\item[Phenotype]
		\item[Genotype]
		\item[Tetrad analysis]
		\item[Meiosis]
		\item[Mitosis]
		\item[Recombination]
		\item[Ascospore]
		\item[(Genetic) marker] % A molecular marker is a site of heterozygosity for some type of silent DNA variation not associated with any measurable phenotypic variation (Popa????)
		\item[Heteroduplex DNA]
		\item[Negative interference]
		\item[Conversion polarity (or polarised recombination)]
		\item[Post-meiotic segregation]
		\item[Gene conversion]
		\item[Gene linkage]
		\item[Crossing-over]
		\item[Gamete]
		\item[Interphase]
		\item[Homologous chromosomes]
		\item[Sister chromatids]
		\item[Chromosome]
		\item[Chromatid]
		\item[Crossing-over]
		\item[Non-crossover]
		\item[Homologous chromosomes]
		\item[Sister chromatids]
		\item[Allele]
		\item[Gene]
		\item[Muller's Ratchet]
		\item[Hill-Robertson]
		\item[C terminus, N terminus]
		\item[Locus (\textit{pl.} loci)]
		\item[Genetic marker]
		\item[Polymorphic]
		\item[Allele]
		\item[Haplotype]
		\item[Pedigree] A family tree drawn with standard genetic symbols, showing inheritance patterns for specific phenotypic characters.
		\item[Polymerase chain reaction]
		\item[ChIP-seq]
		\item[Heterochiasmy] The differential recombination rates between the sexes of a species.
		\item[Achiasmy] The phenomenon where autosomal recombination is completely absent in one sex of a species.


Meiosis-linked definitions
\item[Prophase]
\item[Metaphase]
\item[Anaphase]
\item[Telophase]
\item[Cellular division]
\item[Equatorial plate]





	\end{description}
\end{alwayssingle}
\mtcaddchapter{}


