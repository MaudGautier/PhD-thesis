% First parameter can be changed eg to "Glossary" or something. / List of Abbreviations
% Second parameter is the max length of bold terms.

\begin{mclistof}{List of Abbreviations}{3.2cm}

\item[F1 hybrid] First filial generation of offspring of distinct parental types.
\item[F2] Second filial generation. Results from a F1 $\times$ F1 cross.
\item[F3, F4, etc] Subsequent filial generations.
	% \otherpagedecoration


\item[CO] Crossing-over (or crossover).
\item[NCO] Non crossing-over (or non-crossover).
\item[PMS] Post-meiotic segregation.
\item[DNA] 
\item[DSB] Double-strand break.
\item[NHEJ]
\item[HR]
\item[HRR]
\item[SC]


%CO–crossover, DSB–double strand break, HR–homologous recombination, HRR–homology recognition region, IR–ionizing radiation, NCO–non-crossover, NHEJ–non-homologous end joining, PC–pairing center, SC–synaptonemal complex



\end{mclistof}




% List of definitions
\begin{alwayssingle}\chapter*{List of Definitions}
	\addcontentsline{toc}{chapter}{List of Definitions}
	\thispagestyle{empty}
	\pagestyle{empty}
	\setlength{\baselineskip}{\frontmatterbaselineskip}
	\begin{description}

		\item[Purebred] Bred from members of a recognized breed, strain, or kind without admixture of other blood over many generations.
		\item[Reciprocal cross] Breeding experiment designed to test the role of parental sex on a given inheritance pattern.
		\item[Gene conversion] A non-reciprocal recombination process that results in an alteration of the sequence of a gene to that of its homologue.
	\item[Chiasma] (plural chiasmata) an exchange (crossing-over) between paired chromatids, observed cytologically between diplotene and the first meiotic anaphase, from the Greek word \textit{\textgreek{χίασμα}}, which refers to two lines placed cross-wise, like an “X”.
		\item[Tetrad analysis] Analysis of the four products (gametes) resulting from one single meiosis event.
		\item[Haploid] Organism (or phase) displaying a ploidy of 1 ($n=1$), i.e.\ a single set of chromosomes.
		\item[Diploid] Organism (or phase) displaying a ploidy of 2 ($n=2$), i.e.\ two sets of chromosomes (which are paired).
		\item[Ploidy] The number of complete sets of chromosomes ($n$) in a cell. 
		\item[Phenotype]
		\item[Genotype]
		\item[Tetrad analysis]
		\item[Meiosis]
		\item[Mitosis]
		\item[Recombination]
		\item[Ascospore]
		\item[(Genetic) marker]
		\item[Heteroduplex DNA]
		\item[Negative interference]
		\item[Conversion polarity (or polarised recombination)]
		\item[Post-meiotic segregation]
		\item[Gene conversion]
		\item[Gene linkage]
		\item[Crossing-over]
		\item[Gamete]
		\item[Interphase]
		\item[Homologous chromosomes]
		\item[Sister chromatids]
		\item[Chromosome]
		\item[Chromatid]
		\item[Crossing-over]
		\item[Non-crossover]
		\item[Homologous chromosomes]
		\item[Sister chromatids]
		\item[Allele]
		\item[Gene]
		\item[Muller's Ratchet]
		\item[Hill-Robertson]
		\item[]






	\end{description}
\end{alwayssingle}
\mtcaddchapter{}


