

% Cela fait vingt-sept années que je me demande régulièremenent ce que j'ai bien pu faire pour mériter de voir la chance me sourire si souvent, et ce doctorat n'a été qu'une occasion de plus de m'en étonner.
%
% C'est en effet une chance inouïe que de réaliser une thèse sous la direction de Laurent \textsc{Duret}.
% % Réaliser une thèse sous la direction de Laurent \textsc{Duret} a en effet été un cadeau inestimable.
% Son incomparable disponibilité et son souci de fournir un cadre de travail idéal et une formation complète dépassent largement ce que n'importe quel étudiant aurait pu espérer recevoir.
% Au delà de son encadrement scientifique excpetionnel, Laurent a fait preuve d'une humanité et d'une bienveillance rares et je ne crois pas que j'aurais pu arriver au bout de cette thèse sans son soutien pendant les périodes plus difficiles.
% Je ne crois pas que j'aurais pu arriver au bout de cette thèse sans son soutien et sans sa capacité à mettre en avant
% Il serait illégitime de ma part de ne pas dire que Laurent m'a littéralement appris à faire de la science, et pourtant, je crains qu'il ne me faille encore quelques décennies d'apprentissage à ses côtés pour pouvoir prétendre en faire réellement.
%
% Je voudrais donc lui dire ici toute la gratitude et toute l'admiration que je lui porte et espère ne pas avoir trop abusé de
%
%
%
% Laurent
% - cadre de travail idéal par sa bienveillance et son humanité
% - souci de donner une formation complète
%
%

Cela fait maintenant un peu plus de vingt-sept ans que, régulièrement, 
% je m'étonne d'être si chanceuse tant dans les rencontres que je fais et
% je m'étonne de la chance que j'ai dans les rencontres que je fais et
je m'étonne de ce que la chance me sourie si souvent et
% de la chance que j'ai dans les rencontres que je fais et dans les événements que je vis au quotidien et
% je me demande
% d'où vient
% ce que j'ai bien pu faire pour mériter de voir la chance me sourire autant et
% force est de constater que
force est de constater que ce doctorat a été une occasion de plus de la mesurer.

Pouvoir réaliser ma thèse sous la direction de Laurent \textsc{Duret} a en effet été une aubaine inouïe.
% C'était en effet une aubaine inouïe que de pouvoir réaliser ma thèse sous la direction de Laurent \textsc{Duret}.
Sa disponibilité manifestement infinie, son souci de me fournir un cadre idéal de travail et de vie, sa détermination à me prodiguer une formation complète et l'extraordinaire bienveillance dont il a fait preuve chaque jour sont allés bien au-delà de ce que j'aurais pu espérer.
Je crois qu'il serait même illégitime de ma part de ne pas dire que Laurent m'a littéralement appris à faire de la science.
Mais, au delà de son encadrement scientifique exceptionnel, c'est son humanité que je voudrais saluer.
Je n'arrive pas à m'imaginer comment j'aurais pu arriver au bout de cette thèse sans l'immense générosité dont il a fait preuve ni sans sa capacité à me redonner espoir dans les moments plus difficiles.
%mettre en avant les réussites de ce travail lorsque je pensais avoir 
% qui m'a redonné espoir dans les moments plus difficiles.
Je voudrais donc lui dire ici toute l'admiration que je lui porte ainsi que toute ma gratitude pour les conditions privilégiées dont il m'a fait bénéficier pendant ces trois dernières années.
 % et espère ne pas avoir trop abusé des conditions privilégiées qu'il m'a offertes pour faire ce travail.

Je voudrais aussi exprimer mes sincères remerciements aux personnes qui m'ont fait l'honneur d'accepter de lire et d'évaluer ce travail : Valérie \textsc{Borde}, Adam \textsc{Eyre-Walker}, Bertrand \textsc{Llorente}, Dominique \textsc{Mouchiroud} et Gwenael \textsc{Piganeau}.

Merci également à ceux qui ont accepté de participer à mon comité de suivi de thèse pour leurs conseils tant sur les aspects scientifiques que personnels : 
Nicolas \textsc{Galtier},
Annabelle \textsc{Haudry},
Tristan \textsc{Lefébure},
Bernard \textsc{de Massy} et
Jonathan \textsc{Romiguier}.


Le travail qui est présenté dans ce manuscrit est le fruit de collaborations qui ont impliqué un grand nombre de personnes que je me dois également de remercier.
Merci à Frédéric \textsc{Baudat}, Valérie \textsc{Borde}, Corinne \textsc{Grey} et Bernard \textsc{de Massy} d'avoir réalisé la totalité du travail expérimental qui a été essentiel pour cette thèse, pour la patience dont ils ont fait preuve à mon égard et pour leurs suggestions éclairées.
Merci aussi à Nicolas \textsc{Lartillot} pour ses commentaires avisés et à Brice \textsc{Letcher} pour le travail admirable qu'il a réalisé pendant son stage et dont je me suis allègrement servie pour rédiger cette thèse.\\



La sympathie et la bienveillance de l'ensemble des membres du laboratoire rendent la qualité de vie au laboratoire exceptionnelle.
Ils sont trop nombreux pour être tous remerciés de façon individuelle, mais je voudrais tout de même en mentionner certains.
% Ils sont si nombreux que je risque d'en oublier certains, mais je voudrais exprimer ma reconnaissance

% L'environnement de travail exceptionnel dans lequel j'ai pu réaliser cette thèse est en grande partie due à l'ensemble des membres du laboratoire dont la sympathie et la bienveillance rendent


L'une des personnes qui a le plus compté pour moi pendant ces trois années est Anouk \textsc{Nec\c{s}ulea}.
 % est une des personnes qui a le plus compté pour moi pendant ces trois années passées au laboratoire.
L'intérêt qu'Anouk porte aux autres ne représente qu'une des nombreuses facettes de sa générosité et je dois dire que sans celui qu'elle m'a porté, son soutien, ses conseils et sa gentillesse, j'aurais peut-être abandonné en cours de route.
Je veux la remercier du fond du cœur d'avoir porté mes difficultés avec moi et d'avoir, chaque fois que j'en ai eu besoin, pris le temps de m'écouter.


J'ai également pu compter sur la gentillesse de plusieurs membres du ‘club 13h’ que je tiens à remercier particulièrement pour leur convivialité et leur soutien :
Hélène \textsc{Badouin},
Anna \textsc{Bonnet},
Florian \textsc{Benitière},
Cyril \textsc{Fournier},
Diego \textsc{Hartas\'anchez Frenk},
Thibault \textsc{Latrille},
Alexandre \textsc{Laverré}, Alexia \textsc{Nguyen Trung},
Djivan \textsc{Prentout},
Théo \textsc{Tricou} et
Philippe \textsc{Veber}.




Je voudrais aussi remercier ceux avec qui j'ai eu la chance de partager le bureau dans une bonne humeur quotidienne : Hélène \textsc{Badouin}, Jérémy \textsc{Ganofsky}, Nicolas \textsc{Lartillot}, Thibault \textsc{Latrille}, Michel \textsc{Lecocq} et Aline \textsc{Muyle}.
Merci également à ceux qui m'ont accueillie dans le leur pendant mes dernières semaines de rédaction : Claire \textsc{Gayral}, Alexandre \textsc{Laverré}, Alexia \textsc{Nguyen Trung}, Christine \textsc{Oger}, Philippe \textsc{Veber} et de façon temporaire Marie \textsc{Cariou}.

Je me dois aussi de remercier l'ensemble des doctorants avec qui j'ai partagé mes joies comme mes peines : 
% ceux qui terminaient leur thèse lorsque je commençais la mienne :
% Adr\'ian \textsc{Arellano Dav\'in},
% Wandrille \textsc{Duchemin},
% Cécile \textsc{Fruchard},
Adr\'ian \textsc{Arellano Dav\'in}, 
Samuel \textsc{Barreto},
Guillaume \textsc{Carillo},
Wandrille \textsc{Duchemin},
Cécile \textsc{Fruchard},
Thibault \textsc{Latrille},
Alexandre \textsc{Laverré},
Vincent \textsc{Mérel},
Alexia \textsc{Nguyen Trung},
Djivan \textsc{Prentout},
\'Elise \textsc{Say-Sallaz} et
Théo \textsc{Tricou}.
Merci aussi à ceux avec lesquels j'ai moins interagi mais avec qui il était toujours agréable de discuter : 
Monique \textsc{Aouad},
Magali \textsc{Dancette},
Ghislain \textsc{Durif},
Pierre \textsc{Garcia} et Anne \textsc{Oudart}.
Je fais un clin d'œil particulier à Carine \textsc{Rey} que j'ai mieux connue ces dernières semaines lors de nos soirées communes de rédaction.

Merci aussi aux stagiaires tous plus sympathiques les uns que les autres : Mathieu \textsc{Brevet}, \'Elisa \textsc{Denier}, Anne-Laure \textsc{Fuchs}, Jérémy \textsc{Ganofsky}, Marie \textsc{Guidoni}, Garance \textsc{Lapetoule} et Brice \textsc{Letcher}.

\`A tous ceux qui ont participé à faire de ce laboratoire un lieu privilégié de transmission du savoir au travers des formations passionnantes qu'ils ont animées, merci ! J'ai beaucoup appris gr\^ace à
Bastien \textsc{Boussau}, Marie-Laure \textsc{Delignette-Muller}, Nicolas \textsc{Lartillot}, Fabien \textsc{Subtil} et Philippe \textsc{Veber} lors de la formation en statistiques bayésiennes, Laurent \textsc{Jacob} lors de celle en machine learning et Vincent \textsc{Lanore} et François \textsc{Gindraud} lors de leurs présentations sur C++.



Je ne serais pas allée bien loin dans cette thèse sans l'aide précieuse des membres du pôle informatique : 
Adil \textsc{El-Filali},
Lionel \textsc{Humblot},
Vincent \textsc{Miele},
Simon \textsc{Penel} et
Bruno \textsc{Spataro}.
Je tiens à exprimer en particulier toute ma gratitude à Stéphane \textsc{Delmotte} que j'ai sollicité à une fréquence s'apparentant à du harcèlement et qui a systématiquement trouvé des solutions à tous mes problèmes.

Merci aussi à l'équipe du pôle administratif pour leur efficacité et leur sympathie au quotidien : 
Nathalie \textsc{Arbasetti},
Laetitia \textsc{Mangeot},
Odile \textsc{Mulet-Marquis} et
Aurélie \textsc{Zerfass}.

Finalement, je voudrais remercier les autres membres permanents et contractuels qui créent l'ambiance chaleureuse du laboratoire par leur bienveillance et leur sympathie : 
Céline \textsc{Brochier-Armanet},
Bastien \textsc{Boussau},
Kelly \textsc{Bradley},
Sylvain \textsc{Charlat},
Jonathan \textsc{Corbi},
Vincent \textsc{Daubin},
Jean-Marie \textsc{Delpuech},
Damien \textsc{de Vienne},
Jean-Pierre \textsc{Flandrois},
Amandine \textsc{Fournier}
Guillaume \textsc{Gence},
Manolo \textsc{Gouy},
Dominique \textsc{Guyot},
Laurent \textsc{Guéguen},
Jos \textsc{Käfer},
Daniel \textsc{Kahn},
Bénédicte \textsc{Lafay},
Gabriel \textsc{Marais},
Florian \textsc{Massip},
Dominique \textsc{Mouchiroud},
Guy \textsc{Perrière},
Héloïse \textsc{Philippon},
Franck \textsc{Picard},
Diamantis \textsc{Sellis},
Marie \textsc{Sémon},
\'Eric \textsc{Tannier},
Najwa \textsc{Taib} et
Raquel \textsc{Tavares}.\\




% - les personnes en conf
% Victoire \textsc{Baillet}
% Julie \textsc{Clément}
% Guillaume \textsc{Achaz}
% Catherine \textsc{Breton}\\
%




%%% NOUVELLE PARTIE
Je porte une telle admiration pour le métier d'enseignant que l'un de mes plus grands rêves était de pouvoir l'exercer moi-même un jour.

Je voudrais donc exprimer toute ma reconnaissance à Dominique \textsc{Mouchiroud} de m'avoir permis de le réaliser ainsi que pour la confiance qu'elle m'a accordée dans la réalisation de mes enseignements.
Il a été très agréable de faire ceux-ci en collaboration avec Hélène \textsc{Badouin}, Annabelle \textsc{Haudry}, Héloïse \textsc{Philippon} et Raquel \textsc{Tavares} et je tiens donc à les en remercier.
Merci aussi à mes étudiants de M2 auprès desquels j'ai tant appris et qui m'ont permis de vivre l'une des expériences les plus jouissives et enrichissantes de ma vie.

Beaucoup des enseignants que j'avais moi-même eus par le passé étaient excellents et j'aimerais en remercier tout particulièrement quelques uns dont l'instruction a joué, me semble-t-il, un rôle important dans l'aboutissement de cette thèse : 
merci à M.\ \textsc{Martin} et à Mlle \textsc{Mirmand}, professeurs d'anglais en prépa et au collège, sans la contribution desquels il m'aurait été bien difficile de rédiger mon manuscrit; 
merci à M.\ \textsc{Fontaine}, professeur d'histoire au collège, qui m'a donné goût à l'étude du passé et auquel j'ai beaucoup pensé lors de l'écriture de mon premier chapitre; 
merci à M.\ \textsc{Beaux} et à Mme \textsc{Mollière}, professeurs de biologie en prépa, qui ont su faire de la génétique un sujet passionnant.

Les personnes qui m'ont encadré pendant mes différents stages m'ont largement donné les clés permettant la bonne réalisation de cette thèse et je voudrais donc les en remercier : 
\'Edouard \textsc{Bove},
Claudia \textsc{Chica},
Juliette \textsc{de Meaux},
Florine \textsc{Poiroux},
Loïc \textsc{Rajjou} et
François \textsc{Roudier}.\\

% Loïc Rajjou, en particulier, a eu un impact majeur sur le cours qu'a pris ma vie professionnelle.

% Parmi toutes les personnes que j'ai pu rencontrer, Loïc Rajjou est certainement celui qui a eu le plus d'impact sur le cours de ma vie professionnelle.






%%%% NOUVELLE PARTIE
Je voudrais terminer en remerciant ceux qui m'ont permis de garder une santé mentale à peu près normale en me rappelant que la vie ne s'arrête pas à l'enceinte du laboratoire.

Merci donc à mes amis plongeurs du \textsc{Coolapic} qui sont bien trop nombreux pour être mentionnés individuellement mais qui ont été une vraie bulle d'oxygène lors de nos sorties en mer.
Je voudrais aussi dire toute ma gratitude à mes camarades de thé\^atre, en particulier Vérane \textsc{Lyon}, Marie \textsc{Pinel}, Hannah \textsc{Samama} et Estelle \textsc{Valette}, pour nos éclats de rires ainsi qu'à mes amis parisiens venus passer un ou plusieurs week-ends à mes côtés dans la contrée Lyonnaise : Alicia \textsc{Berret}, Yann \textsc{Boulestreau}, Lucie \textsc{Gomez}, Mathieu \textsc{Laugier}, Ulysse \textsc{Le Goff}, Jordane \textsc{Lelong}, Anne \textsc{Rabault} et Anne \textsc{Schneider}.


Mes parents sont très certainement les deux personnes les plus formidables qu'il m'ait été donné de rencontrer, et je n'ose imaginer la tournure qu'aurait pris ma vie s'ils n'avaient pas été là.
Ils ont toujours placé les intérêts de leurs quatre enfants bien avant les leurs et leur abnégation, leur amour et leur soutien inconditionnel font d'eux des réels modèles de vie.
% et j'espère, un jour, être capable de donner autant qu'eux.

% tout ce dont on pouvait avoir besoin matériellement, intellectuellement et psychologiquement
% On peut difficilement rêver d'avoir de meilleurs parents qu'eux

Mais ce cadre familial n'aurait pas été complet sans mes frères et leurs compagnes, Arthur et Marie-\'Elise, Ronan et Anna, ainsi que ma sœur Lévana que je me dois de remercier pour l'indulgence dont ils ont fait preuve à l'égard de mes moments d'anxiété, pour leur humour et pour la joie qu'ils m'apportent au quotidien.
Merci en particulier à Arthur d'avoir porté mes difficultés sur ses épaules, à Ronan de m'avoir ouvert l'esprit sur d'autres sciences et à Lévana de m'avoir toujours soutenue avec une constance et un dévouement qui forcent l'admiration.

Merci, enfin, à Alexandre \textsc{Chaintreuil} d'avoir su systématiquement transformer mes angoisses en rires, mes larmes en joie et mes craintes en doses de confiance en moi.
Qu'il continue à le faire quelques décennies de plus me comblerait.\\

\hfill \textit{Villeurbanne, le 26 juillet 2019}

% ainsi que pour les innombrables heures qu'il a passées dans le train.
% mes larmes en joie
% mes préoccupations en perspectives
% mes larmes de désespoir
%
% angoisses
% doutes
% peurs
% craintes
%
% rires
%






