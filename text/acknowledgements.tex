Cela fait maintenant un peu plus de vingt-sept ans que, régulièrement, 
je m'étonne de ce que la chance me sourie si souvent et
force est de constater que ce doctorat aura été une occasion de plus de la mesurer.

Pouvoir réaliser ma thèse sous la direction de Laurent \textsc{Duret} a en effet été une aubaine inouïe.
Sa disponibilité manifestement infinie, son souci de me fournir un cadre idéal de travail et de vie, sa résolution à me prodiguer une formation complète et l'extraordinaire bienveillance dont il a fait preuve chaque jour sont allés bien au-delà de ce que j'aurais pu espérer.
\`A travers son inspection minutieuse des détails méthodologiques, son intransigeance à l'égard des imprécisions, l'habitude qu'il a de décortiquer les notions complexes pour les rendre limpides 
et sa détermination à replacer systématiquement nos résultats dans une image plus globale, Laurent m'a littéralement appris à faire de la science.
Mais, au delà de son encadrement scientifique exceptionnel, c'est son humanité que je voudrais saluer.
Je n'arrive pas à m'imaginer comment j'aurais pu arriver au bout de cette thèse sans l'immense générosité dont il a fait preuve ni sans sa capacité à me redonner espoir dans les moments plus difficiles.
Je voudrais donc lui dire ici toute l'admiration que je lui porte ainsi que toute ma gratitude pour les conditions privilégiées dont il m'a fait bénéficier au cours des trois années que j'ai passées à ses côtés.

Je voudrais aussi exprimer mes sincères remerciements aux personnes qui m'ont fait l'honneur d'accepter de lire et d'évaluer ce travail : Valérie \textsc{Borde}, Adam \textsc{Eyre-Walker}, Bertrand \textsc{Llorente}, Dominique \textsc{Mouchiroud} et Gwenael \textsc{Piganeau}~;
ainsi qu'à ceux qui ont participé à mon comité de suivi pour leurs conseils tant sur les aspects scientifiques que personnels : 
Nicolas \textsc{Galtier},
Annabelle \textsc{Haudry},
Tristan \textsc{Lefébure},
Bernard \textsc{de Massy} et
Jonathan \textsc{Romiguier}.


Le travail qui est présenté dans ce manuscrit est le fruit de collaborations qui ont impliqué nombre de personnes envers qui je suis pleinement reconnaissante.
Merci donc à Frédéric \textsc{Baudat}, Valérie \textsc{Borde}, Corinne \textsc{Grey} et Bernard \textsc{de Massy} d'avoir réalisé la totalité du travail expérimental qui a été essentiel pour cette thèse, pour leur patience à mon endroit et pour leurs suggestions éclairées,
à Nicolas \textsc{Lartillot} pour ses commentaires avisés, sa douceur et son indulgence à mon égard, ainsi qu'à Brice \textsc{Letcher} pour le travail admirable qu'il a réalisé pendant son stage et dont je me suis allègrement servie pour rédiger cette thèse.\\



La sympathie et la bienveillance de l'ensemble des membres de l'unité rendent la qualité de vie au laboratoire exceptionnelle.
Ils sont trop nombreux pour être tous remerciés de façon individuelle, mais je voudrais tout de même en mentionner certains.


L'une des personnes qui a le plus compté pour moi pendant ces trois années est Anamaria \textsc{Nec\c{s}ulea}.
L'intérêt qu'Anouk porte aux autres ne représente qu'une des nombreuses facettes de sa générosité et je dois dire que sans celui qu'elle m'a porté, son soutien, ses conseils et sa gentillesse, j'aurais peut-être abandonné en cours de route.
Je veux la remercier du fond du cœur d'avoir porté mes difficultés avec moi et d'avoir, chaque fois que j'en ai eu besoin, pris le temps de m'écouter.


J'ai également pu compter sur l'amitié de plusieurs membres du ‘club 13h’ que je tiens à remercier particulièrement pour leur convivialité et leur soutien :
Hélène \textsc{Badouin},
Anna \textsc{Bonnet},
Florian \textsc{Benitière},
Cyril \textsc{Fournier},
Diego \textsc{Hartas\'anchez Frenk},
Thibault \textsc{Latrille},
Alexandre \textsc{Laverré}, 
Anouk \textsc{Nec\c{s}ulea},
Alexia \textsc{Nguyen Trung},
Djivan \textsc{Prentout},
Théo \textsc{Tricou} et
Philippe \textsc{Veber}.
Je voudrais notamment dire 
à Hélène ma sincère admiration pour sa droiture d'esprit et sa franchise éminentes ; 
à Alexia la sérénité que sa douceur m'a procurée~;
à Florian, Cyril, Alexandre et Djivan le bonheur que leur simple compagnie m'a donné ;
à Théo et Philippe l'apaisement que leur apparente insouciance et leurs plaisanteries m'ont apporté ; 
à Diego combien son entrain et ses sourires m'ont été salutaires ;
à Anna combien son enthousiasme m'a manqué quand elle a quitté le laboratoire ;
et à Thibault à quel point sa fulgurance d'esprit m'a impressionnée et combien son éternel optimisme et son altruisme ont embelli mes journées.


Je tiens ensuite à remercier ceux avec qui j'ai eu la chance de partager le bureau dans une bonne humeur quotidienne : Hélène \textsc{Badouin}, Jérémy \textsc{Ganofsky}, Nicolas \textsc{Lartillot}, Thibault \textsc{Latrille}, Michel \textsc{Lecocq} et Aline \textsc{Muyle}.
Merci également à ceux qui m'ont accueillie dans le leur pendant mes dernières semaines de rédaction : Claire \textsc{Gayral}, Alexandre \textsc{Laverré}, Alexia \textsc{Nguyen Trung}, Christine \textsc{Oger}, Philippe \textsc{Veber} et de façon temporaire Marie \textsc{Cariou}.

De même, je rends gr\^ace aux doctorants avec qui j'ai partagé mes joies comme mes peines :
Adr\'ian \textsc{Arellano Dav\'in}, 
Samuel \textsc{Barreto},
Guillaume \textsc{Carillo},
Wandrille \textsc{Duchemin},
Cécile \textsc{Fruchard},
Thibault \textsc{Latrille},
Alexandre \textsc{Laverré},
Vincent \textsc{Mérel},
Alexia \textsc{Nguyen Trung},
Djivan \textsc{Prentout},
\'Elise \textsc{Say-Sallaz} et
Théo \textsc{Tricou}.
Merci aussi à ceux avec lesquels j'ai moins interagi mais avec qui il était toujours agréable de discuter : 
Monique \textsc{Aouad},
Magali \textsc{Dancette},
Ghislain \textsc{Durif},
Pierre \textsc{Garcia} et Anne \textsc{Oudart}.
Je fais un clin d'œil particulier à Carine \textsc{Rey} que j'ai mieux connue ces dernières semaines lors de nos soirées communes de rédaction.
Merci, enfin, aux stagiaires tous plus sympathiques les uns que les autres : Mathieu \textsc{Brevet}, \'Elisa \textsc{Denier}, Anne-Laure \textsc{Fuchs}, Jérémy \textsc{Ganofsky}, Marie \textsc{Guidoni}, Garance \textsc{Lapetoule} et Brice \textsc{Letcher}.

\`A tous ceux qui ont participé à faire de ce laboratoire un lieu privilégié de transmission du savoir au travers des formations passionnantes qu'ils ont animées, merci ! J'ai beaucoup appris gr\^ace à
Bastien \textsc{Boussau}, Marie-Laure \textsc{Delignette-Muller}, Nicolas \textsc{Lartillot}, Fabien \textsc{Subtil} et Philippe \textsc{Veber} lors de la formation en statistiques bayésiennes, gr\^ace à Laurent \textsc{Jacob} lors de celle en machine learning et gr\^ace à Vincent \textsc{Lanore} et François \textsc{Gindraud} lors de leurs présentations sur C++.



Je ne serais pas allée bien loin dans cette thèse sans l'aide précieuse des membres du pôle informatique : 
Adil \textsc{El-Filali},
Lionel \textsc{Humblot},
Vincent \textsc{Miele},
Simon \textsc{Penel} et
Bruno \textsc{Spataro}.
Je tiens plus spécialement à exprimer ma gratitude à Stéphane \textsc{Delmotte} que j'ai sollicité à une fréquence s'apparentant à du harcèlement et qui a systématiquement trouvé des solutions à tous mes problèmes.

Merci aussi à l'équipe du pôle administratif pour leur efficacité : 
Nathalie \textsc{Arbasetti},
Laetitia \textsc{Mangeot},
Odile \textsc{Mulet-Marquis} et
Aurélie \textsc{Zerfass}.

Finalement, je voudrais remercier les autres membres permanents et contractuels qui créent l'ambiance chaleureuse du laboratoire : 
Céline \textsc{Brochier-Armanet},
Kelly \textsc{Bradley},
Sylvain \textsc{Charlat},
Jonathan \textsc{Corbi},
Vincent \textsc{Daubin},
Jean-Marie \textsc{Delpuech},
Damien \textsc{de Vienne},
Jean-Pierre \textsc{Flandrois},
Amandine \textsc{Fournier}
Guillaume \textsc{Gence},
Manolo \textsc{Gouy},
Dominique \textsc{Guyot},
Laurent \textsc{Guéguen},
Jos \textsc{Käfer},
Daniel \textsc{Kahn},
Bénédicte \textsc{Lafay},
Gabriel \textsc{Marais},
Florian \textsc{Massip},
Dominique \textsc{Mouchiroud},
Guy \textsc{Perrière},
Héloïse \textsc{Philippon},
Franck \textsc{Picard},
Diamantis \textsc{Sellis},
Marie \textsc{Sémon},
\'Eric \textsc{Tannier},
Najwa \textsc{Taib} et
Raquel \textsc{Tavares}.\\






%%% NOUVELLE PARTIE
Je porte une telle admiration pour le métier d'enseignant que l'un de mes plus grands rêves était de pouvoir l'exercer moi-même un jour.

Je voudrais donc exprimer toute ma reconnaissance à Dominique \textsc{Mouchiroud} de m'avoir permis de l'accomplir ainsi que pour la confiance qu'elle m'a accordée dans la réalisation de mes enseignements.
Il a été très agréable de faire ceux-ci en collaboration avec Hélène \textsc{Badouin}, Annabelle \textsc{Haudry}, Héloïse \textsc{Philippon} et Raquel \textsc{Tavares} et je tiens donc à les en remercier.
Merci aussi à mes étudiants de M2 auprès desquels j'ai tant appris et qui m'ont permis de vivre l'une des expériences les plus jouissives et enrichissantes de ma vie.

Beaucoup des enseignants que j'avais moi-même eus par le passé étaient excellents et j'aimerais en remercier tout particulièrement quelques uns dont l'instruction a joué — me semble-t-il — un rôle important dans l'aboutissement de cette thèse~: 
merci à M.\ \textsc{Martin} et à Mlle \textsc{Mirmand}, professeurs d'anglais en prépa et au collège, sans la contribution desquels il m'aurait été bien difficile de rédiger mon manuscrit ; 
merci à M.\ \textsc{Fontaine}, professeur d'histoire au collège, qui m'a donné goût à l'étude du passé et auquel j'ai beaucoup pensé lors de l'écriture de mon premier chapitre ; 
merci à M.\ \textsc{Beaux} et à Mme \textsc{Mollière}, professeurs de biologie en prépa, qui ont su faire de la génétique un sujet passionnant.

Finalement, pour m'avoir donné les clés nécessaires à la bonne réalisation de cette thèse, merci à mes encadrants de stage :
\'Edouard \textsc{Bove},
Claudia \textsc{Chica},
Juliette \textsc{de Meaux},
Florine \textsc{Poiroux},
Loïc \textsc{Rajjou} et
François \textsc{Roudier}.
Je tiens particulièrement à signifier à Loïc la profonde estime que j'ai pour lui et mon indéfectible reconnaissance
pour m'avoir soutenue quand je me croyais abbatue, aiguillée quand je me sentais perdue et 
encouragée à entreprendre un doctorat quand je ne m'en savais pas capable.
\\







%%%% NOUVELLE PARTIE
Je voudrais terminer en remerciant ceux qui m'ont permis de garder une santé mentale à peu près normale en me rappelant que la vie ne s'arrête pas à l'enceinte du laboratoire.

Merci donc à mes amis plongeurs du \textsc{Coolapic} qui sont bien trop nombreux pour être mentionnés individuellement mais qui ont été une vraie bulle d'oxygène lors de nos sorties en mer.
Je voudrais aussi dire toute ma gratitude à mes camarades de thé\^atre, en particulier Vérane \textsc{Lyon}, Marie \textsc{Pinel}, Hannah \textsc{Samama} et Estelle \textsc{Valette}, pour nos éclats de rires ainsi qu'à mes amis parisiens venus passer un ou plusieurs week-ends à mes côtés dans la contrée Lyonnaise : Alicia \textsc{Berret}, Yann \textsc{Boulestreau}, Lucie \textsc{Gomez}, Mathieu \textsc{Laugier}, Ulysse \textsc{Le Goff}, Jordane \textsc{Lelong}, Anne \textsc{Rabault} et Anne \textsc{Schneider}.


Je n'ose imaginer la tournure qu'aurait pris ma vie si mes parents n'avaient pas été tels qu'ils sont.
Ils ont toujours placé les intérêts de leurs quatre enfants bien avant les leurs et leur abnégation, leur amour et leur soutien inconditionnel font d'eux des réels modèles de vie.
Merci donc, Papa, Maman : c'est un cadeau de vous avoir comme parents !
Mais ce cadre familial n'aurait pas été complet sans mes frères et leurs compagnes, Arthur et Marie-\'Elise, Ronan et Anna, ainsi que ma sœur Lévana que je me dois de remercier pour l'indulgence dont ils ont fait preuve à l'égard de mes moments d'anxiété, pour leur humour et pour la gaieté qu'ils apportent à mon quotidien.
Merci en particulier à Arthur de m'avoir consolée et d'avoir soulagé mes difficultés en les prenant sur ses épaules, à Ronan dont l'espièglerie m'a allégé l'esprit et dont la curiosité m'a ouvert les yeux sur d'autres sciences et à Lévana de m'avoir toujours réconfortée et soutenue avec une constance et un dévouement qui forcent l'admiration.

Merci, enfin, à Alexandre \textsc{Chaintreuil} d'avoir su systématiquement transformer mes angoisses en rires, mes larmes en joie et mes craintes en doses de confiance en moi.
Qu'il continue à le faire quelques décennies de plus me comblerait.\\

\hfill \textit{Villeurbanne, le 26 juillet 2019}


