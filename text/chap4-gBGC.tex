\begin{savequote}[8cm]
	% ‘[…] the process of mutation itself is not adaptive. On the contrary, the mutants which arise are, with rare exceptions, deleterious to their carriers, at least in the environments which the species normally encounters. Some of them are deleterious apparently in all environments. Therefore, the mutation process alone, not corrected and guided by natural selection, would result in degeneration and extinction rather than in improved adaptedness.’
	% ‘[…] the mutants which arise are, with rare exceptions, deleterious to their carriers, at least in the environments which the species normally encounters.’
	% — Theodosius Dobzhansky
	% 'On Methods of Evolutionary Biology and Anthropology', American Scientist, 1957, 45, 385.


	%%%%% KIMURA
	% ‘[…] if the neutral or nearly neutral mutation is being produced in each generation at a much higher rate than has been considered before, then we must recognize the great importance of random genetic drift due to finite population number in forming the genetic structure of biological populations.’
	‘Finally, if my chief conclusion is correct, and if the neutral or nearly neutral mutation is being produced in each generation at a much higher rate than has been considered before, then we must recognize the great importance of random genetic drift due to finite population number in forming the genetic structure of biological populations.’
	\qauthor{--- Motoo Kimura, \textit{\usebibentry{kimura1968evolutionary}{title}} \citeyearpar{kimura1968evolutionary} }
	
	% ‘Of course, Darwinian change is necessary to explain change at the phenotypic level — fish becoming man — but in terms of molecules, the vast majority of them are not like that.’
	%
	% ‘This leads us to an important principle for the neutral theory stating that “the neutral mutants” are not the limit of selectively advantageous mutants but the limit of deleterious mutants when the effect of mutation on fitness becomes indefinitely small. This means that mutational pressure causes evolutionary change whenever the negative-selection barrier is lifted.’
	% ‘[…] “the neutral mutants” are not the limit of selectively advantageous mutants but the limit of deleterious mutants when the effect of mutation on fitness becomes indefinitely small. This means that mutational pressure causes evolutionary change whenever the negative-selection barrier is lifted.’
	% ‘[…] “the neutral mutants” are not the limit of selectively advantageous mutants but the limit of deleterious mutants when the effect of mutation on fitness becomes indefinitely small.’
	% \qauthor{--- Motoo Kimura, \textit{\usebibentry{kimura1983neutral}{title}} \citeyearpar{kimura1983neutral} }

	% FROM
	% https://books.google.fr/books?hl=fr&lr=&id=olIoSumPevYC&oi=fnd&pg=PR9&dq=%22The+neutral+theory+of+molecular+evolution%22+by+Motoo+Kimura&ots=P0R5memrff&sig=5w8VhVKBKCo2IxJIhcSPc-36vnA#v=onepage&q=deleterious&f=false
	% — avec recherche du mot "deleterious" (page 113)


	% DEPUIS L'AUTRE
	% ‘Unlike the Darwinian theory of evolution by natural selection, the neutral theory claims that the overwhelming majority of evolutionary changes at the molecular level are not caused by Darwinian natural selection acting on advantageous mutants, but by random fixation of selectively neutral or very nearly neutral mutants through the cumulative effect of sampling drift (due to finite population number) under continued input of new mutations.’
	% \qauthor{--- Motoo Kimura, \textit{\usebibentry{kimura1991neutral}{title}} \citeyearpar{kimura1991neutral} }

\end{savequote}

\chapter{\label{ch:4-gBGC}GC-biased gene conversion, the designer of genomic lanscapes} 
%\otherpagedecoration

\minitoc{}


% DES TITRES POSSIBLES
% The evolution of genomic GC-content through biased gene conversion
% GC-biased gene conversion and the evolution of genomic landscapes (de Duret et Galtier)
% GC-biased gene conversion, the designer of genomic lanscapes
%
% The Role of GC-Biased Gene Conversion in Shaping the Fastest Evolving Regions of the Human Genome
% The role of GC-biased gene conversion in shaping genomic GC-content
% How GC-biased gene conversion shapes the genomic GC-content
%
%
% Biased gene conversion and GC-content evolution
% The evolution of genomic base composition through GC-biased gene conversion
% GC-biased gene conversion and the evolution of base composition
% Gene conversion and GC-content evolution in mammals
% GC-content evolution and biased gene conversion
% GC-biased gene conversion and the evolution of genomic content
% Biased gene conversion and GC-content evolution
% The genomic landscape of GC-content and biased gene conversion



%




% Intro chap4: la recombinaison est mutagenique, et en particulier deamination des cytsines. Comment se fait-il donc que pas d'enrichissment en AT? Cela explique par l'existence du gBGC. 1. MMR BER etc (i.e. comment fonctionne) et isochores. 2. Ressemble à la selection naturelle. 3. Preuves directes et indirects
% CHAPITRE 4:
% BGC suppose pour contrecarrer le mutational load (dans intro?)
%
%
% VOIR SI RAJOUTER PLUS HAUT DANS MUTAGENIC (de capilla)
% LIEN AVEC GENET DIVERSITY
% Genetic diversity has been considered a good predictor of recombination rates; that is, levels of DNA sequence variation are normally reduced in genomic regions with low recombination rates [Begun and Aquadro, 1992; Nachman, 2001; Stevison et al., 2016].
% Four different causes have been considered to explain this correlation [Begun and Aquadro, 1992; Aquadro, 1997]: (1) the mutagenic effect of the recombination process itself, (2) functional constraints, (3) adaptive evolution (such as selective sweeps — the reduction of nucleotide variation due to positive selection), and (4) background selection (i.e., loss of genetic diversity at a non-deleterious locus due to negative selection).
% Among all of these possibilities, the correlation between recombination and genetic divergence (scored as the ratio of rates of substitution at non-synonymous and synonymous nucleotide sites, dN/dS) is, however, more controversial, given that conflicting results have been reported in different organisms [for a review, see Smukowski and Noor, 2011]. According to the Hill-Robertson effect [Hill and Robertson, 1966], natural selection can be less effective in regions of low recombination rates, affecting, as a result, rates of adaptation. Whereas this correlation has been detected neither in Drosophila nor in mice [Begun and Aquadro, 1992], contrasting results have been obtained in great apes [Nachman, 2001; Bussell et al., 2006; Stevison et al., 2016].
%





%% DONNER L'INFORMATION SUR LES CO COMPLEXES (deja un peu dit a la toute fin du chaptire 2)

% Dans chapitre 2 (ou 4?) — les evenements complexes identifies chez la levure
% and exchanges are not always simple, some exchange events are complex, containing both crossover and gene conversion events, interrupted by unconverted mark- ers (Borts and Haber 1989)

% Notons, de plus, que le tract est dit "simple" si, pour un même évènement de recombinaison (CO ou NCO) tous les mésappariements des hétéroduplexes sont convertis dans le même sens (i.e. pas d’alternance conversion / réparation). Dans le cas contraire on parle de tract de conversion "complexe" [Mancera et al., 2008].

% De plus, chez Saccharomyces cerevisiae , environ 11% des CO sont associés à des tracts complexes contre seulement 3,4% des NCO [Mancera et al., 2008]. Ceci est en accord avec le fait que le SDSA, principale voie de formation des NCO, ne donne que rare- ment des tracts complexes. On ignore encore l’origine des tracts complexes. Comme évoqué plus haut, ils pourraient venir d’une réparation des hétérodu- plexes par patches, alternant entre conversion et restauration. Ils pourraient aussi provenir de la résolution de dHj complexes comportant plusieurs hété- roduplexes comme suggéré par [Mancera et al., 2008].






%% INFO MISMATCH REPAIR DANS CHAPITRE 4
% % POUR APRES SI SAUTE:
% MISMATCH REPAIR
% https://www.ncbi.nlm.nih.gov/pmc/articles/PMC86394/
% https://academic-oup-com.inee.bib.cnrs.fr/hmg/article/11/15/1697/636769
% https://cshperspectives-cshlp-org.inee.bib.cnrs.fr/content/7/3/a022657
% https://www-ncbi-nlm-nih-gov.inee.bib.cnrs.fr/pmc/articles/PMC4856981/pdf/gkw059.pdf
%
% DNA mismatch repair in mammals: role in disease and meiosis
% Norman Arnheim* and Darryl Shibatat
%
% http://content.ebscohost.com.inee.bib.cnrs.fr/ContentServer.asp?EbscoContent=dGJyMMTo50SeprM4v%2BbwOLCmr1Gep7NSsKu4TLWWxWXS&ContentCustomer=dGJyMOzpsE21p7JOuePfgeyx9Yvf5ucA&T=P&P=AN&S=R&D=a9h&K=2800408
% Andrew B. Buermeyer1, Suzanne M. Deschenes ˆ 1,Sean M. Baker2, and R. Michael Liskay
% MAMMALIAN DNA MISMATCH REPAIR
%
% Voir Hinch 2019 pour des references en biblio
%
% SAIS PLUS POURQUOI
% https://onlinelibrary.wiley.com/doi/epdf/10.1111/j.1558-5646.1981.tb04864.x
% https://www-sciencedirect-com.inee.bib.cnrs.fr/science/article/pii/S0169534715003286#bib0065
% https://www-annualreviews-org.inee.bib.cnrs.fr/doi/pdf/10.1146/annurev.genet.41.110306.130301





% %% DEPUIS LES Notes_for_cahp4.txt (deleted then)
% Sur le BGC: parler de la restauration/conservation des alleles.
% Hastings and colleagues in S. cerevisiae (46, 125) and in A. immersus(45)totestthehypothesisthatmismatchrepair cancorrecteithertowardstheinformationontheinvading strand(togiveaconversionevent)ortowardstheinforma- tionontherecipientstrand(togivearestorationevent).
% (depuis Orr-weaver and Szostak 1985, fungal recombination).
%
% BGC against load: voir Bengtsson 1985
% Biased conversion as the primary function of recombination
%
% MMR avec role dans BGC
% It is hard to find a single mechanism that is consistent
% with all of the genetic and physical data obtained in
% fungi. In S. cerevisiae, at the ARG4 locus, there is a
% double-strand break located near the high end of the
% conversion gradient 5~ and there is a good correlation
% between the level of gene conversion and the amount of
% single-strand excision 51. Although these results suggest
% that the polarity gradient of ARG4 might reflect only
% heteroduplex formation, E. Alani, R. Reenan and R.
% Kolodner (pers. commun.) have found that the steepness of the polarity gradient at ARG4 is substantially
% reduced by a mutation affecting mismatch repair. At the
% HIS4 locus, the conversion gradient for high PMS alleles is much less steep than that observed for low PMS
% alleles, and the gradient is nearly eliminated by one of
% the mutations affecting mismatch repair ~a. These results
% indicate that mismatch repair strongly influences the
% shape of the conversion gradient at this locus.
% %%%% de conclu de Nicolas and Pete Polarity of meiotic gene conversion in fungi: contrasting views (1994)
%




%%%% NOTES CHAPITRE PRECEDENT


% CHAPITRE 4
% chapitre 4: BGC comme conseqeucne de la recombinaison
% Isochores
% (+ BGC comme sel nat)
% (+ preuves directes et indirectes du BGC)



%%% DANS GENE CONVERSION (CHAPITRE 4)
% Parler de gene conversion + conversion/restauration
% Parler de tract
% Parler de heteroduplex
% MMR et BER


%%%% CHAPITRE 4
% %% QUAND PARLERAI DU MISMATCH REPAIR
% % https://en.wikipedia.org/wiki/Holliday_junction
% Robin Holliday proposed the junction structure that now bears his name as part of his model of homologous recombination in 1964, based on his research on the organisms Ustilago maydis and Saccharomyces cerevisiae. The model provided a molecular mechanism that explained both gene conversion and chromosomal crossover. Holliday realized that the proposed pathway would create heteroduplex DNA segments with base mismatches between different versions of a single gene. He predicted that the cell would have a mechanism for mismatch repair, which was later discovered.[3] Prior to Holliday's model, the accepted model involved a copy-choice mechanism[26] where the new strand is synthesized directly from parts of the different parent strands.[27]
% 3.  Liu Y, West S (2004). "Happy Hollidays: 40th anniversary of the Holliday junction". Nature Reviews Molecular Cell Biology. 5 (11): 937–44. doi:10.1038/nrm1502. PMID 15520813.
% 27.  Advances in genetics. Academic Press. 1971. ISBN 9780080568027.
%


%% CHAPITRE 4
%% GENE CONVERSION DES NCO (SDSA)
% Wikipedia https://en.wikipedia.org/wiki/Synthesis-dependent_strand_annealing
%SDSA is unique in that D-loop translocation results in conservative rather than semiconservative replication, as the first extended strand is displaced from its template strand, leaving the homologous duplex intact. Therefore, although SDSA produces non-crossover products because flanking markers of heteroduplex DNA are not exchanged, gene conversion does occur, wherein nonreciprocal genetic transfer takes place between two homologous sequences.[10]

% CHAPITRE 4
% % https://books.google.fr/books?id=7V0N6Tt8fUwC&pg=PA43&lpg=PA43&dq=murray+1960+polarity&source=bl&ots=mtj-qfJ1ZM&sig=ACfU3U1rKTqzCqEtcJkNw4ex96F_KPI87Q&hl=fr&sa=X&ved=2ahUKEwiG0b39-tfgAhUJ0RoKHRn4CWsQ6AEwB3oECAkQAQ#v=onepage&q=murray%201960%20polarity&f=false
% Sur la polarité des gene conversion DONC des sites precis ou la recombinaison demarre (a mettre dans les points chauds de recombinaison).
%
% N. Saitou, Introduction to Evolutionary Genomics, Computational Biology 17,
% % DOI 10.1007/978-1-4471-5304-7_2, © Springer-Verlag London 2013
% file:///Users/maudgautier/Downloads/9781447153030-c2.pdf
%
% %% GENE CONVERSION (pargraphe de Whitehouse ou Saitou ou autre??)
% Early studies on gene conversion were mostly restricted to fungal genetics. As
% molecular evolutionary studies of multigene family started, unexpected similarity
% of tandemly arrayed rRNA genes was found [ 15  ]. This phenomenon was termed
% ‘concerted evolution,’ and gene conversion or unequal crossing-over was proposed
% to explain this characteristic of some multigene families (e.g., [ 16  ]). New statistical
% methods were developed to detect gene conversion between homologous
% sequences [ 17, 18  ]. Program GENECONV developed by Sawyer [ 19  ] became the
% standard tool for analyzing gene conversions. We now know that conversion can
% occur in any genomic region irrespective of genes (DNA regions having function)
% or nongenic regions (e.g., [ 20  ]). However, ‘gene conversion’ as technical jargon is
% currently widely accepted, and I follow this nomenclature. Gene conversion can be
% classifi ed into two types: intragenic or between alleles and intergenic or between
% duplicated genes.
%
%
% mismatch repair
% https://fr.wikipedia.org/wiki/Mismatch_repair


% Question à moi-même (pour Laurent): si DSB, pourquoi la partie cassée du chromosome ne part pas ailleurs dans le cytoplasme?



% Biblio souris - meiose
% O’Bryan, M. K. & Kretser, D. Mouse models for genes involved in impaired spermatogenesis. Int. J. Androl. 29, 76–89 (2006).


%%% CHAPITRE 4
%% DANS LES ISOCHORES: 
%The genomes of many eukaryotes, including Saccharomyces cerevisiae, are mosaics of regions with high- and low-GC base composition.
%The Isochores as a Fundamental Level of Genome Structure and Organization: A General Overview. Costantini M, Musto H J Mol Evol. 2017 Mar; 84(2-3):93-103.

%DANS BGC: des gens qui disent que BGC pourrait etre une adaptation pour repondre au mutational load de la mutation (dans recombi). Eg: Crossovers are associated with mutation and biased gene conversion at recombination hotspots (Barbara Arbeithuber, Andrea J. Betancourt, Thomas Ebner, and Irene Tiemann-Boege 2015)
% Crossovers are associated with mutation and biased gene conversion at recombination hotspots (Barbara Arbeithuber, Andrea J. Betancourt, Thomas Ebner, and Irene Tiemann-Boege 2015) — ce papier important poir toutes les references sur BGC au debut.

%%% CHAPITRE 4: et notamment, distinguer BGC de mutation
% Dit dans COOP PRZEWORSKi: These observations raise the possibility that recombination might be mutagenic, as reported for mitotic recombination in yeast24. However, a recent analysis suggests that the broad-scale associations are not causal, but instead arise from covariates such as GC content25
% 25. Biased gene conversion plutot que mutagenique. Spencer, C. et al. The influence of recombination on human genetic diversity. PLoS Genet. 2, 1375–1385 (2006).
% A careful examination of the associations between diversity, divergence, genomic features and finescale recombination (inferred from LD patterns) on chromosome 20. The authors find evidence for biased gene conversion in recombination hotspots







% % SISTER
% Hinch: Nevertheless, it appears that some fraction of programmed meiotic DSBs are repaired using the sister chromatid [Hyppa and Smith, 2010].
% sex chromosomes: no homology donc repaired via sister chromatids at late prophase. Silenced (cf Altemose)
%
% % SYNCHRONISATION CELL CYCLE
% Meiotic success also hinges on theability to synchronize the meiotic transcriptional programwith cell cycle progression and cell growth. This isachieved in yeast by coupling double strand break for-mation with progression of the replication fork
% % Borde V, Goldman AS, Lichten M:Direct coupling betweenmeiotic DNA replication and recombination initiation.Science2000,290:806-809.
%
% % REGULATION
% This step regulated: Pachytene checkpoint: avoid defects (Handel Schimenti)
% If error, meiotic silencing (https://en.wikipedia.org/wiki/Synapsis)
%
% % DIFF MEIOSE MITOSE
% attachement des chromosomes par les kinetochores differe de la meiose (https://cshperspectives.cshlp.org/content/7/5/a015859.long)
% Autre difference avec la mitose: le spindle qui peut etre asymetrique.




%%% AUTRES CHAPITRES


%%  A GARDER POUR CHAPITRE 5 ET LES BIAIS DE LA METHODE SI ERREUR DE GENOTYPAGE
% Artefacts for sperm-typing (by Venn??)
% Cross-over results in the transmission of mosaic parental haplotypes to the next gen- eration. Consequently, we can view sibling genomes as independent samples of the parental haplotypes and the action of recombination on those haplotypes. Hence, to detect cross-over in pedigrees the task is to infer the inheritance pattern underlying the genotypes observed in family members (c.f., Thompson, 2000). Note that arte- fact (genotyping errors) and biological process (de novo mutation, which are worthy of their own study) can mimic the action of recombination at local-scales.









%% SOME POSSIBLE QUOTES
% CHAP 5
% “Truth has nothing to do with the conclusion, and everything to do with the methodology.”
% ― Stefan Molyneux

% “As to methods there may be a million and then some, but principles are few. The man who grasps principles can successfully select his own methods. The man who tries methods, ignoring principles, is sure to have trouble.”
% ― Harrington Emerson

% CHAP 
% If the facts do not fit the theory, change the facts! — Einstein.
% “Everything must be taken into account. If the fact will not fit the theory---let the theory go.” 
% ― Agatha Christie, The Mysterious Affair at Styles


% CHAP ABC (6)
% “So my antagonist said, "Is it impossible that there are flying saucers? Can you prove that it's impossible?" "No", I said, "I can't prove it's impossible. It's just very unlikely". At that he said, "You are very unscientific. If you can't prove it impossible then how can you say that it's unlikely?" But that is the way that is scientific. It is scientific only to say what is more likely and what less likely, and not to be proving all the time the possible and impossible.”
% ― Richard P. Feynman

% CHAP 7 (theorie de BGC qui evolue)
% “The whole [scientific] process resembles biological evolution. A problem is like an ecological niche, and a theory is like a gene or a species which is being tested for viability in that niche.”
% ― David Deutsch, The Fabric of Reality: The Science of Parallel Universes--and Its Implications







%%%% LISTE DE QUOTES UTILISEES COMME TEST POUR CHAPITRE 4

% Motoo Kimura
% "This neutral theory claims that the overwhelming majority of evolutionary changes at the molecular level are not caused by selection acting on advantageous mutants, but by random fixation of selectively neutral or very nearly neutral mutants through the cumulative effect of sampling drift (due to finite population number) under continued input of new mutations" (Kimura, 1991)

% Servira pour la partie 3 du chaptire 1
% KIMURA
 % "Looking back, I think that it is a curious human nature, that if a certain doctrine is constantly being spoken of favorably by the majority, endorsed by top authorities in their books and taught in classes, then a belief is gradually built up in one's mind, eventually becoming the guiding principle and the basis of value judgement. At any rate, this was the time when the panselectionist or 'neo-Darwinian' position was most secure in the history of biology: the heyday of the traditional 'synthetic theory' of evolution."
 % http://wasdarwinwrong.com/kortho37.htm


% % QUOTES POSSIBLES
% RONALD FISHER
% I believe that no one who is familiar, either with mathematical advances in other fields, or with the range of special biological conditions to be considered, would ever conceive that everything could be summed up in a single mathematical formula, however complex.
% The evolutionary modification of genetic phenomena. Proceedings of the 6th International Congress of Genetics 1, 165-72, 1932.
%
%
% % CHAP 7?? Fisher
% In relation to any experiment we may speak of this hypothesis as the “null hypothesis,” and it should be noted that the null hypothesis is never proved or established, but is possibly disproved, in the course of experimentation. Every experiment may be said to exist only in order to give the facts a chance of disproving the null hypothesis.
% The Design of Experiments, Edinburgh: Oliver and Boyd, 1935, p. 18
%
% % Stephen Jay gould
% Results rarely specify their causes unambiguously. If we have no direct evidence of fossils or human chronicles, if we are forced to infer a process only from its modern results, then we are usually stymied or reduced to speculation about probabilities. For many roads lead to almost any Rome.
% "Senseless Signs of History", p. 34
%
% % Pour chapitre 3 si je change
% But all evolutionary biologists know that variation itself is nature's only irreducible essence. Variation is the hard reality, not a set of imperfect measures for a central tendency. Means and medians are the abstractions.
%
%
% Just as in a book misprints are more likely to produce nonsense than better sense, so mutations will almost always be deleterious, almost always, in fact, they will kill the organism or the cell, often at so early a stage in its existence that we do not even realize it ever came into being at all. {John C. Kendrew,the Cambridge scientist who is a Nobel laureate for his discovery of the structure of the protein myoglobin, THE THREAD OF LIFE, 1966, pp.106-107}
%
% It is a considerable strain on one's credulity to assume that finely balanced systems such as certain sense organs (the eye of vertebrates, or the bird's feather) could be improved by random mutations. This is even more true of some ecological chain relationships. However, objectors to random mutations have so far been unable to advance any alternative explanation that was supported by substantial evidence. {Harvard biologist Ernst Mayr, SYSTEMATICS & THE ORIGIN OF SPECIES, 1942, p.296}
%
%
%
% Mutations and chromosomal changes arise in every sufficiently studied organism with a certain finite frequency, and thus constantly and unremittingly supply the raw materials for evolution. But evolution involves something more than origin of mutations. Mutations and chromosomal changes are only the first stage, or level, of the evolutionary process, governed entirely by the laws of the physiology of individuals. Once produced, mutations are injected in the genetic composition of the population, where their further fate is determined by the dynamic regularities of the physiology of populations. A mutation may be lost or increased in frequency in generations immediately following its origin, and this (in the case of recessive mutations) without regard to the beneficial or deleterious effects of the mutation. The influences of selection, migration, and geographical isolation then mold the genetic structure of populations into new shapes, in conformity with the secular environment and the ecology, especially the breeding habits, of the species. This is the second level of the evolutionary process, on which the impact of the environment produces historical changes in the living population.
% — Theodosius Dobzhansky
% Genetics and Origin of Species (1937), 13.
%
%
% Mutation is random; natural selection is the very opposite of random
% — Richard Dawkins
% The Blind Watchmaker (1996), 41
%
%
%
%
% The process of mutation is the only known source of the raw materials of genetic variability, and hence of evolution. It is subject to experimental study, and considerable progress has been accomplished in this study in recent years. An apparent paradox has been disclosed. Although the living matter becomes adapted to its environment through formation of superior genetic patterns from mutational components, the process of mutation itself is not adaptive. On the contrary, the mutants which arise are, with rare exceptions, deleterious to their carriers, at least in the environments which the species normally encounters. Some of them are deleterious apparently in all environments. Therefore, the mutation process alone, not corrected and guided by natural selection, would result in degeneration and extinction rather than in improved adaptedness.
% — Theodosius Dobzhansky
% % 'On Methods of Evolutionary Biology and Anthropology', American Scientist, 1957, 45, 385.
%
%
%http://wasdarwinwrong.com/kortho37.htm#Notes
%https://www.nature.com/scitable/topicpage/neutral-theory-the-null-hypothesis-of-molecular-839

% Citation de Dawkins a regarder

