\begin{savequote}[8cm]

	‘The supreme maxim in scientific philosophizing is this: wherever possible, logical constructions are to be substituted for inferred entities.’
	
	\qauthor{--- Bertrand Russell, \textit{\usebibentry{russell1914relation}{title}} \citeyearpar{russell1914relation} }
	
\end{savequote}

\chapter{\label{ch:10-philosophising}A little bit of scientific philosophising}
%\otherpagedecoration


\minitoc{}

% - Russell a evidemment contribue en math: theorie des ensemble et principes mathematique ou il redemontre tout
% - mais aussi important en philosophie des sciences: meme logique: il veut reprendre les elements de base, et “logically construct” a partir des relations entre evenements \citep{braithwaite1970bertrand}
% - preached that logic was the essence of philosophy

% dans le cas des forces evolutives, precisement, ce concept n'est pas quelque chose de tangible, qu'on peut percevoir physiquement, ressentir.
% C'est un concept dont on peut mesurer les effets



% https://archive.org/details/problemsofphilo00russuoft/page/n11

\begin{quote}
	\textit{‘Is there any knowledge in the world which is so certain that no reasonable man could doubt it?}
\end{quote}

This question, which was the first sentence of the book \textit{\usebibentry{russell1912problems}{title}} \citeyearpar{russell1912problems} by Bertrand Russell (1872--1970), summarises rather well his life's quest: the search for truth — which he believed could be attained with logic.
Russell spent all his life working on this topic, both in mathematics and in philosophy, and this made him one of the founding fathers of contemporary logic.
% Russell was a brilliant mathematician who spent all his life working on this topic, both in the context of science and philosophy, and he became one of the founding fathers of contemporary logic.
We can get a small taste of his logical developments in the paradox he discovered in the domain of set theory, and which he himself translated into ‘ordinary language’ \citep{russell1918philosophy} under the form of the barber's paradox:

\begin{quote}
	\textit{‘You can define the barber as “one who shaves all those, and those only, who do not shave themselves”. The question is, does the barber shave himself?’}
\end{quote}

 % le barbier du régiment rase tous ceux qui ne se rasent pas eux-mêmes. Mais alors qui rase le barbier ? S'il se rase lui-même, il ne doit pas se raser et s'il ne se rase pas lui-même, il lui faut se raser. C'est ce qu'on appelle un paradoxe logique.
Answering this question results in a contradiction: if he shaves himself, he cannot shave himself (because the barber shaves only those who do not shave themselves); and if he does not shave himself, he must shave himself (because the barber shaves all those who do not shave themselves). This is a typical ‘logical paradox’.

 % http://sos.philosophie.free.fr/russel.php
For Russell, the solution to such contradictory phenomena is to break down each proposition (scientific or philosophic) into ultimate logical units (or atoms) which can be understood independently of other units: this is what he called ‘logical atomism’.
In his view, to know whether a proposition is true or false comes back to analysing the veracity of each atom and the relationship between them\footnote{This is, by the way, what led him to redemonstrate every simplistic principle of algebra (like the fact that $1+1=2$) in his \textit{\usebibentry{whitehead1912principia}{title}} \citeyearpar{whitehead1912principia}.}.
To further decide on the veracity of a given simple proposition, — which he redefines as the adequacy between a belief and a fact, — 
one must agree to hierarchise the degree of certainty of each ‘known’ fact.
For instance, one can be absolutely certain of the things they directly experimented with their five senses (‘sense-data’), — he calls that ‘knowledge by acquaintance’, — but the confidence one has in ‘knowledge by induction’ (i.e.\ the process of deriving a theory from the repeated observation of events) must be questioned.
To borrow one of his own illustrations of that matter \citep{russell1912problems}, we do not feel the slightest doubt that the sun will rise tomorrow because of the laws of motion.
\textit{‘But the \textit{only} reason for believing that the laws of motion will remain in operation is that they have operated hitherto, so far as our knowledge of the past enables us to judge. […] But the real question is: do any number of cases of a law being fulfilled in the past afford evidence that it will be fulfilled in the future?’}
It is, of course, highly unlikely that the laws of motion would stop tomorrow and that the sun would not rise; though, we cannot \textit{prove} it is impossible and, thus, the degree of confidence we can have in such knowledge is lower than that for the things we are directly acquainted with, like the fact that the paper on which this text is written is white.

% Indeed, we are absolutely certain of things that we can directly experiment with our senses (“sense-data”) but much less certain of inducted knowledge: for instance, because the sun rose every day hitherto, we know that it will rise tomorrow too, though, it would only take one counter-example, one day without the sun rising, for this proposition to be false.
% Even if this is extremely unlikely, this knowledge remains more uncertain than what we can experiment, like the fact that the paper on which this text is written is white.

In the case of an evolutionary force, we are in a typical case of such knowledge learnt by induction: 
we infer its very existence on the basis of the observation of its consequences on genomes and, on top of that, it is not even tangible, but merely a concept useful to theorise how genomes evolve.
To analyse such ideas, Russell systematically started with redefining precisely the terms.
But, what is an evolutionary force, exactly? Is it even a cause (for genome evolution) or a consequence (of the molecular processes taking place in individuals)? And, consequently, at what scale — individual or populational — should it be studied?
In the first section of this chapter, I will try and provide ideas to answer those questions.
In the second section, I will dive into the more general notion of the way scientific knowledge can be obtained and the context in which it arises and last, I will focus on the particular and more recent role of bioinformatics in acquiring such knowledge.



% - atomisme logique=tout decomposer en axiomes simples dont on peut savoir s'ils sont vrais ou faux, et en faire des propositions complexes
% - selon lui, pour savoir si une prop est vraie, il faut donc analyser la veracite de chacun des atomes et les relations entre eux.
% - ainsi, pour atteindre la verite, il redemontre des choses extremement simples comme dnas principia mathematica.
% - et, pour juger de la veracite d'une proposition (= correspondance entre une croyance et un fait), il faut accepter de questionner l'induction, i.e.\ passer des faits a la loi, et d'accepter de hierarchiser nos savoirs: on a une connaissance certaine de ce qu'on a experimente, mais, par exemple, de ce qu'on induit: parce que quelque chose arrive toujours, on dit que ca arrivera encore, mais on ne peut pas en etre parfaitement sur, meme si hautement improbable.
% - C'est typiquement le cas des forces que nous on observe: clairement, on ne les experimente pas, mais on observe des choses plusieurs fois et on induit qu'il existe une force et qu'elle agit de telle ou telle facon.
% C'est typiquement un cas de figure dont on est peu sur.
% Pour analyser ce genre de choses, Russell repasse toujours par une definition precise de ce que c'est: comment peut-on def une force?
%

% Russell supported the idea of a scientific philosophy, which he introduced in \textit{\usebibentry{russell1914relation}{title}} by suggesting to apply logical analysis to any traditional problem.
% In particular, in this book, to know what we are certain to be true, and that which we are less certain of, he introduced a distinction between knowledge by acquaintance, i.e.\ what he calls ‘sense-data’, i.e.\ what we can know directly with our sensations without any process of inference (feel it, touch it, see it, hear it, taste it, smell it…) and knowledge by description, i.e.\ knowledge obtained through acquaintance with
% what we cannot know directly and must therefore know \textit{via} its description




% -mathematicien de la logique mais egalement un philosophe qui introduit la logique et cree la philo analytique
% - une de ces questions majeures epistemologique, qui demarre le livre
%
% Existe-t-il au monde une connaissance dont la certitude soit telle qu'aucun Homme raisonnable ne puisse la mettre en doute?
%
% Grande question epistemologique: existe-til qqch dont on peut etre certain?
% knowledge by acquaintance = via nos sens
% % description = on peut le decrire mais pas le sentir de facon directe, et en ca, il s'agit de choses moins certaines.
% pour lui, on peut etre sur d'une proposition (philo ou scientifique) si on peut decrire de facon logique ces relations avec les choses sensibles.
% Dans le cas des forces evolutives, precisement, il ne s'agit pas de choses tangibles. On en mesure les effets, et on les decrit, mais on en a donc une connaissance indirecte.
% Ceci dit, ca ne veut pas forcement dire que n'existe pas, par contre important de bien definir ce que c'est.
% Je ne pretends pas ici faire ce qu'a fait Russell, j'en suis bien loin, ni meme philosopher, j'ai plutot envie d'essayer de penser ce concept de force: comment il se definit, a quelle echelle faut-il le penser, est-il une cause ou une consequence?
% J'en parlerai dans la section 1
%
% Puis, en epistemologie, pour juger de la validite, une des questions qui revient souvent est le contexte dans lequel les decouvertes ont evolue, j'y reviens en sec 2 avec un gros focus sur l'apport de la bioinfo en sec 3.
%
%
%
%
%
%






% je reprends le titre de Russell qui etait mathematicien et philosophe
% surtout connu pour sa pratique de la logique: reprend les principes et axiomes de bases, et cree les relations entre elles pour les redemontrer logiquement (principia mathematica)
% une des choses les plus demonstratives de la logique est son travail sur la theorie des ensembles et le barbier
% il reprend sa facon de faire a la philophie, en particulier, il anatomise les phrases pour distinguer les elements logiques et comprendre les relations entre
% mais aussi, une de ses grandes questions etaient de savoir s'il peut y avoir des connaissances dont toute personne raisonnable serait certaine?
% C'est plus de l'epistemologie et, dans ce chapitre, je vais tenter de garder un facon de penser logique, pour parler de quelques aspects epistemologiques, et des reflexions plus personnelles. — ni de la philo, ni de l'espistemologie, mais je vais quand meme donner quelques pistes pour definir les forces evolutives + les facteurs autour de la science en general, et focu sur les bonnes pratiques en sciences bioinformatiques.
%
% Ce chapitre, pas vraiment de la philo mais pourrait se rapprocher de l'epistemologie
%
% Epistemo= examen critique des sciences (ou d'une science) pour juger de sa valeur/décider si se rapproche d'une connaissance certaine
% pour cela, savoir comment on definit la science, comment elle s'est constituee (facteurs) et methodes de construction, comment juger de sa validite/verifier qu'elle est vraie
%
% partie 1: j'essaye de definir les forces evolutive, dont le BGC objet de cette these, fait partie: en particulier, a quelle echelle le penser?
% partie 2: plus general sur l'avancee des sciences et les facteurs qui permettent son avancee
% partie 3: je focus sur un facteur actuel que je connais bien: le dvpmt des ressources informatiques et la discipline bioinformatique et comment je vois cette discipline, ainsi que son role dans le cadre de l'evolution moleculaire
% i.e.\ comment faire pour que l'on ait les bonnes pratiques pour que ca soit correct
%
%
%
% intro sur russell master de la logique + exemple du barbier et la theorie des ensembles.
% Pour lui, philosophie moins logique, je vais essayer de ne pas trop manquer de ologique.
% Question epistemologique plutot que philo
\section{About evolutionary forces}
% \subsection{Force: emerging property or consttuants of populations?} Cause ou consequence?
% \subsection{What about forces in physics?} + desir des consommateur
% \subsection{???} Concept of effective population size OU retour sur notre cas: BGC
% dire ici que cette reflexion a demarre lors d'un journal club par Sam et qui a implique X et X
% et que donc, une bonne part des idees developpeees ici vient d'eux

% \subsection{Are forces causes or consequences?}
\subsection{Forces as conceptual frameworks} 

By definition, a force represents an interaction which, if unopposed, can change the motion of an object.
As such, forces are generally viewed as causes driving objects or phenomena in a certain direction and are commonly symbolised as vectors giving their direction and intensity.
But are forces mere conceptual tools useful to better apprehend physical phenomena, or could they exist as real physical entities?
% But are forces real physical entities or nothing more than conceptual tools useful to better apprehend physical phenomena?
% But are they more than their iconographic representation?
% Do forces really exist as physical entities or are they just tools useful to better apprehend physical phenomena?

Gravitation, which ensures the mechanical movement of planets and other celestial bodies, is a most interesting case study to think of the aforementioned interrogation.
Indeed, for over 200 years, the theory formulated by Isaac Newton (1642--1727) — the law of universal attraction stating that a ‘gravitational force’ leads masses to attract one another — had been widely accepted.
But, in the early 1900's, Albert Einstein (1879--1955) established the theory of general relativity which accounted for the physical effects unexplained by Newton's law and contradicted the idea that the gravitational force was even a force at all: instead, gravitational attraction would be the result of the warping of spacetime by large masses.
Since then, gravitation has stopped being considered as a force, but its pictural representation as vectors has nonetheless persited, for it helps conceptualising the physical phenomena it explains.

% But, are forces systematically only a conceptual framework helpful to understand observable phenomena as in the case of gravitation, or could they be regarded otherwise in certain cases?

% In the context of evolution, genetic drift is defined as changes in allelic frequencies in a population due to random sampling and is generally included in the list of evolutionary forces.
% This phenomenon is nothing but the result of a gain or a loss of a given allele at the individual-scale because of mere chance.




% For instance, one of the main forces in evolution, natural selection, pushes genomes to purge deleterious alleles and maintain advantageous ones.
% Similarly, in physics, gravitation, the force which brings all things with mass towards one another, ensures the mechanical movement of planets and other celestial bodies, while another force, pressure, can lead to the displacement of the area on which it is applied, unless another force exerted in the opposite direction compensates it.

% Thus, forces are most commonly viewed as the sources driving change (genome evolution, motion, etc…) of other entities and are generally used as conceptual frameworks helpful to understand the physical world, rather than tangible entities existing outside of their pictural representation.
% Though, if such phenomena exist, they must come from somewhere, and can
% ,therefore, also be viewed as consequences
% from somewhere, and are thus also the consequence

% the mechanical movement of planets and other celestial bodies is ensured by gravitation which brings all things with mass towards one another, and pressure exerted on an area can drive its displacement unless another force exerted in the opposite direction compensates it.

% - forces utilisees comme entites conceptuelles qui aident a penser: representees comme un vecteur de force qui va dans une direction ou dans un autre, mais ne representent pas forcement physiquement qqch de tangible
% - eg: drift: un concept pour penser les changements de frequence allelique, force gravitationnelle (mais n'est peut etre meme pas une force d'apres la theorie de la relativite generale mais seulement une courbure de l'espace-temps)
% dirft: vraiment une force vu que c'est purement du hasard (loi binomiale) pour chaque individu de gagner ou perdre un allele.
% Sel nat: pareil: ca revient a gagner ou perdre avec une deviation epsilon, mais est-ce que vraiment une force tangible qui existerait?

\subsection{Forces as emerging properties of individuals}

Another way to regard forces consists in perceiving them as emerging properties of the individuals (particles, people, cells, etc…) which constitute them, i.e.\ as phenomena resulting from the intrinsic characteristics of their components, but \textit{not} reductible to the latter. 
In other words, a force would be the consequence of the fundamental properties of its components, but somehow more than the mere sum of its parts.

To borrow once again an example taken from physics, pressure corresponds to the mean action of the collision of gas particles on a given area and, thus, arises from the intrinsic properties of its components.
Though, each of these particles moves completely randomly (‘Brownian motion’) and does not cease bumping into other molecules or into the surfaces of the walls. 
As such, pressure \textit{cannot} be seen in any particle by itself (for its trajectory is random and the force it exerts on an area is unpredictable) but it nonetheless \textit{emerges} from the collective action of many.
% contained in each particle on its own (for their trajectories are random) but still emerges from their unceasing motion.

In a totally different context, what is called peer pressure results from the individual choices of single people and can thus be seen as a consequence of the biological processes occurring inside their brains.
When looked at it at the scale of a population though, this phenomenon becomes the root cause of the behaviour, attitude or values of other people to conform to the influencing group.
As such, peer pressure — and the same would apply to other sociological phenomena, like consumer behaviour — can be seen both as a cause or as a consequence, depending on the point of view.\\

Altogether thus, even if forces are most generally used as concepts useful to understand phenomena (whether physical, biological, sociological, or else), they are the result of more fundamental properties emerging from their individual components.
With this in mind, at what scale, — populational or individual, — would it be most meaningful to study them in the context of evolutionary biology?



% pressure results from the random motion of particles (‘Brownian motion’) which bump into one another or over the area of an object:
% the mean action of their collision on the surface results in what is called pressure.
% Thus, pressure arises from the intrinsic properties of gas particles, but it is somehow more than each of them, because they all DEAMBULER in a completely random manner.
% In the context of evolutionary forces, biased gene conversion also results from more fundamental properties: as far as it so far known, it is a side-effect of the molecular machinery repairing DNA mismatches in heteroduplexes.
%

% Au global, meme si sont la plupart du temps utilisees comme des concepts, elles sont aussi et avant tout les conseuqences de proprietes intrinseques des individus.


% - mais, elles peuvent aussi etre pensees comme des proprietes emergente des individus, resultant de l'action individuelle des elements la composant, tout en n'y etant pas reductible
% - eg: pression et le mouvement desordonne des particules, mais qui forment la pression, eg: BGC qui resulte d'un biais de la machinerie,
% eg: pression de groupe les deux (ou comportement des consommateurs)
% - donc sont a lafois utiles comme concepts pour penser, mais aussi des proprietes emergentes des individus, et donc des entites reellement existantes, bien que non tangibles.


% Such phenomena arise from the collective action of their components, but cannot be apprehended at the individual-scale.
% For example, pressure corresponds to the distribution of many small forces exerted by particles over an area of a body, but each of these particles moves completely randomly (‘Brownian motion’): the mean action of their collision on a given area results in what is called pressure, but single particles, on their own, do not carry a
%
% but pressure does not make sense at the scale of single particles.
% and none of them could carry, on its own, the conceptanything that may be called pressure.
% pressure is the mean action of the collision of particles on a given area.
% As for evolutionary forces, genetic drift, for one, corresponds to changes in allelic frequencies in a population due to random sampling.
% As such, this concept too only exists at the scale of a population: each individual either keeps or loses a given allele, and the total sum of the individual properties (loss or gain) results in changes in frequencies for the population, but cannot be thought at the individual scale.
% Last, biased gene conversion results from biases in the molecular machinery responsible for mismatch correction, but

% Therefore, forces are, somehow, emerging properties of individuals, i.e.\ properties that result from fundamental characteristics of individuals (cells, people, particles…) but that cannot be reducted to the latter.
% In other words, forces are consequences of individual features, but more than the sum of its parts.
% But how do individual properties determine the existence of forces at the more global scale?
% can these two scales (individual and populational) be thought and linked together?



% - gbgc resulte d'un mecanisme moleculaire biaise mais a des consequences
% sel: un individu ne se reproduit pas donc gene perdu (pas emergent)
% drift:perte ou gain pour chaque individu qui n'a pas d'effet pour l'individu mais importe a l'echelle de la pop (pas de conseq pour l'individu ou sa lignee)
% le processus est le meme en physique: chaleur n'est pas de chaque molecule, mais vient de leur agitation
% % courant electrique pour chaque particule est plus rapide mais au global, agit dans un sens.
% % donc force est en quelque sorte une propriete emergente des individus qui est causee par des effets existant chez les indiviv (BGC: bias de la machinerie, chaleur: agitation des molecules) donc sont des consequences.
% % Pour autant, le nom meme de force signifie une cause, et on parle d'une force qui pousse a aller dans un sens: la selection naturelle pousse les genomes a perdre les elements deleteres, la pression pousse dans un sens particulier, le drift pousse les genomes a perdre certaines versions des genes.
% pression de groupe qui resulte des choix individuels des personnes et qui poussent les autres a faire de meme
% % la pression resulte d'un mouvement desordonne des particules qui peuvent s'entrechosquer entre elles ou rebondir sur les parois)
% alors, si c'est une propriete emergente et donc une consequence pour l'echelle individuelle, mais en meme temps une force a l'echelle globable qui drive le mouvement dans un sens donne, quelle est la relation entre indiv et pop?
% Comment penser le sdeux conjointement?
%
% proprietes emergentes des individus? Proprietes intrinseques?
% eg BGC vient de biais dans la machinerie de reparation des mismatch
% au contraire, le dirft resulte de proprietes qui n'emergent pas des individus car independantes des genotypes
% Relier avec la physique: pression, bruit thermostatique (chaleur) ne represente pas une propriete intrinseque des particules
%
% Dans un sens donc, c'est une consequence de ce qu'on voit a l'echelle individuelle (mais qu'on ne peut pas percevoir a l'echelle individuelle = prop emergente)
% mais d'un autre cote, la notion meme de force est qqch qui appuie dans un sens et qui va faire aller dans un sens donne.
% Donc on l'emploie plutot comme une cause
%

% + car emergent des processus moleculaires, mais en meme temps causent les evolutions des genomes + dependent de la taille efficace qui, lui-meme, est un concept assez mal defini + forces en physique

% \subsection{Individual scale \textit{versus} population-scale}

% \subsection{Relationship between the individuals and populations}
% \subsection{Individuals \textit{versus} populations}


% Transi: determinisme de l'individuel sur le global
% Homo economicus
% micro evol et macro evol
% collaboration de groupe hamilton
% relier l'infiniment petit et l'infiniment grand est le but ultime d'une theorie physique



% relation avec les forces physiques (vent…) et lien entre le microscopique et le macroscopique en physique. Micro-economie vs macroeconomie (concept de l'homo economicus, mais reste difficile de prevoir l'effet global) aussi, dans le cadre du vivant, difficile de tenir compte des effets de collaboration (voir Hamilton et voir le comportement des votes aux elections ou pas forcement le mieux pour moi, donc concept de selfish gene a voir)
% microevolution et macroevolution et determinisme de l'un sur l'autre

% https://www.pourlascience.fr/sd/epistemologie/hasard-et-determinisme-953.php
% https://www.pourlascience.fr/sd/physique/pour-la-science-385-268.php
%
% https://www.pourlascience.fr/sd/histoire-sciences/la-theorie-synthetique-de-levolution-revisitee-2956.php
% dobhanski: population adaptee pour mesurer et comprendre les mecanismes de l'evolution.
% Puis math haldane, puis integre par Huxley et Mayr geneticiens donc cree le neodarwinisme

% Dans le cas du BGC, il faut penser les deux conjointement: comment ca se fait (moleculaire) et effets deleteres sur pop (??? ou ne pas dire)
% + cas du BGC\@: serait un cas de regulation par le groupe sur la microevolution (comme dans le cas de collaborations, ou processus Hamilton)
% transition sur le fait que les nouvelles avancees scientifiques permettent de repenser les forces (apport tectonique des plaques, transfert de genes… pour dire que hasard a prendre en compte la dedans et donc critique la dichotomie de fonctionnel et historique) — donc comment se font ces avancees?
% Ces methodes — et donc la science/recherce — evolue au gre des avancees techniques
%


% \subsection{Molecular processes \textit{versus} historical patterns}
\subsection{Processes \textit{versus} patterns} 
% \subsection{Processes and patterns}

In the 1930's and 1940's, the modern synthesis (a.k.a.\ neo-Darwinian synthesis), — which was formally defined by \citet{dobzhansky1937genetic}, \citet{huxley1942evolution}, \citet{mayr1942systematics} and \citet{simpson1944tempo} — reconciled Darwin's theory of evolution and Mendel's ideas on heredity (see Chapter~\ref{ch:1-history-genetics}).
%, notably under the impulse of Haldane and Fisher and their mathematical framework of population genetics (see Chapter~\ref{ch:1-history-genetics}).

Since then, the way of considering the objects of study in evolution and their relationships has considerably changed \citep[reviewed in][]{paulin2015epistemologie}: a bipolarisation between \textit{patterns} (i.e.\ the description of the results of evolution, as independently as possible from any explanatory theory) and \textit{processes} (i.e.\ the mechanisms responsible for evolution) emerged.
What is less well known is that this distinction was defensibly already present in Darwin's theory \citep{gayon2018connaissance} as the name he gave it — ‘descent with modification by means of natural selection’ — suggests: the ‘descent with modification’ part would correspond to the \textit{patterns} of evolution and the ‘by means of natural selection’ part to the \textit{processes} leading to it.

This distinction could arguably be applied to the study of evolutionary forces as well.
% In the case of the object of this thesis, — biased gene conversion (BGC), — understanding the process would correspond to the functional study of the way the molecular machinery responsible for the repair of DNA mismatches results in BGC, and studying the pattern would correspond to describing its deleterious consequences on genomes and the extent to which it induces divergence.
In the case of the object of this thesis, — biased gene conversion (BGC), — the \textit{process} would correspond to the functional study of the way the molecular machinery responsible for the repair of DNA mismatches results in BGC, and the \textit{pattern} to describing its deleterious consequences on genomes and the extent to which it induces divergence between them.
As such, the joint study of both aspects seems essential to describe this evolutionary force as a whole.
% In other words, individual or populational scale
% the process would be the functional study of the way the molecular machinery responsible for the repair of DNA mismatches results in BGC, and the pattern would be the description of its deleterious consequences in genomes.

Though, the distinction between \textit{patterns} and \textit{processes} may be too simplistic, and it has been much criticised by Stephen Jay Gould (1946--2002) and Niles Eldredge (born 1943) from the 1970's on \citep[reviewed in][]{dericqles2009quelques}.
Their major objection concerned gradualism (i.e.\ the idea that all evolutionary changes are slow, gradual and cumulative) because this implied that there would be a nearly total determinism of micro-evolution (\textit{processes}) onto macro-evolution (\textit{patterns}) and that almost everything could be explained by the sole action of natural selection and adaptation \citep[reviewed in][]{paulin2015epistemologie}.
Instead, Gould put into perspective the extent to which such deterministic features contributed to macro-evolution by reintroducing historical contingency, i.e.\ the idea that the history of life also depends on a series of historical events that are often random or, at least, unpredictable \citep{gould1989wonderful}.

As such, even though his view is still debated, Gould managed to question parts of a theory which was already widely accepted by the scientific community.
The way through which such novel ideas can spread into the scientific world participates much in the progress of science and represents one of the main questions tackled by epistemologists. 
As such, I will focus on this issue in the following section.

% Aisni, Gould a pu remettre en question une theorie globalement bien etablie (et le choix entre ces deux theories est toujorus sujet a debat)
% - mais comment de tels changment de theories sont-elles rendues possibles? dans quel contexte? Object du prochaine sec




% dans quel contexte de telles avancees scientifiques et changements de paradigmes sont-elle possibles?
% remise en question est le debut des changements de dogme, mais comment ces changements se font-ils


% according to supporters of the modern synthesis,

% research on the historical part of evolution has commonly been distinguished from research on its functional part, i.e.\ the mechanisms leading to it.




% Process (mechanistics) vs patterns (results)
% surtout utile pour la speciation: arbre vs forces qui conduisent a la formation
% mais, au sein meme des forces BGC, on pourrait appliquer cette distinction: d'une part, fonctionnellement comment se fait (biais machinerie) etd'autre part quelles en sont les manifestations (et effets deleteres) dans le genome des individus et les consequences a l'echelle de la pop.

% Mais cette distinction est peut etre trop simpliste: suggere qu'il y aurait une sorte de determinisme du process sur pattern
% critique jay gould + contingence historique + dans quel cadre ce genre de critique peut se faire/ changement de vision scientifique? object prochaine section


%\citet{mayr1959where}, \citet{stebbins1966processes} and \citet{dobzhansky1974chance}, — 

% process: gene flow and selection
% pattern: sympatry
%
% distinction entre les systemacists: diversification of living forms, both past and present, and the relationships among living things through time
% et population biologists: mating + processes by which one lineage splits into two.
% Population biology is an interdisciplinary field combining the areas of ecology and evolutionary biology.[1] Population biology draws on tools from mathematics, statistics, genomics, genetics, and systematics. Population biologists study allele frequency changes (evolution) within populations of the same species (population genetics), and interactions between populations of different species (ecology).
%
% TSE gradualisme avec selection naturelle comme moteur principal: les mutations genetiques (micro evol) sont la cause de la macro-evolution
% Jay Gould et Elredge remettent en question: reintroduit l contingence hitorique, et donc perte de determinisme total.
%
% Dob 1937
% Huxley 42 — Evoluton: the modern synthesis
% Mayr 42 — Systematics and the origin of species
% Simpson 44
%
% - TSE dans les 30 avec difference micro evol et macro evol et mise en place d'un differentiel entre historique vs fonctionnalisme
% -



% Voir these epistemo: historique vs fonctionnaliste et reuni dans TSE

% - BGC a dire pour fonctionnel (bias de la machinerie) vs pattern (population et consequences)
% In the context of evolutionary forces, biased gene conversion also results from more fundamental properties: as far as it so far known, it is a side-effect of the molecular machinery repairing DNA mismatches in heteroduplexes.



% - au depart TSEe: difference entre historique et fonctionnalisme
% - pour BGC, fonctionnel (bias de la machinerie) vs pattern (population et consequences)
% - puis remis en question par Jay Gould: trop de determinisme et de pouvoir explicatif de la selection naturelle alors que choses independantes qui influent
% - donc peut etre que trop distinguer ces deux aspects releve d'une vision trop simpliste
% - cette remise en question cree un changement de dogme. Comment la science avance-t-elle et ces dogmes se succedent?


\section{About scientific advances}
\subsection{Scientific revolutions and paradigm shifts} 

% Kuhn: science evolue par cycle
% dans les periodes de science normale, on travaille dans le cadre d'une theorie qui globalement marche bien, avec des enigmes encore inexpliquees
% mais si des enigmes s'accumulent, survient alros une crise qui debouche sur une revolution scientifique
% apres cette periode agitee, les nouveaux scientifiques regardent le monde autrement: les principes, les methodes, les langages ont change

To face gradualism in the modern synthesis of evolution, Gould and Eldredge put forward another thesis: the theory of punctuated equilibria, according to which periods of rapid change are followed by longer periods of relative stasis, i.e.\ states of little change \citep{gould1972punctuated}.\\

We could draw a parallel between this new theory about evolution and that by Thomas Samuel Kuhn (1922--1996) about scientific progress.
Indeed, when it began in the eighteenth century, history of science was written by scientists who presented the discoveries of their time as the culmination of a long process of advancing knowledge.
Thus, science was perceived as a progressive accumulation of cognition where true theories replaced false beliefs \citep{golinski2008making}.

In contrast, Kuhn portrayed scientific progress as a cyclic process involving paradigm shifts, i.e.\ fundamental changes in the basic principles of a scientific discipline \citep{kuhn1962structure}.
In his view, periods of ‘normal science’ where scientists work under a conceptual framework which works globally well alternate with shorter periods of ‘revolutionary science’ where the repeated detection of anomalies (i.e.\ observations unreconciliable with the paradigm of the time) leads to another paradigm under which the world that scientists perceive, as well as the principles, methods or even language they use, are different.\\
% is perceived differently by scientists, and the principles, methods or even language they use are different.
% under which scientists see the world differently and use distinct principles, methods and even language.

According to Kuhn, the transition from one paradigm to another does not rest solely on rational scientific reasons justifying that the new paradigm would be more accurate: he firmly believes that these major shifts also largely depend on external factors, like the sociological and ideological context of the time. I give examples of these in the following subsection.

% Such identification of anomalies is made possible in a particular context involving
% matrix of intellectual, cultural, economic and political trends



% - pour faire faceau gradualism de l'evolution, Gould propose une theorie des equilibres ponctuees selon laquelle les changements se font lors de periodes de temps courts et sont separes par des stases ou l'on voit peu de changements.
% - on pourrait tirer un parallele avec le progres scientifique, qui etait vu comme accumulation graduelle de savoir, mais Kuhn propose une theorie proche de celle de Gould: une vision du progres scientifique par cangement de paradigme, qui alterne des periodes de burst et des periodes de science normale
% - de tels changements sont la resultante de enigmes qui s'accumulent, de fait inexpliques, et sont donc le plus souvent permis par les nouvelles avancees technologiques. Mais d'autres facteurs, comme les facteurs sociaux et politiques qui entrent en jeu.
%


% - theorie du dogme par Kuhn, ou un changement de paradigme en remplace un autre. — ainsi en va de Darwin. Mais aussi Morgan, qui, meme s'il n'a pas decouvert ni les chromosomes, ni le DL, a tout refonde dans une theorie qui pose reellement les bases de la genetique.






\subsection{The impact of external factors} 


% I show that in the aftermath of Germany's defeat the
% dominant intellectual tendency in the Weimar academic world was
% a neo-romantic, existentialist "philosophy of life," reveling in crises
% and characterized by antagonism toward analytical rationality gen
% erally and toward the exact sciences and their technical applications
% particularly. Implicitly or explicitly, the scientist was the whipping
% boy of the incessant exhortations to spiritual renewal, while the
% concept?or the mere word?"causality" symbolized all that was
% odious in the scientific enterprise.


Paul Forman (born 1937), a former student of Kuhn's, defended the thesis of a cultural conditioning of scientific knowledge.
He developed his proposition with the example of the connection between the culture of Weimar Germany and the emergence of quantum mechanics in the 1920's \citep{forman1971weimar}.
According to him, in the aftermath of the defeat of Germany in World War I, the dominant tendancy was characterised by intellectual revolts against causality, determinism and materialism and welcomed the rise of anti-rationalist movements such as existentialism, i.e.\ a philosophy of life claiming that individuals are faced with the absurdity of life and that the essence of their being lies in their own actions which are \textit{not} predetermined by any kind of theological, philosophical or moral doctrine.

In Forman's view, the concept of quantum \textit{a}causality could spread much more easily into this German scientific world marked by the rejection of determinism and analytical rationality than in other Western countries which did not undergo such crises, and explains why the most prominent advances in that field were made by Germans.\\

On top of the sociological, political and religious context, Barry Barnes (born 1943) argues that the personal interests of researchers also play a major role in determining their actions and, thus, in shaping scientific advances \citep{barnes1977interests}.
Interests at stake in scientific practice may include the use of techniques or theories specific to a given paradigm which they want to promote, or defined by their social, political or ideological position \citep{gingras2017determinants}.
As such, ‘inner’ and ‘outer’ factors are not necessarily distinct.

For instance, in nowadays world where ecological awareness is growing, several scientists promote the creation of a new geological epoch — the so-called ‘Anthropocene’ — that would account for the impact of mankind on Earth's geology and ecosystems \citep{crutzen2002geology} and some geologists and mineralogists have already started doing research in this still unofficial field of investigation \citep{corcoran2014anthropogenic,hazen2017mineralogy}.\\


% - forman, etudiant de Kuhn qui defend la place des facteurs externes dans les changements de paradigmes/avancees scientifiques.
% - prend l'exemple de la physique quantique, qui a pu avoir lieu dans l'aftermath de la defaite de l'Allemagne, ou  le climat etait une philosophie existentialiste, i.e.\ l'etre humain forme l'essence de sa vie par ses propres actions, non predeterminees.
% Ainsi, un climat de rejet du determinisme et de la rationalite analytique qui permet l'emergence de la physique quantique acausale.
% - Aussi, des interets cognitifs: selon ce que les gens defendent: la sphere scientifique et societale pas forcement completement distinct; et anthropocene avec bastien Bousseau
% - transition Gould Eldredge et les nouveaux champs d'investigation qui pemettent de challenge parts of theories
% - mais aussi, evidemment, des facteurs technologiques


% matrix of intellectual, cultural, economic and political trends


In the case of Gould and Eldredge too, their challenging the modern synthesis was made possible thanks to the contemporary creation of additional fields of investigation — including developmental genetics, phylogenetic cladistics, the molecular clock and gene transfers: these provided novel findings or original ways of thinking, which participated a great deal in questioning parts of the modern synthesis \citep{lecointre2009apres}.

Generally, the creation of new domains of study pairs up with the establishment of modern techniques which themselves play a significant role in advancing knowledge. 
I discuss this topic in the next subsection. 

% In the case of Gould and Eldredge, their questioning of the established theory was made possible by the addition of new fields of investigation — including developemental genetics, phylogenetic cladistics, the molecular clock and gene transfers — which provided new findings or ways of thinking challenging parts of the modern theory \citep{lecointre2009apres}.
% The more general context in which such scientific advances arise will be the object of the next section.



\subsection{The contribution of modern techniques}

It goes without saying that scientific knowledge has systematically considerably benefited from both technological advances and the expertise of scientists in using the latter.
Cell biology, for one, would not have existed had microscopy not been invented \citep{bechtel2006discovering} and chromosomes would not have been discovered if it had not been for Frans Janssens's mastery of cell staining (see Chapter~\ref{ch:1-history-genetics}).

Though, the very use of technologies for scientific progress can bring a set of questions of its own.
Indeed, it has been argued that there is often a circular relationship between the pieces of evidence for a phenomenon of interest and the instruments detecting it \citep[reviewed in \citealp{godin2002experimenters}]{collins1975seven,collins1985changing}: 
according to the words of the sociologist who developed this idea, ‘\textit{we won't know if we have built a good detector until we have tried it and obtained the correct outcome. But we don't know what the correct outcome is until… and so on ad infinitum}’ \citep{collins1985changing}.
He termed this pitfall the ‘experimenter's regress’.

On top of that, the belief (or not) in the outcome and the acceptance (or not) of the value given by the instrument somehow depends on the reasearcher's interests: a scientist who believes in the existence of a phenomenon will be willing to accept the announcement of its detection, while one who does not would probably rather question the validity of either the apparatus or the method used \citep{gingras2017determinants}.\\

% The best way to avoid such pitfalls remains yet to be determined, but we already owe sociologists of science a great deal in making us aware of their existence.\\

In genetics, the development of the first sequencing techniques in the 1970's have led to a major upheaval in the way research is carried.
Indeed, the rise of ‘-omics’ (genomics, transcriptomics, metabolomics, proteomics, etc…) as major fields of study, together with the large progresses in computing resources and data storage capacity, has led some to re-think of the interplay between data-driven and hypothesis-driven science \citep{kell2004here, mazzocchi2015could}.

But, from now on, future advances in the field surely depend much more on the ability of bioinformaticians to analyse the deluge of data standing before them rather than on further technological leaps.
% In the last section, I thus share my vision on bioinformaticians and the way I think they can best help in scientific progress.
In the last section, I thus share my vision on the way I believe bioinformaticians can best help scientific progress.

% More recently, thanks to the development of the first sequencing techniques in the 1970's, genomics has become a major field of study.
% But, considering the large progresses in computing resources and data storage capacity, future advances in the field surely depend more on the ability of bioinformaticians to analyse such quantities of data than on further technological leaps, which is why I will focus in the last section on what I believe our role to be in science.

% - techniques de sequencage
% - qui maintenant, couplee aux ressources de calcul et capacites de stockage, a reundu le champ possible a la bioinformatique, et en particulier la genomique.
% - desormais, la limite, plus que les capacites technologiques, la limite reside dans la capacite du bioinfo a analyser ca


% - remet en question une analyse hypothesis-dirven ou data-driven

\section{About bioinformaticians}

\subsection{Biologists before informaticians} 

The word ‘bioinformatics’ is a contraction of ‘biology’ and ‘informatics’ and both facets are of course required in this domain.
Though, it seems to me that, in view of the colossal quantity of data that genomicians are supposed to deal with, it can be tempting to let the informatics side take over.
On top of that, results obtained purely by an automated process involving bioinformatic tools with little or no input from the experimenter are usually perceived as objective, whereas any choice made by the biologist is more often negatively regarded as subjective.

I would like to argue against that line of reasoning by taking an example from machine learning — a set of methods which has begun to be used by bioinformaticians in the last few years.
Basically, machine learning is a subset of artificial intelligence aiming at ‘learning’ from data.
In the vast majority of cases, these programs ‘learn’ on the basis of the correlations they find within the training sets they are provided with.
Retracing how these associations have been made is actually a rather complex process but, in one study, after creating a classifier allowing to distinguish between dogs and wolves, \citet{ribeiro2016why} wanted to understand the reasons why their artificial-intelligence method was so outstandingly accurate.
They analysed the associations made by the program and found out that the main feature used to distinguish between the two animals was the background in the training pictures: wolves were often standing on snow whereas dogs were rather standing on grass.
As such, even if the classifier outputted the correct results, it became obvious that it could not be trusted.
Nevertheless, such caveats originating from automated processes can easily be avoided by human knowledge.
% It thus takes human know-how to avoid such caveats from automated processes.

In the context of this thesis, the method we implemented to detect recombination events from sequencing data rested on identifying and iteratively suppressing sources of error (see Chapters~\ref{ch:5-methodology} and~\ref{ch:8-HFM1}).
In the process leading to it, a considerable amount of time was spent visually inspecting the candidate events and hypothesising on the origin of miscalls.
Automation was only used in a second phase to assess the impact of each possible adjustment to the final outcome.
This is, by the way, how we could identify that mapping biases explained most of the false positive miscalls.

As such, I firmly believe that the input from any savvy human can make analyses much more accurate than the sole work of bioinformatic tools.\\
% Even in cases where the goal is not to implement a method, but to use an alledgedly already optimised one, it seems to me that looking at the data is an absolutely indispensable step.
% I would also add that, even in cases where the process has already been optimised, looking at data is important to adapt parameters to the new study.
% meme dans le cas ou on reproduit un pipeline, essentiel de regarder les donnees, car les choix faits toujours adaptes a un jeu de donnees en particulier.

I would also like to argue in favour of simplicity.
Indeed, considering the extremely wide range of bioinformatic tools — but also statistical and mathematical methods — available today, it is often tempting to create sophisticated processes to tackle biological problems that are generally rather complex.
Though, it seems to me that, except for some specific issues, aiming at the maximal simplicity carries many advantages, including a better reproducibility of analyses, a more straightforward detection of errors, greater smooth in adapting code or methods to other frameworks and much larger clarity in transmitting the ideas.
% All in all, thus, one of the best
% it allows to reproduce analyses and detect mistakes rapidly, to readapt code or methods to other
% Ward Cunningham (born 1949), the programmer who invented the concept of the wiki (i.e.\ knowledge-based websites collaboratively incremented),

% and , it could be tempting to
% Also, SIMPLICITY
% I would also like to argue in favour of simplicity.
% - compte tenu des outils a disposition, on peut faire complique, mais Ward Cunningham juge qu'il faut faire simple dans le code.
% Je dirais que c'est aussi ce qu'il faut faire dans les analyses bioinfo, pour pouvoir reproduire facilement, detecter les erreurs, ou readapter, et aussi trasnsmettre.
% Bay vs inf: mettre une prior ou pas

% - simplicite?
% - on a une masse importante d'outils a disposition, et tente de faire des choses complexes, mais il faut choisir la simplicite, pour la reproductibilite et la verification et ou adaptation du code ou de la procedure
%
% - bien sur, ca ne veut pas dire qu'il faut supprimer la rigueur pour autant







%
% - bioinfo contraction biologiste et informaticien
% - etant donne la large quantite de donnees, tentant de tout automatiser au maximum, mas le meilleur conseil que j'ai recu en bioinfo venait d'un biologiste: regarde les donnees.
% - dans mon cas, c'est en passant des heures a regarder mes alignements et les recombinants potentiels que nous avons pu mettre au point la methode et eliminer els sources d'erreur de facon iterative.
% - autre exemple: ML et detection d'images chiens/loups, et c'est a l'inge de savoir quoi faire, plutot que de laisser l'ordi proceder, meme dans un tel cas d'IA
%
% - l'autre chose que je dirais de mon experience est de chercher la simplicite (cf Ward)


% (divise en 2 si pas gnomics a temps)
% Bioinfo et emergence de info donne possibilite de tout automatiser, mais besoin de l'expertise
% ce que j'ai appris de plusimportant probablement etait de regarder les donnees. Utile pour detecter les pb, voir les recombinants, comprendre d'ou viennent les sources de mauvais genotypage. — autre exemple, ML et le cas de la detection d'images de chiens vs loup, et ML qui ne fait que des correlations donc n'est pas “intelligent”: c'est bien au scientifique de savoir quoi faire, et le mieux, aller au plus simple
% Aussi, en particulier pour methode chap8, fait a partir de ce que je voyais. Mon mot d'ordre: simplicite (cf Ward Cunningham) — ca ne fonctionne peut etre pas dans certains cas, mais beaucoup de fois, c'est ce qu'il y a de mieux a faire + ca permet de rendre les choses claires a expliquer derriere, et faciles a corriger.

\subsection{Training biologists in genomics}

With the ever increasing amount of sequencing data available, one of the major limitations in genomics becomes the ability to process them.
I argued in the previous subsection that the input from humans — biologists in the case of bioinformatics — was crucial to analyse the data correctly.

Though, it is not that easy for biologists to get trained in bioinformatics: to the extent of my knowledge, there is no free website that explains the basic know-how of next-generation sequencing data analysis.
Therefore, I decided to create one (\url{https://gnomics.io/}) to account for this lack.
In it, I try to provide biologists with a global overview of the major steps that one should follow to perform the most common genomic analyses, indicate the tools allowing to complete each of these and the way to use them concretely and, finally, explain the assumptions on which they are based and the way the outcome they render should be interpreted.
% an overall understanding of the way they work and how their outcome should be interpreted.

% which aims at listing the successive steps one should follow to deal with the most common genomic analyses.
% In it, I try to provide biologists with a global knowledge of the
% , how to use the tools allowing to perform these steps, as well as an overall understanding of the way they work and how their outcome can be interpreted.
%
%
%
% - plus de donnees de sequencage (mais d'autres type aussi en biologie) et une des limites devient le traitement de ces donnees
% - je viens de dire que l'apport du biologiste important
% - mais, difficile pour un biologiste de formation qui le souhaiterait, de se former a la bioinfo: on peut trouver les sites des outils mais savoir quels outils utliser pourquoi dans quel ordre et ce qu'ils font est difficile
% - j'ai donc commence un site, gnomics.io, pour tenter de repondre a ce manque: voir pas a pas les etapes d'analyse et les outils utilises et les lignes de commande associees.
%





% Technical know-how
% % - les techniques et outils evoluent rapidement, et necessaire d'y rester formes, y compris bien apres sa formation initiale (quand a eu lieu).
% % - comme, a ma connaissance, pas de site (gratuit) qui redonne de facon precise la facon precise d'utiliser les outils genomiques, j'en ai cree un (avec marche a suivre)
% % - gnomics.io et le nom comme contraction genomics et gnome.
% %
% - besoin de bioinformaticien en grande qte
% - et des biologistes de formation qui souheraient apprendre a utiliser les outils, mais il n'existe pas (a ma connaissance) de site gratuit qui permette de savoir quels outils utiliser en fonction de l'objectif et de savoir pratiquement comment les utiliser (generalement il faut aller sur le site de l'outil) tout en comprenant le format des fichiers (structure) et d'etre capable d'interpreter les sorties
% - donc je cree gnomics.io qui tente de repondre a ce manque — gnomics comme contraction de genomics et gnome, etant donne qu'un bioinformaticien est finalement assez proche d'un gnome cad une creature qui voit assez peu de lumiere (sauf celle de son ecran) et dont le but est de travailler pour gagner en connaissance.
%
% % - pas trouve de site (gratuit) qui permet d'apprendre comment faire de la genomique et en combinant une comprehension des outil et pratiquement leur utilisation (generalement il faut aller sur le site de l'outil) — mais c implique de deja connaitre les outils.
% - donc je cree gnomics qui vise a donner la possibilite a des gens qui n'y sont pas families de connaitre les bases et de s'approprier les outils de base
%

% - dire que avec l'emergence des donnees, pas de pb de stockage, mais pb pour analyser, donc besoin de former.
% - en genomique, de nombreux outils, qu'il faut apprendre sur le tas et peu d'infos (gratuites) sur internet pour avoir une vue d'ensemble de ce qu'il faut faire techniqument.
% - je cree donc gnomic.io
% - Appele gnomics comme contraction de gnome et de genomics, puisque, finalement, un bioinformaticien travaillant en genomique reste enferme et ne voit que peu le jour dans le but de gagner de la connaissance, tout comme un gnome.
%

% Base sur mon experience, certes limitee.



\subsection{A genomician in evolutionary biology}

% - certitudes de l'informatique, parfois binaire
% - en biologie au contraire, bcp d'incertitudes, car la vie est evidemment tres complexe, et pas binaire, et donc on doute: doute des theories, doute des resultats, doute des doutes
% - dans le cas de l'evolution, c'est encore pire, car on ajoute une dimension temporelle, et on sait qu'on ne pourra jamais avoir un acces direct a la verite.
% - ainsi, le genomicien, qui est une sorte de combinaison de stats, info, son travail est justement de quantifier
%


According to the paleontologist Stephen Jay Gould, evolutionary biology is a kind of science somewhat special in the way that it creates knowledge.
Indeed, in most research fields, the best way to know whether a hypothesis is true or false consists in experimentally testing for it and comparing the outcome it predicted to the real one: if they concord, the hypothesis may be true; otherwise, we can be sure that it is false.
Though, this so-called scientific method is not adapted to the study of evolution because the objects of study cannot be reproduced experimentally\footnote{Nevertheless, this is precisely what studies of so-called ‘experimental evolution’ aim to do.}.
Instead, the past is to be \textit{inferred} and, arguably, if there was a past, remnants of it should persist in today's world.
The whole work of the evolutionary biologist thus consists in searching for these relics — which, according to Gould, are often imperfections or incongruities — and to make sense of them in a more global picture of evolution \citep{gould1979turtles}.
% and all the work of the evolutionary biologist consists precisely in looking for those vestiges and making sense of their existence in a more global life history \citep{gould1979turtles}.

In this context, a genomician working in evolutionary biology should scan genomes to try and find vestiges of the past which could help reconstruct indirectly the unobservable evolutionary history.
The discovery of biased gene conversion was typically such a case of evolutionary inferrence based on unexplained incongruities seen in genomes:
it all started with the strange observation that GC-content varies along genomes (see Chapter~\ref{ch:4-gBGC}).
Several hypotheses were then proposed to explain it — one of which being the existence of biased gene conversion.
Since then, a lot of work — including that carried for this thesis, — has been done with the aim of providing evidence for this hypothesis.

Bioinformaticians generally have a training in either informatics, algorithmics, mathematics, statistics or any other field in which certainty is much more widespread than in biology, and especially more than in evolutionary biology.
As such, for them to work in this research field, I would argue that one of the major difficulties may reside in fighting an inner struggle to make room for doubt in the middle of all the apparent objectivity of computer programs.\\
% they would have to fight an inner struggle to make room for doubt in the middle of all their informatic objectivity.\\
% - pour pouvoir bosser en evol, un des plus grand combat a mener est peut etre, justement, d'accepter de faire la place aux incertitudes au milieu de l'objectivite mathematiques, et peut etre plutot d'essayer de quantifier le niveau d'invertitude.





% - on infere le passe a partir de ce qu'on voit, et on ne peut jamais verifier (de meme, le paleontologiste de Gould)
% - dans ce cadre, le travil du genomicien/evolution moleculare est de trouver des traces de ce qu'on pourrait exploiter comme vestiges du passe dans les genomes, pour pouvoir retracer l'histoire evolutive dessus
% - mais le bioinfo a generalement une formation en info, algorithmique, math, ou les choses sont peut etre plus de certitude— donc, une des particularites, peut etre du genomicien en evolution serait, justement, de remettre en question ses certitudes pour accepter l'erreur et, justement, plutot la quantifier.
%


% % - dans ce cadre, le travail du bioinformaticien, qui recoupe des savoirs en stats, math, info, est de
% Dans une colonne du New Scientist, Jay Gould parle de la facon dont les imperfections ont permis a darwin de comprendre l'evolution.
% (D'abord, les recherches se faisaient dans les vestiges donc les imperfections de la nature pour comprendre pourquoi etaient la, cf New Scientist Gould).
% En cela, les biologistes de l'evolution sont des scientifiques un peu particuliers, a la frontiere entre l'histoire et la biologie: rien a voir avec le stereotype du chercheur.
%
%
%
% - role du biologiste de l'evolution: retracer l'histoire de la vie. La biologie est deja extremement complexe, et on y rajoute une dimension temporelle en plus.
% - sachant qu'on ne pourra jamais connaitre la realite cf Gould et les paleontologistes
% - le bioinfo, ajoute des math, stat, info a cela: on peut mesurer, quantifier, connaitre l'incertitude associee a ces phenomenes
% lire les reliquats du genome pour inferer les forces evolutive en jeu (dans le cas de la genomique), au meme titre que les paleontologue lisent les reliquats fossiles pour inferer les evenements de speciation
% -
%
%
% - voir these abdekhader
% - preambule: role du biologiste en evolution est d'inferer le passe, sans pouvoir y acceder rellement
% - le role du bioinfo, essayer de limiter l'erreur pour approcher au meixu la realite
%
% % Bioinfo dans cadre evolution: on fait des inferences}
% + role du bioinfo n'est que de faire des models/inferer/limiter l'erreur
% + boucler la boucle avec le preambule et Jay gould du role de l'
%
% on interprete le passe sans pouvoir le demontrer (cf Jay Gould) et ses imperfections sont utiles poour comprendre de nouvelles choses

% Conclu: objectif de la science est la recherche de la verite, et on cherche a faire ca dans un souci d'objectivite. Toutefois, la science avance au contraire au gre de la contingence et subjectivite des evenements et des chercheurs et dans le cas particulier de la bioinfo, je crois qu'il faut etre capable de ne surtout pas faire fi de cette subjectivite, prendre position basee sur l'expertise humaine et pouvoir inferer les processus ayant eu lieu dans le passe.
% C'est au travers de cette facon de faire que le travail que nous avons fait pdt cette these s'est construit, et il ne me reste plus qu'a conclure dessu.
%



All in all, science is not much different than a quest for truth and scientists generally try and pursue objectivity so as to get to it.
Though, in this chapter where I gathered epistemological, philosophical and sociological thoughts, I showed that scientific progress also depends on the contingency of external events and on the subjective interests of researchers, no matter how neutral they are willing to be.
In the particular case of bioinformatics applied to evolutionary genomics, I believe that the subjectivity of human expertise can be used as an advantage rather than as an obstacle to make further progress.
It was with these thoughts in mind that the work useful to this thesis was carried.
As for now, there is nothing left for me but to conclude about it all.



% Dans epistemo: discuter du fait que la taille efficace de population est un concept qui est assez peu bien defini — difficile a mesurer donc les projets menes dessus sont un peu limites.

% Mettre dedans l'idee que reflechir sur comment le BGC peut etre contreselectionnne a l'echelle de l'individu pour eviter les defauts a l'echelle populationnelle est insoluble. 
% Mais, mettre la citation (Ernst Mayr oiu Jay Gould?) sur : claim to tell that the problems they did not solve are insoluble.
% It is an occupational risk of biologists to claim, towards the end of their careers, that the problems which they have not solved are insoluble. (John maynard Smith)







%OK % FIN CHAP 6
%OK % Page de titre
%OK % CHAP 7: au moins motifs+ hitchhinking
%OK % ce soir: abstract en francais.
%OK % demain: fin chap 7 + chap8 (au moins design + adaptation methode)
%OK % Samedi: fin chap 8 + chap9 en entier
% Dimanche: chap10 (au moins 1 section) + conclusion + preambule
% Lundi: fin chapitre 10
% Mardi + mercredi + jeudi: chap1 section 3
% Vendredi + dimanche : figures sur Inkscape
% Lundi + mardi: resume etendu + abbreviations + definitions + verif les references
% Mercredi + Jeudi: relecture totale.
% Vendredi: remerciements.
% + Preparation de SMBE

% \textbf{NOTE a Laurent: Je me demande s'il est pertinent de comparer ces deux mesures car, dans le cas de l'ABC, on extrapole le taux de recombinaison reel (i.e.\ nb de COs en cM/Mb génomique) alors que dans le cas des fragments informatifs, on obtient un taux de COs en cM/Mb séquencée.}




% Annexes (erreur du jeune est d'en mettre trop)
%% pour chap6
% mettre les figures DMC1 (les deux — correlation par groupe de 10 plus relation asymetrie)
% + position des switch points
%% data availability: mettre lien vers le github pour reproduire les figure et avoir les tableaux d'entree + numero accession SRA
%% pour chap8: mettre les autres images de l'identification du background + image des correlations sur les memes hotspots
%% les listes de hotspots etudies

% Annexe des permissions
% La meilleure:
% [It] is not the nature of things for any one man to make a sudden, violent discovery; science goes step by step and every man depends on the work of his predecessors. When you hear of a sudden unexpected discovery—a bolt from the blue—you can always be sure that it has grown up by the influence of one man or another, and it is the mutual influence which makes the enormous possibility of scientific advance. Scientists are not dependent on the ideas of a single man, but on the combined wisdom of thousands of men, all thinking of the same problem and each doing his little bit to add to the great structure of knowledge which is gradually being erected. 
% — Sir Ernest Rutherford
% Concluding remark in Lecture ii (1936) on 'Forty Years of Physics', revised and prepared for publication by J.A. Ratcliffe, collected in Needham and Pagel (eds.), Background to Modern Science: Ten Lectures at Cambridge Arranged by the History of Science Committee, (1938), 73-74. Note that the words as prepared for publication may not be verbatim as spoken in the original lecture by the then late Lord Rutherford.




% [I]f texts are unified by a central logic of argument, then their pictorial illustrations are integral to the ensemble, not pretty little trifles included only for aesthetic or commercial value. Primates are visual animals, and (particularly in science) illustration has a language and set of conventions all its own.
% De Stephen Jay Gould
% de Jay Gould encore
% God bless all the precious little examples and all their cascading implications; without these gems, these tiny acorns bearing the blueprints of oak trees, essayists would be out of business.
% Questioning the Millennium (second edition, Harmony, 1999), p. 42
% de Einstein 
% When a man after long years of searching chances on a thought which discloses something of the beauty of this mysterious universe, he should not therefore be personally celebrated. He is already sufficiently paid by his experience of seeking and finding. In science, moreover, the work of the individual is so bound up with that of his scientific predecessors and contemporaries that it appears almost as an impersonal product of his generation.
% From the story "The Progress of Science" in The Scientific Monthly edited by J. McKeen Cattell (June 1921), Vol. XII, No. 6. The story says that the comments were made at the annual meeting of the National Academy of Sciences at the National Museum in Washington on April 25, 26, and 27. Einstein's comments appear on p. 579, though the story may be paraphrasing rather than directly quoting since it says "In reply Professor Einstein in substance said" the quote above.
% 
% In science men have discovered an activity of the very highest value in which they are no longer, as in art, dependent for progress upon the appearance of continually greater genius, for in science the successors stand upon the shoulders of their predecessors; where one man of supreme genius has invented a method, a thousand lesser men can apply it. … In art nothing worth doing can be done without genius; in science even a very moderate capacity can contribute to a supreme achievement. 
% — Bertrand Russell
% Essay, 'The Place Of Science In A Liberal Education.' In Mysticism and Logic: and Other Essays (1919), 41.

% It is a wrong business when the younger cultivators of science put out of sight and deprecate what their predecessors have done; but obviously that is the tendency of Huxley and his friends … It is very true that Huxley was bitter against the Bishop of Oxford, but I was not present at the debate. Perhaps the Bishop was not prudent to venture into a field where no eloquence can supersede the need for precise knowledge. The young naturalists declared themselves in favour of Darwin’s views which tendency I saw already at Leeds two years ago. I am sorry for it, for I reckon Darwin’s book to be an utterly unphilosophical one. 
% — William Whewell
% Letter to James D, Forbes (24 Jul 1860). Trinity College Cambridge, Whewell Manuscripts.

% Very few people, including authors willing to commit to paper, ever really read primary sources–certainly not in necessary depth and contemplation, and often not at all ... When writers close themselves off to the documents of scholarship, and then rely only on seeing or asking, they become conduits and sieves rather than thinkers. When, on the other hand, you study the great works of predecessors engaged in the same struggle, you enter a dialogue with human history and the rich variety of our own intellectual traditions. You insert yourself, and your own organizing powers, into this history–and you become an active agent, not merely a ‘reporter.’ 
% — Stephen Jay Gould

% [It] is not the nature of things for any one man to make a sudden, violent discovery; science goes step by step and every man depends on the work of his predecessors. When you hear of a sudden unexpected discovery—a bolt from the blue—you can always be sure that it has grown up by the influence of one man or another, and it is the mutual influence which makes the enormous possibility of scientific advance. Scientists are not dependent on the ideas of a single man, but on the combined wisdom of thousands of men, all thinking of the same problem and each doing his little bit to add to the great structure of knowledge which is gradually being erected. 
% — Sir Ernest Rutherford
% Concluding remark in Lecture ii (1936) on 'Forty Years of Physics', revised and prepared for publication by J.A. Ratcliffe, collected in Needham and Pagel (eds.), Background to Modern Science: Ten Lectures at Cambridge Arranged by the History of Science Committee, (1938), 73-74. Note that the words as prepared for publication may not be verbatim as spoken in the original lecture by the then late Lord Rutherford.

% It is strange that only extraordinary men make the discoveries, which later appear so easy and simple.
% GEORG C. LICHTENBERG, 1742 TO 1799




% REMERCIEMENTS
% You have … been told that science grows like an organism. You have been told that, if we today see further than our predecessors, it is only because we stand on their shoulders. But this [Nobel Prize Presentation] is an occasion on which I should prefer to remember, not the giants upon whose shoulders we stood, but the friends with whom we stood arm in arm … colleagues in so much of my work. 
% — Sir Peter B. Medawar
% From Nobel Banquet speech (10 Dec 1960).


% DEDICACE
% A Loïc Rajjou,
% qui, le premier, m'a donné le goût de la recherche,
% et sans la rencontre duquel,
% l'idée-même d'un doctorat ne m'aurait pas effleuré l'esprit.

% l'idee d'un doctorat ne m'aurait pas meme effleure l'esprit.
% l'idée-même d'un doctorat ne m'aurait pas effleurée.
%

% l'idee d'un doctorat ne m'aurait pas meme effleuree.
% Je lui dois tout, alors merci a lui.
% la possibilite d'un doctorat ne m'aurait pas meme effleure l'esprit.
% l'idee-meme d'un doctorat ne m'aurait pas effleure l'esprit.

% et sans la rencontre duquel l'idee meme d'un doctorat ne m'aurait jamais effleuree.
% et sans le soutien duquel l'idee meme d'un doctorat ne m'aurait jamais effleuree.
% et sans le soutien duquel la possibilite d'un doctorat serait reste une idee vague.
% et sans lequel je n'aurais jamais envisage un doctorat.

% qui, le premier, m'a donne le gout de la recherche
% et sans la rencontre duquel,
%
% sans la rencontre duquel
% cette these n'aurait pas existe
%


%
% \textbf{\\NOTES}
%
% DANS CHAP9 s'assurer que pose bien la question: la question est comment la selection peut se mettre en place en fonction de la taille efficace, qui est une propriete externe aux individus sur laquelle la selection s'applique.
%


% HFM1
% file:///Users/maudgautier/Downloads/StatHotspots.html#pairwise-comparison-of-capture-efficiency
% file:///Users/maudgautier/Library/Containers/com.apple.mail/Data/Library/Mail%20Downloads/0B36F5AF-413F-43AB-BAA8-D254CD7974B3/MapGenotypeF1.html
%
% file:///Users/maudgautier/Documents/These/R_projects/01_identification_of_recombinants.html#perspectives
% file:///Users/maudgautier/Documents/These/R_projects/03_New_method/03_Rmarkdown_recap.html
% file:///Users/maudgautier/Documents/These/R_projects/01b_Mice_Whole_experiment/01_identification_of_recombinants.html
% file:///Users/maudgautier/Documents/These/R_projects/01_identification_of_recombinants.html
%
%
%
% A garder
% https://en.wikipedia.org/wiki/Intragenomic_conflict
% https://www.phil.vt.edu/dmayo/PhilStatistics/b%20Fisher%20design%20of%20experiments.pdf
% https://gdurif.perso.math.cnrs.fr/#contact
% Wu anaotomy of mouse recombination hotspot
% https://www.nature.com/articles/ng0794-420
% https://www.ncbi.nlm.nih.gov/pmc/articles/PMC4978934/
% Yauk EMBO
% https://en.wikipedia.org/wiki/History_of_evolutionary_thought
% https://tel.archives-ouvertes.fr/tel-00431055/document
% https://www.ncbi.nlm.nih.gov/pmc/articles/PMC1832099/
%
%
%
% https://www.nature.com/scitable/topicpage/genetic-drift-and-effective-population-size-772523
% https://www.ncbi.nlm.nih.gov/pmc/articles/PMC2635931/
%
%
% http://www.lequydonhanoi.edu.vn/upload_images/S%C3%A1ch%20ngo%E1%BA%A1i%20ng%E1%BB%AF/Rich%20Dad%20Poor%20Dad.pdf
%
