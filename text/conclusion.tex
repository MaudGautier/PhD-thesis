\begin{savequote}[8cm]

	‘L'ineptie consiste à vouloir conclure.’
	\qauthor{--- Gustave Flaubert, \textit{\usebibentry{flaubert1889correspondance}{title}} \citeyearpar{flaubert1889correspondance} }

\end{savequote}

\chapter*{\label{ch:conclusion}}
\fancyhead[LO]{\emph{Conclusion}}
\fancyhead[RE]{\emph{Conclusion}}

% Add blank lines before starting
\begin{parse lines}[\noindent]{#1\\}



\end{parse lines}

In summary, the aim of this thesis was to better understand the interplay between the intensity of GC-biased gene conversion and the effective population size ($N_e$) within the mammalian clade.
We thus wanted to estimate the parameters on which the gBGC coefficient ($b$) depends — namely the recombination rate $r$, the length of conversion tracts $L$ and the transmission bias $b_0$ — in a species with large $N_e$ (mice) to compare them with those found in a species with lower $N_e$ (humans).\\

To do this, we implemented a method that allowed to detect recombination events at high resolution in the recombination hotspots of single individuals.
Our approach appeared unprecedentedly powerful in detecting such events and we showed that it could be adapted to practically any kind of experimental design, no matter the number of genomic introgressions it may involve.

In the course of our enterprise, we managed to quantify double-strand break-induced biased gene conversion (dBGC) in several hundreds of autosomal recombination hotspots and brought to light the fact that, in cases of structured populations, dBGC hitchhiked past gBGC, thus creating an intrincate interplay between the two forms of biased gene conversion occurring in PRDM9-dependent species.\\

Overall, we found that, in mouse autosomal hotspots, the transmission bias $b_0$ was similar to that measured in humans for single-marker non-crossover (NCO-1) events but extremely reduced for multiple-marker non-crossover (NCO-2+) events and null for crossing-overs (COs).
As, in addition, the recombination rate $r$ and the length of conversion tracts $L$ were smaller in mice, the gBGC coefficient ($b$) was globally much reduced in this species.

Altogether, the globally stable intensity of biased gene conversion at the population-scale ($B$) in \textit{Homo sapiens} ($B = 0.355$), \textit{Mus musculus domesticus} ($B = 0.465$) and \textit{Mus musculus castaneus} ($B = 1.21$) was permitted by the joint decrease of all three parameters on which $b$ depends ($r$, $L$ and $b_0$) in the species with 20- to 70-fold larger $N_e$.
We argued that such large differences in $b$ between the two species in spite of their comparable $B$ was consistent with the hypothesis of a selective pressure restraining gBGC at the population-scale and materialising under the form of an extremely rapid evolution of the molecular machinery leading to it.

If our hypothesis were to be correct, the way the information on the effective population size could be integrated by a selective force to constrain the evolution of the molecular machinery at the scale of single individuals remains a widely open question.
I would have probably even ventured into saying that this conundrum might be indecipherable, it if were not for John Maynard Smith's observation that \textit{‘It is an occupational risk of biologists to claim, towards the end of their careers, that the problems which they have not solved are insoluble’} \citep{smith1988games}.



