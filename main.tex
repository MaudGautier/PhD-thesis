%%%%%%%%%%%%%%%%%%%%%%%%%%%%%%%%%%%%%%%%%%%%%%%%%%%%%%%%%%%%%%%
%% OXFORD THESIS TEMPLATE

% Use this template to produce a standard thesis that meets the Oxford University requirements for DPhil submission
%
% Originally by Keith A. Gillow (gillow@maths.ox.ac.uk), 1997
% Modified by Sam Evans (sam@samuelevansresearch.org), 2007
% Modified by John McManigle (john@oxfordechoes.com), 2015
%
% This version Copyright (c) 2015-2017 John McManigle
%
% Broad permissions are granted to use, modify, and distribute this software
% as specified in the MIT License included in this distribution's LICENSE file.
%

% I've (John) tried to comment this file extensively, so read through it to see how to use the various options.  Remember
% that in LaTeX, any line starting with a % is NOT executed.  Several places below, you have a choice of which line to use
% out of multiple options (eg draft vs final, for PDF vs for binding, etc.)  When you pick one, add a % to the beginning of
% the lines you don't want.


%%%%% CHOOSE PAGE LAYOUT
% The most common choices should be below.  You can also do other things, like replacing "a4paper" with "letterpaper", etc.

% This one will format for two-sided binding (ie left and right pages have mirror margins; blank pages inserted where needed):
\documentclass[a4paper,twoside]{ociamthesis}
% This one will format for one-sided binding (ie left margin > right margin; no extra blank pages):
%\documentclass[a4paper]{ociamthesis}
% This one will format for PDF output (ie equal margins, no extra blank pages):
%\documentclass[a4paper,nobind]{ociamthesis} 


%%%%% SOME PERSONNAL SETTINGS
% Add greek character
% From: https://texblog.org/2012/03/15/greek-letters-in-text-without-changing-to-math-mode/
\usepackage[euler]{textgreek}

% Caption
\usepackage[labelfont=bf]{caption}

% Male and female symbols
\usepackage{wasysym}







%%%%% SELECT YOUR DRAFT OPTIONS
% Three options going on here; use in any combination.  But remember to turn the first two off before
% generating a PDF to send to the printer!

% This adds a "DRAFT" footer to every normal page.  (The first page of each chapter is not a "normal" page.)
\fancyfoot[C]{\color{titlepagecolorsection} \emph{DRAFT Printed on \today}}  

% This highlights (in blue) corrections marked with (for words) \mccorrect{blah} or (for whole
% paragraphs) \begin{mccorrection} . . . \end{mccorrection}.  This can be useful for sending a PDF of
% your corrected thesis to your examiners for review.  Turn it off, and the blue disappears.
\correctionstrue


%%%%% BIBLIOGRAPHY SETUP
% Note that your bibliography will require some tweaking depending on your department, preferred format, etc.
% The options included below are just very basic "sciencey" and "humanitiesey" options to get started.
% If you've not used LaTeX before, I recommend reading a little about biblatex/biber and getting started with it.
% If you're already a LaTeX pro and are used to natbib or something, modify as necessary.
% Either way, you'll have to choose and configure an appropriate bibliography format...

% The science-type option: numerical in-text citation with references in order of appearance.
\usepackage{natbib}
% \usepackage[style=numeric-comp, sorting=none, backend=biber, doi=false, isbn=false]{biblatex}
\newcommand*{\bibtitle}{References}

% The humanities-type option: author-year in-text citation with an alphabetical works cited.
%\usepackage[style=authoryear, sorting=nyt, backend=biber, maxcitenames=2, useprefix, doi=false, isbn=false]{biblatex}
%\newcommand*{\bibtitle}{Works Cited}

% This makes the bibliography left-aligned (not 'justified') and slightly smaller font.
% \renewcommand*{\bibfont}{\raggedright\small}

% Change this to the name of your .bib file (usually exported from a citation manager like Zotero or EndNote).
% \addbibresource{references.bib}


% Uncomment this if you want equation numbers per section (2.3.12), instead of per chapter (2.18):
%\numberwithin{equation}{subsection}

\usepackage{usebib}
\usepackage{hyperref}
\hypersetup{colorlinks, linkcolor=black, citecolor=titlepagecolorsection} % Color citation links in purple
% To use usebib
% \usepackage{keyval}
\bibinput{references}

%
%% FROM = https://tex.stackexchange.com/questions/159551/customizing-part-style-with-tikz
%\usepackage{tikz}
%\usepackage[explicit]{titlesec}
%
%\definecolor{mybluei}{RGB}{0,173,239}
%\definecolor{myblueii}{RGB}{63,200,244}
%\definecolor{myblueiii}{RGB}{199,234,253}
%
%\renewcommand\thepart{\arabic{part}}
%
%\newcommand\partnumfont{% font specification for the number
%  \fontsize{380}{130}\color{myblueii}\selectfont%
%}
%
%\newcommand\partnamefont{% font specification for the name "PART"
%  \normalfont\color{white}\scshape\small\bfseries 
%}
%
%\titleformat{\part}
%  {\normalfont\huge\filleft}
%  {}
%  {20pt}
%  {\begin{tikzpicture}[remember picture,overlay]
%  \fill[myblueiii] 
%    (current page.north west) rectangle ([yshift=-13cm]current page.north east);   
%  \node[
%      fill=mybluei,
%      text width=2\paperwidth,
%      rounded corners=6cm,
%      text depth=18cm,
%      anchor=center,
%      inner sep=0pt] at (current page.north east) (parttop)
%    {\thepart};%
%  \node[
%      anchor=south east,
%      inner sep=0pt,
%      outer sep=0pt] (partnum) at ([xshift=-20pt]parttop.south) 
%    {\partnumfont\thepart};
%  \node[
%      anchor=south,
%      inner sep=0pt] (partname) at ([yshift=2pt]partnum.south)   
%  {\partnamefont PART};
%  \node[
%      anchor=north east,
%      align=right,
%      inner xsep=0pt] at ([yshift=-0.5cm]partname.east|-partnum.south) 
%  {\parbox{.7\textwidth}{\raggedleft#1}};
%  \end{tikzpicture}%
%  }
%
%% END FROM = https://tex.stackexchange.com/questions/159551/customizing-part-style-with-tikz


% FROM = https://tex.stackexchange.com/questions/263130/customizing-part-style

\usepackage{tikz}
\usetikzlibrary{calc}
\usepackage{lipsum}
%
%\makeatletter
%\@addtoreset{chapter}{part}
%\makeatother  
%
%\renewcommand{\thepart}{\arabic{part}}
%\def\topleft{4}
%
%\newcommand{\cpart}[1]{
%\clearpage
%\stepcounter{part}
%\pagestyle{empty}
%\begin{tikzpicture}[overlay,remember picture]
%\coordinate(ptopleft) at ($(current page.north west)+(\topleft,0)$);
%\foreach \i in{0,.1,.2,.3,.5,.6,.9,1.1,1.2,1.3,1.5,1.6,1.8,1.9,2}
%\draw[gray!70]($(ptopleft)+(\i,0)$)--+(0,-\paperheight);
%\node[white,fill=black,minimum size=1.5cm](A) at ($(ptopleft)+(1,-2)$){\huge\thepart};
%\node [anchor=south,ultra thick,yshift=-1mm]at(A.north){\Large PART};
%\node [fill=white,anchor=north,very thick,right=-2mm,inner sep=5mm]at($(ptopleft)+(0,-0.5\paperheight)$){\Huge #1};
%\end{tikzpicture}
%\addcontentsline{toc}{part}{\thepart\hspace{1em}#1}
%\clearpage
%\pagestyle{plain}
%}

%\pagestyle{empty}

% END FROM = https://tex.stackexchange.com/questions/263130/customizing-part-style



%% FROM = https://tex.stackexchange.com/questions/86294/trying-to-do-graphical-decorations-in-classicthesis-style/86310#86310
%
%
%\usepackage[eulerchapternumbers,pdfspacing]{classicthesis}
%\usepackage{arsclassica}
%\usepackage{tikz}
%\usepackage{changepage}
%\usepackage{lipsum}% just to generate text for the example
%
%\strictpagecheck
%
%\definecolor{halfgray}{gray}{0.55}
%\usepackage{titlesec}
%\titleformat{\part}[block]%
%        {\normalfont\Large\sffamily}%
%        {{\color{halfgray}\partNumber\thechapter%
%        \hspace{10pt}\vline}  }{10pt}%
%        {\spacedallcaps}[\partdecoration]
%
%
%
%\newcommand\anglei{-45}
%\newcommand\angleii{45}
%\newcommand\angleiii{225}
%\newcommand\angleiv{135}
%
%\newcommand\partdecoration{%
%\begin{tikzpicture}[remember picture,overlay,shorten >= -10pt]
%\coordinate (aux1) at ([yshift=-15pt]current page.north east);
%\coordinate (aux2) at ([yshift=-410pt]current page.north east);
%\coordinate (aux3) at ([xshift=-4.5cm]current page.north east);
%\coordinate (aux4) at ([yshift=-150pt]current page.north east);
%\checkoddpage
%\ifoddpage
%\else
%\coordinate (aux1) at ([yshift=-15pt]current page.north west);
%\coordinate (aux2) at ([yshift=-410pt]current page.north west);
%\coordinate (aux3) at ([xshift=4.5cm]current page.north west);
%\coordinate (aux4) at ([yshift=-150pt]current page.north west);
%\renewcommand\anglei{-135}
%\renewcommand\angleii{135}
%\renewcommand\angleiii{-45}
%\renewcommand\angleiv{45}
%\fi
%\begin{scope}[halfgray!40,line width=12pt,rounded corners=12pt]
%\draw
%  (aux1) -- coordinate (a)
%  ++(\angleiii:5) --
%  ++(\anglei:5.1) coordinate (b);
%\draw[shorten <= -10pt]
%  (aux3) --
%  (a) --
%  (aux1);
%\draw[opacity=0.6,halfgray,shorten <= -10pt]
%  (b) --
%  ++(\angleiii:2.2) --
%  ++(\anglei:2.2);
%\end{scope}
%\draw[halfgray,line width=8pt,rounded corners=8pt,shorten <= -10pt]
%  (aux4) --
%  ++(\angleiii:0.8) --
%  ++(\anglei:0.8);
%\begin{scope}[halfgray!70,line width=6pt,rounded corners=8pt]
%\draw[shorten <= -10pt]
%  (aux2) --
%  ++(\angleiii:3) coordinate[pos=0.45] (c) --
%  ++(\anglei:3.1);
%\draw
%  (aux2) --
%  (c) --
%  ++(\angleiv:2.5) --
%  ++(\angleii:2.5) --
%  ++(\anglei:2.5) coordinate[pos=0.3] (d);   
%\draw 
%  (d) -- +(\angleii:1);
%\end{scope}
%\end{tikzpicture}%
%}

%% END FROM = https://tex.stackexchange.com/questions/86294/trying-to-do-graphical-decorations-in-classicthesis-style/86310#86310


% used: https://tex.stackexchange.com/questions/85904/showcase-of-beautiful-title-page-done-in-tex
% used: https://latex.org/forum/viewtopic.php?t=22554



\usepackage{titlesec}

%\definecolor{titlepagecolor}{cmyk}{1,.60,0,.40}
%\definecolor{titlepagecolor}{cmyk}{0,0,0,0.55}
%\definecolor{titlepagecolor}{rgb}{0.6,0.2,0.2}

\definecolor{titlepagecolor}{cmyk}{0, 0.5, 0, 0.7} % Rose
\definecolor{titlepagecolorlight}{cmyk}{0, 0.12, 0, 0.2} % Rose
\definecolor{titlepagecolorsection}{cmyk}{0, 0.5, 0, 0.55} % Rose

%\definecolor{titlepagecolor}{cmyk}{0, 0, 0, 0.8} % Gris
\definecolor{greydark}{cmyk}{0, 0, 0, 0.8}

\titleclass{\part}{top} % make part like a chapter
\titleformat{\part}
[display]
{\centering\fontsize{40}{20}\normalfont}
{\color{titlepagecolorsection} \bfseries\filleft\fontsize{46}{0}{\partname} \thepart}
{100pt}
{\color{titlepagecolor} \filleft}[\titlepagedecoration]
%\titlespacing*{\part}{0pt}{0pt}{20pt}

%\titleclass{\chapter}{top} % make part like a chapter
%\titleformat{\chapter}[\otherpagedecoration]
%[display]
%{\centering\fontsize{40}{20}\normalfont}
%{\bfseries\filleft\fontsize{46}{0}{\chaptername} \thechapter}
%{100pt}
%{\filleft}[\otherpagedecoration]
%\titlespacing*{\part}{0pt}{0pt}{20pt}




\usepackage{epigraph}

\renewcommand\epigraphflush{flushright}
\renewcommand\epigraphsize{\normalsize}
\setlength\epigraphwidth{0.7\textwidth}



\DeclareFixedFont{\titlefont}{T1}{ppl}{b}{it}{0.5in}


\makeatletter                       
\def\printauthor{%                  
    {\large \@author}}              
\makeatother
\author{%
    Author 1 name \\
    Department name \\
    \texttt{email1@example.com}\vspace{20pt} \\
    Author 2 name \\
    Department name \\
    \texttt{email2@example.com}
    }


\newcommand\titlepagedecoration{%
\begin{tikzpicture}[remember picture,overlay,shorten >= -10pt]

\coordinate (aux1) at ([yshift=-15pt]current page.north east);
\coordinate (aux2) at ([yshift=-410pt]current page.north east);
\coordinate (aux3) at ([xshift=-4.5cm]current page.north east);
\coordinate (aux4) at ([yshift=-150pt]current page.north east);

\begin{scope}[titlepagecolor!40,line width=12pt,rounded corners=12pt]
\draw
  (aux1) -- coordinate (a)
  ++(225:5) --
  ++(-45:5.1) coordinate (b);
\draw[shorten <= -10pt]
  (aux3) --
  (a) --
  (aux1);
\draw[opacity=0.6,titlepagecolor,shorten <= -10pt]
  (b) --
  ++(225:2.2) --
  ++(-45:2.2);
\end{scope}
\draw[titlepagecolor,line width=8pt,rounded corners=8pt,shorten <= -10pt]
  (aux4) --
  ++(225:0.8) --
  ++(-45:0.8);
\begin{scope}[titlepagecolor!70,line width=6pt,rounded corners=8pt]
\draw[shorten <= -10pt]
  (aux2) --
  ++(225:3) coordinate[pos=0.45] (c) --
  ++(-45:3.1);
\draw
  (aux2) --
  (c) --
  ++(135:2.5) --
  ++(45:2.5) --
  ++(-45:2.5) coordinate[pos=0.3] (d);   
\draw 
  (d) -- +(45:1);
\end{scope}
\end{tikzpicture}%
}


\newcommand\otherpagedecoration{%
\begin{tikzpicture}[remember picture,overlay,shorten >= -10pt]

\coordinate (aux1) at ([yshift=-15pt]current page.north east);
\coordinate (aux2) at ([yshift=-410pt]current page.north east);
\coordinate (aux3) at ([xshift=-4.5cm]current page.north east);
\coordinate (aux4) at ([yshift=-150pt]current page.north east);

\begin{scope}[titlepagecolor!40,line width=12pt,rounded corners=12pt]
\draw
  (aux1) -- coordinate (a)
  ++(225:5) --
  ++(-45:5.1) coordinate (b);
\draw[shorten <= -10pt]
  (aux3) --
  (a) --
  (aux1);
\draw[opacity=0.6,titlepagecolor,shorten <= -10pt]
  (b) --
  ++(225:2.2) --
  ++(-45:2.2);
\end{scope}
\draw[titlepagecolor,line width=8pt,rounded corners=8pt,shorten <= -10pt]
  (aux4) --
  ++(225:0.8) --
  ++(-45:0.8);
\end{tikzpicture}%
}





% Modify chapter color
\colorlet{chaptergrey}{titlepagecolorlight}
\renewcommand*\sectfont{\color{titlepagecolor}}



\titleformat{\section}
{\color{titlepagecolorsection}\normalfont\Large\bfseries}
{\color{titlepagecolorsection}\thesection}{1em}{}

\titleformat{\subsection}
{\color{titlepagecolorsection}\normalfont\large\bfseries}
{\color{titlepagecolorsection}\thesubsection}{1em}{}

\titleformat{\subsubsection}
{\color{titlepagecolorsection}\normalfont\bfseries}
{\color{titlepagecolorsection}\thesubsubsection}{1em}{}

%\titleformat*{\paragraph}{\large\bfseries}
%\titleformat*{\subparagraph}{\large\bfseries}



\usepackage{titletoc} % pourquoi besoin ???

\usepackage{minitoc}
\usepackage{xcolor}
\usepackage{colortbl}

\makeatletter
\arrayrulecolor{titlepagecolorsection}
\def\mtc@bottom@rule{%
  \ifx\mtc@rule\relax\relax\else
      \vskip -2.5ex
        \color{titlepagecolorsection}\rule[2.4\p@]{\columnwidth}{.4\p@}\vspace*{2.6\p@}\fi}    
\makeatother



\renewcommand*\rhead{\textcolor{titlepagecolor}}




%%%%% THESIS / TITLE PAGE INFORMATION
% Everybody needs to complete the following:
\title{Suitably impressive thesis title}
\author{Your Name}
\college{Your College}

% Master's candidates who require the alternate title page (with candidate number and word count)
% must also un-comment and complete the following three lines:
%\masterssubmissiontrue
%\candidateno{933516}
%\wordcount{28,815}

% Uncomment the following line if your degree also includes exams (eg most masters):
%\renewcommand{\submittedtext}{Submitted in partial completion of the}
% Your full degree name.  (But remember that DPhils aren't "in" anything.  They're just DPhils.)
\degree{Doctor of Philosophy}
% Term and year of submission, or date if your board requires (eg most masters)
\degreedate{Michaelmas 2014}


%%%%% YOUR OWN PERSONAL MACROS
% This is a good place to dump your own LaTeX macros as they come up.

% To make text superscripts shortcuts
	\renewcommand{\th}{\textsuperscript{th}} % ex: I won 4\th place
	\newcommand{\nd}{\textsuperscript{nd}}
	\renewcommand{\st}{\textsuperscript{st}}
	\newcommand{\rd}{\textsuperscript{rd}}

%%%%% THE ACTUAL DOCUMENT STARTS HERE
\begin{document}


%
%\begin{titlepage}
%
%\noindent
%\titlefont Hardy's Theorem\par
%\epigraph{Pure mathematics is on the whole distinctly more useful than applied. For what is useful above all is technique, and mathematical technique is taught mainly through pure mathematics.}%
%{\textit{London 1941}\\ \textsc{G. H. Hardy}}
%\null\vfill
%\vspace*{1cm}
%\noindent
%\hfill
%\begin{minipage}{0.35\linewidth}
%    \begin{flushright}
%        \printauthor
%    \end{flushright}
%\end{minipage}
%%
%\begin{minipage}{0.02\linewidth}
%    \rule{1pt}{125pt}
%\end{minipage}
%\titlepagedecoration
%\end{titlepage}
%

%%%%% CHOOSE YOUR LINE SPACING HERE
% This is the official option.  Use it for your submission copy and library copy:
\setlength{\textbaselineskip}{22pt plus2pt}
% This is closer spacing (about 1.5-spaced) that you might prefer for your personal copies:
%\setlength{\textbaselineskip}{18pt plus2pt minus1pt}

% You can set the spacing here for the roman-numbered pages (acknowledgements, table of contents, etc.)
\setlength{\frontmatterbaselineskip}{17pt plus1pt minus1pt}

% Leave this line alone; it gets things started for the real document.
\setlength{\baselineskip}{\textbaselineskip}


%%%%% CHOOSE YOUR SECTION NUMBERING DEPTH HERE
% You have two choices.  First, how far down are sections numbered?  (Below that, they're named but
% don't get numbers.)  Second, what level of section appears in the table of contents?  These don't have
% to match: you can have numbered sections that don't show up in the ToC, or unnumbered sections that
% do.  Throughout, 0 = chapter; 1 = section; 2 = subsection; 3 = subsubsection, 4 = paragraph...

% The level that gets a number:
\setcounter{secnumdepth}{2}
% The level that shows up in the ToC:
\setcounter{tocdepth}{2}


%%%%% ABSTRACT SEPARATE
% This is used to create the separate, one-page abstract that you are required to hand into the Exam
% Schools.  You can comment it out to generate a PDF for printing or whatnot.
\begin{abstractseparate}
	

\section*{Abstract}

During meiosis, recombination hotspots host the formation of DNA double-strand breaks (DSBs). DSBs are subsequently repaired through a process which, in a wide range of species, is biased towards the favoured transmission of G and C alleles: GC-biased gene conversion (gBGC).
The intensity of this fundamental distorter of meiotic segregation strongly varies between species but the factors dictating its evolution are not known.
We thus aimed at directly quantifying the transmission bias in mice and comparing the parameters on which it depends with other mammals.
% To better understand this fundamental distorter of meiotic segregation, we aimed at directly quantifying the transmission bias in mice and comparing the parameters on which in depends with other mammals.
% Better understanding this fundamental distorter of meiotic segregation thus requires to directly quantify the transmission bias and compare the parameters on which in depends across several species.
% To identify the latter and thus better understand this fundamental distorter of meiotic segregation, we aimed at directly quantifying the transmission bias in mice and comparing the parameters on which in depends with other mammals.
% This fundamental distorter of meiotic segregation
% proceeds with distinct intensities
% plays a major role in genome evolutiona
% mimics the action of positive selection and plays a major role in the evolution of base composition in the vicinity of recombination hotspots.

Here, we coupled capture-seq and bioinformatic techniques to implement an approach that proved 100 times more powerful than current methods to detect recombination. With it, we identified 18,821 crossing-over (CO) and non-crossover (NCO) events at very high resolution in single individuals and could thus precisely characterise patterns of recombination in mice.
In this species, recombination hotspots are targeted by PRDM9 and are therefore subject to a second type of biased gene conversion (BGC): DSB-induced BGC (dBGC). Quantifying both dBGC and gBGC with our data brought to light the fact that, in cases of structured populations, past gBGC from the parental lineages is hitchhiked by dBGC when the populations cross.
% Next, we dissociated the hitchhiking effect to directly measure the intensity of gBGC ($b$) in both COs and NCOs.
We next observed that, in male mice, only NCOs — and more particularly single-marker NCOs — contribute to the intensity of gBGC. In contrast, in humans, both NCOs and at least a portion of COs (those with complex conversion tracts) distort allelic frequencies. 
This suggests that the DSB repair machinery leading to gBGC varies across mammals.
% has evolved extremely rapidly within the mammalian clade.
Our findings are also consistent with the hypothesis of a selective pressure restraining the intensity of the deleterious gBGC process at the population-scale: this would materialise through a multi-level compensation of the effective population size by the recombination rate, the length of conversion tracts and the transmission bias.

Altogether, our work has allowed to better comprehend how recombination and biased gene conversion proceed in the mammalian clade.\\

\textbf{Keywords:} Recombination, Biased gene conversion, PRDM9, Hotspots, Genomics, Molecular evolution, Mammals, Sperm-typing.


\newpage
\section*{Résumé en français}

Au cours de la méiose, les points chauds de recombinaison sont le siège de la formation de cassures double-brin de l’ADN. Ces dernières sont ensuite réparées par un processus qui, chez de nombreuses espèces, favorise la transmission des allèles G et C : la conversion génique biaisée vers GC (gBGC). 
L'intensité de cet important distorteur de la ségrégation méiotique varie fortement entre espèces mais les facteurs déterminant son évolution sont toujours inconnus. 
Nous avons donc voulu quantifier directement le biais de transmission chez la souris et comparer les paramètres dont il dépend avec d'autres mammifères.
% Cet important distorteur de la ségrégation méiotique mime les effets de la sélection positive et joue un rôle majeur dans l’évolution de la composition nucléotidique au voisinage des points chauds.

Dans cette étude, en couplant des développements bioinformatiques à une technique de capture ciblée d’ADN suivie de séquençage haut-débit (capture-seq), nous avons réussi à mettre au point une approche qui s’est révélée 100 fois plus performante pour détecter les événements de recombinaison que les méthodes existant actuellement. Ainsi, nous avons pu identifier 18 821 crossing-overs (COs) et non-crossovers (NCOs) à très grande résolution chez des individus uniques, ce qui nous a permis de caractériser minutieusement la recombinaison chez la souris.
Chez cette espèce, les points chauds de recombinaison sont ciblés par la protéine PRDM9 et sont donc soumis à une deuxième forme de conversion génique biaisée (BGC) : le biais d’initiation (dBGC). La quantification du dBGC et du gBGC à partir de nos données nous a permis de mettre en lumière le fait que, au moment où des populations structurées s’hybrident, le gBGC des lignées parentales est propagé par un phénomène d’auto-stop génétique (genetic hitchhiking) provenant du dBGC.
% Ensuite, nous avons dissocié ce phénomène d’auto-stop pour mesurer directement l’intensité du gBGC ($b$) dans les COs et les NCOs.
Nous avons ensuite pu observer que, chez les souris m\^ales, seuls les NCOs — et plus particulièrement les NCOs contenant un seul marqueur génétique— contribuent à l'intensité du gBGC. En comparaison, chez l’Homme, à la fois les NCOs et au moins une part des COs (ceux qui présentent des tracts de conversion complexes) distordent les fréquences alléliques. 
Ceci suggère que la machinerie de réparation des cassures double-brin qui induit le biais de conversion génique (BGC) présente des variations au sein des mammifères.
%a évolué extrêmement rapidement au sein des mammifères. 
Nos résultats sont aussi en accord avec l’hypothèse selon laquelle une pression de sélection limiterait l’intensité de ce processus délétère à l’échelle de la population. Cela se traduirait par une compensation de la taille efficace de population à de multiples niveaux : par le taux de recombinaison, par la longueur des tracts de conversion et par le biais de transmission.

Somme toute, notre travail a permis de mieux comprendre la façon dont la recombinaison et la conversion génique biaisée opèrent chez les mammifères.\\

\textbf{Mots-clés:} Recombinaison, Conversion génique biaisée, PRDM9, Points chauds, Génomique, \'Evolution moléculaire, Mammifères, Sperm-typing.

\newpage
\section*{Résumé étendu en français}

{
\setstretch{1.15}

Lorsque l'on traite de l'évolution des génomes, trois forces sont classiquement invoquées : la mutation, la sélection naturelle et la dérive génétique.
% La première est la source
% La deuxième
% La troisième
%
Toutefois, depuis une vingtaine d'année, une autre force a fait son entrée sur la scène évolutive~: la conversion génique biaisée, que nous noterons ‘BGC’ (de l'anglais \textit{biased gene conversion}).
Ce phénomène est une conséquence directe du processus de recombinaison méiotique chez les espèces à reproduction sexuée.

Chez les mammifères en effet, après s'être fixée à certains loci cibles appelés ‘points chauds de recombinaison’, la protéine PRDM9 recrute la machinerie de formation de cassures double-brin et marque, de ce fait, l'initiation d'un événement de recombinaison \citep{baudat2010prdm9,myers2010drive,parvanov2010prdm9}.
Ce dernier doit ensuite être réparé en utilisant le chromosome homologue comme matrice, ce qui mène à ce qu'on appelle un événement de conversion génique, c'est-à-dire le transfert non-réciproque d'une information de séquence d'ADN\@.

Toutefois, si PRDM9 présente une plus grande affinité de liaison avec la séquence de l'un des deux chromosomes (que nous appellerons ‘haplotype’), la cassure s'initiera préférentiellement sur cet haplotype, et l'événement de conversion génique se fera donc préférentiellement dans un sens donné : c'est ce qu'on appelle le biais d'initiation, aussi appelé conversion génique biaisée induite par cassure double brin et noté ‘dBGC’ (de l'anglais \textit{double-strand break-induced biased gene conversion}).
Du fait de ce phénomène, les points chauds finissent nécessairement par s'éroder : comme l'haplotype portant le motif ciblé par PRDM9 est le siège de la cassure, il est systématiquement converti par l'autre haplotype, et voué à dispara\^itre \citep{boulton1997hotspot}.

Il existe une deuxième forme de conversion génique biaisée : la conversion génique biasée vers GC, que l'on notera ‘gBGC’ (de l'anglais \textit{GC-biased gene conversion}).
En effet, il a été observe chez plusieurs espèces 
de façon directe \citep{mancera2008highresolution, si2015widely, williams2015noncrossover, halldorsson2016rate, keith2016high, smeds2016highresolution}
ou indirecte \citep{escobar2011gcbiased,pessia2012evidence,figuet2014biased}
que la réparation des cassures double-brin favorise les allèles G et C par rapport aux allèles A et T\@.\\


% les événements de recombinaison sont initiés par la formation de cassures double-brin, dont la position est déterminée par la protéine PRDM9.
% Après s'être fixée à certains loci selon son affinité de liaison avec ceux-ci, cette dernière recrute la machinerie de formation des cassures double-brin.
% Suite à cela,
% Cette dernière se lie à certains loci spécifiques d'autant plus fortement que son affinité de liaison avec eux est élevée et
%
%
% En effet, les événements de recombinaison sont initiés par la formation d'une cassure double-brin qui est ensuite réparée gr\^ace à l'action d'une machinerie de réparation de ces cassures.
% Or, la position de ces cassures est déterminée par la protéine PRDM9 qui se lie d'autant plus fortement à certains loci que son affinité de liaison avec ceux-ci est forte, et recrute
% en fonction de son affinité de liaison avec certains loci,

% - une consequence : si des mutations sur un des haplotypes, PRDM9 se lie plus sur celui non mute, et donc, le mute est donneur dans conversion. C'est le paradoxe des hotspots
% - cela mene a dBGC = biais d'initiation, dont une des consequences est lerosion des hotspots

% - aussi une autre forme: lors de la reparation, il a ete observe que l'allele GC est souvent favorise chez many species: on appelle ca le gBGC






La quantification du coefficient de conversion génique biaisée à l'échelle des populations ($B$) chez un grand nombre de métazoaires \citep{galtier2018codon} a mis en évidence un résultat étonnant: 
l'intensité du gBGC ne varie que dans une gamme de valeurs très restreinte.
Par exemple, chez les mammifères placentaires, $B$ reste dans une fourchette de 0 à 7 \citep{lartillot2013phylogenetic}.
\'Etant donné que $B$ correspond au produit de la taille efficace de population ($N_e$) par le coefficient de gBGC ($b$) et que la taille efficace peut varier sur plusieurs ordres de grandeurs parmi les métazoaires, $b$ ne peut mécaniquement pas être identique chez toutes les espèces.
Au contraire, un ou plusieurs des paramètres dont $b$ dépend (le taux de recombinaison $r$, la longueur des tracts de conversion $L$ et le biais de transmission $b_0$) varient nécessairement inversement à la taille efficace.


Cependant, peu de données sont disponibles pour comprendre la base de la dépendance entre $N_e$ et $b$: le biais de transmission ($b_0$) n'a été mesuré que chez quelques espèces \citep{mancera2008highresolution, si2015widely, williams2015noncrossover, halldorsson2016rate, keith2016high, smeds2016highresolution} et, parmi les mammifères, la seule espèce chez qui ce biais a été mesuré de façon directe (\textit{Homo sapiens}) présente une très faible taille efficace d'environ 10,000 \citep{takahata1993allelic,erlich1996hla,harding1997archaic,charlesworth2009fundamental,yu2004nucleotide}.

Afin d'apporter un éclairage nouveau sur l'interaction entre $b$ et $N_e$, nous avons donc voulu quantifier le gBGC chez une autre espèce de mammifères présentant une taille efficace beaucoup plus grande que celle de l'Homme \citep{geraldes2008inferring,phifer-rixey2012adaptive,davies2015factors}: la souris \textit{Mus musculus}.\\


Pour pouvoir quantifier précisément le gBGC, il est nécessaire de disposer d'un grand nombre d'événements de recombinaison.
Or, la méthode généralement utilisée pour détecter ces événements — l'analyse de pedigrees — est extrêmement gourmande en ressources : 
elle requiert le séquençage de génomes complets d'un grand nombre d'individus et permet de détecter seulement un nombre limité de recombinants.
Nous avons donc mis au point une nouvelle approche permettant de détecter plusieurs milliers de recombinants à très haute résolution chez des individus uniques.

% De plus, afin de maximiser le nombre de recombinants détectables, nous avons sélectionné 1 018 points chauds de recombinaison particulièrement denses en marqueurs hétérozygotes
% Concrètement, notre approche repose sur le génotypage de molécules d'ADN uniques issues du sperme de souris hybrides.
% Afin de maximiser le nombre de recombinants détectables, nous avons, au préalable, réalisé une étape de capture ciblée d'ADN provenant de 1 018 points chauds de recombinaison particulièrement denses en marqueurs hétérozygotes.
% Brièvement, notre approche repose sur le génotypage de molécules d'ADN uniques issues du sperme de souris hybrides enrichi en événements de recombinaison gr\^ace au ciblage spécifique de 1 018 points chauds de recombinaison denses en sites polymorphes.
Concrètement, notre approche repose sur deux étapes principales.
Premièrement, puisque la recombinaison n'est identifiable qu'à partir du génotypage de marqueurs hétérozygotes, nous avons croisé deux races de souris (C57BL/6J que nous noterons ‘B6’ et CAST/EiJ que nous appellerons ‘CAST’) issues de deux sous-espèces (\textit{Mus musculus domesticus} et \textit{Mus musculus castaneus}) présentant un fort taux de polymorphisme de 0.74\% \citep{keane2011mouse,yalcin2012nextgeneration}.
Les points chauds de recombinaison chez l'hybride F1 qui résulte de ce croisement (B6xCAST) ont déjà été identifiés par d'autres que nous \citep{baker2015prdm9}.
Afin de maximiser le nombre de recombinants détectables, nous en avons donc sélectionné 1 018 qui sont particulièrement denses en marqueurs hétérozygotes.
Nous avons ensuite enrichi l'ADN du sperme de cet hybride en fragments provenant de ces loci gr\^ace à une technique de ciblage spécifique suivie de séquençage haut-débit (capture-seq).

La deuxième étape de notre procédure consiste à génotyper les molécules séquencées de façon individuelle, et d'identifier, parmi ces dernières, celles correspondant à des événements de recombinaison.
Toute la difficulté de cette analyse réside dans le fait que les molécules sont uniques: dès lors, toute erreur de séquençage ou toute ambiguïté d'alignement peut devenir une source d'erreur à l'origine de faux positifs (i.e.\ de fragments détectés comme recombinants alors qu'ils ne le sont pas).
Lors de la mise en œuvre de notre approche, nous nous sommes rendus compte que les anomalies les plus critiques à cet égard provenaient de l'étape d'alignement car celui-ci est biaisé vers le génome de référence.
L'étape cruciale de notre méthode a donc été d'effectuer la procédure en utilisant successivement les deux génomes parentaux comme référence.

Au final, notre approche s'est révélée extrêmement performante.
A titre de comparaison, les études récentes ayant obtenu des cartes de recombinaison à haute résolution chez l'Homme, la souris ou l'oiseau \citep{halldorsson2016rate,smeds2016highresolution,li2018highresolution} se sont montrées plus de cent fois moins puissantes que notre approche pour détecter ces événements.\\


L'approche que nous avons mise au point nous a permis de détecter 18 821 événements de recombinaison chez la souris et donc de caractériser précisément la recombinaison sur environ un millier de points chauds (jusqu'alors, ceci n'avait été fait que sur une poignée de points chauds).

En premier lieu, nous avons pu observer l'étendue de la variation du taux de recombinaison entre les points chauds et identifier quelques uns de ses déterminants.
En particulier, l'affinité de liaison entre la protéine PRDM9 et son motif cible est parfaitement proportionnelle à l'activité recombinationnelle du point chaud.
Toutefois, les points chauds dont les deux haplotypes (celui venant de B6 et celui venant de CAST) présentent un différentiel d'affinité à PRDM9 important (les points chauds dits ‘asymétriques’) ont un taux de recombinaison fortement réduit (d'un facteur deux à quatre) par rapport à l'attendu basé sur l'intensité du signal PRDM9.

Un certain nombre d'événements de recombinaison (en particulier ceux dont le tract de conversion ne chevauche aucun marqueur polymorphe) ne sont pas détectables.
Dès lors, les paramètres de recombinaison observés — comme la longueur des tracts de conversion, le taux de recombinaison et le ratio de COs et de NCOs — ne sont pas forcément représentatifs des paramètres de recombinaison réels.
Pour pouvoir estimer ces paramètres réels, il est donc nécessaire de passer par des méthodes inférentielles telles que la méthode bayésienne approchée (\textit{approximate bayesian computation}) qui consiste à simuler le processus biologiques avec différents paramètres et à sélectionner les simulations dont le résultat est proche des observations biologiques.
Par ce biais, nous avons pu estimer de façon indirecte les paramètres de recombinaison chez la souris : les tracts de conversion des COs mesurent 450 paires de bases en moyenne contre 35 pour les NCOs, et le taux de recombinaison moyen sur l'ensemble des points chauds que nous avons étudié est de 30 cM/Mb.\\


Ensuite, en cherchant à quantifier le biais de transmission ($b_0$) des allèles GC et donc l'intensité du gBGC ($b$) chez la souris, nous avons remarqué que, dans un dispositif expérimental tel que le nôtre, ce biais était affecté par l'autre forme de conversion génique: le biais d'initiation (dBGC).
En effet, prenons le cas de deux populations possédant deux allèles \textit{Prdm9} distincts évoluant donc de façon indépendante dans leurs lignées respectives.
Dans chacune des lignées, les points chauds ciblés par l'allèle présent s'érodent sous l'effet du dBGC et s'enrichissent en même temps en allèles G et C sous l'effet du gBGC\@.
Lorsque l'on croise deux individus issus de ces deux lignées, l'allèle \textit{Prdm9} initie la cassure double-brin sur l'haplotype pour lequel il a la plus grande affinité, c'est-à-dire l'haplotype de la lignée avec laquelle il n'a \textit{pas} co-évolué, puisque celle dans laquelle il se trouvait a vu ses points chauds s'éroder.
Ainsi, c'est l'haplotype de sa lignée d'origine — qui est localement enrichi en GC — qui sera systématiquement le donneur lors de l'événement de conversion génique.
De ce fait, le gBGC qui a eu lieu dans les lignées parentales est propagé par un phénomène d’auto-stop génétique (\textit{genetic hitchhiking}) provenant du dBGC\@.

Pour pouvoir quantifier le gBGC correctement, il fallait donc contrôler pour cet effet d'auto-stop, ce que nous avons fait en sous-échantillonnant les tracts de conversion analysés pour égaliser le nombre d'événements de conversion ayant un donneur B6 à ceux ayant un donneur CAST\@.
Dès lors, nous avons pu quantifier le gBGC et observer que le biais de transmission ($b_0$) est nul pour les COs et extrêmement faible chez les NCOs contenant plusieurs marqueurs génétiques (NCO-2+). 
En revanche, le biais est très élevé pour les NCOs contenant un seul marqueur (NCO-1) : l'intensité du biais est comparable à ce qui a été observé chez l'humain \citep{halldorsson2016rate}.\\

% En effet, lorsque deux populations possédant deux allèles \textit{Prdm9} distincts évoluent de façon indépendante pendant un laps de temps suffisamment long pour que les points chauds ciblés par chaque allèle s'érodent dans leurs lignées respectives, croiser deux individus issus de ces deux lignées amènera forcément à une situation dans laquelle le g
%
% En effet, si deux populations possédant deux allèles \textit{Prdm9} distincts évoluent de façon indépendante pendant longtemps (relativement à la vitesse d'évolution)
%
% if two populations with distinct Prdm9 alleles have evolved independently during a length of time sufficient for the hotspots targeted by each allele to erode specifically in their lineage, crossing them together will end in dBGC hitchhiking past gBGC
%

% Cette quantification de l'intensité du biais chez la souris, qui est une espèce à forte taille efficace, nous a
A partir de là, nous avons pu comparer la relation entre l'intensité du gBGC ($b$) et la taille efficace de population ($N_e$) chez les deux espèces de mammifères pour lesquelles le biais de transmission ($b_0$) a été quantifié de façon directe : la souris et l'Homme.
Nos analyses indiquent que le taux de recombinaison et la longueur des tracts de conversion participent tous deux à limiter l'intensité du gBGC ($b$) chez la souris par rapport à l'Homme et, bien que les données disponibles à l'heure actuelle soient insuffisantes pour le confirmer, il semblerait le biais de transmission des COs y participe également.
% nous avions de nouveaux éléments permettant d'apporter quelques réponses à la question originelle de cette étude : la relation entre l'intensité du gBGC ($b$) et la taille efficace de population ($N_e$).
% Chez la souris

Globalement, ces observations sont compatibles avec l'hypothèse selon laquelle une pression de sélection limiterait l’intensité de ce processus délétère à l’échelle de la population par le biais d'une compensation de la taille efficace de population à de multiples niveaux : par le taux de recombinaison, par la longueur des tracts de conversion et, peut-être, par le biais de transmission des COs.\\


Enfin, la méthode de détection des recombinants à l'échelle d'individus uniques est tout indiquée pour étudier le rôle individuel de gènes impliqués dans le processus de recombinaison.
Pour ce faire, il faut analyser des individus homozygotes pour une version inactivée du gène d'intérêt mais présentant tout de même un haut niveau d'hétérozygotie pour que la recombinaison soit détectable.
Comme des individus F2 issus du croisement de trois lignées distinctes peuvent présenter de telles caractéristiques alors que des individus F1 issus d'un unique croisement ne le peuvent pas, il nous a fallu adapter notre méthode à un tel schéma de croisement.
% Concrètement, nous avons dû distinguer les marqueurs génétiques

Suite à cela, nous avons pu analyser le rôle du gène \textit{Hfm1}, une hélicase d'ADN essentielle à la résolution des cassures double-brin en COs : nous avons observé que son inactivation menait à un taux de recombinaison plus élevé et à des tracts de conversion de COs sensiblement plus courts que chez les individus non mutants.\\



Somme toute, notre travail a mené à la mise au point d'une approche originale de détection de la recombinaison à haute résolution et à faible coût chez des individus uniques.
Cette approche ouvre la voie à l'étude plus poussée des gènes impliqués dans le processus de recombinaison et nous a permis de mieux comprendre la façon dont la recombinaison et la conversion génique biaisée opèrent chez les mammifères.


% Ainsi, notre étude a permis d'observer l'étendue de la variation du taux de recombinaison entre les points chauds et

}





%
%
% inkscape
% contour rouge
% 310a33ff
% couleur rouge
% 993333ff
% couleur jaune
% efbc00ff
% contour jaune
% d59c00ff
% Image souris
% https://www.google.com/imgres?imgurl=https%3A%2F%2Funixtitan.net%2Fimages%2Fmice-clipart-silhouette-2.png&imgrefurl=https%3A%2F%2Funixtitan.net%2Fexplore%2Fmice-clipart-silhouette%2F&docid=WP3tqoNXVwdgBM&tbnid=lJVmSkpwVUXs6M%3A&vet=12ahUKEwjLgcqR1IbjAhVOzYUKHdNQC8c4rAIQMygVMBV6BAgBEBY..i&w=591&h=410&bih=848&biw=1860&q=house%20mouse&ved=2ahUKEwjLgcqR1IbjAhVOzYUKHdNQC8c4rAIQMygVMBV6BAgBEBY&iact=mrc&uact=8#h=410&imgdii=lJVmSkpwVUXs6M:&vet=12ahUKEwjLgcqR1IbjAhVOzYUKHdNQC8c4rAIQMygVMBV6BAgBEBY..i&w=591
% https://omaharentalads.com/explore/mice-clipart-silhouette/
%
%
%
%
%
 % Create an abstract.tex file in the 'text' folder for your abstract.
\end{abstractseparate}


% JEM: Pages are roman numbered from here, though page numbers are invisible until ToC.  This is in
% keeping with most typesetting conventions.
\begin{romanpages}

% Title page is created here
% \maketitle

%%%%% DEDICATION -- If you'd like one, un-comment the following.
%\begin{dedication}
%This thesis is dedicated to\\
%someone\\
%for some special reason\\
%\end{dedication}

%%%%% ACKNOWLEDGEMENTS -- Nothing to do here except comment out if you don't want it.
\begin{acknowledgements}%\otherpagedecoration
 	

% Cela fait vingt-sept années que je me demande régulièremenent ce que j'ai bien pu faire pour mériter de voir la chance me sourire si souvent, et ce doctorat n'a été qu'une occasion de plus de m'en étonner.
%
% C'est en effet une chance inouïe que de réaliser une thèse sous la direction de Laurent \textsc{Duret}.
% % Réaliser une thèse sous la direction de Laurent \textsc{Duret} a en effet été un cadeau inestimable.
% Son incomparable disponibilité et son souci de fournir un cadre de travail idéal et une formation complète dépassent largement ce que n'importe quel étudiant aurait pu espérer recevoir.
% Au delà de son encadrement scientifique excpetionnel, Laurent a fait preuve d'une humanité et d'une bienveillance rares et je ne crois pas que j'aurais pu arriver au bout de cette thèse sans son soutien pendant les périodes plus difficiles.
% Je ne crois pas que j'aurais pu arriver au bout de cette thèse sans son soutien et sans sa capacité à mettre en avant
% Il serait illégitime de ma part de ne pas dire que Laurent m'a littéralement appris à faire de la science, et pourtant, je crains qu'il ne me faille encore quelques décennies d'apprentissage à ses côtés pour pouvoir prétendre en faire réellement.
%
% Je voudrais donc lui dire ici toute la gratitude et toute l'admiration que je lui porte et espère ne pas avoir trop abusé de
%
%
%
% Laurent
% - cadre de travail idéal par sa bienveillance et son humanité
% - souci de donner une formation complète
%
%

Cela fait maintenant un peu plus de vingt-sept ans que, régulièrement, 
% je m'étonne d'être si chanceuse tant dans les rencontres que je fais et
je m'étonne de la chance que j'ai dans les rencontres que je fais et
% de la chance que j'ai dans les rencontres que je fais et dans les événements que je vis au quotidien et
% je me demande
% d'où vient
% ce que j'ai bien pu faire pour mériter de voir la chance me sourire autant et
force est de constater que ce doctorat n'aura été qu'une occasion de plus de la mesurer.

C'était en effet une aubaine inouïe que de pouvoir réaliser ma thèse sous la direction de Laurent \textsc{Duret}.
Sa disponibilité manifestement infinie, son souci de fournir un cadre idéal de travail et de vie à ses étudiants, sa détermination à leur prodiguer une formation complète et l'extraordinaire bienveillance dont il fait preuve chaque jour vont bien au-delà de ce que tout doctorant normalement constitué aurait pu espérer recevoir.
Je crois qu'il serait même illégitime de ma part de ne pas dire que Laurent m'a littéralement appris à faire de la science.
Mais, au delà de son encadrement scientifique exceptionnel, c'est son humanité que je voudrais saluer.
Je n'arrive pas à m'imaginer comment j'aurais pu arriver au bout de cette thèse sans l'immense générosité dont il a fait preuve ni sans sa capacité à me redonner espoir dans les moments plus difficiles.
%mettre en avant les réussites de ce travail lorsque je pensais avoir 
% qui m'a redonné espoir dans les moments plus difficiles.
Je voudrais donc lui dire ici toute l'admiration que je lui porte ainsi que toute ma gratitude pour les conditions privilégiées dont il m'a fait bénéficier pendant ces trois dernières années.
 % et espère ne pas avoir trop abusé des conditions privilégiées qu'il m'a offertes pour faire ce travail.

Je voudrais aussi exprimer mes sincères remerciements aux personnes qui m'ont fait l'honneur d'accepter de lire et d'évaluer ce travail : Valérie \textsc{Borde}, Adam \textsc{Eyre-Walker}, Bertrand \textsc{Llorente}, Dominique \textsc{Mouchiroud} et Gwenael \textsc{Piganeau}.

Merci également à ceux qui ont accepté de participer à mon comité de suivi de thèse pour leurs conseils tant sur les aspects scientifiques que personnels : 
Nicolas \textsc{Galtier},
Annabelle \textsc{Haudry},
Bernard \textsc{de Massy},
Jonathan \textsc{Romiguier} et
Tristan \textsc{Lefébure}.


Le travail qui est présenté dans ce manuscrit est le fruit de collaborations qui ont impliqué un grand nombre de personnes que je me dois également de remercier.
Merci à Frédéric \textsc{Baudat}, Valérie \textsc{Borde}, Corinne \textsc{Grey} et Bernard \textsc{de Massy} d'avoir réalisé la totalité du travail expérimental qui a été essentiel pour cette thèse, pour la patience dont ils ont fait preuve à mon égard et pour leurs suggestions éclairées.
Merci aussi à Nicolas \textsc{Lartillot} pour ses commentaires avisés et à Brice \textsc{Letcher} pour le travail admirable qu'il a réalisé pendant son stage et dont je me suis allègrement servie pour rédiger cette thèse.\\



La sympathie et la bienveillance de l'ensemble des membres du laboratoire rendent la qualité de vie au laboratoire exceptionnelle.
Ils sont trop nombreux pour être tous remerciés de façon individuelle, mais je voudrais tout de même en mentionner certains.
% Ils sont si nombreux que je risque d'en oublier certains, mais je voudrais exprimer ma reconnaissance

% L'environnement de travail exceptionnel dans lequel j'ai pu réaliser cette thèse est en grande partie due à l'ensemble des membres du laboratoire dont la sympathie et la bienveillance rendent


L'une des personnes qui a le plus compté pour moi pendant ces trois années est Anouk \textsc{Nec\c{s}ulea}.
 % est une des personnes qui a le plus compté pour moi pendant ces trois années passées au laboratoire.
L'intérêt qu'Anouk porte aux autres ne représente qu'une des nombreuses facettes de sa générosité et je dois dire que sans celui qu'elle m'a porté, son soutien, ses conseils et sa gentillesse, j'aurais peut-être abandonné en cours de route.
Je veux la remercier du fond du cœur d'avoir porté mes difficultés avec moi et d'avoir, chaque fois que j'en ai eu besoin, pris le temps de m'écouter.


J'ai également pu compter sur la gentillesse de plusieurs membres du ‘club 13h’ que je tiens à remercier particulièrement pour leur convivialité et leur soutien :
Hélène \textsc{Badouin},
Anna \textsc{Bonnet},
Florian \textsc{Benitière},
Cyril \textsc{Fournier},
Diego \textsc{Hartas\'anchez Frenk},
Thibault \textsc{Latrille},
Alexandre \textsc{Laverré}, Alexia \textsc{Nguyen Trung},
Djivan \textsc{Prentout},
Théo \textsc{Tricou} et
Philippe \textsc{Veber}.




Je tiens à remercier tout particulièrement ceux avec qui j'ai eu la chance de partager le bureau dans une bonne humeur quotidienne : Hélène \textsc{Badouin}, Jérémy \textsc{Ganofsky}, Nicolas \textsc{Lartillot}, Thibault \textsc{Latrille}, Michel \textsc{Lecocq} et Aline \textsc{Muyle}.
Merci également à ceux qui m'ont accueillie dans le leur pendant mes dernières semaines de rédaction : Claire \textsc{Gayral}, Alexandre \textsc{Laverré}, Alexia \textsc{Nguyen Trung}, Christine \textsc{Oger}, Philippe \textsc{Veber} et de façon temporaire Marie \textsc{Cariou}.

Je me dois aussi de remercier l'ensemble des doctorants avec qui j'ai partagé bien des joies et des peines : 
% ceux qui terminaient leur thèse lorsque je commençais la mienne :
% Adr\'ian \textsc{Arellano Dav\'in},
% Wandrille \textsc{Duchemin},
% Cécile \textsc{Fruchard},
Adr\'ian \textsc{Arellano Dav\'in}, 
Samuel \textsc{Barreto},
Guillaume \textsc{Carillo},
Wandrille \textsc{Duchemin},
Cécile \textsc{Fruchard},
Thibault \textsc{Latrille},
Alexandre \textsc{Laverré},
Vincent \textsc{Mérel},
Alexia \textsc{Nguyen Trung},
Djivan \textsc{Prentout},
\'Elise \textsc{Say-Sallaz} et
Théo \textsc{Tricou}.
Merci aussi à ceux avec lesquels j'ai moins interagi mais avec qui il était toujours agréable de discuter : 
Monique \textsc{Aouad},
Magali \textsc{Dancette},
Ghislain \textsc{Durif},
Pierre \textsc{Garcia} et Anne \textsc{Oudart}.
Je fais un clin d'œil particulier à Carine \textsc{Rey} que j'ai mieux connue ces dernières semaines lors de nos soirées communes de rédaction.

Merci aussi aux stagiaires tous plus sympathiques les uns que les autres : Mathieu \textsc{Brevet}, \'Elisa \textsc{Denier}, Anne-Laure \textsc{Fuchs}, Jérémy \textsc{Ganofsky}, Marie \textsc{Guidoni}, Garance \textsc{Lapetoule} et Brice \textsc{Letcher}.


Je tiens à remercier plus particulièrement tous ceux qui ont fait de ce laboratoire un lieu privilégié de transmission du savoir au travers des formations passionnantes qu'ils ont animées : 
Bastien \textsc{Boussau}, Marie-Laure \textsc{Delignette-Muller}, Nicolas \textsc{Lartillot}, Fabien \textsc{Subtil} et Philippe \textsc{Veber} pour la formation en statistiques bayésiennes, Laurent \textsc{Jacob} pour celle en machine learning et Vincent \textsc{Lanore} et François \textsc{Gindraud} pour la transmission de leurs connaissances en C++.



Je ne serais pas allée bien loin dans cette thèse sans l'aide précieuse des membres du pôle informatique : 
Adil \textsc{El-Filali},
Lionel \textsc{Humblot},
Vincent \textsc{Miele},
Simon \textsc{Penel} et
Bruno \textsc{Spataro}.
Je tiens à exprimer en particulier toute ma gratitude à Stéphane \textsc{Delmotte} que j'ai sollicité à une fréquence s'apparentant à du harcèlement et qui a systématiquement trouvé des solutions à tous mes problèmes.

Merci aussi à l'équipe du pôle administratif pour leur efficacité et leur sympathie au quotidien : 
Nathalie \textsc{Arbasetti},
Laetitia \textsc{Mangeot},
Odile \textsc{Mulet-Marquis} et
Aurélie \textsc{Zerfass}.

Finalement, je voudrais remercier les autres membres permanents et contractuels qui créent l'ambiance chaleureuse du laboratoire par leur bienveillance et leur sympathie : 
Céline \textsc{Brochier-Armanet},
Bastien \textsc{Boussau},
Kelly \textsc{Bradley},
Sylvain \textsc{Charlat},
Jonathan \textsc{Corbi},
Vincent \textsc{Daubin},
Jean-Marie \textsc{Delpuech},
Damien \textsc{de Vienne},
Jean-Pierre \textsc{Flandrois},
Amandine \textsc{Fournier}
Guillaume \textsc{Gence},
Manolo \textsc{Gouy},
Dominique \textsc{Guyot},
Laurent \textsc{Guéguen},
Jos \textsc{Käfer},
Daniel \textsc{Kahn},
Bénédicte \textsc{Lafay},
Gabriel \textsc{Marais},
Florian \textsc{Massip},
Dominique \textsc{Mouchiroud},
Guy \textsc{Perrière},
Héloïse \textsc{Philippon},
Franck \textsc{Picard},
Diamantis \textsc{Sellis},
Marie \textsc{Sémon},
\'Eric \textsc{Tannier},
Najwa \textsc{Taib} et
Raquel \textsc{Tavares}.\\




% - les personnes en conf
% Victoire \textsc{Baillet}
% Julie \textsc{Clément}
% Guillaume \textsc{Achaz}
% Catherine \textsc{Breton}\\
%




%%% NOUVELLE PARTIE
Je porte une telle admiration pour le métier d'enseignant que l'un de mes plus grands rêves était de pouvoir l'exercer moi-même un jour.

Je voudrais donc exprimer toute ma reconnaissance à Dominique \textsc{Mouchiroud} de m'avoir permis de le réaliser ainsi que pour la confiance qu'elle m'a accordée dans la réalisation de mes enseignements.
Il a été très agréable de farei ceux-ci en collaboration avec Hélène \textsc{Badouin}, Annabelle \textsc{Haudry}, Héloïse \textsc{Philippon} et Raquel \textsc{Tavares} et je tiens donc à les en remercier.
Merci aussi à mes étudiants de M2 auprès desquels j'ai tant appris et qui m'ont permis de vivre l'une des expériences les plus jouissives et enrichissantes de ma vie.

Beaucoup des enseignants que j'avais moi-même eus par le passé étaient excellents et j'aimerais en remercier tout particulièrement quelques uns dont les enseignements ont joué, me semble-t-il, un rôle important dans l'aboutissement de cette thèse : 
merci à M.\ \textsc{Martin} et à Mlle \textsc{Mirmand}, professeurs d'anglais en prépa et au collège, sans la contribution desquels il m'aurait été bien difficile de rédiger mon manuscrit; 
merci à M.\ \textsc{Fontaine}, professeur d'histoire au collège, qui m'a donné goût à l'étude du passé et auquel j'ai beaucoup pensé lors de l'écriture de mon premier chapitre; 
merci à M.\ Beaux et à Mme Mollière, professeurs de biologie en prépa, qui ont sur faire de la génétique un sujet passionnant.

Les personnes qui m'ont encadré pendant mes différents stages m'ont largement donné les clés permettant la bonne réalisation de cette thèse et je voudrais donc les en remercier : 
\'Edouard \textsc{Bove},
Claudia \textsc{Chica},
Juliette \textsc{de Meaux},
Florine \textsc{Poiroux},
Loïc \textsc{Rajjou} et
François \textsc{Roudier}.\\

% Loïc Rajjou, en particulier, a eu un impact majeur sur le cours qu'a pris ma vie professionnelle.

% Parmi toutes les personnes que j'ai pu rencontrer, Loïc Rajjou est certainement celui qui a eu le plus d'impact sur le cours de ma vie professionnelle.






%%%% NOUVELLE PARTIE
Je voudrais terminer en remerciant ceux qui m'ont permis de garder une santé mentale à peu près normale en me rappelant que la vie ne s'arrête pas à l'enceinte du laboratoire.

Merci donc à mes amis plongeurs du \textsc{Coolapic} qui sont bien trop nombreux pour être mentionnés individuellement mais qui ont été une vraie bulle d'oxygène lors de nos sorties en mer.
Je voudrais aussi dire toute ma gratitude à mes camarades de thé\^atre, en particulier Vérane \textsc{Lyon}, Marie \textsc{Pinel}, Hannah \textsc{Samama} et Estelle \textsc{Valette}, pour nos éclats de rires ainsi qu'à mes amis parisiens venus passer un ou plusieurs week-ends à mes côtés dans la contrée Lyonnaise : Alicia \textsc{Berret}, Yann \textsc{Boulestreau}, Lucie \textsc{Gomez}, Mathieu \textsc{Laugier}, Ulysse \textsc{Le Goff}, Jordane \textsc{Lelong}, Anne \textsc{Rabault} et Anne \textsc{Schneider}.


Mes parents sont très certainement les deux personnes les plus formidables qu'il m'ait été donné de rencontrer, et je n'ose imaginer la tournure qu'aurait pris ma vie s'ils n'avaient pas été là.
Ils ont toujours placé les intérêts de leurs quatre enfants bien avant les leurs et leur abnégation, leur amour et leur soutien inconditionnel font d'eux des modèles de vie pour moi.

% tout ce dont on pouvait avoir besoin matériellement, intellectuellement et psychologiquement
% On peut difficilement rêver d'avoir de meilleurs parents qu'eux

Mais ce cadre familial n'aurait pas été complet sans mes frères et leurs compagnes, Arthur et Marie-\'Elise, Ronan et Anna, ainsi que ma sœur Lévana que je me dois de remercier pour l'indulgence dont ils ont fait preuve à l'égard de mes moments d'anxiété, pour leur humour et pour la joie qu'ils m'apportent au quotidien.
Merci en particulier à Arthur d'avoir porté mes difficultés sur ses épaules, à Ronan de m'avoir ouvert l'esprit sur d'autres sciences et à Lévana de m'avoir toujours soutenue avec une constance et un dévouement qui forcent l'admiration.

Merci, enfin, à Alexandre \textsc{Chaintreuil} d'avoir su systématiquement transformer mes angoisses en rires, mes larmes en joie et mes craintes en doses de confiance en moi.
Qu'il continue à le faire quelques décennies de plus me comblerait.\\

\hfill \textit{Villeurbanne, le 26 juillet 2019}

% ainsi que pour les innombrables heures qu'il a passées dans le train.
% mes larmes en joie
% mes préoccupations en perspectives
% mes larmes de désespoir
%
% angoisses
% doutes
% peurs
% craintes
%
% rires
%







\end{acknowledgements}

%%%%% ABSTRACT -- Nothing to do here except comment out if you don't want it.
\begin{abstract}%\otherpagedecoration
	

\section*{Abstract}

During meiosis, recombination hotspots host the formation of DNA double-strand breaks (DSBs). DSBs are subsequently repaired through a process which, in a wide range of species, is biased towards the favoured transmission of G and C alleles: GC-biased gene conversion (gBGC).
The intensity of this fundamental distorter of meiotic segregation strongly varies between species but the factors dictating its evolution are not known.
We thus aimed at directly quantifying the transmission bias in mice and comparing the parameters on which it depends with other mammals.
% To better understand this fundamental distorter of meiotic segregation, we aimed at directly quantifying the transmission bias in mice and comparing the parameters on which in depends with other mammals.
% Better understanding this fundamental distorter of meiotic segregation thus requires to directly quantify the transmission bias and compare the parameters on which in depends across several species.
% To identify the latter and thus better understand this fundamental distorter of meiotic segregation, we aimed at directly quantifying the transmission bias in mice and comparing the parameters on which in depends with other mammals.
% This fundamental distorter of meiotic segregation
% proceeds with distinct intensities
% plays a major role in genome evolutiona
% mimics the action of positive selection and plays a major role in the evolution of base composition in the vicinity of recombination hotspots.

Here, we coupled capture-seq and bioinformatic techniques to implement an approach that proved 100 times more powerful than current methods to detect recombination. With it, we identified 18,821 crossing-over (CO) and non-crossover (NCO) events at very high resolution in single individuals and could thus precisely characterise patterns of recombination in mice.
In this species, recombination hotspots are targeted by PRDM9 and are therefore subject to a second type of biased gene conversion (BGC): DSB-induced BGC (dBGC). Quantifying both dBGC and gBGC with our data brought to light the fact that, in cases of structured populations, past gBGC from the parental lineages is hitchhiked by dBGC when the populations cross.
% Next, we dissociated the hitchhiking effect to directly measure the intensity of gBGC ($b$) in both COs and NCOs.
We next observed that, in male mice, only NCOs — and more particularly single-marker NCOs — contribute to the intensity of gBGC. In contrast, in humans, both NCOs and at least a portion of COs (those with complex conversion tracts) distort allelic frequencies. 
This suggests that the DSB repair machinery leading to gBGC varies across mammals.
% has evolved extremely rapidly within the mammalian clade.
Our findings are also consistent with the hypothesis of a selective pressure restraining the intensity of the deleterious gBGC process at the population-scale: this would materialise through a multi-level compensation of the effective population size by the recombination rate, the length of conversion tracts and the transmission bias.

Altogether, our work has allowed to better comprehend how recombination and biased gene conversion proceed in the mammalian clade.\\

\textbf{Keywords:} Recombination, Biased gene conversion, PRDM9, Hotspots, Genomics, Molecular evolution, Mammals, Sperm-typing.


\newpage
\section*{Résumé en français}

Au cours de la méiose, les points chauds de recombinaison sont le siège de la formation de cassures double-brin de l’ADN. Ces dernières sont ensuite réparées par un processus qui, chez de nombreuses espèces, favorise la transmission des allèles G et C : la conversion génique biaisée vers GC (gBGC). 
L'intensité de cet important distorteur de la ségrégation méiotique varie fortement entre espèces mais les facteurs déterminant son évolution sont toujours inconnus. 
Nous avons donc voulu quantifier directement le biais de transmission chez la souris et comparer les paramètres dont il dépend avec d'autres mammifères.
% Cet important distorteur de la ségrégation méiotique mime les effets de la sélection positive et joue un rôle majeur dans l’évolution de la composition nucléotidique au voisinage des points chauds.

Dans cette étude, en couplant des développements bioinformatiques à une technique de capture ciblée d’ADN suivie de séquençage haut-débit (capture-seq), nous avons réussi à mettre au point une approche qui s’est révélée 100 fois plus performante pour détecter les événements de recombinaison que les méthodes existant actuellement. Ainsi, nous avons pu identifier 18 821 crossing-overs (COs) et non-crossovers (NCOs) à très grande résolution chez des individus uniques, ce qui nous a permis de caractériser minutieusement la recombinaison chez la souris.
Chez cette espèce, les points chauds de recombinaison sont ciblés par la protéine PRDM9 et sont donc soumis à une deuxième forme de conversion génique biaisée (BGC) : le biais d’initiation (dBGC). La quantification du dBGC et du gBGC à partir de nos données nous a permis de mettre en lumière le fait que, au moment où des populations structurées s’hybrident, le gBGC des lignées parentales est propagé par un phénomène d’auto-stop génétique (genetic hitchhiking) provenant du dBGC.
% Ensuite, nous avons dissocié ce phénomène d’auto-stop pour mesurer directement l’intensité du gBGC ($b$) dans les COs et les NCOs.
Nous avons ensuite pu observer que, chez les souris m\^ales, seuls les NCOs — et plus particulièrement les NCOs contenant un seul marqueur génétique— contribuent à l'intensité du gBGC. En comparaison, chez l’Homme, à la fois les NCOs et au moins une part des COs (ceux qui présentent des tracts de conversion complexes) distordent les fréquences alléliques. 
Ceci suggère que la machinerie de réparation des cassures double-brin qui induit le biais de conversion génique (BGC) présente des variations au sein des mammifères.
%a évolué extrêmement rapidement au sein des mammifères. 
Nos résultats sont aussi en accord avec l’hypothèse selon laquelle une pression de sélection limiterait l’intensité de ce processus délétère à l’échelle de la population. Cela se traduirait par une compensation de la taille efficace de population à de multiples niveaux : par le taux de recombinaison, par la longueur des tracts de conversion et par le biais de transmission.

Somme toute, notre travail a permis de mieux comprendre la façon dont la recombinaison et la conversion génique biaisée opèrent chez les mammifères.\\

\textbf{Mots-clés:} Recombinaison, Conversion génique biaisée, PRDM9, Points chauds, Génomique, \'Evolution moléculaire, Mammifères, Sperm-typing.

\newpage
\section*{Résumé étendu en français}

{
\setstretch{1.15}

Lorsque l'on traite de l'évolution des génomes, trois forces sont classiquement invoquées : la mutation, la sélection naturelle et la dérive génétique.
% La première est la source
% La deuxième
% La troisième
%
Toutefois, depuis une vingtaine d'année, une autre force a fait son entrée sur la scène évolutive~: la conversion génique biaisée, que nous noterons ‘BGC’ (de l'anglais \textit{biased gene conversion}).
Ce phénomène est une conséquence directe du processus de recombinaison méiotique chez les espèces à reproduction sexuée.

Chez les mammifères en effet, après s'être fixée à certains loci cibles appelés ‘points chauds de recombinaison’, la protéine PRDM9 recrute la machinerie de formation de cassures double-brin et marque, de ce fait, l'initiation d'un événement de recombinaison \citep{baudat2010prdm9,myers2010drive,parvanov2010prdm9}.
Ce dernier doit ensuite être réparé en utilisant le chromosome homologue comme matrice, ce qui mène à ce qu'on appelle un événement de conversion génique, c'est-à-dire le transfert non-réciproque d'une information de séquence d'ADN\@.

Toutefois, si PRDM9 présente une plus grande affinité de liaison avec la séquence de l'un des deux chromosomes (que nous appellerons ‘haplotype’), la cassure s'initiera préférentiellement sur cet haplotype, et l'événement de conversion génique se fera donc préférentiellement dans un sens donné : c'est ce qu'on appelle le biais d'initiation, aussi appelé conversion génique biaisée induite par cassure double brin et noté ‘dBGC’ (de l'anglais \textit{double-strand break-induced biased gene conversion}).
Du fait de ce phénomène, les points chauds finissent nécessairement par s'éroder : comme l'haplotype portant le motif ciblé par PRDM9 est le siège de la cassure, il est systématiquement converti par l'autre haplotype, et voué à dispara\^itre \citep{boulton1997hotspot}.

Il existe une deuxième forme de conversion génique biaisée : la conversion génique biasée vers GC, que l'on notera ‘gBGC’ (de l'anglais \textit{GC-biased gene conversion}).
En effet, il a été observe chez plusieurs espèces 
de façon directe \citep{mancera2008highresolution, si2015widely, williams2015noncrossover, halldorsson2016rate, keith2016high, smeds2016highresolution}
ou indirecte \citep{escobar2011gcbiased,pessia2012evidence,figuet2014biased}
que la réparation des cassures double-brin favorise les allèles G et C par rapport aux allèles A et T\@.\\


% les événements de recombinaison sont initiés par la formation de cassures double-brin, dont la position est déterminée par la protéine PRDM9.
% Après s'être fixée à certains loci selon son affinité de liaison avec ceux-ci, cette dernière recrute la machinerie de formation des cassures double-brin.
% Suite à cela,
% Cette dernière se lie à certains loci spécifiques d'autant plus fortement que son affinité de liaison avec eux est élevée et
%
%
% En effet, les événements de recombinaison sont initiés par la formation d'une cassure double-brin qui est ensuite réparée gr\^ace à l'action d'une machinerie de réparation de ces cassures.
% Or, la position de ces cassures est déterminée par la protéine PRDM9 qui se lie d'autant plus fortement à certains loci que son affinité de liaison avec ceux-ci est forte, et recrute
% en fonction de son affinité de liaison avec certains loci,

% - une consequence : si des mutations sur un des haplotypes, PRDM9 se lie plus sur celui non mute, et donc, le mute est donneur dans conversion. C'est le paradoxe des hotspots
% - cela mene a dBGC = biais d'initiation, dont une des consequences est lerosion des hotspots

% - aussi une autre forme: lors de la reparation, il a ete observe que l'allele GC est souvent favorise chez many species: on appelle ca le gBGC






La quantification du coefficient de conversion génique biaisée à l'échelle des populations ($B$) chez un grand nombre de métazoaires \citep{galtier2018codon} a mis en évidence un résultat étonnant: 
l'intensité du gBGC ne varie que dans une gamme de valeurs très restreinte.
Par exemple, chez les mammifères placentaires, $B$ reste dans une fourchette de 0 à 7 \citep{lartillot2013phylogenetic}.
\'Etant donné que $B$ correspond au produit de la taille efficace de population ($N_e$) par le coefficient de gBGC ($b$) et que la taille efficace peut varier sur plusieurs ordres de grandeurs parmi les métazoaires, $b$ ne peut mécaniquement pas être identique chez toutes les espèces.
Au contraire, un ou plusieurs des paramètres dont $b$ dépend (le taux de recombinaison $r$, la longueur des tracts de conversion $L$ et le biais de transmission $b_0$) varient nécessairement inversement à la taille efficace.


Cependant, peu de données sont disponibles pour comprendre la base de la dépendance entre $N_e$ et $b$: le biais de transmission ($b_0$) n'a été mesuré que chez quelques espèces \citep{mancera2008highresolution, si2015widely, williams2015noncrossover, halldorsson2016rate, keith2016high, smeds2016highresolution} et, parmi les mammifères, la seule espèce chez qui ce biais a été mesuré de façon directe (\textit{Homo sapiens}) présente une très faible taille efficace d'environ 10,000 \citep{takahata1993allelic,erlich1996hla,harding1997archaic,charlesworth2009fundamental,yu2004nucleotide}.

Afin d'apporter un éclairage nouveau sur l'interaction entre $b$ et $N_e$, nous avons donc voulu quantifier le gBGC chez une autre espèce de mammifères présentant une taille efficace beaucoup plus grande que celle de l'Homme \citep{geraldes2008inferring,phifer-rixey2012adaptive,davies2015factors}: la souris \textit{Mus musculus}.\\


Pour pouvoir quantifier précisément le gBGC, il est nécessaire de disposer d'un grand nombre d'événements de recombinaison.
Or, la méthode généralement utilisée pour détecter ces événements — l'analyse de pedigrees — est extrêmement gourmande en ressources : 
elle requiert le séquençage de génomes complets d'un grand nombre d'individus et permet de détecter seulement un nombre limité de recombinants.
Nous avons donc mis au point une nouvelle approche permettant de détecter plusieurs milliers de recombinants à très haute résolution chez des individus uniques.

% De plus, afin de maximiser le nombre de recombinants détectables, nous avons sélectionné 1 018 points chauds de recombinaison particulièrement denses en marqueurs hétérozygotes
% Concrètement, notre approche repose sur le génotypage de molécules d'ADN uniques issues du sperme de souris hybrides.
% Afin de maximiser le nombre de recombinants détectables, nous avons, au préalable, réalisé une étape de capture ciblée d'ADN provenant de 1 018 points chauds de recombinaison particulièrement denses en marqueurs hétérozygotes.
% Brièvement, notre approche repose sur le génotypage de molécules d'ADN uniques issues du sperme de souris hybrides enrichi en événements de recombinaison gr\^ace au ciblage spécifique de 1 018 points chauds de recombinaison denses en sites polymorphes.
Concrètement, notre approche repose sur deux étapes principales.
Premièrement, puisque la recombinaison n'est identifiable qu'à partir du génotypage de marqueurs hétérozygotes, nous avons croisé deux races de souris (C57BL/6J que nous noterons ‘B6’ et CAST/EiJ que nous appellerons ‘CAST’) issues de deux sous-espèces (\textit{Mus musculus domesticus} et \textit{Mus musculus castaneus}) présentant un fort taux de polymorphisme de 0.74\% \citep{keane2011mouse,yalcin2012nextgeneration}.
Les points chauds de recombinaison chez l'hybride F1 qui résulte de ce croisement (B6xCAST) ont déjà été identifiés par d'autres que nous \citep{baker2015prdm9}.
Afin de maximiser le nombre de recombinants détectables, nous en avons donc sélectionné 1 018 qui sont particulièrement denses en marqueurs hétérozygotes.
Nous avons ensuite enrichi l'ADN du sperme de cet hybride en fragments provenant de ces loci gr\^ace à une technique de ciblage spécifique suivie de séquençage haut-débit (capture-seq).

La deuxième étape de notre procédure consiste à génotyper les molécules séquencées de façon individuelle, et d'identifier, parmi ces dernières, celles correspondant à des événements de recombinaison.
Toute la difficulté de cette analyse réside dans le fait que les molécules sont uniques: dès lors, toute erreur de séquençage ou toute ambiguïté d'alignement peut devenir une source d'erreur à l'origine de faux positifs (i.e.\ de fragments détectés comme recombinants alors qu'ils ne le sont pas).
Lors de la mise en œuvre de notre approche, nous nous sommes rendus compte que les anomalies les plus critiques à cet égard provenaient de l'étape d'alignement car celui-ci est biaisé vers le génome de référence.
L'étape cruciale de notre méthode a donc été d'effectuer la procédure en utilisant successivement les deux génomes parentaux comme référence.

Au final, notre approche s'est révélée extrêmement performante.
A titre de comparaison, les études récentes ayant obtenu des cartes de recombinaison à haute résolution chez l'Homme, la souris ou l'oiseau \citep{halldorsson2016rate,smeds2016highresolution,li2018highresolution} se sont montrées plus de cent fois moins puissantes que notre approche pour détecter ces événements.\\


L'approche que nous avons mise au point nous a permis de détecter 18 821 événements de recombinaison chez la souris et donc de caractériser précisément la recombinaison sur environ un millier de points chauds (jusqu'alors, ceci n'avait été fait que sur une poignée de points chauds).

En premier lieu, nous avons pu observer l'étendue de la variation du taux de recombinaison entre les points chauds et identifier quelques uns de ses déterminants.
En particulier, l'affinité de liaison entre la protéine PRDM9 et son motif cible est parfaitement proportionnelle à l'activité recombinationnelle du point chaud.
Toutefois, les points chauds dont les deux haplotypes (celui venant de B6 et celui venant de CAST) présentent un différentiel d'affinité à PRDM9 important (les points chauds dits ‘asymétriques’) ont un taux de recombinaison fortement réduit (d'un facteur deux à quatre) par rapport à l'attendu basé sur l'intensité du signal PRDM9.

Un certain nombre d'événements de recombinaison (en particulier ceux dont le tract de conversion ne chevauche aucun marqueur polymorphe) ne sont pas détectables.
Dès lors, les paramètres de recombinaison observés — comme la longueur des tracts de conversion, le taux de recombinaison et le ratio de COs et de NCOs — ne sont pas forcément représentatifs des paramètres de recombinaison réels.
Pour pouvoir estimer ces paramètres réels, il est donc nécessaire de passer par des méthodes inférentielles telles que la méthode bayésienne approchée (\textit{approximate bayesian computation}) qui consiste à simuler le processus biologiques avec différents paramètres et à sélectionner les simulations dont le résultat est proche des observations biologiques.
Par ce biais, nous avons pu estimer de façon indirecte les paramètres de recombinaison chez la souris : les tracts de conversion des COs mesurent 450 paires de bases en moyenne contre 35 pour les NCOs, et le taux de recombinaison moyen sur l'ensemble des points chauds que nous avons étudié est de 30 cM/Mb.\\


Ensuite, en cherchant à quantifier le biais de transmission ($b_0$) des allèles GC et donc l'intensité du gBGC ($b$) chez la souris, nous avons remarqué que, dans un dispositif expérimental tel que le nôtre, ce biais était affecté par l'autre forme de conversion génique: le biais d'initiation (dBGC).
En effet, prenons le cas de deux populations possédant deux allèles \textit{Prdm9} distincts évoluant donc de façon indépendante dans leurs lignées respectives.
Dans chacune des lignées, les points chauds ciblés par l'allèle présent s'érodent sous l'effet du dBGC et s'enrichissent en même temps en allèles G et C sous l'effet du gBGC\@.
Lorsque l'on croise deux individus issus de ces deux lignées, l'allèle \textit{Prdm9} initie la cassure double-brin sur l'haplotype pour lequel il a la plus grande affinité, c'est-à-dire l'haplotype de la lignée avec laquelle il n'a \textit{pas} co-évolué, puisque celle dans laquelle il se trouvait a vu ses points chauds s'éroder.
Ainsi, c'est l'haplotype de sa lignée d'origine — qui est localement enrichi en GC — qui sera systématiquement le donneur lors de l'événement de conversion génique.
De ce fait, le gBGC qui a eu lieu dans les lignées parentales est propagé par un phénomène d’auto-stop génétique (\textit{genetic hitchhiking}) provenant du dBGC\@.

Pour pouvoir quantifier le gBGC correctement, il fallait donc contrôler pour cet effet d'auto-stop, ce que nous avons fait en sous-échantillonnant les tracts de conversion analysés pour égaliser le nombre d'événements de conversion ayant un donneur B6 à ceux ayant un donneur CAST\@.
Dès lors, nous avons pu quantifier le gBGC et observer que le biais de transmission ($b_0$) est nul pour les COs et extrêmement faible chez les NCOs contenant plusieurs marqueurs génétiques (NCO-2+). 
En revanche, le biais est très élevé pour les NCOs contenant un seul marqueur (NCO-1) : l'intensité du biais est comparable à ce qui a été observé chez l'humain \citep{halldorsson2016rate}.\\

% En effet, lorsque deux populations possédant deux allèles \textit{Prdm9} distincts évoluent de façon indépendante pendant un laps de temps suffisamment long pour que les points chauds ciblés par chaque allèle s'érodent dans leurs lignées respectives, croiser deux individus issus de ces deux lignées amènera forcément à une situation dans laquelle le g
%
% En effet, si deux populations possédant deux allèles \textit{Prdm9} distincts évoluent de façon indépendante pendant longtemps (relativement à la vitesse d'évolution)
%
% if two populations with distinct Prdm9 alleles have evolved independently during a length of time sufficient for the hotspots targeted by each allele to erode specifically in their lineage, crossing them together will end in dBGC hitchhiking past gBGC
%

% Cette quantification de l'intensité du biais chez la souris, qui est une espèce à forte taille efficace, nous a
A partir de là, nous avons pu comparer la relation entre l'intensité du gBGC ($b$) et la taille efficace de population ($N_e$) chez les deux espèces de mammifères pour lesquelles le biais de transmission ($b_0$) a été quantifié de façon directe : la souris et l'Homme.
Nos analyses indiquent que le taux de recombinaison et la longueur des tracts de conversion participent tous deux à limiter l'intensité du gBGC ($b$) chez la souris par rapport à l'Homme et, bien que les données disponibles à l'heure actuelle soient insuffisantes pour le confirmer, il semblerait le biais de transmission des COs y participe également.
% nous avions de nouveaux éléments permettant d'apporter quelques réponses à la question originelle de cette étude : la relation entre l'intensité du gBGC ($b$) et la taille efficace de population ($N_e$).
% Chez la souris

Globalement, ces observations sont compatibles avec l'hypothèse selon laquelle une pression de sélection limiterait l’intensité de ce processus délétère à l’échelle de la population par le biais d'une compensation de la taille efficace de population à de multiples niveaux : par le taux de recombinaison, par la longueur des tracts de conversion et, peut-être, par le biais de transmission des COs.\\


Enfin, la méthode de détection des recombinants à l'échelle d'individus uniques est tout indiquée pour étudier le rôle individuel de gènes impliqués dans le processus de recombinaison.
Pour ce faire, il faut analyser des individus homozygotes pour une version inactivée du gène d'intérêt mais présentant tout de même un haut niveau d'hétérozygotie pour que la recombinaison soit détectable.
Comme des individus F2 issus du croisement de trois lignées distinctes peuvent présenter de telles caractéristiques alors que des individus F1 issus d'un unique croisement ne le peuvent pas, il nous a fallu adapter notre méthode à un tel schéma de croisement.
% Concrètement, nous avons dû distinguer les marqueurs génétiques

Suite à cela, nous avons pu analyser le rôle du gène \textit{Hfm1}, une hélicase d'ADN essentielle à la résolution des cassures double-brin en COs : nous avons observé que son inactivation menait à un taux de recombinaison plus élevé et à des tracts de conversion de COs sensiblement plus courts que chez les individus non mutants.\\



Somme toute, notre travail a mené à la mise au point d'une approche originale de détection de la recombinaison à haute résolution et à faible coût chez des individus uniques.
Cette approche ouvre la voie à l'étude plus poussée des gènes impliqués dans le processus de recombinaison et nous a permis de mieux comprendre la façon dont la recombinaison et la conversion génique biaisée opèrent chez les mammifères.


% Ainsi, notre étude a permis d'observer l'étendue de la variation du taux de recombinaison entre les points chauds et

}





%
%
% inkscape
% contour rouge
% 310a33ff
% couleur rouge
% 993333ff
% couleur jaune
% efbc00ff
% contour jaune
% d59c00ff
% Image souris
% https://www.google.com/imgres?imgurl=https%3A%2F%2Funixtitan.net%2Fimages%2Fmice-clipart-silhouette-2.png&imgrefurl=https%3A%2F%2Funixtitan.net%2Fexplore%2Fmice-clipart-silhouette%2F&docid=WP3tqoNXVwdgBM&tbnid=lJVmSkpwVUXs6M%3A&vet=12ahUKEwjLgcqR1IbjAhVOzYUKHdNQC8c4rAIQMygVMBV6BAgBEBY..i&w=591&h=410&bih=848&biw=1860&q=house%20mouse&ved=2ahUKEwjLgcqR1IbjAhVOzYUKHdNQC8c4rAIQMygVMBV6BAgBEBY&iact=mrc&uact=8#h=410&imgdii=lJVmSkpwVUXs6M:&vet=12ahUKEwjLgcqR1IbjAhVOzYUKHdNQC8c4rAIQMygVMBV6BAgBEBY..i&w=591
% https://omaharentalads.com/explore/mice-clipart-silhouette/
%
%
%
%
%

\end{abstract}

%%%%% MINI TABLES
% This lays the groundwork for per-chapter, mini tables of contents.  Comment the following line
% (and remove \minitoc from the chapter files) if you don't want this.  Un-comment either of the
% next two lines if you want a per-chapter list of figures or tables.
\dominitoc % include a mini table of contents
%\dominilof  % include a mini list of figures
%\dominilot  % include a mini list of tables

% This aligns the bottom of the text of each page.  It generally makes things look better.
\flushbottom

% This is where the whole-document ToC appears:
\tableofcontents%\otherpagedecoration

\listoffigures%\otherpagedecoration
	\mtcaddchapter
% \mtcaddchapter is needed when adding a non-chapter (but chapter-like) entity to avoid confusing minitoc

% Uncomment to generate a list of tables:
\listoftables
	\mtcaddchapter

%%%%% LIST OF ABBREVIATIONS
% This example includes a list of abbreviations.  Look at text/abbreviations.tex to see how that file is
% formatted.  The template can handle any kind of list though, so this might be a good place for a
% glossary, etc.
% First parameter can be changed eg to "Glossary" or something. / List of Abbreviations
% Second parameter is the max length of bold terms.

\begin{mclistof}{Abbreviations}{3.2cm}

\item[F1 hybrid] First filial generation of offspring of distinct parental types.
\item[F2] Second filial generation. Results from a F1 $\times$ F1 cross.
\item[F3, F4, etc] Subsequent filial generations.
	% \otherpagedecoration


\item[CO] Crossing-over (or crossover).
\item[NCO] Non crossing-over (or non-crossover).
\item[PMS] Post-meiotic segregation.
\item[DNA] 
\item[DSB] Double-strand break.
\item[NHEJ]
\item[HR] Homologous recombination.
\item[HRR] Homologue recognition region.
\item[SC] 
\item[NE] Nuclear enveloppe.
\item[A]
\item[C]
\item[G]
\item[T]
\item[MMR]
\item[BER]
\item[kb]
\item[Mb]
\item[Gb]
\item[INM] Inner nuclear membrane.
\item[ONM] Outer nuclear membrane.
\item[PC] Pairing centre.
\item[DSB]
\item[ssDNA]
\item[dsDNA]
\item[LCA] Last common ancestor.
\item[PCR] Polymerase chain reaction.
\item[COA] Crossing-over assurance.
\item[COI] Crossing-over interference.
\item[COH] Crossing-over homeostasis.
\item[MHC] Major histocompatibility complex.
\item[RNA]
\item[RNAi] RNA interference.
\item[NDR] Nucleosome-depleted region.
\item[TSS] Transcription start site.
\item[DNM] \textit{De novo} mutation.
\item[UTR]
\item[CpG island]
\item[H3K4me3]

	Synaptonemal complex associated
\item[LE] Lateral element.
\item[CE] Central element.
\item[TF] Transverse filaments.
\item[SCP1,2,3]
\item[SYCE1,2]


	Recombination proteins
\item[ATM (kinase)] Ataxia telangiectasia mutated (kinase).
\item[MEI1,4]
\item[RPA]
\item[DMC1]
\item[RAD50, RAD51]
\item[MRE11]
\item[NBS1]
\item[HORMAD1] HORMA domain-containing protein 1.
\item[MER2,3]
\item[REC114]
\item[SPO11]
\item[REC8]
\item[BLM] Bloom syndrome RecQ helicase-like.
\item[TEX11] Testis-expressed sequence 11.
\item[ZIP3,4]
\item[RNF212] RING finger protein 212.
\item[MCM8,9] Minichromosome maintenance deficient 8, 9.
\item[MSH4,5] MutS protein homologue 4, 5.
\item[MLH1] MutL protein homologue 1.
\item[TEX11]
\item[HFM1]
\item[MUS81]
\item[MMS4]
\item[SRS2]


Recombination models
\item[HJ] Holliday junction.
\item[dHJ] Double-Holliday junction.
\item[SDSA]
\item[DSBR]
\item[NHEJ]
\item[SEI] Single-end invasion.
\item[D-loop]


	Genetic distances
\item[M] Morgan.
\item[cM] Centimorgan.
\item[SNP]


%CO–crossover, DSB–double strand break, HR–homologous recombination, HRR–homology recognition region, IR–ionizing radiation, NCO–non-crossover, NHEJ–non-homologous end joining, PC–pairing center, SC–synaptonemal complex



\end{mclistof}




% List of definitions
\begin{alwayssingle}\chapter*{Definitions}
	\addcontentsline{toc}{chapter}{Definitions}
	\thispagestyle{plain}
	\pagestyle{fancy}
	\fancyhead[LO]{\emph{Definitions}}
	\fancyhead[RE]{\emph{Definitions}}
	\fancyhead[RO,LE]{\emph{\thepage}}
	\setlength{\baselineskip}{\frontmatterbaselineskip}
	\begin{description}

		\item[Purebred] Bred from members of a recognized breed, strain, or kind without admixture of other blood over many generations.
		\item[Reciprocal cross] Breeding experiment designed to test the role of parental sex on a given inheritance pattern.
		\item[Gene conversion] A non-reciprocal recombination process that results in an alteration of the sequence of a gene to that of its homologue.
	\item[Chiasma] (plural chiasmata) an exchange (crossing-over) between paired chromatids, observed cytologically between diplotene and the first meiotic anaphase, from the Greek word \textit{\textgreek{χίασμα}}: “X-shaped cross”.%, which refers to two lines placed cross-wise, like an “X”.
		\item[Tetrad analysis] Analysis of the four products (gametes) resulting from one single meiosis event.
		\item[Haploid] Organism (or phase) displaying a ploidy of 1 ($n=1$), i.e.\ a single set of chromosomes.
		\item[Diploid] Organism (or phase) displaying a ploidy of 2 ($n=2$), i.e.\ two sets of chromosomes (which are paired).
		\item[Ploidy] The number of complete sets of chromosomes ($n$) in a cell. 
		\item[Phenotype]
		\item[Genotype]
		\item[Tetrad analysis]
		\item[Meiosis]
		\item[Mitosis]
		\item[Recombination]
		\item[Ascospore]
		\item[(Genetic) marker] % A molecular marker is a site of heterozygosity for some type of silent DNA variation not associated with any measurable phenotypic variation (Popa????)
		\item[Heteroduplex DNA]
		\item[Negative interference]
		\item[Conversion polarity (or polarised recombination)]
		\item[Post-meiotic segregation]
		\item[Gene conversion]
		\item[Gene linkage]
		\item[Crossing-over]
		\item[Gamete]
		\item[Interphase]
		\item[Homologous chromosomes]
		\item[Sister chromatids]
		\item[Chromosome]
		\item[Chromatid]
		\item[Crossing-over]
		\item[Non-crossover]
		\item[Homologous chromosomes]
		\item[Sister chromatids]
		\item[Allele]
		\item[Gene]
		\item[Muller's Ratchet]
		\item[Hill-Robertson]
		\item[C terminus, N terminus]
		\item[Locus (\textit{pl.} loci)]
		\item[Genetic marker]
		\item[Polymorphic]
		\item[Allele]
		\item[Haplotype]
		\item[Pedigree] A family tree drawn with standard genetic symbols, showing inheritance patterns for specific phenotypic characters.
		\item[Polymerase chain reaction]
		\item[ChIP-seq]
		\item[Heterochiasmy] The differential recombination rates between the sexes of a species.
		\item[Achiasmy] The phenomenon where autosomal recombination is completely absent in one sex of a species.


Meiosis-linked definitions
\item[Prophase]
\item[Metaphase]
\item[Anaphase]
\item[Telophase]
\item[Cellular division]
\item[Equatorial plate]





	\end{description}
\end{alwayssingle}
\mtcaddchapter{}




% The Roman pages, like the Roman Empire, must come to its inevitable close.
\end{romanpages}


%%%%% CHAPTERS
% Add or remove any chapters you'd like here, by file name (excluding '.tex'):
\flushbottom
%\cpart{Bibliography}
\part{Introduction}
\begin{savequote}[8cm]
“[…] if there is one event in the whole evolutionary sequence at which my own mind lets my awe still overcome my instinct to analyse, and where I might concede that there may be a difficulty in seeing a Darwinian gradualism hold sway throughout almost all, it is this event — the initiation of meiosis.”
% “[…] if there is one event in the whole evolutionary sequence at which my own mind lets my awe still overcome my instinct to analyse […], it is this event — the initiation of meiosis.”
	
\qauthor{--- W. D. “Bill” Hamilton, \textit{\usebibentry{hamilton1996narrow}{title}} \citeyearpar{hamilton1996narrow} }

	% \textlatin{Neque porro quisquam est qui dolorem ipsum quia dolor sit amet, consectetur, adipisci velit...}

% There is no one who loves pain itself, who seeks after it and wants to have it, simply because it is pain...
  % \qauthor{--- Cicero's \textit{de Finibus Bonorum et Malorum}}
\end{savequote}

\chapter{\label{ch:2-recombination-mechanistics}Meiotic recombination, the essence/substrate of heredity} 
%\otherpagedecoration

\minitoc{}


% \subsection*{Preamble — Laying the foundations for research on recombination}
% \subsection*{Preamble — Laying the grounds for research on recombination}
% \subsection*{Preamble — From understanding heredity to discovering recombination}
% \subsection*{Preamble — From the mystery of heredity to the discovery of recombination}
% \subsection*{Preamble — The discovery of recombination}
% \subsection*{Preamble — How the concept of recombination emerged (a little of history)}
% \subsection*{History of sciences — How the concept of recombination emerged}

\subsection*{Preamble — How the concept of recombination emerged}

\paragraph{A unexplained exception to Mendel's laws of heredity.}
Between 1857 and 1864, the Austrian monk Gregor Mendel (1822--1884) undertook a series of hybridisation experiments on the garden pea plant \textit{Pisum sativum}. This led him to describe the idea of an “independent assortment of traits” \citep{mendel1996experiments}, thereby proving the existence of paired “elementary units of heredity” (i.e.\ genes) and establishing the statistical laws governing them.
His work remained unrecognised by the scientific community for several decades but was finally rediscovered in the early twentieth century when three botanists (Hugo de Vries (1845--1935), Carl Correns (1864--1963) and Erich von Tschermak (1871--1962)) independently confirmed his findings \citep{dunn2003gregor}.
Meanwhile, William Bateson (1861--1926) fiercely defended Mendel's thesis in \textit{\usebibentry{bateson1902mendel}{title}} \citep{bateson1902mendel} against his contemporary biometricians \citep[reviewed in][]{bateson2002william}, thus spreading Mendel's view into the scientific world.
% Meanwhile, as Mendel's attempts to explain the mechanisms of heredity lacked scientific support and contradicted Galton's “Law of Ancestral Heredity”, many scientists were skeptiks

A few years later, Bateson noticed exceptions to Mendel's principles of independent assortment: some crosses generated certain phenotypes in far excess from the expected Mendelian ratios \citep{bateson1905experimentalpea}. This led him and his collaborators to propose that certain traits were somehow coupled with one another, although they did not know how \citep{bateson1905experimental}. 
%They had discovered what is now called “genetic linkage”.

% Conclure: methode d'exploration de la genetique nee + desequilibre de liaison = on voit des differences aux lois donnees par Mendel. donc a comprendre.

\textbf{Ajouter la decouverte de l'ADN comme support de l'information genetique?}

% \citet{suzuki1986introduction}
% \paragraph{Genetic information is encoded in DNA.}
% In the meantime,

% PLAN: DNA SUPPORT DE L'INFORMATION GENETIQUE:
% 1869: Friedrich Miescher: discovery of nuclein (=DNA) + Altmann 1889: named “nucleid acid”
% + 1883: Eduard Van Beneden: discovery of chromatin (=DNA+RNA+proteins that make up a chromosome)
% 1929: Phoebus Levene: DNA composed of four bases (ATCG)+phosphate+sugar
% + 1941: Chargaff's rule: A=T and c=G
% + 1953: James Watson and Francis Crick: model of double-helix DNA






\paragraph{The chromosomal theory of inheritance.}
In the meantime, it had been understood that cells derived from other cells, but the exact process was unknown. 
To understand it, Walther Flemming (1843--1905) used stains to intensify the contrasts of cell contents observed through microscopy and identified a substance located within the nucleus, which he named “chromatin” (from the greek word \textit{\textchi\textrho\textomega\textmu\textalpha}: “color”).
He described precisely the movements of chromosomes during cell division (which he termed “mitosis”), thus providing a mechanism for the distribution of nuclear material into daughter cells during mitosis \citep{flemming1879contributions}.

Theodor Boveri (1862--1915) went one step further by demonstrating the individuality of chromosomes in the roundworm \textit{Ascaris megalocephala}, which allowed him to suggest that the chromosomes of the germ cells are involved in heredity \citep{boveri1888zellen}.
In addition, he showed that the egg and the spermatozoon contribute the same number of chromosomes to the new individual, thus providing the first descriptions of meiosis \citep{boveri1890zellen}.
Walter Sutton (1877--1916) independently came to the same conclusion at about the same time: he enunciated the chromosomal theory of inheritance with the following words closing his \usebibentry{sutton1902morphology}{year} paper: “I may finally call attention to the probability that the association of paternal and maternal chromosomes in pairs and their subsequent separation during the reducing division […] may constitute the physical basis of the Mendelian law of heredity” \citep{sutton1902morphology}.

However, this theory was debated in the scientific community, because there was yet no direct proof of a link between the inheritance of traits and the segregation of chromosomes.\\ 

In parallel, based on cytological observations of chromosomes, Frans Janssens (1863--1924), a priest also known as the “microscopy wizard” for he mastered the process, developed the idea that the chromosomes' “filaments [chromatids] are involved in contacts that can modify their organization from one segment to the next” which “will generate new segmental combinations” in his \textit{Chiasmatype Theory} \citep{janssens1909theorie}.



\paragraph{Thomas Hunt Morgan's theory of gene linkage and crossing-over.}
% Morgan ne croit à aucune des theories. Pourtant, va etre le lien de ces trois. 
% Morgan decouvre une mutation chez la droso (explication globale dessus) => suspecte la liaison.
% Il decrit le mecansisme dans livre 1915.
% 
In 1909, Thomas Hunt Morgan (1856--1945) expressed his strong skepticism of the Mendelian theory of inheritance in his very derisive article \textit{\usebibentry{morgan1909factors}{title}} \citep{morgan1909factors} and doubted the chromosomal basis of heredity \citep[reviewed in][]{koszul2012centenary}.
Little did he know at the time that he was to become the main craftsman of the reconciliation of these two theories.




\begin{figure}[h]
	\centering
	\includegraphics[width = 1\textwidth]{figures/chap2/morgan-drosophila-cross-results.eps}
	% \includegraphics[width = 0.7\textwidth, trim = 0cm 0cm 11.65cm 0cm, clip]{figures/chap2/morgan-drosophila-cross-results.eps}
	\caption[Explanation of the results from reciprocal crosses between red-eyed and white-eyed \textit{Drosophila}]
	{\textbf{Explanation of the results from reciprocal crosses between red-eyed (red) and white-eyed (white) \textit{Drosophila}.} 
		\par In the first cross (left), a red-eyed purebred female is crossed with a white-eyed male, resulting in F1 hybrids made of heterozygous red-eyed females bearing both the dominant (w\textsuperscript{+}) and the recessive (w) alleles and red-eyed males bearing only the dominant (w\textsuperscript{+}) allele. The inbreeding of F1 individuals results in a F2 generation with a 3:1 ratio of red-eyed:white-eyed individuals, all white-eyed individuals being males.
		\par In the second cross (right), a white-eyed female is crossed with a red-eyed purebred male, resulting in F1 hybrids made of heterozygous red-eyed females bearing both the dominant (w\textsuperscript{+}) and the recessive (w) alleles and white-eyed males bearing only the recessive (w) allele. The inbreeding of F1 individuals results in a F2 generation with a 2:2 ratio of red-eyed:white-eyed individuals, half of white-eyed being males and half being females.
		\par The results of these two crosses show that the gene coding for eye color is located on the female sexual chrosome (X). The fact that results in the F2 progeny differ according to the direction of the cross (($\frac{w\textsuperscript{+}}{w\textsuperscript{+}}$) $\times$ (w) or ($\frac{w}{w}$) $\times$ (w\textsuperscript{+})) is a typical signature of linkage disequilibrium between the observed trait (eye color) and the sex chromosomes.
		\par This figure was reproduced from \citet{suzuki1986introduction} (Rights: WH Freeman \& Co).
	}
\label{fig:morgan-drosophila-cross-results}
\end{figure}

\clearpage




In his famous “fly room” where he bred \textit{Drosophila melanogaster} fruit flies, he found an unusual male white-eyed individual. Crossing it with purebred red-eyed females yielded red-eyed males and females F1 hybrids, — a typical result proving that the white eye color is a recessive trait. Unexpectedly, after inbreeding the heterozygous F1 progeny, he discovered that the traits of the F2 offspring did not assort independently: all white-eyed flies were males (Figure~\ref{fig:morgan-drosophila-cross-results}, left). 
However, when he crossed the white-eyed male with F1 daughters, he found both male and female white-eyed flies (Figure~\ref{fig:morgan-drosophila-cross-results}, right), thus showing that the white eye color was not lethal for females.

He immediately hypothesised that eye color was connected to the sex determinant \citep{morgan1910sex} and, as these findings were consistent with the idea that genes were physical objects located on chromosomes, Morgan soon came up with the idea of genetic linkage, i.e.\ the fact that two genes closely associated on a chromosome do not assort independently \citep{morgan1911random}. 
He also suggested that this coupling dependended on the distance between genes: “we find coupling in certain characters, and little or no evidence at all of coupling in other characters; the difference depending on the linear distance apart of the chromosomal material that represent the factors.”

With three of his students (Alfred Sturtevant (1891--1970), Hermann Muller (1890--1967) and Calvin Bridges (1889--1938)), he summarized all the evidence in \usebibentry{morgan1915mechanism}{title} which constitutes one of the most important books in the whole history of genetics \citep{gayon2016mendel}. 
There were two major propositions in that book. 
First, the recognition that Mendelian factors — Morgan would soon call them “genes” — are physical portions of chromosomes. This brought a mechanistic support to Mendel's “law of segregation” (according to which the zygote inherits only one version of each gene from each parent) and to the so far unexplained exception to Mendel's “law of independent assortment fo traits”: when two genes are located on the same chromosome, they have to segregate together — and thus the law does not apply to this special case. 
Second, they proposed that the linkage between genes located on the same chromosome could sometimes break, through the process of what Morgan called ”crossing over” (Figure~\ref{fig:morgan-CO}). This was to take place at the positions of the chiasmata previously observed by Janssens \citep{janssens1909theorie}. Morgan, together with Edgar Wilson (1908--1992) later crafted structures of crossing overs with clay \citep{wilson1920chiasmatype}.\\

Altogether, with the idea of recombination and crossing over, Morgan had fused three theories: gene linkage (the major exception to Mendel's laws of heredity), the chromosomal theory of inheritance and the chiasmatype theory. This triggered a real revolution in biology and marked the real commencement of genetics. His major contribution through his work on \textit{Drosophila} won him the \textit{Nobel Prize in Physiology or Medicine} in 1933.

It was only ten years later that Harriet Creighton (1909--2004) and Barbara McClintock (1902--1992) would bring the first proof of that theory by correlating cytological and genetic exchanges in the maize \citep{creighton1931correlation}.

\begin{figure}[h]
	\centering
	\includegraphics[width = 1\textwidth]{figures/chap2/morgan-CO-1916.eps}
	% \includegraphics[width = 0.7\textwidth, trim = 0cm 0cm 11.65cm 0cm, clip]{figures/chap2/morgan-drosophila-cross-results.eps}
	\caption[Original drawing of crossing over in \textit{\usebibentry{morgan1915mechanism}{title}} \citep{morgan1915mechanism}]
	{\textbf{Original drawing of crossing over in \textit{\usebibentry{morgan1915mechanism}{title}} \citep{morgan1915mechanism}.} 
		\par Original legend: “At the level where the black and the white rod cross in A, they fuse and unite as shown in   D. The details of the crossing over are shown in B and C.”
		This drawing symbolises the reconciliation between Mendel's and the chromosomal theories of inheritance.
	}
\label{fig:morgan-CO}
\end{figure}

\clearpage


% Conclure: on comprend que il existe des chromosomes et des recombinaisons, qui expliquent donc les deviations par rapport a l'attendu de Mendel. Le concept de recombinaison est ne.












\paragraph{The study of fungal products of meiosis led to the key concepts of the recombination mechanism.}


% Conclure: on voit des differences dans la recombinaison. Donc a expliquer. En particulier, BGC.



\paragraph{And research still goes on…}

% Des inconnues sur l'origine evolutive de la meiose. Des inconnues sur le role de la recombinaison...






%%%%% Mon plan (Ordonne par theme)
%HEREDITY:
1865: Mendel: experience des pois a expliquer.
+ Bateson 1905: linkage diesquilibrium in certain cases. (ou dans meiose et recombinaison)

%DNA SUPPORT DE L'INFORMATION GENETIQUE:
1869: Friedrich Miescher: discovery of nuclein (=DNA) + Altmann 1889: named “nucleid acid”
+ 1883: Eduard Van Beneden: discovery of chromatin (=DNA+RNA+proteins that make up a chromosome)
1929: Phoebus Levene: DNA composed of four bases (ATCG)+phosphate+sugar
+ 1941: Chargaff's rule: A=T and c=G
+ 1953: James Watson and Francis Crick: model of double-helix DNA

%MEIOSE AND RECOMBINATION:
1911/1913 Thomas Hunt Morgan: concept de linkage desiquilibrium + crossover. Overall, fusion of three theories: “the chromosome theory of inheritance,” imparted by Wilhelm Roux, Walther Flemming, Theodor Boveri, and Walter Sutton; “gene linkage,” an exception to Mendel’s law of independent assortment, first reported by Carl Correns; and the “chiasmatype theory,” derived from Frans Janssens' cytological observations of meiotic chromosomes.
+ Les trois autres theories avant. 
1931: Proof of the crossover theory came from Harriet Creighton and Barbara McClintock (Creighton and McClintock 1931), who were able to correlate cytological and genetic exchanges in maize.

%CONSEQUENCES(BGC):
fungal genetics experiments (to get all four products of meiosis)
our key concepts that formed the foundation of molecular models of recombination:gene conversion, an exception to Mendel’s principle of segregation, signaled a local nonreciprocal transfer of genetic information (Winkler 1930; Lindergren 1953; Mitchell 1955); postmeiotic segregation (PMS) indicated the presence of heteroduplex DNA (Olive 1959; Kitani et al. 1962); polarity gradients of gene conversion lead to the idea that recombination initiated from pseudofixed sites (Lissouba and Rizet 1960; Murray 1960); and the strong correlation between gene conversion/PMS events and crossing-over led to the proposal that these processes were mechanistically linked (Kitani et al. 1962; Perkins 1962; Whitehouse 1963).


%MINI CONCLU:
Les modeles toujours que des modeles (cf partie 3)
Encore bcp de points d'interrogation sur la meiose et la recombinaison et le role precis des molecules impliquees. Donc toujours de la recherche dessus.
En particulier, le BGC qu'on etudie dans cette these
Role de la recombinaison comme reparation des cassures pour replication  (cf topo Daniel)


% Dans partie meiose (partie 2):
Origine evolutive de la meiose debattue (transfo bacterienne ou mitose?)
Greek word






\section{Cytological aspects of meiosis}
+ 2 possibilities. EIther meiosis comes from mitosis, or from transfomration https://academic.oup.com/bioscience/article/60/7/498/234118

Stages + chromatin state + checkpoints + chiasmata + recombination nodules + synaptonemal complex

\section{Chronology of meiotic recombination}
cf Baudat et de Massy + ma presentation a ce sujet
\subsection{Initiation of recombination}
\subsection{Meiotic DSB repair}
\subsection{Resolution}

\section{Models of recombination}

\section{The importance of meiotic recombination}


+ Parler de Gene conversion
+ Parler de Male vs Female meiosis  https://cellbiology.med.unsw.edu.au/cellbiology/index.php/Meiosis
+ errors in meiosis
+ regulation of meiois (cyclins)






\section*{Sorte de plan / Liste d'idees}
historic
initiation of recombination: 
- Homologue pairing / interhomolg interactions
- determinsation localisation
- formation DSB / programmed DSB formation
Meiotic DSB repair:
- Homoology search
- synapsis between homologues
- models of recombination
- Resolution into CO/NCO
- Dissolution of the SC
Crossover control: 
- assurance / interference
- differentiation CO/NCO
Chromatin state: shapes the recombination landscape (ou dans position des hotspots): nucleosome occupancy + meiotic chromosome architecture. 
Importance of meiotic recombination: Genetic disorders otherwise + exemple de l'un qui a perdu PRDM9 mais qui n'est pas stérile pour autant. 
Checkpoints
strand asymmetry
DNA polymerases
Gene conversion ici? + non-allelic gene conversion (et un impact sur la détection de)



Deuxieme chapitre:
Methodological approaches to study recombination
Variation of recombination rates wtithin genomes and among species
evolvability of recombination rates



\section*{Notes temporaires}

%%%%% Modele de Odenthal-Hesse (chapitre sur recombinaison)

% \section{Chronology of meiotic recombination}
% \subsection{Programmed DSB formation}
% \subsection{Strand invasion and junction molecule formation}
% \subsection{Mismatch repair}
% \subsection{Resolution}
%
% \section{Models of recombination}




%%%%% Modele de Papier Baudat de Massy 2013

% \section{Initiation of recombination}
% \subsection{Homologue pairing}
% \subsection{Programmed DSB formation}
%
% \section{Meiotic DSB repair}
% \subsection{Homology search}
% \subsection{Synapsis between homologues}
% \subsection{Models of DSB repair}
%
% \section{Resolution CO/NCO}
% \subsection{Differentiation CO/NCO}
% \subsection{CO interference}



Plan chronologique
\begin{itemize}
	\item Mammalian meiosis (overview of the cycle)
	\item (Zoom sur Prophase 1)
	\item Leptotene stage: Initiation of recombination (Homologue pairing before DSB + Determination localisation DSB + DSB) + Miotic DSB repair (homology search)
	\item Zygotene stage: Meiotic DSB repair (Synapsis between homologues + Start resolution CO/NCO)
	\item Pachytene stage: Meiotic DSB repair (Resolution CO/NCO)
	\item Pachytene + Zygotene stages: Preparation to metaphase I (dissolution of SC)

\end{itemize}

Plan Neil Hunter the essence of heredity
\begin{itemize}
	\item Meiosis and the roots of recombination research
	\item Molecular models of meiotic recombination
	\item Interhomolog interactions
	\item Programmed DSB formation
	\item Crossover control (Assurance and interference, differentiation CO/NCO, pro-CO role of the synaptonemal complex, recombination associated DNA synthesis)
	\item Resolving, disolving and unwinding joint molecules to implement CO and NCO fates (differential timing and regulation of CO and NCO formation, MutL and EXO1=CO-specific resolving factor, MUS81 enzymes = role in meiotic joint molecule processing, STR/BTR ensemble as master regulators of meiotic joint molecule metabolism, SLX4-associated endonucleases and he GEN1 resolvase, SMC complex facilitates joint molecules formation and resolution, implementing NCO formation)
	\item Clinical significance of meiotic recombination (Aneuploidy CO and advancing maternal age, meiotic recombination and genomic disorders, defective recombination and infertility)


\end{itemize}

Plan Mammalian Meiotic Recombination: A Toolbox for Genome Evolution (https://www.karger.com/Article/FullText/452822):
\begin{itemize}
	\item Recombination and he repair of DSBs (Organization of meiotic chromosomes: importance of chromosomal axes, molecular events involved in he formation and repair of DSBs)
	\item Methodological approaches to he study of recombination
	\item Genetic and epigenetic marks of DSBs and recombination hotspots 
	\item Variation of recombination rates within genomes and among species (Variability at the chromosomal level, variation of fine-scale recombination maps)
	\item Evolvability of recombination rates (Chromosomal rearrangements as recombination modifiers)
\end{itemize}

Plan de Hotposts for initiation of meiotic recombination (https://www.ncbi.nlm.nih.gov/pmc/articles/PMC6237102/)
\begin{itemize}
	\item Defining DSB hotspots
	\item Chromatin shapes the meiotic DSB landscape (Nucleosome occupancy, meiotic chromosome architecture)
	\item Meiotic DSB and crossover distributions
	\item PRDM9 and H3K4me3
	\item The hotspot paradox
	\item Recombination initiation in repetitive sequences
	\item Byond hotspots: DSB-dependent spatial regulation


\end{itemize}


Mechanismes moleculaires precis + molecules impliquees
\begin{itemize}
	\item Appariement des chromosomes
	\item Formation du DSB
	\item Reparation CO/NCO
	\item Tous les modeles de resynthese des brins
	\item Observation des parametres de recombinaison chez la levure, souris
	% \item https://www.cell.com/current-biology/pdf/S0960-9822(06)01257-7.pdf: surveillance of breaks = checkpoints
	\item interference des CO
	\item Notion de strand asymmetry
	\item DNA polymerases (https://www.ncbi.nlm.nih.gov/pmc/articles/PMC5295669/)
	\item https://www.karger.com/Article/FullText/452822: mammalian eiotic recombination: a toolbox for genome evolution

\end{itemize}


Molecules tres importantes sur lesquelles insister:
\begin{itemize}
	\item DMC1
	\item RAD51
	\item (PRDM9)
	\item Spo11
	\item MUS81
	\item MLH1
	\item HFM1
\end{itemize}

Autre:
\begin{itemize}
	\item Non-allelic gene conversion
	\item Recombining without hotspots (https://www.ncbi.nlm.nih.gov/pmc/articles/PMC4684701/)
	\item Knockout of PRDM9 (http://science.sciencemag.org.inee.bib.cnrs.fr/content/352/6284/474)

\end{itemize}





%\cpart{PART 1}
% \include{text/ch1-intro}
%\cpart{PART 2}
% \part{The name of that part is that}
%\cpart{PART 2}
% \include{text/ch2-litreview}


%% APPENDICES %% 
% Starts lettered appendices, adds a heading in table of contents, and adds a
%    page that just says "Appendices" to signal the end of your main text.
% \startappendices
% Add or remove any appendices you'd like here:
% \include{text/appendix-1}


%%%%% REFERENCES

% JEM: Quote for the top of references (just like a chapter quote if you're using them).  Comment to skip.
% \begin{savequote}[8cm]
% The first kind of intellectual and artistic personality belongs to the hedgehogs, the second to the foxes \dots
%   \qauthor{--- Sir Isaiah Berlin \cite{berlin_hedgehog_2013}}
% \end{savequote}
%
\setlength{\baselineskip}{0pt} % JEM: Single-space References

% {\renewcommand*\MakeUppercase[1]{#1}%
% \printbibliography[heading=bibintoc,title={\bibtitle}]}

\renewcommand\bibname{References}
\bibliographystyle{apalike}
\bibliography{references}


\end{document}
